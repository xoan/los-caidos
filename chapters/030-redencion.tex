No sólo cambiar es difícil, tampoco lo es que otros asuman ese cambio de manera externa. La incomodidad, la sensación de estar frente a otro, frente a un desconocido, es muchas veces un obstáculo insalvable.

Tanto como el odio y el rencor debidos a hechos pasados\dots

\fancyparbreak~---Esto está abocado a fracasar ~---se limitó a decir Sky, mirando la celda del Aquerón donde estaba aún inconsciente el que una vez fue uno de sus peores enemigos, si no el peor.

~---Entiendo tu recelo, James ~---interpeló Scream, mirando al villano en decadencia, pensativo, preguntándose si Starr Miles no le hubiera mirado a él mismo de manera similar cuando le sacó de las calles.

~---¿Cómo te sentirías si hubieran intentado reclutar a Silenciador para que formara parte de Los Caídos, John? ~---dijo Sky con rabia poco contenida.

~---Él no le hizo nada a tu hermano, James. Estoy seguro de ello. Llámalo intuición, llámalo corazonada. Pero creo que Shockman no es tanto lo que parece.

~---No le llames por su apellido, llámalo como siempre se le ha conocido. Éxeter. Uno más de los villanos que amenazaban Ernépolis en los viejos tiempos.

~---Esos tiempos ya no existen, y las cosas no son siempre cuestión de blanco o negro. Escucha, James. Sé perfectamente que esto no es fácil para ti. Pero confía en mi criterio, por favor. Permíteme intentar que Los Caídos puedan realmente servir para poner fin a las amenazas que se ciernen sobre la ciudad de un modo más filosófico que limitándonos a romper mandíbulas de criminales de poca monta.

Sky dejó que el silencio hablara durante un rato. Su sonrisa irónica estaba muy lejos de surgir en algún momento de aquella tensa conversación.

~---Está bien, John. Confiaré en ti por esta vez. Aunque preferiría que nunca me lo hubieras dicho.

~---No quería que te enteraras por terceros ni de manera más brusca. Yo mismo te prometo que si Shockman vuelve a las andadas le pondremos bajo custodia de nuevo, y si hace falta le encerraremos en el Aquerón hasta que exista una celda especial para él.

~---¿Qué hay de la seguridad en estos momentos?

~---Tranquilo, el asunto está controlado. Ahora ya sabes lo que toca.

~---Sí ~---dijo Sky reflexivo, recordando un pasado cada vez más lejano en su memoria~---. Todos hemos pasado por algo similar.

\parbreak
Warren Shockman despertó y se encontró con que estaba en el interior de una celda diáfana por completo. Sin muebles, sin ventanas. Las luces incrustadas tras largos paneles acristalados, sin apenas resquicios por ninguna parte. La única entrada era una robusta compuerta que se hundía varios centímetros por debajo del nivel del suelo.

Un olor que conocía bien envolvía el ambiente. Un olor que, si bien no suponía peligro real para él, era sin duda la manifestación de que estaba metido en un problema gordo.

Insecticida.

Le habían calado pero bien. Joder, sabían lo que podía hacer. Aunque eso sí, tenía muy claro que su captor no era ninguna clase de supervillano ni nada parecido. Si así fuera, se hubiera limitado a matarle sin más. Nada de encerrarle, nada de hacerle salir de su inconsciencia. El enemigo bueno es el enemigo muerto.

A no ser que pensara como él, claro. Pero de estar en su lugar, nunca hubiera sido tan evidente al respecto como su ausente enemigo.

Era una obviedad comprobar algo así, pero como era de esperar, no tenía armas. Tampoco su rata estaba con él, otro arma tan útil como cualquier otra. Pero el chisme de emergencia sí estaba. Podía notarlo entre los dedos del pie, camuflado como una segunda piel. Un ligerísimo chip, un circuito de alcance limitado, pero que podía resultarle útil en ese momento.

Debido a su sencillez y simplicidad, no podía elaborar más que una orden con el mismo, y a criaturas muy simples. La orden era que fueran hacia él, y las criaturas cucarachas.

Volvió a oler a su alrededor. El insecticida penetraba en el ambiente y, sin duda, hubiera provocado mareos y tal vez vómitos a alguien menos acostumbrado que él, alguien que en muchos sentidos, en muchos aspectos, ya era en parte un insecto, en términos mentales y de comportamiento.

Se habían empleado a fondo, sin duda. Pero sabía bien que era insuficiente por completo. No se puede acabar con sus pequeños aliados de negro caparazón queratinoso. Están en todas partes, en todos los rincones, por limpios y ordenados que puedan parecer. Cuando el mundo termine, cuando todas las batallas entre humanos dejen de tener sentido, ellos estarán allí, conquistadores, vencedores e imperecederos, alzándose sobre los restos de los que creían ser los amos de la tierra que pisaban.

Activó el diminuto circuito y esperó, apoyado en una pared. Aquella era la parte que más odiaba, estar quieto, inactivo. Sólo dejando que los acontecimientos fluyeran por sí solos, a su favor.

El primer ruido que anunció la llegada provino de una de las paredes contiguas a su posición. Notaba cómo pasaban por el otro lado, se retorcían, luchaban por acercarse aunque el insecticida hacía mella en muchas de ellas. Alguna que otra lograba penetrar, y se acercaba a él hasta quedarse más o menos quieta en torno a la punta de su zapato, origen de la señal que la había llevado a realizar aquel sorprendente peregrinaje de motivos desconocidos para ella.

Pero las que venían de las paredes no eran las que le interesaban a Shockman. No, eran las que sonaban por encima de su cabeza las que atraían por completo su atención.

Se suele pensar en lo insignificante que un insecto puede llegar a ser. Pero a su escala, esos seres son auténticos colosos gargantuescos de fuerza, agilidad y destreza incomparables. Y gracias a un mero asunto de decoherencia la suma combinada de muchos de ellos puede llegar a hacer cosas inimaginables.

Por ejemplo, combar una chapa de metal por mera acumulación de cientos de ellos sobre un mismo punto del techo.

Cuando la estructura se hubo separado por un lado unos veinte centímetros de su posición, formó una rampa por la que el ejército de cucarachas comenzó a deslizarse y caer lentamente hacia la celda. Shockman esperó unos segundos hasta que sus rescatadores dejaron el acceso libre, y luego de eso desconectó el dispositivo. Las criaturas se marcharon, desconcertadas, y le dejaron solo de nuevo en la celda, con alguna pequeña excepción. Después de eso dio un salto esforzado y alcanzó el borde de la chapa. Esa acción le produjo cortes en las manos, y entrar por la chapa magulló considerablemente su torso, pues el espacio era muy reducido. Pero no era nada que no hubiera tenido que experimentar en algún momento del pasado.

El conducto estaba sorprendentemente cuidado teniendo en cuenta lo fácilmente que esa clase de sitios solían acumular porquería, y más en una ciudad como aquella. De hecho, a medida que avanzaba por él, Shockman empezó a tener una sospecha que su mente era, sencillamente, incapaz de asimilar.

Cuando encontró una rejilla que accedía a una sala oscura hizo palanca con los pies y empujó con fuerza hasta saltarla de sus goznes. No había sido hecha para durar, sin duda.

Una vez en la sala, carente también de mobiliario, aunque era difícil discernir algo así ante tanta penumbra, empezó a entender lo que había estado haciendo.

Jugar al juego. Seguir las normas. Superar la prueba.

\emph{Impresionante} ~---comentó la sombra desde algún lugar indeterminado de la habitación~---. \emph{No tenía claro que llegaras a encontrar el conducto. Me pregunto qué hubiera pasado bajo circunstancias más inflexibles.}

~---No me engañas, justiciero ~---proclamó Shockman, furioso por haber estado siendo manipulado~---. No eres ninguna clase de criminal sofisticado ni nada por el estilo. Me hubieras matado de lo contrario.

\emph{Deberías hablar por ti también, entonces. He examinado tu arma, Shockman. Modalidad de aturdir. No es propio de una amenaza social.}

~---Sí si es lista y astuta, justiciero. Si no mato a nadie, los delitos en mi contra son de menor categoría.

\emph{Tal vez eso convenza a otros, pero no a mí. Sé cómo funciona la mente de los criminales. No les importan las consecuencias de sus actos. Son viciosos, perversos y sicopáticos. Son un cáncer que asedia mis dominios. Tú no eres así. Lo sé. Apostaría que nunca has matado a nadie.}

~---Ten cuidado con lo que dices, amigo ~---amenazó Shockman cerrando los puños.

\emph{¿Qué ocurre, te he enfadado acaso? ¿Temes que se lo cuente a otros y se vaya al garete tu reputación?}

~---¡Cállate! ~---insistió Shockman, saltando hacia la fuente del sonido. Pero no había nada allá de donde provenía, al menos a simple vista.

\emph{Tú nunca has sido un villano de verdad, Shockman. En tu interior fluye la rebeldía, sin duda, y un profundo cinismo. Pero no eres como ellos.}

~---¿Qué soy, entonces? ~---replicó el aludido, cansado después de tanto esfuerzo y juego mental.

\emph{Eres como yo. Una sombra. Un desarraigado. No perteneces a ningún lugar. El mundo que te vio nacer ya no existe, y no formas parte del nuevo orden.}

Shockman vio cómo su interlocutor salía de las sombras, y por primera vez una sensación extraña de incertidumbre invadió su frío y duro corazón.

\emph{Eres un caído, Warren Shockman. Pero no estás solo para afrontar ese viaje sin retorno.}

\parbreak
Si normalmente había que tomar grandes precauciones a la hora de entrenar e incorporar un nuevo miembro al equipo de Los Caídos, en el caso de Shockman esas precauciones rozaban el recelo más absoluto. Durante mucho tiempo sólo Scream fue el enlace de Shockman con el resto de la organización, pues no era aún el momento de introducirle a la misma. Como era lógico, muchas de las lecciones que Scream recibió de Miles trató de usarlas con Shockman, del mismo modo que había hecho con otros como Sam Grove. Sin embargo Shockman era un alumno complicado. Tremendamente aventajado, sin duda. Listo, aunque no demasiado inteligente. Experto en su campo y dispuesto a aprender y analizar su entorno. La astucia, una de sus mejores armas.

Pero Shockman tenía su propio aprendizaje interno, y era difícil de eludir. Para él la debilidad era algo que nunca debía manifestarse, no sólo al enemigo, ni siquiera al aliado, porque en su mundo, hasta ese momento, tal cosa nunca había existido. Todo potencial aliado era también un potencial traidor. En un reino de depredadores, debía ser otro más, o al menos parecerlo a la perfección.

Scream no pudo evitar pensar que en ese sentido tenía ya mucho aprendido a la hora de convertirse en miembro de Los Caídos. Shockman se había pasado la vida entera fingiendo, aparentando ser algo que no era del todo por instinto de supervivencia. Esa parte del entrenamiento estaba más que asimilada.

Era lo referente al trabajo en equipo lo que preocupaba a Scream. Llegó finalmente el día en que tuvo que sacarle de las salas de entrenamiento, dar el paso definitivo de confianza y mostrarle cómo funcionaba la organización. Algo que podía tener nefastas consecuencias si Shockman decidía huir y divulgar ese conocimiento.

Pero era un riesgo que como líder debía asumir. Mirar más allá de las apariencias, no dejarse llevar sólo por los ecos del pasado. Si juzgaban a un hombre sólo a partir de sus errores, entonces Los Caídos no eran mejores que sus muchos y muy despiadados enemigos.

Pero eso no quería decir que Shockman aceptara de buen grado lo que se estaba haciendo por él. De hecho una primitiva y básica emoción le dominó cuando Scream le ofreció que se uniera a ellos. Una que el propio Scream conocía muy bien.

Orgullo. Orgullo más allá de todo razonamiento.

~---Me ofreces ser uno de vosotros, dices ~---replicó Shockman en el hemiciclo del Aquerón, sin mirar a ninguna parte en particular~---. Yo, que fui un enemigo de algunos de tus amigos, ahora podría ser parte de tu organización.

~---No es mi organización, Shockman. Aquí no hay individualismos, al menos no en la escala estratégica. Eso no quiere decir que formes parte de una secta, ni nada parecido. Ya hay quien abandonó la organización por su propia voluntad, y nadie se lo ha impedido.

~---El Jefe Sky\dots\ ~---susurró Shockaman. Scream no había dado nombres, pero no era tampoco muy difícil llegar a tal deducción si se poseían las pistas e incentivos adecuados~---. ¿Qué me dices de ellos, eh? ~---señaló a lo lejos, donde varios miembros de escuadrón miraban hacia ellos dos con recelo~---. Nunca aceptarán a uno como yo. Métetelo en la cabeza, John Scream. Somos de universos distintos.

~---Tú eres el que estás empeñado en pertenecer a otro mundo ~---interpeló Scream moviéndose hacia sus subordinados~---. ¿Ocurre algo? ~---preguntó enfadado.

~---No, jefe, nada, lo sentimos ~---dijeron mientras seguían caminando.

~---Un equipo, dices. Ya lo veo, ya.

~---No esperarás que confíen ciegamente en alguien como tú después de todo lo que han vivido. Tendrás que ganarte su respeto.

~---Yo no quiero ni deseo el respeto de nadie, entérate.

~---¿Ni siquiera el mío? Te saqué de aquel callejón varado, y sabíamos quién eras cuando participabas en el torneo. Pero no tratamos de ir a por ti ni nada parecido. La gente puede cambiar. Nosotros lo hemos hecho.

~---De héroes ingenuos que patrullan por separado y al descubierto, a héroes ingenuos que patrullan juntos y en las sombras. Un gran cambio.

~---No te insistiré más, entonces. Si tan seguro estás de poder comerte el mundo tú solo, ahí tienes la puerta ~---señaló a los múltiples pasillos ramificados del Aquerón~---. Pero si sales por esa puerta y descubro que has vuelto a las viejas costumbres, iré a por ti sin género alguno de dudas.

~---Me parece perfecto ~---dijo Shockman entrando por una de ellas al azar, sin importarle demasiado dónde pudiera llevarle.

Scream no se movió de donde estaba, sólo se quedó de pie, allí, pensativo, hasta que Razorclaw se acercó a sacarle de sus conjeturas internas.

~---¿Le seguimos, John?

~---No hará falta ~---dijo Scream con un gesto de mano apenas esforzado~---. Yo también pensé muchas veces en irme, ¿sabes? Pensaba que con Aryn muerta ya no había nada por lo que mereciera la pena luchar.

~---Creo que ese sujeto no tiene a ninguna Aryn por la que dirigir un pensamiento noble.

~---No todos los héroes tienen que estar cortados por el mismo patrón. Algunos pueden estar donde menos te lo esperas.

Apenas unos segundos después de esas palabras Shockman reapareció de nuevo en el módulo principal, no sin cierta altivez en la manera de andar. Nuevamente fingiendo, pensó Scream. Fingiendo hasta el mismísimo fin de sus días.

Del bolsillo de Shockman salió su rata y corrió por su brazo hasta posarse junto a su hombro. Junto a Hades, uno de los pocos seres vivos que habían logrado burlar las defensas del cuartel.

~---Si me quedo, se queda conmigo ~---dijo señalando al roedor.

John Scream sonrió. Una sonrisa irónica que no desmerecía en nada a las que solía esbozar Sky.

~---No me opondré a ello. De hecho, tengo cierta sugerencia al respecto.

\parbreak
La idea de Scream era algo que no pilló por sorpresa a Shockman una vez el líder de Los Caídos le explicó con todo lujo de detalles cómo eran los dispositivos que solían llevar, desde los hologramas hasta los anuladores de fotones.

~---Quieres que haga un modelo universal de mi dispositivo para el traje ~---dedujo Shockman sin mucha dificultad.

~---Así es. Soy consciente de las limitaciones de fabricarlo a gran escala, y de que no será tan sofisticado como los que tú sueles llevar. Pero siempre has tenido que soportar el usar tus poderes sin mostrarlos abiertamente. Ahora eso puede ser una ventaja para todos nosotros.

Aunque a Shockman no le gustaba trabajar en equipo, la tarea propuesta por Scream era todo reto a sus habilidades y conocimientos como biólogo, por lo que se puso manos a la obra para diseñar un prototipo que cumpliera las exigencias básicas requeridas, a saber: atraer animales comunes en los callejones de la ciudad y manipularles para que acometieran acciones básicas y elementales. Para ello contaría con el grupo de laboratorio y todo material que pudiera necesitar dentro de los límites de lo alcanzable por la organización. El prototipo, además, tenía que ser diseñado para poderse fabricar en masa sin problemas y sin costes exagerados ni piezas demasiado extrañas, aunque tenía a su disposición todo material que tuviera que ver con la fabricación de naves espaciales.

A medida que la fabricación entraba en fase terminal empezaron a surgir los primeros modelos de prueba, así como los primeros problemas: interferencias con los comunicadores y entre los propios dispositivos, así como dificultades en el manejo para los no expertos. Hubo que realizar muchos rediseños, muchos tests adicionales.

Durante ese periodo que duró varias semanas Shockman se sintió el hombre más extraño que había existido jamás. Por un lado tenía la sensación de pertenecer a un sitio, de haber encontrado un lugar; por otro, no había noche en que no pensara obsesivamente en marcharse y dejarlo todo atrás. Un par de veces ensayó la manera de escapar sin que nadie se diera cuenta de ello, pero nunca llegó a poner el plan en práctica. Siempre lo abortaba cuando se daba cuenta de que nadie le estaba impidiendo largarse, y el mero hecho de no sentirse un prisionero quitaba todo valor a la planificación, por mucho que pensara furtivamente en ella para que nadie notara su ausencia hasta mucho más tarde de haberles abandonado.

En gran parte esas tribulaciones se quedaron atrás cuando el modelo estuvo definitivamente terminado. Sin embargo, ahí empezaron nuevas dudas para Shockman. Ya les había otorgado lo que podían obtener de él, de seguro a partir de ese momento le echarían o le harían el vacío.

O quién sabe, le matarían. Las cosas no siempre eran lo que parecían ser.

Pero lo que ocurrió fue que Matthew Swind le ofreció un hueco en su escuadrón, donde nadie tenía reparos en que peleara con ellos. Uno de los suyos deseaba integrarse en tareas de investigación en el laboratorio, y era el momento propicio para encontrar un sustituto.

La respuesta de Shockman no se hizo esperar. Negativa, por supuesto. Hacer las cosas de manera sencilla no figuraba en su diccionario personal.

~---Prefiero salir solo ~---fue su escueta explicación.

~---Te juegas la vida y la de todos si haces eso ~---objetó Scream, reunido con él y con Swind, a la espera de hacerle recapacitar.

~---Tranquilo, Scream ~---comentó Shockman con arrogancia~---. No pondré en peligro los preceptos de vuestra organización. Ya me tomé la pastillita, y llevaré siempre mis chismes impregnados con acelerante. Si me matan arderé como una cerilla, no te preocupes por mí. Preocúpate mejor por los novatos.

Después de aquello se marchó como si él fuera el líder del grupo.

~---¿Estás seguro de lo que haces, John? ~---agregó Swind viéndole marchar, ya con el traje de Los Caídos pero su rata al hombro. Scream notó que no le llamó \emph{jefe}, como solía hacer en broma a menudo.

~---Descuida, Matt. Todo está bien ~---fue lo único que Scream acertó a decir, aunque fuera más una expectativa que una certeza.

Y así pasaron las semanas, y Los Caídos contaron con un nuevo miembro, un escuadrón de un solo hombre como se suele decir. También contaron con una nueva habilidad que, tal vez, aún era pronto para entrenar, pero que en todo caso Shockman solía usar a menudo en sus incursiones solitarias, aunque más perfeccionada por medio de aparatos que sólo él era capaz de manejar, dada su condición de inventor del instrumento. Pero a pesar del paso del tiempo las obras continuaron por todo Ernépolis y las luces públicas aún aparecían y se desvanecían, parpadeaban como luciérnagas metálicas e infernales a lo largo y ancho de las más despreciables callejuelas de la sucia y ennegrecida ciudad.

Alma Espejo también estaba allí fuera, peleando, luchando por cuenta propia. Sólo que allá donde él iba, no había oscuridad posible. Tan pronto arrestaba a una panda de ladrones de deslizadores como atrapaba a un cártel de distribuidores de Valis. Pero Los Caídos sabían que ese sólo sería su comienzo como héroe. Pruebas mayores y de gran importancia aún estaban por acontecerle, y no sólo pensaban en la aparición de amenazas acordes a sus cualidades. También pensaban en su relación con otros elementos que rondaban al margen de la ley y no eran exactamente criminales.

Y entre los cuales estaban ellos mismos.

Al fin, la respuesta a una incógnita latente no se hizo esperar. Una vez Alma Espejo hubo patrullado la ciudad y repartido justicia entre los delincuentes de poca monta un periodista, en una rueda de prensa en la que estaba también el Presidente Scatter, le hizo la pregunta que muchas veces antes había ignorado, y por fin se dignó a contestar.

~---¿Qué opina de ese ser que vaga por las calles de la ciudad como una sombra, y que popularmente es conocido como el Caído? ~---fue la sencilla manera en que la expresó, levantándose de su asiento.

Alma Espejo no tardó en contestar, flotando como estaba en el aire, sin que tampoco hubiera ninguna luz encendida en la sala en la que estaban, a todas luces innecesarias.

~---De todos los criminales de esta ciudad me parece el más despreciable, detestable y deleznable que existe. Hasta ahora ha hecho su voluntad en todos los sentidos, y es posible que sea el responsable de la desaparición de muchos de los héroes de antaño. Es hora de pararle los pies de una vez por todas.

Al mismo tiempo que el héroe luminoso pronunciaba estas palabras, que se emitieron en todos los medios y reprodujeron en los monitores de las calles, en el hemiciclo del Aquerón la práctica totalidad de los miembros de Los Caídos asimilaban con preocupación las nuevas noticias.

Al fondo, sin embargo, dos sujetos estaban apartados de los demás, reflexionando en silencio. Uno de ellos era Shockman, a quien no solía agradarle la compañía, y cuya rata estaba apoyada sobre la palma abierta de su mano.

Miró al otro lado de la sala y vio a Scream, a medias entre la luz, a medias entre las sombras, como él apartado de los demás.

Y por primera vez en mucho tiempo Warren Shockman, conocido como Éxeter en tiempos ya olvidados, pensó que tal vez no estaba tan solo como él pensaba.

\endinput
