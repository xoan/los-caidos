El camino que eligieron no siempre les llevaba a tomar las decisiones que hubieran deseado en circunstancias acontecidas. Muchas veces tenían que fingir, hacerse pasar por lo que no eran, y eso era algo que habían aprendido a asumir con el paso del tiempo.
Pero a veces eso podía llevarles a adquirir oponentes que nunca hubieran deseado tener que enfrentar\dots

\fancyparbreak
John Scream sintió un profundo pesar cuando escuchó la decisión de Alma Espejo de ir a por ellos. A por él, bajo su punto de vista. Era algo que comprendía y respetaba, que hubiera incluso apoyado, seguramente, si hubiera sido un ciudadano más de la ciudad, ignorante de las distintas capas de complejidad que se cernían ante ellos por el control y preservación de Ernépolis~I. Era algo que incluso, en el pasado, pudo llegar a admirar.

Pero al mismo tiempo, era una decisión que tenía que impedir a toda costa.

Y nuevamente las dudas e incertidumbres eternas afloraban al exterior otra vez. ¿Salir o esconderse? Bastaba con que Alma Espejo capturara a uno solo de ellos y lo llevara ante las autoridades para que todo se viniera abajo sin remedio. Sólo una afortunadísima combinación de casualidades ~---la intervención a tiempo de Sky, por ejemplo~--- impedirían que se supiera el secreto. Por otro lado era patente que no cabía en estrategia razonable alguna pelear contra el nuevo héroe. Sus poderes parecían ir más allá de toda capacidad, siendo destacables incluso en la era perdida de protectores de la ciudad. En aquella ocasión no podrían plantar cara de manera directa. ¿O tal vez sí? Siempre pasaba lo mismo, por otro lado. Cada nuevo peligro parecía más fuerte que el anterior y sin embargo habían logrado superarlos todos.

No, no, no te engañes, eso es un espejismo, una mentira, pensó Scream apesadumbrado, mirando las grabaciones de Alma Espejo en acción, solo y de pie en el hemiciclo del Aquerón. Derrotaron a Armor sin reservas, en efecto. Pero no fue así cuando regresó, y ni siquiera era él del todo ni estaba a plena potencia. Lo de Hades tampoco fue una victoria rotunda, ni mucho menos. Por no hablar de otras amenazas igualmente temibles y perseverantes.

Sin embargo Alma Espejo derrotó a Armor sin prácticamente despeinarse, y desde su llegada a la ciudad el crimen había sufrido un duro golpe sin precedentes en mucho tiempo. Aun a pesar de los continuados esfuerzos de Los Caídos.

Muy bien, pensó Scream. Es fácil entonces. Contémosle la verdad. Que luchamos por proteger la ciudad desde las tinieblas que tanto nos esforzamos por despejar.

Era fácil, sin duda.

¿O no?

¿Podían fiarse de Alma Espejo? ¿Dónde estuvo todo ese tiempo? ¿Quién era en realidad? ¿Todo era tan sencillo como aparentaba?

Si le decían la verdad y resultaba ser una trampa\dots\ entonces estaban acabados, y dudaba mucho que ningún otro héroe pudiera hacerle frente cara a cara.

Tal vez todo era una cuestión de paranoia suya, exceso de celo. Los años le volvían sombrío, pesimista y taciturno. O tal vez sus dudas eran legítimas. En todo caso, arriesgarse sin tapujos era algo que le costaba mucho hacer teniendo en cuenta que en sus manos estaba el destino de toda la organización.

Además tenía una corazonada. Una terrible sospecha que alertaba su instinto. La manera en que habló de ellos, en que les amenazó\dots\ había algo personal, algo de naturaleza visceral, a tener en cuenta. No eran una tarea más para él, eran parte importante de su planificación. También era cierto que al ser una parte indiscutible de las leyendas urbanas de la ciudad, tenía sentido que Alma Espejo hubiera enfocado sus iras directamente en ellos.

Demasiadas cosas en las que pensar. Demasiadas tribulaciones internas.

Cuando miró al fondo de la oscura sala y le vio allí, sentado en una silla y con los pies puestos en la delantera, con su rata correteando por los respaldos, pensó que tal vez no era el único que tenía conflictos internos.

~---Deberías pasar más tiempo con los demás ~---argumentó Scream, acercándose hacia su posición.

~---Yo no tengo interés en estar con ellos, y ellos no tienen interés en estar conmigo. Creo que todos salimos ganando con la situación actual.

~---Bien, no insistiré. Eres libre de hacer lo que quieras.

~---¿Fuiste como él en el pasado? ~---preguntó Shockman mirando a la pantalla, donde Alma Espejo deslumbraba a unos rateros y les noqueaba sin que les diera prácticamente tiempo a reaccionar.

~---Eso ya da igual. El pasado no vuelve, y si lo hace es sólo para traer malas noticias.

~---Mira quién es el pesimista ahora ~---dijo Shockman dejando que la rata volviera al bolsillo de su abrigo.

~---¿Cómo te hiciste lo del ojo? ~---preguntó Scream de repente, tratando de sacar algo más que comentarios cínicos de aquel peculiar nuevo miembro de la organización.

~---¿Acaso te importa?

~---En realidad, no. Pero estoy seguro que en tu vida se lo has contado a nadie.

Scream comenzó a marcharse, a velocidad más lenta de lo normal. No esperaba que Shockman reaccionara de repente y contestara a su pregunta. Tendría que haber más momentos como aquel, más situaciones fingidas y de altivez por parte de su interlocutor.

Shockman se quedó solo en la sala, dejando divagar la mente. Un tipo curioso este Scream, pensó. Quién sabe si alguna vez nos cruzamos en el pasado.

\parbreak
La reacción de los demás miembros de Los Caídos a la declaración de Alma Espejo fue una insólita mezcla de admiración y temor. Admiración porque sus palabras parecían evidenciar la veracidad de sus actos.

Temor porque tal vez fueran el inicio del fin para todos ellos. Y después de tantos esfuerzos y sacrificadas batallas, ser vencidos por uno de los suyos tenía un indudable sabor agridulce en la derrota.

Al menos no les mataría ni nada parecido. Alma Espejo no era un asesino, eso estaba claro. Ya fuera por convicción o necesidad, eso hacía que muchos de ellos se sintieran más relajados al respecto.

Scream había dado órdenes estrictas al respecto para aquellos días. Nada de intervenir a nivel de calle a menos que fuera absolutamente necesario. Tenían que tomar a Alma Espejo como lo que era dadas las circunstancias: un rival, un adversario. Todo enfrentamiento directo con él debía ser evitado. Eso, por otro lado, entraba también en consonancia con la filosofía de la organización. Pero claro, las normas eran muy claras con la basura inmoral. Meterles miedo. Hacerles temblar. Que se sintieran perdidos desde el primer momento.

¿Pero cómo reaccionar ante alguien que se supone que tiene tus mismos objetivos, sigue los mismos límites, pero está totalmente en tu contra?

Los grandes veteranos, como Razorclaw, Saw o Swind, se limitaban a seguir las órdenes de Scream. Pero eso no era tan sencillo para alguien como Sam Grove. Joven e idealista, mala combinación. Un gran soldado, un gran jefe de escuadrón. Eso no es lo que era, sino lo que sería si seguía comportándose como hasta ese momento.

Pero la duda le carcomía por dentro, como si en el interior de su alma se estuviera forjando una división. No podía seguir órdenes que pensaba que no llevaban a ninguna parte. Estaba seguro de que Scream tenía sus motivos para actuar así. Era obvio que Saw, desde su puesto privilegiado de ayudante del Presidente Scatter, tenía la oreja pegada a todo dato que pudiera resultar interesante de escuchar, y que muchos de los miembros estaban recabando rumores de las calles y poniéndolos a disposición del grupo de investigación del Aquerón para que ellos pudieran hacer algo al respecto.

Pero eso no le bastaba. Tenía que hacer algo, quería hacer algo. Tampoco es que fuera impetuoso, pero sí voluntarioso. Sin embargo no podía salir con su escuadrón, les pondría en peligro sin motivo alguno. Si Scream se enteraba, además, le tendría limpiando celdas hasta el fin de sus días.

De repente vio a Shockman salir del hemiciclo, con aires de completa indiferencia. Apenas había hablado con él, aunque no tenía reparos en hacerlo puesto que, dado que él no había sido un héroe en los viejos tiempos, no tenía prejuicios ni recelos a su respecto. Además, el nuevo dispositivo que gracias a su llegada se había instalado en el traje le parecía como poco muy útil para provocar el temor en sus adversarios.

~---Eh ~---le dijo Grove cuando pasó cerca de él, así por lo bajo, como si tuviera miedo de interrumpir su avance hacia alguna clase de potencial lugar importante.

~---¿Qué? ~---fue la única respuesta de Shockman, sin añadir ningún apelativo. Lacónico y directo. Sólo le faltó mover los hombros en tono despectivo.

~---Quiero salir fuera a espiar a ese Alma Espejo. Por lo que Ellis dice, suele intervenir menos a estas mismas horas. Tal vez nos enteremos de algo.

~---Yo salgo solo, novato ~---contestó Shockman escuetamente.

~---Si me ayudas haré tus informes de todo el mes próximo.

~---¿Todo el mes?

~---Todo el mes.

~---Se van a dar cuenta enseguida. Básicamente, porque no me tomo muchas molestias para rellenarlos.

~---Vamos, estoy seguro de que los detestas.

~---Ella no ~---dijo sacando un trozo de papel de un bolsillo y metiéndolo furtivamente en el otro~---. Está bien, jefe de novatos. Me has convencido.

~---Yo diría que estabas convencido desde el principio y sólo lo haces por fastidiar.

~---Quién sabe, tendrás que quedarte para siempre con la duda. ¿De cuánto tiempo disponemos?

~---Cuatro horas. A partir de ahí nos calan seguro.

~---¿Y por dónde tenía vuesa merced pensado que patrulláramos?

~---Me he colado en ocasiones a fisgar en los datos de investigación y alguna que otra vez se le ha visto por un distrito concreto pero no se sabe de actuaciones suyas por la zona. Así que o conoce a alguien o hace algo personal, y eso siempre será información.

~---Bien, chico listo ~---dijo Shockman cruzado de brazos, esperando que se moviera para seguirle~---. Veamos de lo que eres capaz.

\parbreak
Sam Grove esperaba que nadie descubriera la pequeña escapada que él y Shockman estaban llevando a cabo y que al mismo tiempo pudieran averiguar algo sobre el centro de atención de todo el mundo en la ciudad, y por eso su decisión a la hora de dirigir a Shockman por las calles fue tan enérgica que el antiguo villano no pudo por más que sentirse ciertamente sorprendido. De modo que así es como se mueven los héroes cuando van juntos, pensó. Avanza por delante de mí y me da la espalda de manera indudable. No le preocupa que pueda dejarle abandonado, o ponerle en evidencia, o bajo circunstancias propicias, acabar con él sin que tenga siquiera tiempo de reaccionar.

Qué gran debilidad, razonó al principio. Pero luego empezó a plantearse que de ese modo actuaban más como un solo guerrero, pues la mano izquierda de una persona nunca debe preocuparse por ser traicionada por la derecha, aunque seguro que en los viejos tiempos tuvo algún aliado al que no podía aplicarse esa frase en sentido literal.

No tardó en reconocer la zona hacia la que estaban llegando como aledaña al Distrito Financiero, como casi cualquier ciudadano de Ernépolis no hubiera tardado en reconocer. En su caso, si tardó un poco más, se debía a su condición de recién llegado. Conocía la ciudad, por supuesto, pero desde que se marchó muchas cosas habían cambiado.

Los crímenes habituales que infectaban las calles no era una de ellas. Shockman no estuvo presente durante el reinado tiránico de Ellen Gorgon, y por eso sólo conocía la A y la Z de la delincuencia urbana, sin haber pasado por los estratos intermedios. Eso les obligó a detenerse un par de veces, actuar con la mínima discreción, incluso tratando de no mostrarse en caso necesario, y continuar con su trayecto hasta llegar a la zona del avistamiento.

Una vez por allí subieron hasta un edificio de quince plantas y echaron un vistazo panorámico a la zona, bien cubiertos y distanciados. Un equipo de sujetos individuales, se podía decir. Un par de rebeldes por motivos divergentes que habían encontrado un objetivo que hacía converger sus intenciones.

Lo bueno de buscar a Alma Espejo es que encontrarle era fácil si uno trataba de buscarle con esmero. La luz que irradiaba era como un faro entre las tinieblas, un destello de esperanza en mitad del más insondable y profundo infierno.

Lo malo era que si bien no resultaba difícil encontrarle, más complicado se ponía seguirle de cerca, teniendo en cuenta que a su alrededor no había sombra ni esquina tras la que resguardarse, ni siquiera por medios artificiales. Un resquicio de sombra allá donde sólo debería haber luz pura y continua resultaba tan evidente como el punto de iluminación de un foco en las más absolutas tinieblas.

\emph{Bien, ya hemos llegado} ~---dijo Shockman plantándose en la esquina contigua del tejado a la que estaba Grove. Vistos desde lejos, si alguien hubiera estado ahí para detectarles a ambos, hubiera pensado de ellos que eran un par de gárgolas resucitando ante un mundo nuevo y diseñado a su perfecta medida.

\emph{Esto es lo más cerca que Scream nos dejaría estar del objetivo} ~---comentó Grove, como si tratara de constatar una obviedad~---. \emph{Es complicado que nos acerquemos más sin que nos detecte.}

\emph{Habla por todos esos estirados jefes de escuadrón} ~---agregó Shockman mientras la rata salía de un bolsillo interior de la gabardina~---. \emph{Afortunadamente para nosotros yo tengo otros métodos.}

La rata reptó a lo largo de la pierna de Shockman hasta el suelo y caminó a pasos cortos pero muy rápidos hasta alcanzar un canalón por el que se deslizó con precisión de equilibrista. Fue cuestión de minutos lo que tuvieron que esperar hasta verla desaparecer de su punto de vista, y cosa de un cuarto de hora más hasta que regresó. En el interregno, totalmente silencioso y sin que ninguno de los dos hiciera clase alguna de comentario, pudieron comprobar cómo la luminosidad descendió gradualmente hasta desaparecer del todo.

Cuando la rata regresó el mensaje que Shockman logró sonsacar era lo que más o menos esperaban, con ciertos matices de concreción.

\emph{Se ha metido en un edificio. Puede llevarnos hasta allí.}

Por lo que Los Caídos sabían Alma Espejo podía graduar sólo en parte la intensidad de la luz que irradiaba, teniendo eso como resultado que nunca dejaba de brillar, ni siquiera de manera consciente. Eso debió de tener sin duda consecuencias devastadoras para su vida personal, pero de qué modo concreto pudo haberle afectado, no estaban ni cerca de poder imaginárselo. Para empezar, mirarle al rostro era algo que resultaba, como poco, incómodo y perturbador. Aunque la impresión general que se tenía de él en todos los medios era la de ser un hombre apuesto. No podía esperarse menos de una criatura que simbolizaba con su luz autogenerada arquetipos tan evidentes como la bondad, la esperanza o en fin de la noche eterna.

Cuando llegaron al edificio comprobaron, como Grove ya había corroborado sin que nadie en el Aquerón se diera cuenta de sus furtivas indagaciones, que la zona que la rata señalaba formaba parte de manera territorial de la Plaza Wave, lugar más que esencial al ser punto de ubicación de muchos edificios notables de la ciudad. La ventana que señalaba, más concretamente, pertenecía a la fachada trasera de un gran complejo público que, en principio, y desde aquel punto de vista, ninguno de los dos logró reconocer.

Se colaron por el interior de la ventana, teniendo el máximo cuidado de mantener el incansable engaño de varios que parecen uno, y no tardaron en ubicar la posición de todas las cámaras de seguridad. En todo caso tales aparatos nunca habían supuesto una amenaza a la estrategia de Los Caídos estando como estaban provistos de la capacidad de provocar sombras y generar falsas imágenes, entre otras manipulaciones sutiles del entorno.

El pasillo estaba oscuro como sólo puede estarlo cuando un edificio público ha sido abandonado por todo el mundo salvo unos pocos guardias de seguridad apostados en entradas principales que ningún criminal trataría de emplear. Más adelante en su trayecto se cruzaron con algún vigilante ocasional, pero nada ni nadie que pudiera suponer para ellos clase alguna de amenaza.

Fue entonces cuando vieron al fondo el brillo. Alma Espejo estaba allí, o cerca de la zona al menos. Parecía que no tratara de ocultarse, y en verdad no lo estaba haciendo, pues no tardaron en ver a un par de guardias conversando acerca de ello en tono intrascendental. De modo que no sólo no les resultaba sorprendente, sino que iba por allí a menudo.

\emph{Puede que nuestro amigo tenga algo que ocultarnos, al fin y al cabo} ~---comentó Shockman mientras avanzaba por los pasillos, siguiendo a Grove a distancia pero caminando con pasos seguros, como si fuera él en realidad el que estuviera a la cabeza de la incursión.

\emph{Sigo sin saber dónde estamos} ~---dijo Grove por el intercomunicador~---. \emph{No hay manera de leer ninguno de los carteles sin llamar la atención.}

\emph{Sigue entonces el camino de baldosas amarillas} ~---comentó Shockman señalando al fondo del largo pasillo, donde la luz seguía irradiando con bastante claridad.

Cuando doblaron la esquina comprobaron que estaban en un callejón sin salida, al menos desde esos corredores, y que la luz de Alma Espejo volvía a verse desde fuera del edificio. Tan pronto como había llegado, se había marchado. Tal vez ni había tratado de hablar con nadie, sólo estaba de paso por allí. Era lo bueno de estar bajo el ala del Presidente de Ernépolis~I, difícilmente nadie iba a negarle el paso si no se extralimitaba en sus exigencias.

En aquella diminuta sección del pasillo sólo había un acceso a una sala, custodiada por una robusta puerta de roble llena de ribetes y adornos tallados con sumo cuidado.

\emph{Creo que hasta aquí hemos llegado} ~---declaró Grove frente a la puerta~---. \emph{Podemos volvernos ya y sugerir a los demás que podría ser una buena idea vigilar esta zona. No tardarán en completar lo que hemos empezado.}

\emph{Vamos, boy scout, no lo dirás en serio} ~---comentó Shockman limitándose a abrir la puerta. A partir de ese momento él llevaba la voz cantante de cara al exterior, y Grove tendría que limitarse, le gustara o no, a darle apoyo logístico.

La sala a la que accedieron, uno a cara descubierta y el otro no, era una especie de despacho muy cómodo y voluminoso que conectaba por medio de un arco con otra sala más grande y destinada sin duda a albergar a una o varias personas importantes en potenciales discusiones diplomáticas o de asuntos burocráticos. La iluminación era muy tenue, apenas un par de lámparas situadas de manera más o menos concienzuda, y gran cantidad de documentos atestaban el por otro lado pulcramente ordenado escritorio. Sobre una pared había gran cantidad de diplomas, títulos y otros reconocimientos.

Entre ellos podía verse claramente uno de forma extraña que parecía ser una suerte de mención de honor de una especie alienígena hacia el interesado, de ahí lo peculiar del documento. Grove no podía leerlo desde donde estaba.

Sin embargo, cuando sobre el escritorio pudo ver una placa que ponía \emph{Juez Supremo}, aunque no pudo ver el resto del nombre, un escalofrío recorrió su cuerpo de arriba abajo.

Le hizo a Shockman la señal universal que no muchas veces había usado pero que sí que había entrenado en numerosas ocasiones. Peligro. Peligro de alto nivel. Huir cuanto antes, y reportar.

Shockman no se movió del sitio, aunque sí advirtió la señal de Grove. Pero por un lado no la supo descifrar pues nunca tuvo interés en aprender ninguno de aquellos procedimientos de comunicación en equipo, y por otro sin duda no hubiera movido un solo músculo, por mero orgullo, aunque hubiera discernido su significado.

Lejos de largarse, Shockman sacó adelante su mejor arsenal de trucos para impresionar al dueño de aquella habitación, al que logró vislumbrar vagamente al fondo, con una copa en la mano. Sabiendo que ya había sido detectado convocó a los insectos del lugar. En edificios de madera como aquel a veces se podían obtener resultados ciertamente espectaculares, por limpios que pudieran parecer de un vistazo preliminar.

Varias moscas y mosquitos comenzaron a revolotear alrededor del cuerpo de Shockman, y algunas hormigas y cucarachas aparecieron a sus pies. La perturbadora visión de conjunto, que se unía a la ya habitual provocada por el traje de Los Caídos, llegó a su máximo nivel de perturbación cuando un miriápodo, en concreto una escolopendra, como Shockman dedujo a partir de sus largas y numerosas patas y sus colmillos delanteros, empezó a enroscarse alrededor de su pierna y a subir con extrema y enervante lentitud.

~---Vaya, vaya. Nueva presentación para el mismo formato ~---dijo el hombre caminando lentamente, copa en mano, desde el otro lado de la habitación. También había en cierto modo insectos a su alrededor, diez en total, pero eran de otra categoría. Fríos, tenebrosos, revoloteando como coleópteros, de formas redondas y aceradas.

Metálicos.

Sam Grove comprobó cómo el Juez Nitram se plantaba frente al acceso al despacho y comprendió que se habían colado en las fauces del lobo hasta llegar a sus mismísimas entrañas.

\endinput
