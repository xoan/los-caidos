\noindent
Nunca pensó que llegaría a hacer algo así. Siempre creyó que cuando estuviera con otros héroes serían admirados. Valorados. Que su labor tendría recompensa. Pero al fin se dio cuenta de que su promesa de defender a los que más quería podía hacer que sintieran miedo de él y los suyos\dots

\parbreak\noindent
Por primera vez en bastante tiempo volvía a caer aguaceniza en las calles oscuras de Ernépolis~I. La combinación de ambas lluvias provocaba una sustancia plastosa que permanecía por varios días en el suelo de la ciudad hasta que nuevas lluvias sólo de agua contribuían a reblandecer la mezcla. Era por eso que Roger Thunder suponía que aquella iba a ser una noche inútil. Allá en los Túneles, nadie sería tan estúpido como para deambular a dichas horas en mitad de una lluvia de aguaceniza. En ese sentido el negocio iba cada vez peor, y Thunder lo sabía. Tendría que empezar a plantearse mudarse de barrio si es que quería cumplir con la cuota monetaria semanal para su jefe. Si no lo hacía ya sabía lo que le tocaría: llamarían a los hombres de Wolf y éstos le meterían en la cárcel tras una interminable sucesión de torturas. Recordó una vez en que encontró a un excompañero de correrías que acabó así. Había huido, decía, en un descuido de sus captores. Su aspecto no parecía evidenciar tal cosa. Thunder supuso que las cárceles de la poli estaban tan llenas de tipos que habían intentado oponerse con la razón a Gorgon que bien podían soltar a un inepto más por la ciudad.

Sumido en sus pensamientos, de repente vio a lo lejos la silueta de una mujer corriendo para guarecerse bajo uno de tantos arcos del lugar. Conocedor de su territorio, Thunder se acercó lentamente y la observó. Tenía pinta de haberse perdido. Treinta años, quizá. Estaba buena. Muy buena. Miró a los alrededores. Nadie. Se frotó las manos y pensó que tal vez era su día de suerte. Reflexionó un poco, sin perderla de vista, y se dijo a sí mismo que sólo sería un polvo rápido, pues no creía que ella tuviera suficiente dinero para amortizar toda la noche. Pero se conocía. Sabía que en esas circunstancias, a veces, perdía el control. Más de una vez tuvo que recurrir a la pala y un paseo a las afueras para cubrirse las espaldas.

Avanzó unos cuantos pasos y la acechó, sintiéndose como un tigre a la caza de una gacela herida. Se acababa de agachar para quitarse un momento uno de los zapatos. La ocasión perfecta. Thunder salió de su escondrijo y sacó la navaja.

~---El dinero ~---dijo sin más.

La mujer estaba atemorizada. Se sentía indefensa por la postura en que la habían sorprendido. Eso le puso a cien a Thunder. Rebuscó en su bolso con la mano temblorosa, el cuerpo mojado y embarrado. Thunder no esperó y se lo arrancó de las manos. No llevaba apenas nada.

~---Es una broma, ¿verdad?

~---Por favor, por eso venía andando. No podía tomar un taxi deslizador\dots

~---Vas a tener que pagarme entonces de otra forma ~---dijo acercándose a ella y arrancando parte de su vestido de un tirón. La mujer gritó desesperada.

~---No te molestes. Nadie puede oírte aquí.

\emph{Te equivocas.}

Thunder se quedó quieto un momento y miró alrededor. Un rayo recortó el final del túnel. No había nadie. Pero no se había inventado aquella voz. Pensó que seguramente los jefes habían mandado a alguien para asegurarse de que hacía el trabajo adecuadamente. Bien. Siempre podía clavarle la navaja por la espalda, empotrarle contra los pinchos de una verja y decir que hubo un desafortunado accidente.

~---Si te mueves te mato ~---dijo a la mujer, con el maquillaje arruinado por las lágrimas y la lluvia.

Thunder avanzó hacia el túnel, cubriendo todos los posibles puntos de emboscada. No podrían sorprenderle.

\emph{Te deberías sentir afortunado. Te he escogido.}

~---¿Quién eres?

Otro rayo iluminó de nuevo el túnel. Thunder vio la silueta enorme de un hombre con gabardina y sombrero. Cuando la luz desapareció, así hizo su misterioso interlocutor. No pudo ver cómo, pero lo hizo.

Thunder empezó a asustarse. Aquel tipo no era uno de los suyos. Le bajó la erección y se concentró en el filo de metal de su arma.

~---¡Desaparece, tío, o te mataré!

\emph{No lo creo} ~---la voz provenía ahora de su espalda.

Thunder se dio la vuelta todo lo deprisa que pudo y aplicó un corte al aire. Justo en el momento en que se giró perdió por completo la visión. No había nada en qué enfocar para orientarse, nada que le diera sustento. Empezó a perder el equilibrio, a tambalearse. Al fin encontró un rincón de luz y se acercó a él. Respiró aliviado.

Una mano enguantada le agarró de la cabeza, tapándole la boca, impidiéndole chillar. La navaja se le cayó al suelo.

\emph{Vas a ser mi mensajero. Habla de mí. Si no, te mataré. Y yo, Roger Thunder\dots}

La presión cedió. Thunder miró al fondo y vio una sombra recortada por los siniestros rompientes del túnel. Estaba muy lejos, demasiado, pero sabía que era su atacante.

\emph{Yo sí que hablo en serio.}

Thunder se olvidó tanto de la navaja como de la mujer. Echó a correr y no paró hasta llegar a zona segura, donde habló con su capo. Le contó todo lo ocurrido, paso a paso, punto por punto.

Un tiro en la cabeza le silenció cuando acabó su historia.

La mujer miró atemorizada a su alrededor. Tapándose la parte rota del vestido, echó a andar, tratando de convencerse de que no había visto ni oído nada. Que si fingía no haber visto a ese ser de tinieblas no le haría nada.

\emph{Corre, mujer, corre.}

Asumió que se había equivocado.

Echó a correr por los Túneles todo lo deprisa que pudo, a pesar incluso de llevar tacones. Alguna vez resbalaba y caía, pero se volvía a levantar con gran celeridad. En ocasiones miraba atrás. No le veía, pero sentía que estaba ahí. Vigilando. Acechando. Por un momento, un brevísimo momento, se preguntó si no estaría en realidad protegiéndola, cuidando que salía a salvo de aquel barrio de mala muerte. Olvidó ese pensamiento y siguió su avance frenético en busca de luces de comercios.

\bigskip\noindent
Albert Fox se despertó y observó su despacho empantanado. Levantó los pies de la mesa y se preguntó dónde estaba su café. Un vistazo al suelo le confirmó sus peores temores. Maldijo por lo bajo y de una patada lo echó contra una esquina. Al cabo de un rato escuchó por radio una llamada.

~---Nueve uno, nueve uno.

Atraco en la zona presidencial. Territorio privado de Gorgon. Se sentó otra vez y agradeció su suerte. En ocasiones tenía que salir a la calle para hacer prevalecer la ley y el orden. Mientras Wolf no se lo mandara, él y los suyos podían seguir tranquilamente prolongando su siesta nocturna.

Pensó que hacía tiempo que no tenían misiones especiales. Le gustaba jugar al poli duro y partirle la cara a los mocosos que se pensaban que tenían algo que protestar contra Ellen Gorgon. Le había cogido el gusto a ser el que soltaba las palabras amenazantes mientras ordenaba con leves gestos esforzados, como si no quisiera hacerlo, que les golpearan una vez más. Sí, era una vida agradable la que llevaba. Estar con el bando ganador era lo mejor que le había pasado a Fox en mucho tiempo. Su mujer se lo agradeció. Al principio trató de hacer prevalecer la justicia. La justicia. Rió por lo bajo y pensó que justicia y violencia eran dos palabras imposibles de separar.

Otro comunicado. Un insulto recorrió las paredes de su despacho. Al otro lado, difuminadas por los cristales, varias cabezas se giraron en su dirección. Idiotas, pensó.

~---Tres cinco en los alrededores, repito, tres cinco en los alrededores.

Fox sonrió. Sabía bien que no existía el tres cinco en el código oficial de la policía de Ernépolis~I, pero entre los compañeros, un tres cinco era un grupo de vagabundos borrachos cerca con los que poder entrenar los puños. Salió del despacho con la mayor solemnidad y pasó por delante del despacho de Wolf, cerrado a cal y canto. Mientras el viejo no les pillase todo iría bien. Tampoco es que lo desaprobara, pensó, pero con su carácter bien podía empapelarles por un tiempo.

Se acercó a los suyos y les hizo gestos para que salieran. Todos sabían a lo que se refería.

~---Vamos, Cracker, Warp, Sky. Todos fuera. Os quiero listos ya.

Avanzaron en el mismo orden en que Fox los llamó, él a la cabeza. Fue por eso que ninguno vio la enigmática sonrisa en el rostro de James Sky.


Cuando llegaron al lugar encontraron lo que esperaban encontrar: un par de borrachos desvariando e insultando a Wolf en particular y a la policía en general. Fox les apartó a un lado de un empujón y miró las botellas que llevaban. Absenta. Las lanzó contra una pared.

~---No me gusta que andéis así por la calle, chicos ~---dijo a los viejos, tan borrachos que le ignoraron por completo. Miró alrededor y se cercioró de que estaban solos~---. Esta es bebida de mala calidad. ¿No crees, Jack?

~---Sí, creo que sí ~---respondió Warp acercándose a un callejón próximo. Al cabo de un rato hizo un gesto afirmativo.

~---Ve a ayudarle, Sky ~---dijo Fox con desgana.

Sky fue para allá tan rápido como pudo. Fox le observó lentamente. El remilgado de James Sky, pensó. Algún día tendrían que ajustar cuentas con él. Al principio todos eran como él, todo honradez y rectitud. Nunca quería su parte cuando había que callar, ni se prestaba dispuesto a usar los puños cuando era necesario. Deseó que Wolf se lo encomendara como próximo trabajito especial. Tal vez lo sugiriera él mismo.

De repente escuchó un par de gritos provenientes del callejón. Mandó a Cracker vigilar a los viejos y desenfundó el arma mientras iba él mismo al callejón, bañado por densa oscuridad y sombras aisladas. No encontró a Sky ni a Warp por ninguna parte.

~---¿Jack? ~---dijo olvidándose por completo de Sky.

\emph{Jack no está} ~---oyó a sus espaldas.

Fox se dio la vuelta y con la experiencia de años de gatillo fácil disparó. Las balas atravesaron una silueta de un tipo con gabardina. El tipo ni se inmutó. Trató de iluminarle con la linterna, pero era inútil. Era como si la luz pasara de largo al encontrarse con él.

Fox esperaba un discursito, palabras amedrentadoras. No encontró nada de eso.

~---Fuera de mi camino ~---dijo disparando otra vez. Nada. Se giró rápidamente, como si esperara una emboscada, y al volverse la silueta ya no estaba. Guardó el arma.

Y entonces sintió el frío cañón apuntándole en la nuca.

\emph{Dile que venga.}

Fox trató de atacarle por sorpresa, suponiendo que su adversario no tendría el valor para disparar. En cierto modo tenía razón.

Antes de volarle la oreja primero le rompió el brazo.

Fox gritó y miró el bidón de gasolina que Sky y Warp debían haber llevado fuera del callejón. Junto a él se encontraba Cracker apuntando al hombre de la gabardina.

~---Las balas\dots\ no\dots\ le afectan ~---dijo Fox lloriqueando, la mano en la oreja.

\emph{Me gustan tus hombres, Fox. Me ahorran el trabajo.}

En un abrir y cerrar de ojos, la silueta desapareció. Cracker miró a todos lados, apuntando convulsivamente el arma de un sitio a otro, como si fuera una pistola de corcho y tuviera frente a sí un montón de patos de feria.

~---¡Idiota! ~---gritó Fox como pudo~---, ¡aléjate de la gasolina!

Cracker echó a correr fuera del callejón, soltando el arma para ganar velocidad. Al cabo de un momento, cuando su enemigo disparó al bidón, no sólo ganó velocidad sino también altura.

Cayó en plena calle entre un montón de escombros. Algunos curiosos se habían asomado desde sus ventanas. Los vagabundos ya no estaban.

Pero él sí.

Recortado entre las llamas, Cracker pudo mirar sus ojos. No había vida en aquellos ojos. Con las llamas a su alrededor y la Nube sobre sus cabezas, aquella fue para él una visión de auténtica pesadilla.

\emph{A partir de ahora sólo yo tengo derecho a quemar vagabundos en mi territorio} ~---dijo sin más.

Cracker tosió y miró al suelo, extenuado. De repente escuchó a Fox chillar. Levantó la vista. Ya no había nadie con él. Se incorporó como pudo y fue hacia el callejón de nuevo. Allí estaban Sky y Warp, el primero recuperando el conocimiento, el segundo lejos aún de hacerlo. No había señal de Fox por ninguna parte. Rebuscaron entre las paredes, en el suelo. Bajaron incluso a la alcantarilla. Nada.

~---Surgió de la ceniza ~---dijo Sky expresando pánico en su rostro~---. Nunca he visto nada igual.

Cracker también estaba asustado, pero no sabía a qué tenía más miedo: si al desconocido de la gabardina o a la ira del Jefe Wolf.

Brian Wolf estaba revisando documentos en su amplio despacho, mucho más amplio que el de Albert Fox, cuando le llamaron por teléfono para comunicarle la noticia. Una patrulla había encontrado al equipo de Fox en la calle, masacrados, con evidente estado de ansiedad. Fox había desaparecido sin dejar rastro. Al parecer, incluso tuvieron que huir a toda prisa, pues la multitud, envalentonada por lo sucedido, se había propuesto lincharles.

Wolf no estaba contento. Nada contento.

Se quedó un buen rato mirando por la ventana, fijándose bien en una ciudad que no parecía ya ser suya. Un extraño había llegado. Aquello no iba a gustar a los altos cargos. Otra llamada. Hablando de altos cargos\dots

~---Mátele, Wolf. No me importa a cuántos hombres tenga que usar. Ignore media ciudad si es necesario.

~---Así se hará, presidenta Gorgon.

~---No me suelte frases complacientes. Sólo hágalo.

~---Sí, presidenta.

Colgaron. Gorgon nunca se había caracterizado por andarse por las ramas precisamente. Más le valía no fallarla. Pensó en cuántos hombres tenían disponibles y cómo los distribuiría de modo que la población no se diera cuenta de que un solo hombre tenía en jaque a toda la policía de Ernépolis~I. Tal vez preparando una trampa le capturaran.

En esa clase de pensamientos divagaba cuando las luces del edificio de policía se apagaron. Otra maldita avería del generador, pensó Wolf. Resultaba gracioso que en una ciudad principalmente exportadora como Ernépolis~I a veces los recursos propios resultaran ser más que insuficientes. Esperó un rato a que alguno de los agentes del sótano activara el conmutador de emergencia. Tenían que reinstalar uno que saltara solo. De todos modos admitió que nunca se había preocupado mucho al respecto. Mantener a la ciudad hibernada y durmiente ya ocupaba la mayor parte de su tiempo.

Pasaron varios minutos y las luces no volvieron. Algo iba mal. Agarró el arma, una linterna y salió a los pasillos. Cuando torció un par de esquinas se encontró con varios policías inconscientes en el suelo. Seguían vivos. Aquel fue el paisaje durante un buen rato, aderezado con un inquietante silencio. Sintió que todas las sombras que le rodeaban se movían. Recordó los informes, lo que decían. Que se fundía con las sombras. Que parecía estar en todas partes. Siguió avanzando y se cruzó con uno de sus hombres, que corría aterrorizado en sentido contrario.

~---¡Huya, no hay escapatoria, no\dots!

Wolf le disparó por la espalda. No soportaba la cobardía.

\emph{Brian Wolf.}

La voz sonaba lejos, como si le estuviera esperando. Wolf sabía que era una trampa, pero él no era un cobarde como otros. Iría a los fusibles, se enfrentaría a su enemigo y lo derribaría. Y así Ellen Gorgon estaría sumamente satisfecha con él. Tal vez incluso accedería a deshacerse de aquel guardaespaldas suyo que tenía un arma multiusos.

Bajó al sótano y barrió las direcciones con la linterna. Nada. Acopló la linterna al arma y se dio cuenta de que estaba tratando de emboscar a un hombre que había reducido en cuestión de minutos a una gran parte de sus hombres. Se mintió a sí mismo y se dijo que no era tan difícil para el intruso teniendo en cuenta que sus oponentes estaban en régimen de oficinas y, por tanto, desentrenados. Cuando todo acabara les pondría a trabajar más duro. Haría revisión de plantilla, se acabaron las escapadas nocturnas para ajusticiar a la podredumbre de la ciudad.

\emph{Brian Wolf} ~---la voz estaba frente a él. Wolf disparó aun sabiendo que no había nada a lo que darle.

\emph{¿Cuántas balas, Wolf? ¿Cuántas quedan?}

La voz venía de todas partes. Cuando recuperó la compostura Wolf pensó que quedaban cinco. No había recargado el arma. Más que suficientes, pensó.

Cuando la inmensa sombra se abalanzó sobre él pensó que ni un ejército de balas hubiera sido suficiente.

Una sombra enorme, puntiaguda, llena de aristas y bordes. Alta, demasiado alta para ser humana. Disparó una, dos, tres, cuatro veces. Fue incapaz de hacerlo una quinta vez. Un golpe le arrebató el arma de las manos, en lo que una mano le levantó en vilo. La sombra se recompuso y volvió a ser, vagamente, un hombre. O por lo menos algo con aspecto de tal.

\emph{No me gusta tu policía. A partir de ahora ya no trabajas para Ellen Gorgon. Trabajas para mí. Te vigilaré para asegurarme de ello. Mientras comes, mientras duermes. Cada vez que veas una sombra, cada vez que mires a la Nube, yo estaré allí. Ernépolis está en guerra. Escoge bien el bando, Wolf.}

Wolf cayó al suelo y cuando se levantó ya nadie estaba frente a él. Miró alrededor. No. Sí estaba. Estaba en todas partes. Corrió por el arma, no para usarla, para sentirse seguro al tenerla entre sus manos. Como si con eso fueran a desaparecer todos sus problemas. Estaba en medio de una batalla campal. E hiciera lo que hiciera, se iba a ganar poderosos enemigos.

\bigskip\noindent
La negrura de la Nube era aún más espesa en la zona de las factorías. No en vano éstas eran las principales responsables de que dicha Nube se hiciera cada vez más grande. El progreso, decían. La acelerada fabricación de naves por parte de Gorgon Enterprises había acelerado también la densidad de la Nube. La lluvia de ceniza era cada vez más frecuente por la zona.

Claro que no todas las fábricas se dedicaban a piezas aeroespaciales.

Por ejemplo aquella en la que Silenciador se encontraba en aquel momento, supervisando a los hombres de Gorgon en lo que cargaban Valis, una nueva droga que Farmacéuticas Gorgon había diseñado en secreto. Para los nuevos ricos, una forma más sutil de dominio. La violencia y las amenazas ya no funcionaban en un estrato social donde por un poco de dinero podían contratarse eficientes guardaespaldas. Silenciador ya no era el único personaje pintoresco de la ciudad. Aunque no siempre fue así, pensó. Estuvo Reflector, su peor enemigo. Aquél cuya última derrota le fue robada por la presidenta. Sin embargo\dots\ había oído los rumores. Era imposible sustraerse a ellos. Al principio, cuando sólo se trataban de rateros de poca monta quienes los contaban, no pensaba que eran más que estúpidas leyendas urbanas en una ciudad cuyo aspecto se prestaba fácilmente a generarlas. Un tipo sin nombre. Sin identidad. No era lo habitual en los héroes a los que se había enfrentado en el pasado, más bien era un recurso propio de manipuladores como Ellen Gorgon. Sin embargo, cuando aquel misterioso justiciero ~---o tal vez rival mafioso~--- continuó su guerra de un solo hombre contra Wolf y los suyos comenzó a prestarlos más atención. Una cosa era lo que decían los navajeros iletrados y otra muy distinta lo que un poli dijera, pues aunque los consideraba un atajo de ratas cobardes tendían menos a la exageración. En todo caso, a la numérica. Pero poco podían distorsionar los hechos en ese sentido si el atacante era un solo hombre. Sí, tenía interés por conocer a aquel sujeto, concluyó en lo que acariciaba los múltiples cañones de su arma.

Cuando los focos de luz estallaron, se alegró de ver que no tardaría en presentarse.

No le importaba una mierda la droga de Gorgon. Ya fabricaría más. Lo mismo pensaba de sus hombres, un atajo de inútiles incapaces de acertar a un ladrillo a diez metros.

~---Es él ~---murmuraban por lo bajo buscando las linternas.

~---Nadie sabe qué hace con los que desaparecen.

~---Dicen que nunca descansa\dots\ que nace de las cenizas\dots

Silenciador escuchó atentamente la sarta de supersticiones que formulaban sus hombres y se lamentó de comprender que estaba igual que si estuviera solo. Su rival ya tenía la mitad del trabajo resuelto. Se ocultó y esperó pacientemente. No iba a auxiliar a aquellos hombres, no iba a efectuar ni un solo disparo. No dejaría que el olor a ozono le delatara. Él también sabía jugar en las sombras.

Cientos de ruidos inundaron el amplio almacén. Siempre libraba las batallas más importantes en almacenes, pensó. Los sicarios de Gorgon estaban desconcertados. Esperaban un silencio total, no una explosión de extraños e inquietantes sonidos, tan extraños que no sabían si eran o no producidos por ser vivo alguno. Silenciador les miró, sudando, llenos de pánico. No le eran de ninguna utilidad, sólo podía aprender de ellos cómo hacer frente a su enemigo.

Pero su enemigo no era alguien que se mostrara fácilmente. Por lo que pudo ver, parecía extremadamente rápido, anormalmente incluso. No recordaba haberse enfrentado a ningún héroe disfrazado que tuviera tal poder. Al mismo tiempo parecía controlar las sombras y alterar su aspecto y forma, además de no resultarle desconocida la lucha. Los gritos inundaron la sala en lo que, con una eficacia que Silenciador encontró admirable, inutilizó a todos y cada uno de sus oponentes. Luego, entre sombras, apenas el ala del sombrero visible, se quedó quieto. Silenciador sonrió. Había cometido un error.

\emph{Sal} ~---dijo la silueta.

Silenciador le hizo caso y, emergiendo de su escondite, disparó una andanada letal contra su enemigo. El rayo le atravesó limpiamente y cayó al suelo. No gritó. No dijo una sola palabra. Bajó y se acercó hacia él.

Cuando llegó comprobó estupefacto que no quedaban más que los ropajes allá donde su enemigo había caído. Se acercó a examinarlos y de repente comenzaron a arder. Se consumieron muy lentamente, y sólo quedó ante él una fina capa de ceniza. Sorprendido por primera vez, se quedó mirando los supuestos restos de su enemigo.

Y entonces se dio cuenta de que el error lo había cometido él.

Recibió un potente puñetazo en plena mandíbula, y antes de pudiera reaccionar un montón de ellos lo tumbaron al suelo, cada uno más potente que el anterior, proveniente de todas direcciones. Como si su atacante poseyera cientos de brazos.

Su arma cayó al suelo, y cuando trató de ir a por ella se encontró con aquella silueta, y con aquellos ojos negros que lo miraban con furia, como si acabaran de volver del mismo infierno.

\emph{Se acabó, Silenciador. No eres nada sin tu arma.}

~---Por eso ~---dijo Silenciador con calma~--- nunca me deshago de ella.

Apretó un botón disimulado y el arma volvió magnéticamente a su mano. Levantó el cañón en modo letal y apuntó a su enemigo. Se sentía raro, como si estuviera pasando algo delante de sus narices, algo que no podía discernir, distinguir con claridad entre todo el espectáculo de sombras y humo.

\emph{Dispara si quieres.}

~---Si lo dices es porque no quieres que lo haga ~---replicó Silenciador. Pero sabía que habría otras batallas, y bien aquella podía acabar en empate. Bajó el arma y salió corriendo del almacén. Nada ni nadie se lo impidió.

~---Volveremos a vernos ~---dijo con calma antes de perderse en las calles industriales.

La silueta no respondió.

\bigskip\noindent
Ellen Gorgon no interrumpió a Silenciador mientras le contaba lo ocurrido. Ni siquiera después, ocupada como estaba en mejorar las alas del último modelo que Gorgon Enterprises estaba a punto de sacar al mercado. Una vez descubrió la curva más aerodinámica tomó una breve nota al pie del plano en lo que su mano alienígena tamborileó sobre la mesa. Llamó a dos ingenieros, les dio las instrucciones y se retiraron igual que los más eficientes guardaespaldas. Acto seguido, como si fuera la menor de sus preocupaciones, se centró en Silenciador.

~---Has fallado ~---se limitó a decir.

~---Yo nunca fallo. Lo hubiera hecho de haberme quedado allí.

~---Valis no ha llegado a su destino. Ese es el único fracaso que me importa.

~---Subestima el poder de ese tipo. Es como los héroes de antaño.

~---Ya no hay héroes como los de antaño. Y ni siquiera contra esos eras capaz de medirte.

Silenciador empuñó el arma contra su propia líder. Gorgon no se movió. Se limitó a mirar el cañón de arma que la estaba apuntando.

~---Parece que, sea quien sea ese tipo, no juega al mismo juego que tú. Mira bien tu arma.

Silenciador sufrió una punzada de temor. Recordó que el arma cayó de sus manos. Sólo fue un momento, pero suficiente tal vez para\dots

No era su arma. Era falsa. Una magnífica imitación, tan buena que no notó la diferencia en el fragor de la batalla. Tal vez tardaron años en hacerla.

Le conocía muy bien. Tan bien que sabía qué tenía que quitar y dónde para que al atraer magnéticamente el arma no fuera ésta sino otra la que iba a recibir. Por eso no se movió, por eso no temía que le disparara. Pero ya le había disparado antes. Le había matado. Aquel cabrón, concluyó en medio de toda la confusión, era muy listo.

Y había tenido suerte. Obviamente no era la única arma que guardaba. Por precaución, obligó a aquel tipo que la fabricó años atrás a hacer dos, por si aquel fatídico momento llegaba alguna vez, antes de destruir él mismo el asteroide del cual provenía el material necesario. Iría a por ella. Volvería a ser él mismo. Pero se había acabado trabajar para Ellen Gorgon. Era hora de ver mundo.

Bajó el arma y se dio la vuelta para salir por la puerta. Gorgon sacó un arma del cajón y apuntó a la espalda del que había sido su más leal soldado. Sabía que debería matarle. Pero por un momento, dudó. Silenciador sabía por qué.

~---Si me mata ahora, nunca sabrá dónde la escondo.

Ellen Gorgon le miró de arriba abajo. Tenía razón. Si le disparaba jamás encontraría la copia del arma. Sin embargo, tras mucho reflexionar, se dio cuenta de que Silenciador era un recuerdo de un tiempo perdido de héroes y villanos, una pieza de coleccionismo que había guardado durante demasiados años. Si lo dejaba marchar bien podría volverse en su contra, desarrollar viejos ideales, unirse a nuevos enemigos. No, hacía mucho que Ellen Gorgon había dejado de jugar al ajedrez con Ernépolis~I. Además, aunque fuera una posibilidad remota, siempre podría tratar de recuperar la original.

Apuntó al corazón y disparó.

Ellis Saw estaba planeando la agenda del día siguiente de la presidenta Gorgon cuando primero escuchó el disparo y luego vio que ésta salía del despacho. Parecía vagamente alterada, como si acabara de tener un percance. Saw no preguntó. Hacía mucho que había dejado de hacerlo. Tal vez por ello Gorgon no prescindía de él.

~---Retira de la nómina a mi guardaespaldas personal, Ellis. Ha decidido abandonar nuestra gran familia.

Siguió avanzando hacia la sala de producción hasta que desapareció de la vista de Saw. Éste se acercó a ponerse en contacto con el departamento de economía de Gorgon Enterprises, ya que el sueldo de Silenciador no procedía de las arcas de la ciudad. Sin embargo, antes de llevar a cabo el cometido hizo una llamada más personal.

~---Todo ha salido como se esperaba ~---se limitó a decir antes de colgar.

\endinput
