Estaban en todas partes. Observando. Acechando. Vigilando incansables cada rincón de la ciudad, al tanto del devenir de los acontecimientos.

Exactamente igual que ellos.

Entre los pliegues de una rebelión civil, una segunda insurrección se estaba gestando. Una batalla entre fuerzas rivales que estaba a punto de eclosionar\dots

\fancyparbreak
Emma Blades terminó su jornada laboral en el distrito financiero de Ernépolis~I y miró por la ventana antes de apagar el ordenador y, seguidamente, las luces de su despacho. Estaba en el margen exacto de la Nube y podía notar cómo un poco más arriba, desde otras torres más afortunadas, se percibía una vaga claridad que algunos llegarían a identificar con la luz del día. Era una luz pobre, apagada y viciada, refractada por capas de ceniza, polvo y contaminación.

Pero era luz, sustento, vida. La razón de ser de mundos enteros, de generaciones a la deriva. Levantarse cada mañana para ver un nuevo amanecer.

Pero no allí, no en Ernépolis. Una ciudad tan cubierta por la polución que ésta había bajado al nivel de la calle, al menos temporalmente, durante el llamado Descenso. Un fenómeno climatológico sin precedentes que precedía a desastres de tráfico aéreo y terrestre tan grandes que todo acceso a la ciudad había sido restringido a lo largo de cinco días.

Y ahora estaban en el segundo de ellos, aislados del exterior. Aunque en realidad la ciudad nunca había gozado de demasiado atractivo turístico, a no ser que se fuera un declarado amante de la arcaica decadencia postcyberpunk.

Tomó el ascensor y, nada más pisar la calle, notó cómo la pesadez de la Nube se hacía patente en sus pulmones. Las autoridades habían recomendado el uso de mascarillas, aunque debido a la alta densidad y masa de las partículas contaminantes tampoco era algo realmente vital. No en vano no se trataba más que de la misma porquería que caía a menudo en el pavimento, sólo que flotando vagamente en el aire en vez de pegándose al calzado.

Blades miró hacia arriba, hasta donde alcanzaba la mirada. Algunas torres estaban por encima de todo aquello. Por encima de polución, contaminación y podredumbre. Durante cinco días, en términos reales y metafóricos. Cuando acabara el Descenso sólo en términos metafóricos.

Comenzó a caminar por las calles, y el sonido de sus propios pasos en el suelo fue la primera advertencia de que el entorno estaba cambiando, al menos de manera temporal. Las calles se encontraban muy tranquilas, incluso las más céntricas, y las manifestaciones se habían disipado por fuerza, debido a la espesura de la niebla. Aún eran frecuentes en barriales alejados del centro, principalmente porque los que vivían allí apenas tenían cuatro maderos a los que llamar hogar, y por tanto no perdían nada de echarse a la calle de nuevo.

Parecía como si un fantasma se hubiera instalado sobre la ciudad, o algo parecido. Un ente que estuviera absorbiendo su esencia vital, retrotrayéndola a tiempos antiguos, de brujas y hogueras de Salem.

No había apenas vehículos deslizadores y nadie hablaba con nadie. Parecía increíble que la ciudad hubiera estado al borde de la explosión social hacía apenas veinticuatro horas.

Pero bajo techo, en locales resguardados, el odio fermentaba.

Blades lo sabía bien, y no era ajena a ello. Tampoco el sujeto al que tenía que sacar al estrado, un ratero de poca monta que decía haber visto un asesinato con sus propios ojos. Uno de tantos charlatanes exagerados a los que seguramente no tardaría en sacar la escasa verdad de sus palabras con unas cuantas preguntas hábiles.

Su trabajo no era agradable, pero no podía concebir el derecho sin la rama de lo penal, y menos en lugar tal como Ernépolis~I. Aun imperfecta, aun corrompida, la justicia era la justicia, y en ella había decidido creer, al menos hasta que la realidad se impusiera a lo contrario.

No se trataba de altruismo. Tampoco de ingenuidad. Blades distaba mucho de ser ingenua. Conocía muchos abismos, grandes, enormes pozos de oscuridad moral sin límite. Pero precisamente por ello no cejaba en pensar que tienen que existir normas, reglas escritas para detener las tinieblas ocultas en el género humano.

Aunque esas reglas no siempre sean del agrado de todos.

Tras mucho caminar, incapaz de tomar un taxi deslizador, llegó al local que solía frecuentar por aquellos días, una sala de juego que era, además, un hervidero de jugosa e interesante información. El punto débil de Blades: la necesidad de conocer, de saber, no para acumular una posición de poder, sino por la información misma.

La curiosidad al servicio de la curiosidad. El conocimiento, supeditado a la acumulación de datos.

No con eso ocurría que no llegara a obtener información útil, claro. Mucho de lo que sabía le había servido para conocer a fondo los trapos sucios en los que quería meterse, sobre los que quería levantar un auto procesal, y también la manera adecuada de blindarse contra amenazas del exterior.

Nadie se libraba de su curiosidad. Policía, empresas, sindicatos, criminales. Combinaciones de todo lo anterior.

Sombras furtivas que se deslizaban en la oscuridad.

Cuando llegó al local, el tipo de la puerta la miró inquisitivamente un segundo y acto seguido la dejó pasar. En realidad, el acceso de aquel lugar era libre, pero tener un portero que fingía no dejar pasar a alguien implicaba un cierto toque exótico de exclusividad y alejaba a los más indeseables, que no se atrevían a intentar entrar con semejante armario de tres cuerpos en la puerta.

Bajó las escaleras del estrecho acceso, una a una, tacón después de tacón, y aún no había descendido del todo cuando ya estaba sacando un cigarrillo y colocándolo en la boquilla.

Una vez abajo miró con interés las mesas de ruleta, así como las de Blackjack. Aquel era uno de tantos locales de juego de la ciudad, bastante llenos en esos días. Cualquier vicio era bueno con tal de que los días del Descenso pasaran cuanto antes, pensó dando una calada al cigarrillo.

Como quien no quiere la cosa se acercó a una mesa donde estaba su soplón, el elegido para la gloria y, con suerte, para la protección de testigos. Poniéndolo todo a par. Rezando para que no saliera impar.

La ruleta paró, y salió el cero.

~---Si te ven aquí jugando, tu testimonio valdrá tanto como tu suerte en este momento ~---arguyó Blades acercándose a su posición.
 
~---No lo entiendo. De verdad que no lo comprendo. Aquel tipo acaba de irse y le estaba yendo fantásticamente bien. Además, acabo de llegar, si la ruleta estuviera trucada, ¿para qué iban a echarme del juego tan pronto?

~---Tal vez porque él tenía más dinero que tú para apostar y andaba tras una mayor recompensa ~---dijo Blades mirando hacia donde señalaba su ludópata cliente. Nada más mirar se quedó quieta, tanto que el cigarrillo se consumió y cayó al suelo por su propio peso.

Sky se acercó hacia ella y la miró con una vaga sonrisa. Ya hacía tiempo que había aprendido a distinguirlas, y aquella era, sin duda, de las más falsas que había esbozado desde que le conocía.

~---Dime que no me estabas siguiendo ~---replicó Blades, inquieta.

~---No a ti directamente, pero sí a él ~---señaló al testigo, aún en la mesa.

~---¿Él? Pero si sólo ha visto un ajuste de cuentas, y de los más anodinos, además.

~---Las cosas no son como imaginas. Resulta que el muerto, si bien estaba en la calle, había sido un pez gordo en el pasado.

~---¿Cómo de gordo?

~---De notables dimensiones. Trabajaba en el distrito comercial, en Qubit, Inc.

~---¿En QI? ¿Cómo pudo acabar en la calle un tipo así?

~---Drogas, Valis al parecer, pero su vida laboral está blindada y no sabemos tampoco nada de ningún paracaídas dorado. El caso es que podría haber detrás de esto más de lo que parece.

~---Si tú lo dices\dots

~---Yo mismo he venido a custodiarle, aunque no lo parezca aún soy un poli de calle cuando hay que serlo. Además, irónicamente, a mí me tienen menos visto que a los patrulleros que suelen pasar por la zona.

~---Eso es que te estás convirtiendo en todo un poli gordo de los de comer rosquillas caramelizadas ~---comentó ella con tono de burla, jugando de manera sensual con la mirada.

~---Sugiero que saques a tu testigo de este lugar y le acompañes a casa. Yo os seguiré de cerca. Es mejor no levantar sospechas. Si el tipo sospecha que hay detrás de esto algo complicado podría querer callar la boca por miedo a represalias, ya sean reales o imaginadas.

~---¿Y qué hay de mí? ¿Cuándo pensabas decírmelo?

~---¿Acaso eso te hubiera detenido a la hora de llevarle al estrado?

~---¿Acaso hubieras mandado a un subalterno que le siguiera en otro caso?

Ninguno de los dos contestó. No hacía falta. Conocían bien las respuestas a ambas preguntas.

Se separaron y Blades esperó pacientemente a que su protegido terminara de perder unas cuantas vueltas más. Después de eso le largó de allí con tanta premura que bien parecía que le estuviera sacando a rastras.

~---¡Aún no he terminado de jugar! ~---inquirió.

~---Tienes razón. Ahora mismo estás jugando con mi paciencia ~---fue la lapidaria respuesta de Blades. El tipo no se atrevió a hacer un solo comentario en un buen rato, en lo que ambos caminaban por la desértica calle. Al fin, más por necesidad humana que otra cosa, comenzó a hablar.

~---¿Conocías a aquel tipo?

~---Estaba conchabado con los de la mesa. Te hubieran dejado sin empastes si no te hubiera sacado de allí.

~---¿Cómo lo supiste?

~---No había más que verle aparentar. Fingía tan mal que parecía policía ~---dijo permitiéndose un chiste personal.

Blades se preguntó si Sky les estaría siguiendo, tal y como había comentado que haría. Su duda vino del hecho de que, si así era, no podía ni verle con aquella espesa niebla. Algo que no hizo sino inquietarla profundamente.

El Descenso. Un momento ideal para arreglar asuntos pendientes, reflexionó.

A mitad de camino vislumbraron, por primera vez en varias manzanas, una silueta que avanzaba desde la niebla lejana hacia ellos. Se veía como si fuera un espectro irreal, un espejismo de otro mundo, hasta que se cruzó con ellos y comprobaron que no era más que una mujer cualquiera de la ciudad, con una vestimenta entre lo informal y lo elegante y un bolso de factura bastante amplia.

Podría haber sido un encuentro casual cualquiera. Uno de tantos transeúntes que uno puede cruzarse en un espeso día neblinoso. Pero la mujer se detuvo y Blades comprendió que la dinámica urbana se había alterado por completo.

~---¿Emma Blades? ~---dijo sin preámbulos intermedios, como si acabara de caer en algún detalle repentino.

~---¿Nos conocemos? ~---fue la única respuesta de Blades, entre preocupada e intrigada. Tenía buena memoria para las caras y la de aquella mujer joven, de largo pelo rubio y mirada torcida, no hubiera sido una que olvidara fácilmente.

~---No nos conocemos, permítame que me presente. Me llamo Perséfone.

~---¿Perséfone qué más?
 
~---Perséfone basta. Llevo tiempo siguiendo su trabajo. En realidad no por gusto, más bien por órdenes de mi superior.

~---Escuche ~---dijo Blades, nerviosa, sacando un cigarrillo del bolso y colocándolo sobre la boquilla~---, no la conozco y tengo que llevar a mi amigo a casa en este momento. Ha bebido demasiado.

~---No se preocupe, no quería entretenerla, ya nos cruzaremos en alguna otra ocasión. Permítame ~---dijo metiendo la mano en su ancho bolso, presumiblemente para buscar un mechero.

~---Alto ~---escucharon desde detrás, al tiempo que Sky salía de la niebla apuntando con una pistola a la mujer~---. Saque la mano del bolso, señorita, por favor.

Perséfone hizo como la mandaron y su mano se deslizó del bolso lentamente. No había nada en ella, aunque sus dedos estaban arqueados, como si hubiera sostenido algo entre las manos.

~---Tranquilo, ¿agente? No quería hacer nada, como ve, no tengo nada en las manos. Al menos\dots

De repente dobló el índice hacia dentro, y el sonido de un disparo cruzó la calle como si se hubiera tratado de una ilusión auditiva. Cuando quisieron darse cuenta, el testigo yacía en el suelo, con un impacto de bala en el pecho. Blades se agachó hacia él, intentando reanimarle.

~---\dots\ nada que pueda verse, claro ~---terminó levantando la mano.

~---¡No se mueva! ~---gritó Sky, completamente anonadado por lo sucedido~---. Un solo movimiento y dispararé, ¿me escucha?

~---Le escuchó, aunque no necesito hacerlo. Sus frases son ya muy previsibles para mí. Deberían cambiar el guión de vez en cuando.

~---Se lo advierto, señorita\dots

Perséfone hizo un ademán de levantar la mano, y Sky disparó. Para su sorpresa, la mujer extendió la mano desnuda y las balas rebotaron en un muro aparentemente inexistente, aprovechando la confusión para salir corriendo.

~---¡Llama a una ambulancia! ~---gritó Sky a Blades en lo que salió corriendo tras la misteriosa mujer. Sin duda poseía poderes, y lo peor era que no estaba seguro de dónde estaban sus limitaciones.

Las suyas, sin embargo, las conocía bien. Sólo era un hombre.

Recordó lo que Starr Miles solía decir. Un hombre puede ser suficiente.

Pero mejor ser un hombre que acecha en las sombras ayudado de gran cantidad de aparatos disuasorios, pensó.

La niebla le obligó a probar a ciegas, y de ese modo llegó hasta un callejón sin salida salvo por una escalera de metal que estaba situada al fondo, inutilizada por gran cantidad de bolsas de basura.

Se acercó a la base de la escalera y notó que las bolsas estaban rotas. Había subido por ahí. Elevó la pistola y la tuvo al alcance del objetivo, varios metros por encima.

~---¡No te muevas! ~---gritó, preguntándose qué era lo que ocurriría entonces.

~---No lo haré, palabra. De hecho, admiro tu perseverancia, y por eso te diré que deberías ser tú quien lo haga.

Sky no sabía a qué podía referirse, pero esa advertencia hizo que agudizara los oídos aún más. Y entonces lo escuchó.

Clonc, clonc, clonc. Un sonido que hacía una eternidad que no escuchaba, desde su formación como policía.

El sonido de una granada cuando chocaba contra una superficie de metal. En ese caso concreto, al caer desde una escalera oxidada.

Salió corriendo como alma que lleva el Diablo, justo a tiempo para saltar detrás de unos cubos de basura cercanos que le protegieron de la metralla de la granada al estallar. Tardó un buen rato en levantarse, sobre todo porque el salto no había sido precisamente uno que se pudiera calificar de olímpico.

Cuando se levantó, como era de esperar, Perséfone ya se había ido, aunque aún la escuchó despedirse.

~---Adiós, Jefe Sky. No olvide decirle a nuestros amigos comunes que Hades les saluda desde el Inframundo.

Dio parte a la central, a la espera de que bloquearan el paso de la fugitiva.

Como era de esperar, él llegó antes.

\emph{¿Qué ha ocurrido, James?} ~---preguntó Scream desde las sombras~---. \emph{Un escuadrón escuchó una explosión cercana.}

Sky miró a su alrededor, a los sitios donde sabía que tenía que mirar. Otros cinco miembros de Los Caídos estaban alrededor, bien parapetados, pero mostrándose visibles para dejar constancia de su solidaridad para con su antiguo compañero de andanzas.

~---Ella lo sabe ~---contestó apoyándose contra la asquerosa pared, cubierta de mugre y polvo. Scream le ayudó a incorporarse del todo, rompiendo por completo el espejismo de silueta implacable que siempre estaba acostumbrado a ejercer~---. Sabe quiénes sois y cómo operáis. Maneja objetos invisibles, es muy peligrosa. No he estado ni siquiera cerca de suponer una amenaza para ella en ningún momento.

\emph{¿Nombre?} ~---preguntó Scream nuevamente. Se notaba la rabia en su voz, aun a pesar del modulador.

~---Se llamó a sí misma\dots\ Perséfone. También habló de un tal Hades, que saludaba desde el Inframundo. ¿Sabes a qué puede referirse?

\emph{Lo sé} ~---añadió Scream al comentario, sin apenas dudarlo.

Sky le miró fijamente, esperando que continuara.

\emph{Se refiere a la ciudad, James. Concretamente, nos está diciendo que ahora estamos en su terreno} ~---terminó en lo que se alejaban juntos del callejón.

\endinput
