Muchas preguntas ya tenían respuesta. Pero al mismo tiempo, nuevas incógnitas salían a la luz. Misterios que hacían tambalearse la estructura misma de la organización y arrojaban nuevas y sorprendentes pistas sobre su origen.

Y gran parte de esas respuestas estaban en manos del que había sido hasta ese momento su enemigo parapetado en el anonimato\dots

\fancyparbreak
Sam Grove le dijo que no se lo creería cuando lo supiera. No se hacía una idea de cuánta razón tenía.

El Juez Nitram y Starr Miles, aliados. Amigos. O al menos, algo muy similar. Las circunstancias concretas de esa relación eran aún desconocidas para ellos, pero sin duda distaban mucho de ser de naturaleza meramente casual. La justicia había movido los actos de aquellos dos hombres desde antes que él mismo pusiera los pies sobre una nave espacial. Eran perros viejos en un mundo más viejo aún.

Y ahora el pasado regresaba para ponerlo todo a prueba. Establecer la duda, la incertidumbre sobre en qué han estado basando realmente todo aquello que han construido. Quién era realmente Starr Miles. Él mismo decía que el pasado no importaba, sólo el presente. Pero lo que por aquel entonces era el presente ya se había convertido en el pasado para todos ellos.

Ya habría tiempo de indagar en detalle sobre lo que acababa de salir a la luz, por otro lado. Era el momento de la acción. De atacar con fuerza, rapidez y contundencia.

Como Echo había anticipado, los suyos no habían perdido el tiempo en su ausencia. Ya tenían los datos de la inmensa mansión en la que Nitram se había retirado, un lugar que al parecer ocupaba de manera habitual para alejarse de las cámaras que atormentaban a otros jueces y otras personalidades públicas. No había cifras exactas, pero la mansión podía estar fácilmente vigilada por treinta o cuarenta guardaespaldas, y eso sin contar con torretas de seguridad en pasillos y puntos muertos.

El concierto de The Jammers no había empezado aún, además. Grove estaba allí, infiltrado entre el público, esperando dar la buena noticia de que las comunicaciones se habían restablecido de la manera más clara posible: usando como prueba el propio mensaje enviado. Su escuadrón también estaría por los alrededores, de modo que se pudiera restablecer la normalidad en la medida de lo posible.

Según los datos de Razorclaw habían sucumbido veinticuatro hombres en las calles desde el comienzo de la vendetta. Diez habían sido descubiertos por cazarrecompensas y obligados a fingir ser imitadores, cinco habían sacrificado su puesto para llamar la atención de manera voluntaria y añadir confusión a los casos anteriores, cinco habían sido heridos de gravedad y dos parejas habían muerto, una de ellas a causa de las letales heridas y otra al autoinmolarse para proteger el secreto.

En total, la organización había perdido a la cuarta parte de sus componentes, pero tendrían que aprender a surgir de las cenizas. Lo primero de todo sería que no se ampliaría la formación. Casi todos los que habían muerto eran novatos, y no se perdonaría que algo así pudiera pasar de nuevo. Aquella experiencia le había enseñado que, si bien eran un equipo, cada hombre tenía que estar sobradamente preparado para actuar como si estuviera solo frente a todas las adversidades existentes en el mundo.

De todos modos, Los Caídos no podrían ser de mucha ayuda en el asalto al interior de la casa. Dado que las comunicaciones aún no se habían restablecido, lo que harían sería asegurar el exterior y los alrededores y dejar la parte de la infiltración a él y los cazarrecompensas, más que ansiosos por llegar hasta el epicentro del asunto.

Cuando se reunió con ellos comprobó que, pese a que Krexon se había escabullido, no habían estado perdiendo el tiempo precisamente. El socio de Dobleseis, Códec, tenía un buen arsenal que no habían desaprovechado en lo más mínimo, y pudo comprobar cómo estaban armados hasta los dientes con toda clase de lanzarrayos, gran parte de potencia no letal, pero muchos otros, pensados para las defensas automáticas, más que demoledores.

Dobleseis, sin embargo, apenas cogió armas. Estaba extasiado con su juguete nuevo, y planeaba toda clase de maniobras que realizar a dos manos con aquellos revólveres multiusos. Scream no quería ni pensar qué hubiera podido hacer Silenciador en su momento si hubiera tenido la capacidad del cazarrecompensas para poder dividir su concentración y puntería con simétrica eficacia.

Los cazarrecompensas, dentro del hecho de que eran tipos que actuaban por separado, se habían logrado compenetrar bastante bien. La planificación fue sencilla, rápida y ágil. No había sitio a las sutilezas en aquella ocasión. Además, la improvisación sería una parte esencial del ataque. Gracias a Saw, que había estado dentro de aquella casa en el pasado, cuando trabajaba para Gorgon, conocían el emplazamiento de los lugares más importantes de la misma, pero la disposición, en mayor o menor medida, podía haber cambiado.

La mansión disponía de tres plantas y estaba realizada en estilo precolonial, con una fachada adornada con complicadas escalinatas y columnas de orden corintio. Los ventanales poseían amplios dinteles, pero el blindaje de los cristales era muy resistente y difícil de atravesar. El techo permanecía también bajo vigilancia, y en términos globales el lugar apenas tenía entradas secundarias de tipo alguno, pues el garaje era subterráneo y se accedía desde un pasadizo situado muchos metros atrás. En su momento, de hecho, se intentó conectar una de las entradas del Aquerón con el interior de la mansión, pero resultó a todas luces imposible, si bien Scream no dudaba que la cantidad de pasillos secretos que debía poseer aquel lugar sin duda sería más que considerable dada su importancia política.

De todos modos, aunque no hubiera una entrada, no hubo problemas al respecto. Repulsor no tardó en fabricar una, tratando de calibrar al máximo el disparo para así no suponer un peso muerto una vez lo hubiera generado.

Los primeros en pasar fueron Dobleseis, Silencio y el propio Scream. Barrera cubriría a Repulsor, y Batería estaba al cargo de hacer que se recuperara cuanto antes, así como recargar de nuevo su equipo al completo. Fue por ello que obviamente se quedaron atrás en comparación con sus más veloces compañeros, que salieron disparados escaleras arriba para ganar una planta antes de que la movilización y la confusión les obligaran a pelear por cada palmo que recorrieran hacia la estancia principal, donde suponían que estaría Nitram, tal vez incluso esperándoles.

Las primeras defensas se cruzaron en el camino de los tres furtivos, y Dobleseis las despachó de un par de tiros sin dejarlas siquiera realinearse. Por desgracia la cosa empezó a ponerse al rojo vivo, y no tardaron en encontrarse con cinco hombres que les cortaban el paso. La batería de disparos fue encarnizada, pero uno por uno lograron derribarles, derrotando Scream a uno de ellos, Dobleseis a tres y Silencio al último, una vez logró colocarse a su espalda.

~---Perfecto, mudito ~---fue la única respuesta de Dobleseis. Silencio se acercó de nuevo a ellos, su actitud tranquila y pausada, evaluando la situación. Sin embargo de repente, sin previo aviso, apartó a Dobleseis de un golpe y recibió una descarga de una de las defensas automáticas abatidas, que había disparado con parte de energía residual que había logrado acumular.

El disparo no era fatal, pero el disparado estaba muy malherido. Dobleseis se quedó mirando, sin saber qué decir. Era la primera vez desde que era cazarrecompensas que alguien recibía de manera voluntaria un tiro en su lugar. Como única respuesta, frió la torreta con un disparo de sobrada potencia procedente de una de las antiguas armas del peor enemigo de Reflector.

Silencio les miró y les hizo un gesto para que continuaran.

~---¿Estarás bien? ~---preguntó Scream.

Escucharon ruido de disparos al fondo, y supieron que se trataba de sus compañeros rezagados. Silencio se limitó a hacer un gesto con la mano abierta para que se marcharan. No había que ser ningún lince para deducir que se ocultaría hasta que ellos alcanzaran su posición.

~---De acuerdo, seguiremos entonces.

Dobleseis y Scream prosiguieron a lo largo del pasillo, y si bien se encontraron con otros guardaespaldas por el camino, quedó claro que el grueso de la artillería ya había sido lanzado ya, algo más lógico que ir aumentando de manera escalonada la dificultad de los oponentes a medida que llegaban a su objetivo. Las defensas no supusieron problemas tampoco, aunque habían recalibrado su puntería en función de los movimientos de sus blancos y resultaba más complicado esquivar sus disparos, sobre todo para Scream, que confiaba más en engañarlas con sus dispositivos y hacer que sus rastreos de imagen, calor y cinética se confundieran para dispararse entre ellas.

~---Ingenioso ~---se limitó a decir Dobleseis al tiempo que recargaba sus cuatro lanzarrayos, con sus respectivas bocas humeantes y rojizas.

Ante ellos se alzaba la puerta de doble batiente que les separaba del despacho en el que sin duda se hallaba el Juez. Scream hubiera optado por una entrada pausada, solemne, dejar que las puertas se abrieran con lentitud para después proyectar una silueta anormalmente grande escurriéndose por los pliegues de luz del suelo.

Dobleseis se limitó a echarlas debajo de una patada en lo que tenía listos y preparados los seis gatillos de sus armas.

El Juez Nitram estaba ante ellos, al otro lado de un pulido escritorio de roble, en una sala con una fastuosa decoración heredera de tiempos extintos. Llevaba un traje negro de corte antiguo, y tenía una mano en el bolsillo mientras la otra se apoyaba en el respaldo de la silla, tan arcaica como el resto del mobiliario.

Dobleseis no dudó ni siquiera un segundo. Descargó los cuatro lanzarrayos directos hacia el Juez. Sin embargo cuatro discos salieron de detrás del escritorio, veloces como insectos enfurecidos, y se interpusieron en la trayectoria de los disparos. Los otros seis no tardaron en mostrarse también a la vista de los asaltantes.

~---Es de sujetos como usted de lo que estas máquinas me protegen ~---apuntó Nitram, indiferente, como si no estuviera hablando en persona con ellos sino a través de una pantalla~---. Pero toda precaución es poca ~---acabó apretando un botón del escritorio, justo después de que sus discos avanzaran algo más de medio metro de distancia, lo que hizo que Dobleseis y Scream se alejaran hasta una pared cercana con una estantería llena de libros. Ninguno de los dos notó nada extraño, pero estaba claro que había activado alguna clase de protección invisible, tal vez energética.

\emph{Se acabó, Juez Nitram. Será cuestión de tiempo que lleguen los otros y se unan a la pelea.}

~---Eso no es problema para mis ayudantes ~---dijo señalando a los discos con un gesto de la palma abierta~---. Sus circuitos tienen bien integrados todos los trucos de estos mercenarios.

~---¿Qué es lo que quiere de los cazarrecompensas? ~---preguntó Dobleseis~---. ¿Exterminarnos? ¿Inculparnos y encerrarnos?

~---Aunque así fuera, aunque considerara que son la escoria de la sociedad, no procedería de tal manera, pues ahora sirven a un bien mayor, un objetivo mucho más honorable que perseguir delincuentes y matones por una despreciable suma de dinero. Ahora son parte de un gran plan, una estrategia global.

\emph{¿Tiene algo que ver ese plan con Starr Miles?}

Nitram miró a Scream, lleno de asombro. Al parecer, dedujo, no sólo ellos buscaban respuestas concretas a preguntas complejas.

~---De modo que mi sospecha era cierta ~---comentó~---. Miles está detrás de ti, de tu creación. Eso no hace más que confirmar la utilidad de todo lo que ha estado pasando.

\emph{¿Qué buscaba con todo lo sucedido?}

~---¿Acaso no lo sabes? Pensé que al menos tú, que pareces tener sobradas aptitudes para la investigación, acabarías por averiguarlo. Miles y yo teníamos objetivos comunes, buscábamos alcanzar un mismo fin: la forja de un ser especial, único. Aunque con fines muy distintos, claro, tan distintos como nuestras respectivas personalidades. En el caso de Miles, él estaba obsesionado con la justicia, con encontrar una figura que impusiera respeto, miedo y obediencia.

»En mi caso, buscaba comenzar la Purga.

Dobleseis y Scream miraron fijamente a aquel hombrecillo en apariencia inofensivo, de no haber sido por esos discos endiablados que flotaban por delante de su cuerpo. En sus ojos se reflejó un brillo de malevolencia insondable e infinita.

~---¿A qué se refiere? ¿Un alzamiento?

\emph{No} ~---dijo Scream, encogiendo los hombros~---. \emph{Religión. Un advenimiento.}

~---Parece que sí eres una creación de Miles después de todo, y al menos estás bien informado ~---fue el único comentario de Nitram.

\emph{La Purga} ~---explicó Scream a Dobleseis, pero sin dejar de mirar a su enemigo común~--- \emph{es una vieja creencia de los Gilock. Ellos creen que vendrá un superser, o algo similar, que traerá una nueva era al Universo, que acabará con la corrupción y el caos. Que se enfrentará a otros similares a él, pero inferiores en realidad, y los exterminará, para convertirse en el único y verdadero, aquel cuyos designios todos seguirán, por ser éstos justos y auténticos.}

~---Explicado de manera tosca y simple, así es ~---apuntó Nitram.

~---De modo que por eso traernos aquí, enfrentarnos los unos a los otros, montar un torneo por toda la ciudad\dots\ para que sólo quedara el mejor.

~---Así es, cazarrecompensas. Mis intereses se habían puesto en el Caído, y por eso puse una recompensa por su cabeza. Una vez lo derrotaste, dejó de tener interés para mí, y tú pasaste a ser mi objetivo.

\emph{Por eso interfirió las comunicaciones, para impedir que los cazadores se informaran entre ellos y no pudieran hacer equipo.}

~---En efecto.

\emph{No ha funcionado, Juez. Nos hemos rebelado, y su experimento ya no tiene éxito.}

~---Eso es lo que crees, ¿no? Pero la mente de los seres es voluble, y sus intenciones son maleables. Demuéstraselo, ya ~---ordenó Nitram, sin que Scream supiera a quién se estaba dirigiendo.

No tardó en averiguarlo cuanto unos brazos violáceos surgieron de la pared para agarrar a Dobleseis del cuello y apuntar a su cabeza con un láser.

~---Volvemos a vernos ~---dijo Krexon guiñando los ojos.

~---Como puedes ver, ahora mismo tengo al cazarrecompensas en mi mano. Hay una gran suma aún en pie para quien me lo traiga vivo. Está inmovilizado, quieto. Si mueve un músculo, es hombre muerto. Ah, ya vienen ~---dijo Nitram escuchando alboroto en el pasillo~---. Justo a tiempo.

Los otros cazadores irrumpieron en la sala y cruzaron las puertas aún abiertas para colarse en su interior. Silencio estaba apoyado en el hombro de Batería, pero no había soltado las armas ni tenía intención de hacerlo, y Batería, además de sus propias pistolas recargables, portaba en la cintura unas granadas de pulso eléctrico que después de pasar por sus manos podían resultar como poco devastadoras. Barrera llevaba a la espalda un rifle de asalto militar, y Repulsor sostenía una de las defensas de la propia casa que había arrancado de cuajo para usar de improvisada arma automática a dos manos.

~---Tú eres el que está detrás de todo esto, ¿verdad? ~---preguntó Repulsor sin demasiada cortesía.

~---Escucha atentamente, pues te interesa lo que voy a decir, y vosotros también. A partir de ahora retiro la recompensa por todos vosotros, y sólo la ofrezco por ese justiciero de ahí ~---señaló a Scream~---. Nueve millones si acabáis con él. Tocáis a más de dos millones por cabeza.

~---Juez, ¿sigue estando en pie la recompensa por esta araña humana? ~---preguntó Krexon, aún con su presa entre los brazos.

~---Así es, Krexon. No prometo en vano cuando ofrezco algo, y habrás sacado a un forajido de las calles.

Los demás cazarrecompensas no dijeron nada. Parecía claro que no iban a aceptar la propuesta, pero hubo un momento de silencio, como si cada uno de ellos esperara que fuera otro el que dijera la primera palabra de rechazo al respecto. Scream, sin embargo, no tenía en ese momento la atención puesta en ellos sino en el incorpóreo y viscoso alienígena.

\emph{Te ha utilizado, Krexon. Como a todos nosotros. ¿Por qué crees que no ha puesto un precio por ti?}

~---Porque colaboro con él, estúpido ~---replicó el alien con desprecio.

\emph{Eso sólo es parte de la verdad. Ya hace tiempo que te derroté, y él lo sabe. Por eso ya no formas parte de su plan. Tú no eres el elegido que busca. ¿Sabías que fue embajador de los Gilock?}

Los demás presentes, incluido Nitram, no dijeron nada. No sabían qué quería decir con todo aquello.

~---¿Los Gilock?

\emph{Los Gilock, sí. Creo que los Axcronianos tuvieron más que palabras con ellos en el pasado, ¿verdad? Los planetas de ambas especies quedaban bastante cerca, en el mismo sistema, ¿no es así, Juez?}

El Juez no respondió, pero un atisbo de ira empezaba a aflorar en su interior.

\emph{Un sistema binario, si no me equivoco. Y mientras que tu especie evolucionó para que al volverse incorpórea pudieran sobrevivir a las constantes lluvias de meteoritos, los Gilock optaron por desviarlos. Desviarlos, Krexon.}

\rquoti Dime, Krexon, los tuyos no lograron predecir la llegada del meteorito que los fulminó, ¿verdad? Como si su comportamiento no fuera el esperado según su trayectoria inicial, ¿o no?

~---¡Basta! ~---fue la única palabra que pronunció Nitram antes de lanzar sus discos flotantes contra Scream, que a duras penas lograba esquivarlos, engañándolos por medio de ilusiones ópticas. El grupo de cazarrecompensas asistía, insólito, a poco menos que una acusación velada de genocidio por parte de una especie hacia otra de su mismo sistema.

~---¿Usted lo sabía, Juez? ~---dijo Krexon, soltando la presa de Dobleseis, quien al instante, furioso, trató de derribarle, pero sus puños le atravesaron como si fuera un fantasma~---. ¿Es cierto lo que dice?

~---Deberías sentirse orgulloso, Axcroniano. Fuiste elegido, sé que te consideraron el único digno de aspirar a pasar la primera fase de la Purga en tu planeta. Otros que como tú no estaban en él en ese momento no fueron tan afortunados.

~---Morirá por lo que acaba de decir y ninguno de sus aparatos podrá impedírmelo ~---fue el contundente comentario del alienígena, caminando directo hacia el Juez.

El invisible campo de baja frecuencia que estaba entre ellos, sin embargo, le impidió dar un solo paso más hacia la consecución de su objetivo.

Los Axcronianos no podían atravesar ciertas energías, y eso era algo que, como antiguo embajador de los Gilock, el Juez Nitram sabía bien. Por eso, la defensa que había bajado tras la aparición de Dobleseis y Scream, más que para detenerles a ellos, servía de cara a obtener una potencial defensa contra su subalterno aún escondido.

Krexon comenzó a convulsionarse como si le hubieran dado una sacudida de miles de voltios, y cayó al suelo echando humo y emanando un olor como ningún otro que los presentes hubieran percibido antes. Estaba aún vivo, pero desde luego también fuera de combate por un tiempo prolongado.

~---Mira lo que me has obligado a hacer ~---dijo Nitram mirando a su peón caído~---. Te mataré por eso.

~---Me temo que no, amigo ~---replicó Repulsor apuntando hacia los discos de Nitram con toda su potencia~---. Ya hemos visto lo que vale tu palabra con bastante claridad.

~---Insensatos ~---continuó Nitram, enfurecido~---. No sabéis lo que estáis haciendo. Estos discos son la cumbre de la tecnología de los Gilock, y pueden enfrentarse contra ejércitos enteros.

~---Ya lo veremos ~---dijo Dobleseis disparando la red electrificada de uno de los revólveres de Silenciador y atrapando a cuatro de ellos en la misma.

~---Esa red no vale de nada, cazador. Su energía no basta para detener a mis máquinas. Deberías haber supuesto que estaría preparado para esa posibilidad.

~---Habrá que usar la recámara, entonces ~---terminó Dobleseis sacando el segundo revólver.

~---¿Otro arma? ~---comentó Nitram~---. Da igual, aun teniendo este imprevisto en cuenta, ni con una descarga doble podrías dañarlos.

~---¿Y qué tal si la descarga es mayor? ~---añadió Batería tocando el arma justo antes de que Silenciador disparara. Una segunda red de energía concentrada, capaz de freír a cualquiera que entrara en contacto con ella, cayó sobre los discos atrapados y al soltar toda su potencia sobre ellos los dejó tan inservibles que apenas eran capaces de separarse malamente unos centímetros del suelo.

~---Pagarás por eso ~---fueron las únicas palabras de Nitram antes de lanzar a sus restantes discos contra ellos.

En teoría era una pelea justa. Eran seis, y quedaban seis de aquellos cacharros. Sin embargo, aunque pomposas en exceso, las palabras de Nitram no estaban ausentes de verdad. Algunos de los presentes, como Batería y Silencio, no eran rivales frente a aquellos instrumentos de tortura flotantes, y pronto otros más poderosos como Dobleseis o Barrera tuvieron que probar ración doble del enemigo. Repulsor era incapaz de atinar a unos objetivos tan escurridizos, y Scream poco podía hacer más que evitar que le impactaran, un baile que podía estar manteniendo durante bastante tiempo, pero no eternamente.

Sólo Dobleseis parecía tener alguna ligera ventaja frente a los chismes flotantes de Nitram, pero aunque logró igualar un poco la balanza, pronto todos ellos estuvieron o bien en el suelo, o bien arrinconados contra alguna pared. Más de uno deseó en ese momento haber pertenecido a la raza de Krexon, fuera de combate sin posibilidad de despertar a corto plazo.

~---Se acabó, cazarrecompensas. Última oportunidad de pelear entre vosotros en pro de un bien mayor para el Universo.

\emph{Si tanto buscas a un ser último y definitivo, ¿por qué no te ofreces para el puesto?} ~---argumentó Dobleseis con cierta ironía.

~---No he sido yo quien os ha derrotado. Cualquiera con estos discos en su poder podría haberlo hecho, aunque sólo yo los poseo entre toda la humanidad. No, una vez que el verdadero poder del vencedor despierte, será una fuerza contra la que nadie nada podrá hacer. En todo caso, está claro que no sois ninguno de vosotros, y no vais a colaborar bajo ninguna circunstancia. Da igual, aún quedan muchos otros cazarrecompensas en la ciudad.

De repente Scream notó un ruido familiar en uno de los bolsillos de su gabardina. Era un ruido que llevaba tiempo sin escuchar, y que había aprendido a valorar como el más preciado de los tesoros. La estática de su comunicador, mal configurado por descuido y la falta de uso.

Apretó el botón disimuladamente, de modo que nadie le viera hacerlo, y escuchó la voz de Sam Grove, llena con la sensación de júbilo y regocijo propia del deber cumplido.

~---Capitán, hace tiempo que acabaron, pero las comunicaciones han tardado un tiempo el volver. ¡Capitán! ¿Me oye?

Scream no contestó. Sólo tenía ojos para el disco que flotaba a la altura de su cabeza, con un láser colocándose en línea con su rostro.

~---¡Capitán, ellos están allí!

Casi al momento de que Grove terminara de hablar, el disco empezó a emitir una serie de extraños brillos y cayó al suelo, como si se hubiera cortocircuitado.

Al otro lado de la puerta, seguido del resto de su grupo, Scream pudo ver a Distorsión estirando la mano hacia el aparato.

~---Tú nos contrataste, ¿verdad? ~---fueron sus palabras, señalando a Nitram~---. Puedes considerar nuestro acuerdo liquidado.

Nitram no se molestó en contestar aquellas palabras y se limitó a lanzar el resto de sus discos contra The Jammers. Sin embargo aquellas máquinas no eran rivales contra el alcance de sus poderes, y enseguida los otros discos acabaron girando sobre sí mismos, moviéndose con aparatosa lentitud, cayendo al suelo casi drenados de energía o disparándose los unos a los otros al rebotar sus láseres en la componente femenina del grupo.

\emph{Se acabó, Juez} ~---dijo Scream levantándose~---. \emph{Las comunicaciones han regresado, su experimento ha fallado.}

~---Sólo por ahora. Sólo por ahora. Pero nada podéis hacerme. La inmunidad de mi cargo me protege.

~---Veamos si puede protegerle de esto ~---dijo Dobleseis disparando cuatro descargas de láser hacia Nitram, pero el campo de fuerza que abatió a Krexon las absorbió tras combarse ligeramente y demostrar así su presencia invisible.

~---Adiós, cazadores ~---terminó el Juez Nitram, marchándose por una puerta que había en su lado de la sala. Al poco rato los discos comenzaron a levantarse, unos más aparatosamente que otros, y salieron volando pasillo abajo, sin que fuera posible atinarlos o detenerlos.

~---Son aparatos impresionantes si han logrado que mi amigo dedos rápidos no logre apagarlos por completo ~---dijo Distorsión mirando a Overdrive.

\emph{Gracias por la ayuda} ~---agregó Scream levantándose, al tiempo que los cazarrecompensas se incorporaban también.

~---¿Amigos tuyos? ~---preguntó Distorsión como si acabara de hacer el comentario más normal del mundo en el momento más propicio para el mismo.

\emph{Dejémoslo así.}

~---Bien, con esto nosotros hemos cumplido. La próxima vez, nos andaremos con cuidado y no nos volveremos a fiar de desconocidos.

~---¿Sois cazarrecompensas? ~---preguntó Dobleseis.

~---¿No sabes quienes somos? ~---dijo Distorsión, con sincera sorpresa.

~---No ~---fue la escueta respuesta del multibrazos.

~---Permíteme, entonces ~---dijo acercándole un par de entradas que tenía en el bolsillo~---. Son para nuestro próximo concierto en la colonia de Bludgor, por si\dots

Antes siquiera de que acabara de hablar, el cazarrecompensas ya estaba reduciendo a pedacitos las dos entradas.

~---Supongo que no te interesa mucho la música.

~---El único ruido que me interesa es el que yo genero ~---contestó mostrando los lanzarrayos.

Scream recibió un mensaje de radio donde le decían que tenían que salir de allí cuanto antes, pues la policía se acercaba a acordonar la zona. Al fin habían vuelto los buenos viejos tiempos, pensó.

\emph{Por éste no dan una recompensa} ~---dijo señalando a Krexon~---, \emph{pero espero poder confiar en vosotros para que lo entreguéis a las autoridades.}

~---Descuida ~---comentó Repulsor~---. De hecho Silencio, Barrera y yo estuvimos hablando y estábamos considerando asociarnos. Habíamos pensado como nombre Fortaleza, ¿qué te parece?

\emph{Pensadlo un poco más} ~---contestó con sinceridad Scream.

~---¿Qué hay de ti, vieja gloria? ¿Tú y Códec os uniríais a nosotros?

~---No, gracias. Demasiado a repartir ~---dijo Dobleseis saliendo de la habitación y marchándose pasillo abajo.

~---¿Siempre es así de amigable? ~---preguntó Distorsión~---. Por cierto, ¿os interesa a vosotros venir a algún concierto de nuestra gira? Es lo menos a cambio de todas las molestias que hemos causado.

~---¿Qué edad tienes, hijo?

~---Eso no importa, abuelo. ¿Le interesa o no?

~---No, no me interesa. Si todavía fuerais Balamb Garden\dots\ esos sí que eran buenos, y muy legales cuando los conocí, en sentido figurado, claro.

~---¿Conoció a Balamb Garden? ~---preguntó Echo impresionada, subiéndose la visera, distinta a la de unas horas antes y dedicada al grupo Garbage.

~---Antes de que tú nacieras, muchacha ~---contestó Repulsor mientras todos se marchaban pasillo abajo, con el ruido de fondo de las sirenas de los deslizadores patrulla.

\parbreak
Como era de esperar, el Juez Nitram declaró que había sufrido un intento de atentado frustrado en su propia residencia temporal. Como era de esperar también, cuando le preguntaron por el aspecto de sus atacantes no recordaba del todo bien cuál era éste, y algo similar le pasaba al resto de sus guardaespaldas. El asunto había quedado como parte de una guerra secreta. Las recompensas se retiraron oficialmente sin que la opinión pública supiera quién las había emitido. Krexon fue el único arrestado, y su presencia sospechosa en el lugar del asalto bastó para realizar un estudio más severo de su caso, además de recluirle en un régimen especial. Su colaboración con las autoridades fue nula, por decir que colaboró de alguna manera. El único tema que parecía producirle algo de interés era la extinción de su especie, y siempre parecía a punto de tener la intención de añadir algo al respecto, pero no tardaba en caer de nuevo en el hermetismo y limitarse a abrir y cerrar por turnos sus ojos dispersos alrededor del rostro.

Los Caídos volvieron a reorganizarse y realizar un informe de daños. La vorágine de aquellos días hizo que se olvidara temporalmente al Caído, aunque no tardaron en señalarle como culpable indirecto de la afluencia de cazadores de bonificaciones en las calles de la ciudad. De todos modos, no era nada por lo que no hubieran pasado antes.

John Scream no tardó en volver a sus ocupaciones habituales, aunque de vez en cuando se permitía unos segundos de solitaria reflexión. En uno de ellos le cogió Sky un día, efectuando una visita sorpresa a Gorgon Enterprises, donde Scream estaba realizando bocetos preliminares de prototipos destinados al año siguiente.

~---Ha ido de poco, por lo que me han contado.

~---Así es. Esta vez hemos estado cerca de ser descubiertos. Pero el Juez Nitram no sabía nuestra verdadera naturaleza. Cree que somos uno, aunque tiene una pieza del rompecabezas que puede llevarle a averiguar la verdad.

~---De modo que conocía a Miles. Quién sabe qué clase de relación se había forjado entre ellos dos\dots\ tal vez fueron enemigos, amigos sinceros, o incluso aliados en la lucha contra el crimen.

~---Puede que nunca lo sepamos. Y aunque lo hagamos, hasta entonces la única persona que nos podía ayudar a entender mejor a nuestro mentor es, por desgracia, uno de nuestros más declarados enemigos ~---terminó Scream llevándose la mano a la boca, en actitud reflexiva.

\endinput
