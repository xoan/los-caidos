La persecución a la que habían sido sometidos había terminado, pero siempre había algo de lo que preocuparse en aquella ciudad. Nuevas situaciones, nuevos conflictos. Y no en todas las ocasiones de naturaleza criminal.

En realidad era la sombra de los tiempos pasados lo que en breve tendrían que afrontar, en más sentidos de los que imaginaban\dots

\fancyparbreak
Las semanas habían pasado y las comunicaciones habían sido restablecidas en su práctica totalidad a lo largo de las cenicientas calles de Ernépolis. No en todas partes a idéntica velocidad, claro. En aquellos barrios donde se concentraba una mayor renta media fueron revisadas con mayor celeridad que en las zonas más deprimidas, como podían ser los Túneles o en general todo el sector Sur. Se trataba de una forma de corrupción leve, no demasiado fatal para el destino de la población, pero prueba presente de que incluso en las más pequeñas acciones la influencia del poder permanecía latente y operante a las acciones de los ciudadanos.

Como resultado del fallo generalizado de transmisión, que nunca fue del todo esclarecido, aunque se creyó que tuvo que ver con alguno de los múltiples cazarrecompensas que visitaron la ciudad, se procedió a una revisión completa de todas las instalaciones con el fin de reforzar su seguridad contra ese tipo de ataques. Tal acción, modificar toda la infraestructura eléctrica de Ernépolis, en muchos casos vieja y corroída por efecto de la dejadez y el adverso clima, trajo consigo numerosas consecuencias en términos de obras públicas.

Las calles y callejones se llenaron de socavones. Agujeros enormes, del tamaño de vehículos deslizantes en muchos casos, empezaron a dominar el paisaje urbano allá donde se posara la mirada. Además de eso, era habitual tener que efectuar cortes ocasionales en otras infraestructuras.

La más afectada e importante fue, sin lugar a dudas, la instalación lumínica. Era habitual que, fuera de los horarios laborales, barrios enteros se quedaran casi sin luz, con unos escasos focos de emergencia que, unidos a la Nube eterna que siempre secuestraba el ya de por sí escaso brillo lunar, retrotraían a la ciudad a tiempos en los que las tinieblas eran aún más poderosas de lo que solían ser en la época actual.

Muchos pensaban en el Caído y en su regocijo no sólo por haber escapado del acoso al que había sido sometido, sino también al comprobar que la ciudad era cada vez más territorio de la penumbra, un juguete nuevo para sus siniestros designios.

Nada más lejos de la realidad. Para Los Caídos las tinieblas eran un medio. Pero nunca, jamás, serían un fin, y eso era algo que Scream siempre tuvo claro desde que entró en la organización. Ser parte de la oscuridad, fundirte con ella. Pero no confundirte en ella, perder la personalidad. Ese era el error que por regla general acababan cometiendo muchos de los justicieros tenebrosos de antaño.

Era el tiempo para la reflexión en la quietud de su pequeño despacho del Aquerón, un lugar casi inaudito teniendo en cuenta la imagen de inhumanidad que ofrecían de cara al exterior. La lista de enemigos que poseían era ya, cuanto menos, importante y a tener en cuenta. Había monstruos, tanto física como psicológicamente hablando, y también déspotas y tiranos, sujetos que se creían superiores a todo y todos y con la capacidad para decidir por los demás. Unos se movían por poder, otros por odio, y también por ideales, incluso. Por supuesto no faltaban los enemigos de la vieja escuela, los que buscaban dinero o fama. Y eso sin contar toda una serie de personajes de ambiguo comportamiento y complicadas reacciones de cara al futuro.

Entre esos, estaba el desaparecido Starr Miles.

Scream no había dejado de pensar en ello desde que supiera que él y Nitram se conocieron en el pasado. Tal vez la organización no había empezado tan de cero como pensaban en un principio. Puede que Miles ya hubiera hecho antes pruebas, experimentos. O llegado a pactos que no les gustaría conocer.

En todo caso, algún día acabaría por averiguar las respuestas a esa clase de preguntas. El Juez Nitram había salido indemne por completo del asalto a la mansión, como no podía ser menos dada su posición oficial, mientras que unos peones fueron encarcelados, como Krexon, y otros se retiraron de la partida con el objetivo de no volver a ella jamás, como la mayor parte de los cazarrecompensas.

Sin embargo aquel no era el asunto más inmediato que requiriera su atención. La extraña novedad había surgido con las noticias que Saw le traía, jugosas y recién proclamadas, aunque no tardarían en ser difundidas.

~---¿Dices que el Presidente Scatter busca aspirantes a protectores de la ciudad? ~---comentó levantando la vista de los informes del día de los escuadrones.

~---Así es, John. A mí también me costó creerlo cuando me comunicó su idea de manera personalizada.

~---¿Crees que alguien puede haberle\dots\ sugerido tal eventualidad?

~---No lo sé. Aunque paso con él la mayor parte del tiempo que no estoy aquí, se mantiene en permanente contacto con toda clase de estratos de poder en la ciudad, y algunos de ellos son enemigos declarados nuestros.

~---En todo caso, sea como fuera la manera en que se le ha ocurrido esta idea, lo que está claro es que no puede traer nada bueno para nuestra estrategia. Puede que quieran volver a ir a por nosotros, o quién sabe qué puede pasar. Además de eso la ciudad se llenará de novatos inexpertos que puede que se pongan en peligro a sí mismos y a otros con ellos\dots\ y eso sólo en el mejor de los casos.

~---Piensa ser lo más cuidadoso posible al respecto de eso, pero al mismo tiempo pretende volverse todo lo políticamente correcto que la situación le permita. Creo que quiere pasar a la historia como el Presidente que reinstauró el orden en la ciudad con nuevos símbolos de justicia.

~---Un héroe estatal\dots\ un policía con poderes.

~---No es algo que nos suene del todo desconocido, si lo pensamos un momento ~---observó Saw.

~---Creo que esto no va a gustarle demasiado a Sky, si le conozco lo suficiente.

~---Yo también lo creo.

~---Mantenme informado de todo lo relacionado con este asunto. No te preocupes por tus tareas como jefe de escuadrón, alguien os sustituirá. Este asunto tiene prioridad al menos hasta que esclarezcamos exactamente qué clase de espectáculo pretende el Presidente montar y presentar.

~---Así lo haré, descuida.

Saw salió del despacho y se quedó un momento reflexionando. No le gustaría estar en el pellejo de Scream, pensó con frialdad. Siempre tenía que mostrar la máxima preocupación ante cualquier cosa que ocurriera en la ciudad, incluso hechos que podían luego no trascender en nada importante, como era ese concurso, esa suerte de oposición a héroe que tal vez se quedara en saco roto.

 

Al día siguiente, a primera hora de la mañana, con las luces aún no restablecidas en muchos barrios del centro de la ciudad, Saw pudo comprobar con aflicción que la ocurrencia del Presidente Scatter distaba mucho de ser una peregrina y pasajera planificación fruto del aburrimiento o una fallida tormenta de ideas.

Nada más llegar a su despacho le fue notificado que dejara pendientes todas sus tareas burocráticas y cancelara las citas del Presidente, ya que tenían que ponerse en marcha hacia uno de los cuarteles de defensa de la ciudad, remodelados y rehabilitados desde la Guerra de las Ocho Colonias. No sabía muy bien qué podía llevarles a un lugar así, pero un pálpito le decía que algo tenía que ver con lo que había estado charlando con Scream.

~---Señor Presidente, dígame, ¿qué estamos haciendo en este lugar? ~---preguntó en lo que ambos salían del deslizador presidencial, un blindado diseñado específicamente para uso de altos dirigentes.

~---¿Recuerdas la idea que te comenté sobre buscar un policía especial para la ciudad? ~---comentó con cierto júbilo en su voz~---. He hablado con ciertos conocidos de las fuerzas armadas y me han ofrecido un prototipo que puede encajar bien en este contexto.

El Coronel en funciones salió a recibirles y les acompañó hasta el recinto interior, donde varios soldados hacían guardia. Las luces exteriores parpadeaban como si estuvieran en una nave recién estrellada.

~---Es un poco molesto, pero gracias a esa luz oscilante de nuestro generador de emergencia podemos garantizar su seguridad mientras están aquí dentro ~---comentó el Coronel mientras avanzaban por los largos pasillos.

Tras toda una retahíla de presentaciones y visitas a varios departamentos, a las que tanto el Presidente como Saw estaban más que habituados e inmunizados, llegaron por fin al lugar que era el objetivo principal de la visita, un laboratorio de pruebas físicas y cinéticas. En una vidriera había una especie de guantelete, fabricado en alguna clase de aleación imposible de distinguir a simple vista y reposando sobre una vara de acero, de modo que sus cinco dedos apuntaban hacia el cenit como si fuera un objeto de exhibición.

~---Este es el prototipo que hablamos, Presidente ~---explicó el Coronel con orgullo~---. Dispara descargas no letales de energía, y otorga a su poseedor una considerable fuerza aumentada, proporcional a la que tuviera en circunstancias normales. Puede remodelarse para manos, digamos, especiales, como pueda ser la de un posible alienígena. Ya hace mucho que diseñamos este prototipo, pero la imposibilidad de fabricarlo en masa detuvo la producción, además de otros problemas de tipo\dots\ político, digamos.

~---Pero esos problemas ya no son tales, por lo que no debe preocuparse por ellos ~---agregó Scatter con convicción~---. Este guantelete sería un gran símbolo a portar por el héroe que defendiera la ciudad, y una muestra de su compromiso a acatar las normas tanto de las fuerzas de la ley como del Estado de la Defensa.

~---Con el debido respeto, Coronel ~---preguntó Saw entrando en la conversación de improviso~---, ¿han hecho pruebas de campo con este arma?

~---Gran cantidad de ellas, y todas con resultados excelentes. Errores menores fueron perfeccionados, dando como consecuencia que este guantelete es único en su especie, y seguramente lo será para siempre. Entonces, señor ~---continuó mirando a Scatter e ignorando a Saw~---, ¿piensa convocar una especie de plaza de aspirante a héroe?

~---Así es, para lo cual espero contar también con expertos entre sus filas.

~---Puede disponer de ellos. Muchos sin duda querrán también poner a prueba sus aptitudes para tan emblemática tarea.

~---No me cabe duda, y es algo que me llena de orgullo también. Esta ciudad ha sufrido el acoso continuo de gran cantidad de enemigos y amenazas, ha llegado el momento de demostrar a los habitantes de Ernépolis que no tienen nada que temer al respecto. Yo mismo me involucraré en el proceso de selección y elección de aspirantes, pero por supuesto habrá que ser respetuoso en todo este asunto. No queremos que haya potenciales problemas relacionados con discriminación por motivo alguno, ya sea sexo, raza, religión o especie. Debemos contemplar todas las posibilidades y preparar una defensa verbal a todo aquello que la oposición pueda decir al respecto.

Saw se planteó hasta qué punto el Presidente estaba metiéndose en aquel berenjenal imbuido por un deseo deformado de llevar el orden a las calles, y hasta qué punto por el deseo de pasar a la historia y conseguir, con suerte, más votos de cara a futuras elecciones, sobre todo después de los problemas económicos derivados de la posguerra. Finalmente concluyó que lo más probable era que ambas cosas movieran los hilos de sus actos, cada una a su maquiavélica manera, complementándose de modo que cada punto de vista ofrecía lo mejor de sí mismo y hacía de cortina de humo a las pegas que pudieran ponerse al otro.

En todo caso, una cosa sí que tenía clara y más que evidente. Entretenimiento no iba a faltarle a Los Caídos mientras durase todo aquel despropósito.

\parbreak
No pasaron muchos días hasta que se hizo público el certamen por el cual se buscaba un héroe para defender la ciudad. Las críticas paralelas también llegaron rápidas y demoledoras. Un absurdo sinsentido, un signo de debilidad, la ocasión perfecta para poner la seguridad en manos de un déspota, una nueva ocasión de llenar la ciudad de extraños.

Hubo también quienes dijeron que así ese nuevo héroe acabaría con el reinado de terror del Caído, y en otra línea opuesta de pensamiento quien decía que Ernépolis~I ya tenía un héroe, aunque no fuera el más abierto del mundo y nadie quisiera admitírselo a sí mismo. Ambas posturas radicales, por otro lado, no sonaron demasiado fuerte en el agitado mar de opiniones que llenaron muchos de los tabloides y medios digitales en ese momento.

Pero la noticia había calado hondo, sin duda. Había un plazo de una semana para inscribirse como aspirante y había que pasar gran cantidad de prerrequisitos previos. Edad mínima, papeles en regla, historial delictivo impoluto, o casi, una vez las primeras quejas afloraron reseñándose que lo que querían era crear al boy scout descerebrado perfecto y no había lugar para quienes hubieran aprendido de sus errores y se hubiesen rehabilitado en la sociedad.

Hubo un hombre que prestó especial atención a este último aspecto del evento. Un desengañado de la vida que rió ante la peculiar circunstancia de los acontecimientos cuando los vio en los medios de comunicación.

La vida no había sido fácil para él, pero tampoco se había quejado nunca por ello. Quejarse, en su entorno, era la manera más fácil de acabar recibiendo una bala en la cabeza.

Aquel hombre no estaba rehabilitado. Tampoco, ni mucho menos, arrepentido de los actos execrables que en su momento pudo llegar a efectuar. Pero sin duda, algo tenía que hacer con su vida. Alterarla de alguna manera radical.

~---¡Eh, Éxeter! ~---escuchó que le gritaban desde el otro lado del sombrío callejón donde se encontraba~---. ¿Seguro que esta tubería está asegurada?

~---Te he dicho que no me llames así ya ~---fue su único comentario, mirando aún hacia la lejanía, perdido en sus pensamientos.

~---Si no me has dicho tu verdadero nombre ¿cómo puedo llamarte entonces?

~---No me llames ~---fue su sencilla respuesta.

~---¿Está esto asegurado o no?

~---Lo está.

~---¿Nadie podría colarse por aquí?

~---No.

~---¿Y cómo has hecho para\dots?

~---Nada de preguntas. Ése era el trato. Tienes mi garantía, y la fama que precede mi trabajo. Nadie entrará por esa tubería a fisgar vuestros asuntos. Si no me crees prueba a hacerlo tú mismo, aunque puede que no te guste lo que encuentres.

El otro tipejo no insistió más en el asunto. Se limitó a mirar al único ojo sano de su interlocutor.

~---¿Cómo te hiciste eso? ~---comentó.

~---Preguntaba demasiado. ¿Quieres que te muestre qué más pasó?

Aquel comentario fue definitivo para dejar atrás la conversación. Regresó de nuevo a sus pensamientos sobre las ironías del destino. Él que fue un gran villano, aliado ocasional de otros, había pasado a ser contratado para asegurar almacenes clandestinos. Un tipo astuto, inteligente, perseverante. Nunca atrapado por la policía, aunque había ciertos\dots\ flecos sueltos en lo relativo a proteger su identidad. Pero la caída de los héroes había supuesto, sorprendentemente, la suya propia también. Él no se movía por poder. No se movía por dinero.

Estaba en el juego por el juego mismo.

La emoción. La adrenalina de poder impedir el paso con sus métodos a otros de igual condición a la suya. Y al desaparecer los grandes héroes de antaño, él también dejó atrás la ciudad, en busca de un horizonte mejor.

Pero finalmente había sido acorralado. Poco a poco las mafias le habían cercado, le habían considerado como una bala perdida que era peligroso dejar silbando de calle en calle, y eso le había devuelto de nuevo al comienzo del laberinto, la ratonera inicial. Ernépolis~I, ciudad de triunfos, ciudad de tragedias.

Su cabeza estaba en la picota, y lo sabía. Esos tipos, después de asegurar la zona, tratarían de matarle. Tendría que largarse, pero correría la voz, hasta que no tardaran en dejarle como un colador. Lleno hasta arriba de ratas, cucarachas y otros bichos. Otra ironía del destino, pensó.

Aquel concurso podía, sin embargo, ser su salvación. No porque tuviera interés en convertirse en el héroe de la ciudad, no, ni mucho menos. Pero al menos le otorgaría el respiro que necesitaba para elaborar un plan de huída. Los aspirantes permanecerían en el anonimato, lo cual sería perfecto para él. Podría incluso usar su nombre verdadero, Warren Shockman. Lo necesitaría, de hecho, para pasar los requisitos previos. No dejaba de resultarle gracioso, por otro lado, que su nombre sonara en cierto modo como el de un superhéroe, lo que le valió motivo de no pocas burlas cuando era un crío.

Una rata salió del interior de la tubería y se quedó mirando a Shockman. Tenía, como él, un ojo tuerto. Shockman se inclinó, extendió la mano y la rata sorteó los surcos de ceniza del suelo para subirse a ella. La acarició con calma.

Aquella era una buena ocasión para escapar, sin duda, pensó dejándola de nuevo corretear por la pared exterior del almacén.
