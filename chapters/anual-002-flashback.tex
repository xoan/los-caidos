Siempre hay un origen antes del origen. Un instante preliminar que nos motiva a dar el siguiente paso. Y no importa cuán decidido se pueda estar en apariencia, todo el mundo ha pasado por una situación de tales características.

A veces, de hecho, el pasado motiva la toma de decisiones por largo tiempo postergadas\dots

\fancyparbreak
El Detective miró desde la ventana de su despacho, situado en la zona Sur de la ciudad. Observó con calma el aguaceniza que caía sin descanso aparente, ensuciando las calles, cubriéndolas con una masa plastosa y repugnante que dificultaba el avance de todo aquel que no tuviera el calzado adecuado.

Lo que el Detective pensaba, de todos modos, era que la ciudad estaba mucho más llena de una clase de podredumbre muy distinta. Una que no se pegaba a los zapatos, pero sí al alma. Un vestigio de putrefacción, de decadencia intangible, un halo de tinieblas y moral perdida que envolvía hasta la última esquina cochambrosa de las desconchadas calles de Ernépolis~I.

El Detective no siempre había pensado así. Su mente, en el pasado, fue un sorprendente crisol de opiniones encontradas. Había estado en los dos lados del espejo, y había aprendido que para algunas personas esos espejos, esos reflejos de su personalidad, podían ser cóncavos o convexos, aunque siempre dirían a sus familiares, a sus amigos, a la opinión pública, que seguían siendo planos, en nada deformantes.

La cantidad de vivencias por las que había pasado para tener que llegar hasta allí era larga, demasiado para ser mencionada tan siquiera de manera anecdótica. Pero había llegado, visto y vencido, aunque también se había marchado, invisible y derrotado. No había lugar relevante de la ciudad donde sus suelas no hubieran hollado a lo largo de sus múltiples investigaciones, ni abismo tan profundo como para ser incapaz de comenzar de nuevo la subida.

Pero lo que leyó aquel día en el Crepuscular, el periódico más importante e influyente de Ernépolis, acabó por derrumbar sus aspiraciones de cambio, a pesar de que era una noticia que sentía como largamente anticipada.

El titular era corto, sencillo y contundente, a pesar de la poca información que podía ofrecer al respecto: \textsc{¿Dónde está Reflector?} El héroe había sido visto volando hacia la zona de las factorías, patrullando la ciudad, seguramente, o tal vez respondiendo a alguna clase de aviso de emergencia. Pero los días pasaron, y las semanas con ellos, y el héroe de Ernépolis~I, como muchos otros antes que él, se esfumó sin dejar rastro alguno de su paradero, al menos en apariencia.

El Detective tenía sus propias ideas acerca de qué podía haber sido de Reflector. Muerto. Roto y quebrado como una marioneta sin hilos. Tal vez con el cuello roto en un callejón ausente de los túneles, o disuelto en los materiales de fundición de las factorías. Había tantas posibilidades como demonios pululaban sueltos por la ciudad, expandiendo sangre, mentiras y balas a partes iguales e intervalos regulares.

Cerró el periódico sobre el escritorio, con calma, como si fuera una notificación bancaria o una orden de desalojo. En otros tiempos no hubiera tenido tantos miramientos. Lo hubiera lanzado al suelo, o tal vez roto en pedazos, como si con eso erradicara los sucesos de sus columnas, toda aquella siembra de maldad que anidaba en su interior esperando a eclosionar e infectar la esperanza de los ciudadanos cada vez que se abrían sus páginas malditas.

Pero los años eran sabios, y otorgaban sabiduría a los que realizaban el viaje con ellos. Por eso, el Detective se limitó a reflexionar, a pensar en la situación actual. Reflector, posiblemente, nunca volvería. Pero sus enemigos seguían ahí, como los de tantos otros.

Los héroes sin villanos no son nada. Sin embargo, los villanos sin héroes lo son todo.

Y en eso se estaba convirtiendo la ciudad, en el Valhalla de los inmorales, un nido de cucarachas, fétido, hinchado, ennegrecido y prácticamente imparable en su avance y expansión hacia las ciudades vecinas.

Hacía falta un cambio. Uno serio, crucial. Uno que arrojara todo lo viejo, lo inútil, lo que había fracasado, y lo sustituyera por algo nuevo, brillante. Una luz entre tantas y tan espesas tinieblas.

El Detective nunca se lo habría reconocido a nadie, pero era un idealista declarado. Hasta la médula. Algo lógico, teniendo en cuenta su pasado de sorprendentes altibajos personales.

Sus reflexiones se detuvieron cuando vio a la mujer entrar en su despacho, sin necesidad alguna de llamar pues la puerta estaba abierta. Eso no quería decir que el Detective no tuviera enemigos, ni que la zona fuera segura en términos del nivel de calle. Pero los rateros del lugar ya habían aprendido a largarse de allí cuando la primera bala silbó sobre sus cabezas.

Por otro lado, el motivo por el que la puerta estaba abierta aquel día era porque hacía calor. La ciudad ardía como las calderas del mismísimo Satanás, y la Nube era el aroma en que se cocían lentamente los pecados de los hombres.

La mujer era hermosa como pocas podían serlo en ciudad tan desahuciada como la primera de las Ernépolis. Su belleza era tan evidente, y sus ropas tan descuidadas, que no era necesario preguntar cuál era su ocupación laboral. Pero el entorno había oscurecido sus ojos, agrietado su cabello, amarilleado sus uñas y seguramente también petrificado su corazón.

~---Me llamo Anne Strud, y necesito sus servicios ~---fue todo lo que acertó a comentar nada más sentarse delante del escritorio, después de que la invitaran a hacerlo con un gesto de mano. Sacó un pitillo y el Detective se lo encendió con un mechero gastado y mellado.

~---La escucho ~---fue todo lo que el sabueso comentó. Usar la lengua para lo necesario y la oreja para todo lo demás. Ésa era su máxima como investigador.

~---Mi hijo está siendo captado por las mafias de los Túneles. Necesito que me ayude a encontrarlo y sacarle de allí.

~---¿Ha cometido alguna clase de delito?

~---No de un tiempo a esta parte, que yo sepa. Estuvo un tiempo en un reformatorio. Una de esas cárceles a escala infantil que abundan por todas partes en la ciudad.

Aquella mujer tenía una manera de hablar que evidenciaba una educación desaprovechada e improrrogable. Tal vez vino de fuera para buscar una oportunidad en la ciudad, sin saber que había colonias infernales donde la vida se antojaba más apetecible que en aquella urbanópolis de noche y polución eterna.

~---Se llama Alex ~---continuó la mujer, visiblemente nerviosa, con los dedos casi estrangulando el cigarrillo~---. Alex Strud. Tiene unos doce o trece años.

~---¿Doce o trece? ¿Acaso no lo sabe?

~---No, no lo sé. No he sido una madre modelo, pero espero compensarle a partir de ahora.

~---¿Qué hay del padre? ~---el Detective dejó un espacio en blanco para que se rellenara con una contestación, pero permaneció muerto y vacío~---. No hay padre reconocido, ¿verdad?

~---No. Lo único que sé de él es que era alienígena, pues el crío nació con un defecto de genoma que no se ha manifestado aún. Con suerte, puede que nunca lo haga.

~---De modo que un mestizo alien\dots\ si la gente que le tiene lo sabe, nunca le dejarán ir por su propia voluntad. O bien explotarán sus posibles cualidades, o bien le matarán si deciden que son una amenaza para ellos.

~---Por eso recurro a usted. ¿Me ayudará?

~---Sospecho que no lo va a tener fácil para pagarme.

~---Hay muchas maneras de saldar una deuda.

El Detective la miró fijamente, y por primera vez en mucho tiempo puso cara a la decadencia que poblaba las calles de su ciudad.

~---Ya me pagará más adelante. De momento déjeme sus datos y yo contactaré con usted en caso necesario.

~---Aquí los tiene ~---dijo ella sacando una tarjeta del bolsillo y dejándola sobre el escritorio~---. Gracias por su ayuda.

~---No me las dé aún, pues puede que la próxima vez que me vea sea con su hijo muerto entre mis brazos ~---fue la durísima respuesta del sabueso en lo que se metió la tarjeta en la solapa, el arma en el bolsillo, cogió su gabardina y sombrero y salió junto con su cliente hacia el destartalado y ruinoso pasillo.

\parbreak
Lo primero que el Detective hizo fue dirigirse a los tribunales a hablar con el fiscal asignado al caso del joven Alex Strud, para saber si podía darle alguna pista sobre qué había sido del chico desde que volvió a pisar las viscosas calles de Ernépolis~I. Pero se encontró con que no estaba allí en aquel momento, y se tuvo que contentar con hablar con su secretaria, una rubia que empleaba palabras amables al mismo tiempo que gestos no verbales que evidenciaban que no le dejaría pasar a hablar con su jefe.

~---Lo lamento señor, pero el fiscal no le atenderá en este momento. Está muy ocupado, acaba de firmar su traslado a otro planeta para llevar a cabo funciones de embajador y el papeleo tiene completamente absorbido su tiempo.

~---Apuesto a que me hubiera dicho algo similar en trasfondo aunque no estuviera efectuando ningún traslado ~---contestó el detective encendiendo un cigarrillo.

~---Puede hablarlo con su próximo sustituto en el cargo, los informes de los casos ya están en sus manos.

~---Lo que necesito no viene en ningún papel, sino que es fruto de la experiencia y el interés personal de su jefe.

~---Tal vez yo pueda ser de ayuda.

~---¿Le suena de algo un chaval llamado Alex Strud?

~---Sí, en efecto. Era un joven solitario y con problemas, pero buen chico en el fondo. El fiscal tenía la esperanza de que se rehabilitara completamente después de su ingreso en el reformatorio.

~---Debo entender que no fue así como sucedió.

~---No del todo. El chico era muy voluntarioso y trataba de reincorporarse a los estudios, pero su madre le estaba siempre metiendo en problemas.

~---¿Dónde está ahora?

~---No lo sabemos, señor. No podemos vigilar a esos chicos una vez han cumplido con su régimen interno.

~---Pensé que solía asignarse un tutor a esta clase de delincuentes juveniles.

~---Eso hubiera sido en caso de disponer de medios, y no es el caso, por desgracia. El fiscal decidió que no era un asunto prioritario.

~---Fue una mala decisión por parte de su jefe, entonces ~---se limitó a decir el Detective~---. ¿Quién fue asignado para su custodia?

~---Su madre.

~---Supongo que se estará equivocando.

~---No, en absoluto. Lo recuerdo a la perfección porque nosotros mismos nos opusimos a ello.

El Detective cogió su cigarrillo con dos dedos y miró con templanza cómo se gastaba y emanaba una columna ondulada de aire prohibido. Lo lanzó al suelo y lo pisó con velocidad, pero sin imprimir aceleración a la punta de su zapato.

~---Entiendo. Debo marcharme, si me disculpa.

~---Espero haberle sido de utilidad.

~---Lo hubieran sido más hace unos cuantos años, cuando aquel chico todavía tenía una segunda oportunidad ~---fue su contestación antes de regresar a las calles húmedas y rezumantes de la ciudad.

\parbreak
El Detective no solía enfadarse a menudo. De hecho, resultaba bastante difícil conseguir que perdiera los estribos. Había visto demasiado, y también había sido testigo mudo y cómplice de demasiadas cosas a su alrededor.

Aun así todo hombre tenía un límite, un punto a partir del cual ponía freno a su permisividad y dejaba de fingir que lo que había a su alrededor no le importaba, que la mierda que le envolvía no era tan espesa como pensaba.

No tenía que ser algo terrible, ni tampoco horrendo, que impidiera dormir a aquellos con menos estómago para soportarlo de haberlo vivido o presenciado. A veces, podía ser algo tan simple como darse cuenta de estar siendo utilizado contra su voluntad.

El Detective llegó a la dirección que le había dado la mujer, un bloque de apartamentos que le hizo considerar el edificio de su despacho como un lujoso hotel de cinco estrellas, y llamó a la puerta. Nadie contestó. No creía que fuera una dirección falsa, por otro lado. Pegó la oreja a la endeble pared, casi tan fina como papel de liar, y escuchó respiraciones al otro lado. Sacó el arma, se puso frente a la puerta y la abrió usando la planta del pie, la llave maestra que por aquel entonces en Ernépolis otorgaba acceso privilegiado a la mayor parte de las puertas de los bajos fondos.

Nada más hacerlo vio el cañón de la escopeta, frío como el tacto de la muerte, apuntando directamente hacia su posición, y saltó como pudo lejos de su alcance, apoyado en el muro contiguo. No tardó en darse cuenta de que de poco serviría ese parapeto, pues el yeso viejo quedaría como un colador, y se lo jugó todo a doble o nada asomándose de nuevo por la puerta, esperando ser más rápido que aquella plaga horizontal de insectos metálicos y estriados que le arrollarían como si fuera un insignificante mosquito trazando un vuelo equivocado.

Para ofrecer un blanco menos evidente antes de asomarse se apoyó en la rodilla, esperando también con ello mejorar la puntería. Como resultado logró herir a su rival en la pierna derecha, y éste al caer disparó el segundo cartucho de modo que la lluvia de plomo pasó por encima de la cabeza del Detective, bien parapetado en su esquina. Trató de percatarse de que no hubiera nadie más en el interior del apartamento, pero en cuanto comprobó que desde su ángulo era poco menos que imposible estar seguro de ello se limitó a entrar a toda prisa y rezar por ser más rápido que cualquiera que pudiera cruzarse en su camino.

Revisó las habitaciones y en una de ellas encontró a la mujer tirada en el suelo y con la cara hecha un cromo, mientras que escuchó cómo otro de los asaltantes escapaba por una ventana que daba al exterior. Trató de seguirle pero era demasiado tarde. Aquellas callejuelas estaban llenas de bifurcaciones, y nadie había visto ni escuchado nada, y ya hacía tiempo que había aprendido que no importaba qué podía intentar decir para convencer a los testigos presenciales, antes se beberían sus propios meados que delatar a un matón de los bajos fondos de Ernépolis~I.

Regresó a la casa y encontró a la mujer tirada en el suelo, hecha un guiñapo. Nunca la hubiera pegado, pero por un sombrío momento se alegró de que otros lo hubieran hecho en su lugar.

~---Usted dejó que se lo llevaran, ¿verdad?

~---Necesitaba dinero, lo necesitaba ~---fue la respuesta reiterada de la mujer.

~---Para qué, ¿eh? ¿Drogas? ¿Un guardaespaldas, quizás?

~---Ellos me prometieron que no le pasaría nada y yo les creí. Fui tan estúpida de creerles.

~---Y por eso vino a mí, para que le sacara las castañas del fuego, en vez de encararse con sus propios problemas.

~---Ayúdele, por favor, es todo lo que tengo.

~---Es una lástima que a su hijo no le pase lo mismo con usted ~---replicó dejando ahí a la mujer, tirada, sin siquiera ayudarla a levantarse.

\parbreak
No fue fácil seguir el rastro del chico. Aquellos tipos tapaban sus movimientos con envidiable habilidad. Aquello debería haberle puesto sobre aviso de lo que podía encontrarse en su camino. Pero hizo caso omiso de los consejos de su vocecilla y siguió con su objetivo. Ya hacía tiempo que no escuchaba demasiado la sabiduría que emanaba de su propio interior. Pero al menos, gracias a ello, tenía una pista de un lugar donde tal vez podía encontrarle.

Vendido, pensó ajustándose el sombrero y metiéndose las manos en los bolsillos de la gabardina. Vendido como un trasto viejo, sólo para sustentar una miserable existencia un poco más. Pensaba con amargura que personas como aquella no deberían tener derecho a engendrar descendencia sobre el mundo. Para ellos sólo debería restar la putridez y la corrupción de una tumba llena de flores marchitas y presidida por un epitafio borroso y oxidado.

Tuvo que cobrarse muchos favores para llegar a aquel local que rodeó hasta dar con el acceso trasero. Siempre había uno en aquella zona de la ciudad donde una salida de emergencia nunca estaba de más. La puerta estaba cerrada, y en aquella ocasión no convenía hacerse de notar en exceso, por lo que dio buen uso al silenciador de su arma por primera vez en mucho tiempo. Una vez que la cerradura fue sólo un vago vestigio del pasado inmediato, descendió lentamente los tenebrosos escalones que se mostraron ante él, amparado por la oscuridad. La oscuridad siempre acababa llegando en todos sus casos. En muchos lugares de los mismos, y también en varios estadios del alma.

Una solitaria bombilla era todo lo que había allí abajo, parpadeante e intermitente como las luces ludopáticas de las tragaperras de un casino. El chico estaba allí abajo, ante él, apoyado en un rincón, acurrucado. Le bastaron dos segundos para darse cuenta de que no estaba solo en aquel zulo fantasmal, pero a su agresor le sirvió uno para salir de las sombras y golpearle en pleno estómago.

Después del primer golpe vino un segundo, y después de ese un tercero. Para cuando se giró lo primero que vio fue la placa de policía que brillaba como una luciérnaga entre las tinieblas que empezaban a devorar su campo de visión.

Miró fijamente a Alex Strud, y dedicó sus esforzadas palabras a él.

~---Si sabes\dots\ hacer algo\dots\ hazlo ahora, Alex.

Pero el chico estaba completamente paralizado, y era incapaz de reaccionar aunque hubiera sido capaz de matar con la mirada al agresor del Detective. Fue entonces cuando el sabueso se dio cuenta de que a él nadie le iba a sacar las castañas del fuego, nunca había sido así y así nunca sería, y por eso en cuanto iba a recibir una nueva patada, más esforzada y lenta de puro cansancio, agarró el pie de su agresor con ambas manos y lo retorció como si fuera el cuello de un animal enloquecido. El policía metido a futbolista cayó al suelo de dolor y se llevó las manos al pie de puro instinto, pero para cuando trató de alcanzar su arma el Detective había empezado a aplicarle un tratamiento similar al que había padecido. Una vez le pateó repetidamente sólo por puro desprecio, le pisó las manos para asegurarse de que no repetía el mismo truco qué él había ejecutado, además de imposibilitarle a la hora de apretar gatillo alguno.

Después de aquello la reserva de adrenalina se vació por completo y cayó rendido al suelo, pero al mismo tiempo, como si la adrenalina, al igual que la energía, ni se creara o destruyera, sólo se transformara, el chico reaccionó y le ayudó a levantarse.

~---Sería gracioso que ahora me dijeras que no eres Alex Strud ~---fue el comentario del Detective, buscando con la mirada algo que pudiera hacer las veces de atadura para el defensor de la ley que yacía en el suelo con los dedos rotos y varias costillas fracturadas. El Detective prefirió no examinar las suyas propias, por si habían sufrido un destino similar.

De repente escucharon voces provenientes de la entrada del local, y el cuerpo del Detective se erizó como si fuera un gato psicópata a punto de saltar sobre la cara de su perseguidor. Sacó el arma y ambos empezaron a caminar hacia las escaleras para emprender el camino de vuelta hacia la luz y la libertad.

Cuando estaban varios escalones por encima del sótano la puerta se abrió y el Detective vio entrar al que había sido un grano en el culo en muchos de sus casos anteriores, del mismo modo que él lo había sido también en sentido contrario. El Jefe de Policía de Ernépolis~I, Brian Wolf. Si le hubieran dado una moneda por cada asunto turbio en el que le creía involucrado podría haber alcanzado la Luna apilándolas una encima de la otra.

~---¿Qué coño\dots? ~---comentó escuetamente en cuanto vio al policía hecho puré en el suelo, pero se detuvo en cuanto se fijó en el Detective, apuntándole con la pistola y con el chico a su lado~---. Le mataré por esto ~---vaticinó.

~---No sería la primera vez que dice eso, ni tampoco la primera que lo incumple.

~---Esta vez será distinto, ya lo verá. Las cosas han cambiado. Pronto, muy pronto, habrá un giro radical en esta ciudad, y nos podremos encargar con impunidad de estorbos y vejestorios como usted.

Al Detective no le gustaron nada esas palabras, pues eran la confirmación de que algo feo, muy feo se estaba cociendo en la trastienda de lugares como aquel a lo largo de todo Ernépolis~I. no le pasaba desapercibido que pronto habría elecciones, o algo que pretendía pasar por ese nombre, y siempre cabía la posibilidad de que las cosas pasaran de feas a horrendas sin aviso previo.

~---Siempre ha sido un pomposo bocazas, Jefe Wolf ~---dijo el Detective, esperando tirar de la lengua al seboso policía con sus palabras ofensivas.

~---Ella lo cambiará todo, despojo. Hoy te llevas al chico, pero habrá más sótanos, y no siempre podrás devolver los golpes.

El Detective había tenido suficiente de aquello, por lo que junto con el chico fueron subiendo muy lentamente las escaleras, sin que dejara de apuntar al seboso alto cargo de la policía ni un solo momento. No hubo disparos, ni amagos de sacar el arma. El Detective sabía que Wolf les hubiera matado en persona si se les hubiera tan siquiera ocurrido poner en riesgo su gordo y sudoroso pescuezo.

Cuando salieron al exterior, por fortuna las mismas calles que servían a los matones para desaparecer a toda prisa tenían también idéntica función para los del bando contrario, y así el sabueso y el chico lograron mezclarse con los transeúntes y largarse de allí cuanto antes.

~---¿Estás bien? ~---dijo al fin el Detective, sacando un cigarrillo.

El chico no respondió. Aún seguía como ausente.

~---¿Se te ha comido la lengua el gato o algo así?

~---¿Vamos a ir a ver a mi madre?

~---Depende. ¿Quieres?

El chico negó con la cabeza.

~---Escucha, no te tienes que quedar con tu madre si no quieres, pero tenemos que advertirla para que se largue de esta ciudad cuanto antes. Al fin y al cabo no estarías aquí conmigo si no fuera por ella.

~---Lo entiendo.

~---Lo sé ~---se limitó a contestar el Detective atinando por fin a encender el cigarrillo.

Cuando llegaron al apartamento corroboró que habían llegado antes que los hombres que estaban con Wolf, ya fueran policías o matones callejeros, para el caso era lo mismo. Sin embargo había un silencio gélido que imperaba en el ambiente y no presagiaba nada bueno.

~---Espera aquí ~---dijo, desenfundando el arma.

El chico estuvo en el comedor, allí, de pie, durante algo más de dos minutos. No protestó ni dijo una sola palabra. Se limitó a hacer lo que le habían dicho, paciente y callado.

Al fin el Detective regresó a su lado.

~---¿Dónde está mi madre?

~---No está ~---fue la seria y escueta respuesta.

~---Se ha marchado, ¿verdad?

El Detective se dio cuenta de que el chico sabía perfectamente lo que estaba preguntando, y también lo que él iba a contestar.

~---Esos hombres ya nunca volverán a hacerla daño ~---contestó haciéndole señas para que ambos se marcharan de aquel edificio con la intención de no regresar jamás.

\parbreak
Después de aquel caso, el Detective pensó que ya iba siendo hora de hacer algo para cambiar la situación. No sabía qué, pero lo que sí que tenía claro era que no le gustaba la vida que estaba llevando. No sólo en su propio fuero interno, sino también en lo que concernía al exterior. Llevaba la decadencia pegada a la piel, y tendría que quitársela de encima si no quería que acabara por absorberla por todos los poros de la misma.

El fiscal sí estuvo dispuesto a recibirle en aquella ocasión. Bastaba con pronunciar las palabras mágicas, corrupción policial, y todo el mundo saltaba como un conejo fuera de la chistera, aunque para llevar a cabo trucos muy distintos según el espectáculo al que se dedicara de manera habitual.

El fiscal era una persona con fama de abrirse a toda clase de culturas, aunque también conocido por su tremenda severidad y defensa peculiar de la justicia. Era de la clase de sujetos que se arrimaban al árbol que más cobijaba y que con el paso de los años daría mucho de lo que hablar.

~---Celebro conocerle al fin, fiscal Nitram ~---dijo el Detective con calma, apoyado en una de las impolutas paredes de los tribunales, como si fuera una presencia extraña en aquel lugar incorrupto y no manchado por la ambigüedad.

~---Mi secretaria me habló de usted y recordé el caso de Alex Strud en cuanto me lo dijo. Espero que esté bien.

~---Todo lo bien que puede estar un chico que ha perdido a su madre. Sabe que las cosas podían haber sido de otra manera, ¿verdad?

~---Ya es tarde para eso, y tampoco es mi misión meterme en ello. No puedo ir por ahí haciendo de justiciero por el mundo, aunque a veces me gustaría, ¿no cree?

~---No lo sé, en verdad. En esta ciudad a veces ya no sé dónde está la frontera entre lo legal y lo ilegal.

~---¿Sabe una cosa? Secretamente comparto con usted ese anhelo, esa sensación de que hace falta algo, o alguien, que cambie las cosas. He estudiado al pueblo con el que voy a entrar en contacto, y tienen en sus culturas una concepción interesante al respecto.

~---Supongo que estará fuera por mucho tiempo.

~---No tanto como cree. Volveré en unos años, pues he sido promocionado como juez. Sin embargo no he querido desaprovechar esta ocasión de pasar un tiempo fuera y conocer el concepto de justicia que tienen otras especies del Universo.

~---Ya veo. Espero que haya arreglado los papeles que le mandé.

~---Está hecho. No debería tener ningún problema al respecto. Ahora, si me disculpa, debo seguir con los preparativos del viaje, aunque volveré a menudo por aquí, y espero que nos veamos en esas visitas.

~---Ya sabe dónde encontrarme.

~---Ha sido un placer, Detective Miles ~---agregó Nitram mientras le estrechaba la mano.

~---Llámeme Starr ~---sugirió el Detective.
