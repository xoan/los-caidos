Nuevos proyectos para la ciudad. Nuevas posibles decepciones, fracasos, esperanzas perdidas.

Nuevas conspiraciones urdiéndose en las sombras\dots

\fancyparbreak
Scream y el resto de Los Caídos no tardaron en conocer de primera mano cuál era el principal proyecto que se estaba preparando para devolver la confianza en la ciudad. Lo supieron de manos de Saw, que en calidad de ayudante del Presidente Scatter era quien solía tener acceso más rápido y sencillo a toda información que supusiera un cambio de infraestructura pública de gran envergadura.

El proyecto en cuestión era una autopista. Pero no una cualquiera, sino un inmenso tramo elevado que atravesaría la ciudad de parte a parte, colándose entre los edificios más altos. Su forma principal sería de dos grandes rectas, unidas por un quiebro en forma de \textsc{u} cerrada, de las que emergerían numerosas bifurcaciones a niveles inferiores. La idea básica del proyecto era tender un puente intermedio entre el tráfico aéreo y el del nivel de calle, y de ese modo dinamizar las visitas ocasionales de naves que pudieran recalar ocasionalmente en la ciudad.

El quiebro rodearía el distrito financiero, con los edificios más altos, y sería donde las ramificaciones se volverían más complejas. Ciertos subtramos se replicarían a lo largo de edificios emblemáticos, como las torres empresariales, bordeándolas a corta distancia, y convirtiéndose de ese modo en la línea principal de transporte para muchos trabajadores. Ya se había firmado un convenio con muchos edificios para que fueran reformados y habilitaran pasarelas y accesos directos al entramado de carreteras.

Habría algunos tramos de tipo subterráneo, también, no demasiados pero suficientes para dotar al conjunto de la capacidad de desarrollarse a tres niveles distintos de la ciudad. Esos tramos, de hecho, serían los primeros en comenzarse junto con segmentos aislados que salieran de manera simbólica de los edificios del distrito financiero, para sellar de manera pública el compromiso suscrito entre la ciudad y los empresarios a la hora de impulsar su desarrollo.

La cantidad de dinero que se iba a invertir en ese proyecto era, cuanto menos, demencial. Sin embargo todo el mundo estaba de acuerdo en que supondría un gran avance para la ciudad una vez se llevara a buen término. Podría casi decirse que, junto a la eterna Nube, se convertiría en uno de los estandartes icónicos de la misma.

Ese era el motivo por el que muchos militares habían sido destinados a la ciudad. Su misión era proteger ese y otros proyectos de desarrollo a costa de lo que fuera. La prioridad en ese momento, sin embargo, estaba en la autopista. Y también lo era en el cuartel de Los Caídos.

~---Será un desastre ~---vaticinó Swind en la asamblea general, mostrándose pesimista al respecto~---. Ernépolis está demasiado endeudada para aguantar una reforma de tanta envergadura.

~---No tiene por qué ~---replicó Razorclaw~---. Esta ciudad ya ha sufrido lo suficiente, tal vez sea desarrollo social lo que necesita y no mano dura en términos de seguridad urbana.

~---¿Qué hay de los militares, Ellis? ~---preguntó de repente Scream, pensativo~---. ¿Qué jurisdicción les ha sido otorgada?

El aludido se tomó un momento antes de contestar, como solía hacer antes de presentar al Presidente Scatter en ruedas de prensa oficiales.

~---Tienen las mismas obligaciones legales que un agente de policía, pero funcionan como una célula independiente. Tampoco tienen por qué seguir órdenes del Jefe Sky.

~---Es decir, que pueden hacer, mientras no llamen demasiado la atención, lo que les venga en gana ~---observó Scream.

~---Más o menos, sí.

~---Eso traerá problemas ~---fue el comentario, elaborado de una ú otra manera, de muchos de los miembros de Los Caídos en ese momento. Y Scream estaba más o menos de acuerdo con ellos. Otra vez el ejército merodeando por la ciudad, aunque al menos eran menos numerosos y tenían un propósito concreto, proteger las obras de la autopista y subsiguientes iniciativas de desarrollo, que no tardarían en ser mostradas a la luz.

Sin embargo no dudaba que habría consecuencias difíciles de vaticinar en un análisis preliminar. Hechos que traerían aparejados nuevos problemas, o quizá ventajas, con los que no se podía contar de primeras.

Los primeros meses de construcción, no obstante, se desarrollaron bajo aparente y relativa normalidad. En lo que el ejército y sus hombres se limitaron a ejercer de guardaespaldas de las zonas donde se estaban levantando los primeros pilares y sujeciones de seguridad, Los Caídos se movieron por terrenos más procelosos y al margen de las normas establecidas. Tanto desde las calles como desde despachos, salas privadas y lugares de reunión, hicieron todo lo posible por apartar el crimen y la corrupción de la elaboración de ese titánico proyecto. Saw estuvo en primera línea del asunto político, vigilando que no se formalizaran sobornos o concesiones empresariales fraudulentas. El resto de los miembros, cada uno a su nivel, trataron de aportar su granito de arena para vigilar todos los estratos sociales en los que el proyecto podía verse comprometido.

Recabaron mucha información. Analizaron todos los posibles intereses creados que se podían estar moviendo. Presionaron a constructores especuladores para que sintieran que alguien estaba vigilando sus acciones por si decidían sacar tajada extra a la hora de obtener beneficios. Hicieron, en definitiva, lo que todo ciudadano honrado de a pie desearía hacer de tener el poder, la voluntad y los medios.

Pero si en algún sitio su labor resultaba impecable, era sin duda a la hora de moverse por los bajos fondos. Y fue allí precisamente donde uno de los más recientes miembros obtuvo noticias de lo más preocupantes al respecto de lo que se estaba cociendo.

~---¡No me mates! ~---gritaba el criminal, reconvertido en víctima, en lo que huía despavorido por las callejuelas mugrientas y cubiertas de ceniza de Los Túneles.

Su perseguidor le seguía de cerca, su gabardina ondeando en el aire mortecino, calándose el sombrero y clavando su mirada siniestra en su objetivo. No habló. No solía hacerlo a menudo. Él en su caso menos aún que los líderes de los escuadrones.

Usaba poco los aparatos de holografía, también, aunque de vez en cuando los empelaba para mantener la ilusión de ser uno con los otros. Su rango de actuación era más biológico, más acorde con aprovechar la fauna del entorno.

El atemorizado ratero llegó a un callejón sin salida, jadeando como un pollo a punto de ser descabezado. Tocó la pared repugnante con desesperación, como si no pudiera creer que estuviera en su camino. Se giró, y el final del callejón, la única salida posible, se le antojó lejano e inalcanzable como las mismísimas puertas del Paraíso.

Durante unos segundos no vio nada al otro lado. Como si se hubiera quedado solo por completo. El corazón dejó de palpitarle como si fuera una bomba de relojería a punto de estallar, recuperó el pulso, se detuvo el hormigueo que discurría por los dedos de sus manos. De manera muy tímida se planteó dar un paso adelante, proporcionalmente inmenso en términos de esfuerzo si se comparaba con los siguientes.

Fue más o menos en ese momento cuando vio a los gatos escapar.

Callejeros. Grises, el color estándar de la ciudad. Salieron de dentro de un cubo de basura, erizaron el lomo como si acabaran de cruzarse con la silueta de Satanás y salieron despavoridos, emitiendo un maullido desagradable como los chillidos de un cerdo camino del matadero.

El criminal no movió un músculo. Se quedó quieto, como de piedra. Paralizado. La silueta de su perseguidor se perfiló entre él y la libertad y caminó hacia su posición de manera uniforme, como si el suelo no fuera el medio que empleaba para desplazarse.

Avanzó hasta colocarse frente al delincuente, sin hacer comentario de clase alguna. Dejó pasar unos pocos segundos en los que el tipo, de haber continuado, podría haber sudado kilos enteros de grasa corporal. Estaba rígido como una estatua de sal, contrayendo todos los músculos del cuerpo. El estómago le daba vueltas como una montaña rusa enloquecida y sin frenos.

Su perseguidor le agarró de las solapas y le separó veinte centímetros del suelo. Le empotró con la pared con tanta violencia que debido a la inercia su nuca golpeó con el frío y ennegrecido ladrillo, pero ni se inmutó de ello, pues la palpitación que sentía en las sienes en ese momento eclipsaba cualquier otra sensación.

Su atacante se dirigió a él al fin. Una sola palabra, fácil de entender.

\emph{Habla.}

El delincuente empalideció por completo. Podían sucederle muchas cosas malas si hablaba. Pero tenía la sensación de podían ser aún peores si no lo hacía.

Una rata comenzó a subir por un canalón y se paró justo a la altura de su cabeza. Aquello era ya demasiado para soportarlo.

~---Vienen\dots\ vienen de fuera ~---empezó a recitar~---. Disidentes. Saboteadores, dicen ellos.

\emph{Más.}

~---Antiguos militares, descontentos con el trato que se les reportó en su momento tras la guerra. No sé más, lo juro.

Le dejaron caer y se estrelló contra el suelo, aún con escalofríos. No quiso ni apartar la mirada del suelo. Cuando lo hizo, estaba solo en el callejón. Se puso en pie y miró al fondo del mismo. Aparentemente vacío. O quizás no.

Le llevó una hora decidirse a dar el primer paso para salir de allí, aunque su atacante ya hacía tiempo que había abandonado aquella zona de la ciudad.

\parbreak
Las noticias frescas que Shockman trajo al cuartel no hicieron más que confirmar los peores temores de muchos de los miembros de Los Caídos. Saboteadores. Renegados, o algo peor. Ya era bastante malo tener militares en la ciudad, encima también paramilitares que actuaban por cuenta propia y, tal vez, sin importarles demasiado poner en peligro a civiles.

Desde el comienzo de las obras sólo unas pocas plataformas habían sido levantadas, concretamente las partes del trazado que se erigían frente a los edificios más icónicos del Distrito financiero y parte de los túneles subterráneos del tramo inferior. Aun con todo la planificación iba a la velocidad adecuada y se esperaba que, al menos a efectos de mostrar cómo sería el resultado final, en los meses siguientes habría varios kilómetros de autopista finalizados y transitables.

Hubo muchos mercenarios a los que se subcontrató para proteger ciertos sectores de autopista. El ejército empezaba a estar desbordado ante la imposibilidad de vigilar todos los lugares a la vez, y Filo Omega dio luz verde a que se dispusiera de los servicios de soldados de alquiler, aunque no fuera esa una decisión popular entre los suyos. Ella tampoco ocultaba su desagrado por tener que trabajar con sujetos que, aunque operaban dentro de la legalidad, para ellos el único precio a pagar por la victoria era el que imponían a su contratante.

Para Los Caídos, sin embargo, eso supuso una buena noticia. Al menos, cuando empezaron a ver que el primer equipo al que contrataron no eran, ni de lejos, unos desconocidos.

Cuando les localizaron la mitad de ese equipo se encontraba en la parte superior de una de las plataformas, vigilando el horizonte lejano y el espacio aéreo, y la otra mitad estaba en la base, al pie de las columnas de hormigón armado que sostenían todo el conjunto, con centenares de metros separando a unos de otros. Su líder estaba abajo, apoyado en una de esas columnas. Sus guanteletes estaban listos para la acción, así como su mochila cibernética tenía la energía al máximo. Y si no, siempre su otro compañero de turno de guardia, al otro lado de la estructura, podía darle una recarga adicional.

~---Odio este encargo ~---comentó Repulsor en voz alta, mirando el retorcido entramado de cables que surgían de sus manos y se dirigían hacia su espalda.

\emph{Yo lo prefiero si con eso ya no tratas de cazarme} ~---escuchó que le contestaban por encima de su cabeza.

~---¿Cuánto tiempo llevas ahí?

\emph{Batería ya me había visto, si era eso lo que te preocupa} ~---dijo Scream bajando y comunicando a los otros miembros del escuadrón que mantuvieran la posición.

~---Bien, nuestros destinos vuelven a cruzarse. Ya imaginaba que no tardarías en dejarte caer.

\emph{Literalmente, de hecho} ~---matizó Scream con ironía.

~---Déjame preguntarte una cosa. ¿Es verdad que acabaste con Alma Espejo?

Scream no contestó.

~---Ya veo. Toqué un tema espinoso.

\emph{Algo que es mejor olvidar} ~---se limitó a agregar su interlocutor.

~---Ahora es cuando me dices qué te trae por aquí. A nosotros algo tan sencillo como sacar algo de pasta para ir al otro lado del planeta, donde nos esperan bocados más grandes.

\emph{Vengo a advertirte.}

~---¿Advertirme? No vamos tras de ti, ya te lo he dicho.

\emph{No contra mí, contra un grupo de paramilitares que al parecer se han infiltrado en la ciudad.}

~---¿Y de dónde has sacado pruebas de algo así, si puede saberse?

Antes de contestar, Scream sólo pudo fijarse en aquello que iba directo hacia ellos, como si lo viera a cámara lenta. No era un vehículo, ni un enemigo. Algo más contundente, más directo.

Un misil.

Apenas tuvieron tiempo de apartarse antes de que cayera a varias decenas de metros de su posición, lo que hizo que ambos cayeran al suelo estrepitosamente. No provocó más daños que materiales, pero bastó y sobró para que cundiera el caos por toda la zona y la gente empezara a correr despavorida, mirando hacia arriba. Fue cuando se dieron cuenta de que la amenaza en realidad provenía de las alturas.

\emph{¿Te vale como prueba?} ~---preguntó Scream saltando a la escalera de incendios del edificio contiguo, para subir más deprisa. A su vez Repulsor comenzó el ascenso por la estructura interior, viendo que Batería le seguía de cerca, arma en mano y un piso por debajo de su posición.

Cuando llegó arriba observó que varios militares estaban cercando a Barrera y Silencio. La vara del primero hacía de escudo de energía y les protegía de las armas automáticas que llevaban sus enemigos, pero la andanada continua de descargas no tardaría en doblegar sus defensas y ponerles a merced de los asaltantes.

Nada más poner un pie en la tarima superior, aún llena de huecos de la estructura y por lo tanto un escenario peligroso para combatir, Repulsor lanzó una descarga a mínima potencia contra uno de los soldados y le tumbó de inmediato. Otro se giró para disparar al recién llegado, pero antes siquiera de que pudiera recargar el arma una sombra cayó sobre él y le tumbó de un sonoro puñetazo.

~---Gracias ~---se limitó a decir Repulsor. Scream no contestó~---. Poco hablador, sin duda ~---se limitó a añadir.

Más soldados estaban saltando desde los edificios contiguos, convirtiendo su posición en más que comprometida. Silencio empezó a guiar a los obreros y constructores lejos de la línea de fuego, hacia niveles inferiores del armazón, en lo que Barrera cubría su salida y Scream y Repulsor trataban de liberar presión a su vez contra Barrera. El escuadrón de Scream empezó a atacar en la sombra, con maniobras tácticas destinadas no a revelar su presencia sino a desequilibrar y tomar por sorpresa a su enemigo. Cubrir a su líder con hologramas, atacar y disparar por la espalda a soldados solitarios. Una maniobra complicada dado que aquel escenario era completamente inédito para ellos, por mucho que hubieran previsto la necesidad de pelear sobre él.

Repulsor y Scream no tardaron en notar cómo iban cediendo terreno y una hilera de cinco soldados comenzaba a avanzar lentamente hacia ellos. Barrera se unió con ellos a la refriega, pero aun así estaban demasiado al descubierto.

Fue entonces cuando los refuerzos llegaron, pero lo primero que de ellos notaron fue un espectacular tajo invisible que tumbó la fila de soldados, momento que aprovecharon para lanzarse sobre ellos, obligándoles a retroceder y escapar.

Un instante después Filo Omega apareció, seguida por más soldados, sin duda un factor motivador extra en la huida de los asaltantes. Su espada estaba desenvainada, y era aún más ancha y portentosa de lo que Scream había imaginado al verla dentro de su funda. El sello del símbolo omega resaltaba con sorprendente claridad, grabado en el arma con un espesor de varios milímetros.

~---Iremos tras ellos ~---comentó con sencillez, mirando a su alrededor y fijándose en Scream, visible como el que más después de toda la refriega~---. ¿Qué hace ése aquí?

~---Nos ha ayudado a contenerlos ~---explicó Repulsor. Quedó claro que era a ella a quien debía dar explicaciones en términos contractuales.

~---Es un fugitivo, y tiene cuentas con la justicia ~---dijo sin inmutarse~---. Aturdidlo ~---ordenó.

Los soldados calibraron sus armas hacia el Caído y dispararon, pero todos los rayos le atravesaron, tras lo que el blanco se limitó a saltar por el borde la plataforma y desaparecer ante la vista de todos los presentes.

~---Ya habrá otras ocasiones ~---comentó Filo Omega con frialdad~---. ¿Todo el mundo bien por aquí? ¿Bajas?

Fue entonces cuando Repulsor se dio cuenta de la ausencia prolongada de uno de los suyos, algo en lo que no había caído antes debido a lo sorpresivo del ataque.

~---No está Batería. ¿Barrera?

~---No le vi subir ~---se limitó a contestar el aludido.

Silencio, que acababa de llegar de poner a salvo a todo el mundo, se limitó a negar con la cabeza.

~---Genial ~---dijo Repulsor~---. Una maniobra de distracción.

~---¿Qué puede hacer ese tal Batería, recargar armas o algo así?

~---No sólo armas, máquinas de toda clase, gracias a su brazo biónico.

~---Un recurso útil, sin duda, para un grupo de traidores que necesitan reservas de energía ~---apuntó Filo Omega con claridad.

Debajo de los presentes, escondido entre las sombras de las vigas, John Scream indicó a los suyos que regresaran al cuartel, y mientras les seguía se planteó cómo su instinto de protección de la plataforma les había hecho olvidar que existen otras maneras de atacar a un enemigo que disparar hacia el objetivo más obvio e inmediato.

\parbreak
Lejos de allí, en un lugar oscuro y desconocido, Batería despertó después de haber estado largo rato inconsciente. Le cogieron por sorpresa y no le dieron la menor oportunidad. Quienes lo hicieron llevaban largo tiempo planeando algo así, tal vez incluso estudiando sus movimientos y los posibles imprevistos.

No veía bien, pero estaba preso contra una pared, en cruz y sujeto con grilletes convencionales, nada de cerraduras electrónicas. Conocían bien quién era y lo que podía hacer. Al fondo vio varias siluetas de militares, unos observándole, otros ocupados en otras tareas, como cargar cajas con, supuso, armamento. Misiles, explosivos, no había manera de saberlo.

Trató de mirar a su alrededor, pero estaba muy oscuro y su visión era demasiado borrosa. No logró captar detalles concluyentes de su lugar de cautiverio, y al final, extenuado por el esfuerzo y la incómoda posición, cayó inconsciente de nuevo.

~---Perfecto ~---dijo alguien acercándose y observándole~---. Quiero verle en perfectas condiciones cuanto antes. Le necesitamos en plena forma para la activación. Mientras, veamos hasta dónde llega el alcance de sus cualidades.

Después de eso Batería se volvió a quedar solo, crucificado, sumido en un letargo sin sueños ni pesadillas de clase alguna.

\endinput
