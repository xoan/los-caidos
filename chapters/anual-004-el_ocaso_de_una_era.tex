Porque todo lo que tiene un principio, tiene un final. Aunque a veces, el final no es más que el principio de algo nuevo tan interesante o más que lo que para siempre se ha perdido.
 
\fancyparbreak
\section*{No se admiten devoluciones}

No me hagas esto joder, pensó Jim Swart volviendo a mirar las componentes descompuestas de su armadura, lanzando chispazos a lo largo de sus junturas más esenciales e importantes. Ahora no, no en este momento, cuando menos quedan de nosotros.

Ya le había pasado eso antes, pero nunca los daños habían resultado tan graves. En el fondo sabía que algún día tenía que pasar así. Aquel equipo no había sido diseñado por él, y por tanto tenía fecha de caducidad desde el primer momento que cayó en sus manos. Un mal golpe, una caída a destiempo, y bye bye heroicidades, bienvenido al mundo real.

En las primeras ocasiones, trasteando con el equipo, había logrado repararlo. Más adelante, ante daños más severos, tuvo que recurrir a la ayuda de tipos más expertos que él. Leches, si hasta una vez obligó a uno de sus peores enemigos a repararlo delante de sus narices. Todo lo éticamente posible con tal de poder seguir haciendo aquello que sabía hacer mejor.

Pero ya no había marcha atrás para el desastre que tenía ante sus ojos. Había partes que ya no podía ni ponerse sin que le dieran toda clase de insoportables calambres. Se lo había llevado a cantidad de expertos de ambos bandos, arriesgando su identidad en el intento, ya que al fin y al cabo si no se arreglaba poco iba a importar que la conservara intacta. Todo inútil. Todo una pérdida de tiempo, dinero y favores, según el caso.

Vamos, arréglate, joder. Arréglate. Tienes que funcionar. Funciona.

Pero no pudo ni ponerse el casco que tapaba su identidad. Maldijo en voz alta y propinó un sonoro golpe a la mesa sobre la que estaban todos aquellos trastos que se habían convertido en su mayor parte en chatarra inservible.

De repente Swart se sintió como si su apartamento fuera pequeño, insignificante, y él fuera un fracasado sin lugar alguno en la corrupta ciudad en que le había tocado vivir. Se sentía furioso, cabreado hasta la misma médula. Otro en su lugar tal vez se hubiera llevado las manos a la cabeza, o asomado por la ventana para reflexionar en terrible y tenebrosa soledad. Él no tenía paciencia para esa clase de actos melancólicos. En ese momento, de hecho, lo único que podía relajarle era salir a la calle y partirle la cara a uno o dos de los múltiples ladrones, violadores, estafadores, secuestradores o asesinos que poblaban las calles en ese momento.

Uno por uno, fue desechando todos los aparatos que alguna vez habían sido estandarte de sus actos, bien por estar para los restos, bien por resultar inútiles por sí solos. Sólo un guante sobrevivió a la criba, el resto era tan útil allí como en un cubo de basura en la calle, cubriéndose de ceniza con lentitud.

Se ajustó el guante, el único resto de su otro yo, casi con tanta parsimonia como si estuviera haciendo lo propio con el equipo completo. Buscó algo con lo que taparse la cara y encontró una gorra y gafas oscuras, y se preguntó de dónde demonios habría sacado esos objetos y para qué los habría comprado en una ciudad de mierda como aquella, donde nunca brillaba el sol. Amargado como estaba, salió a la calle en busca de bronca. Sólo desahogarse era lo que buscaba, tampoco era mucho más lo que podía hacer en ese momento.

Se topó, dos callejones más abajo, con tres quinquis de la pandilla local que estaban apaleando a un rival demasiado tonto para no darse cuenta de dónde se estaba metiendo. Nada de sutilezas ni entradas espectaculares. Se plantó en mitad de la calle y se encaró con los tres agresores.

~---Meteros con uno de vuestro tamaño ~---soltó sabiendo que, aunque él no era precisamente de su tamaño, tenía por otro lado una notable desventaja numérica.

Los puños nunca le fallarían, al menos, aunque se dio cuenta de que por mucho que el guante mejorara y aumentara la fuerza de su pegada eso no era ni la sombra de lo que no hacía mucho había sido capaz de hacer. Y ahí estaba, encarándose con unos matones del tres al cuarto, en la decadencia más absoluta. Pero al menos se largaría como siempre había hecho, con los puños por delante.

Los tipos no tenían la menor idea de pelear cuando se trataba de hacerlo contra alguien a quien no pillaban indefenso. Todo lo que sabían hacer era juego sucio. Patadas, puñetazos mal golpeados. Eran una nulidad en combate, no tenían ni idea de que lo importante del golpe es la aceleración que se le imprime, no la distancia que el puño recorre. Uno de ellos sacó una cadena, pero tampoco es que le sirviera de mucho. Swart la enredó en el guante, tiró de él y le golpeó tal cual el atacante se le acercó contra su voluntad.

Entonces vino el destino y desequilibró la balanza a favor de los pandilleros. El guante también empezó a soltar descargas que le pillaron por sorpresa y se dobló como un folio, cayendo de rodillas al suelo. Sólo tenía que quitarse el guante, aunque no fuera más que un hombre después de renunciar al último trocito de su alter ego. Pero ni de eso tuvo tiempo. La patada en el estómago que recibió fue demoledora, y después de eso vinieron muchas otras. Si no se lo cargaron allí mismo fue porque la poli apareció por la zona, patrullando para meter miedo, y se largaron con pies en polvorosa.

El tipo al que había salvado, por supuesto, ya hacía tiempo que no estaba allí, por lo que se quedó solo de nuevo. Se levantó apoyándose en una repugnante pared y caminó tambaleante en dirección a su aún más repugnante piso. Tenía que llegar sin desmayarse en el camino, so pena de que alguien viera el guante y sumara dos más dos. Ya no le daba calambres, y pensó que igual algún día podía arreglarlo. Algún día.

Entró en su casa de nuevo, donde los restos de su vida se esparcían por toda la mesa, y se desmayó al fin, consciente de que cuando despertara ya nunca sería el mismo de nuevo.

\section*{El enemigo más poderoso}

Su nombre es Matthew Swind, aunque no es con el que muchos, sobre todo muchos que le odiaban, le conocieron. Sus poderes eran algo entre increíble y hermoso. Ilusiones. Hologramas. Mostrar cosas que no se podían ver, mundos que no existían en realidad.

A veces parecía omnipotente, inmortal. Que pudiera hacer cualquier cosa, y le encantaba que así fuera, de hecho. No en vano, por eso había elegido ser mago como profesión, además de suponer una excelente coartada para sus trucos.

¿Estoy aquí? No, estoy allí. \emph{Desaparecer} en plena calle, en mitad de una tormenta de aguaceniza. Escapar de los lugares más sorprendentes e inimaginables, como un deslizador sin piloto en pleno movimiento. Para sus enemigos, situaciones y trucos más increíbles aún, que por desgracia para él nunca podría efectuar en sus actuaciones.

Pero aun con ese tremendo poder, esa capacidad de cambiar y alterar las leyes de la física en apariencia, toda su vulnerabilidad quedó patente y al descubierto cuando se vio a sí mismo sentado junto al lecho de muerte de su hijo.

Tenía ocho años y había tenido la desgracia de padecer una enfermedad como los médicos no habían visto jamás. Su mujer y él quedaron destrozados al saber la noticia. No se lo dijeron el uno al otro en ese momento, pero eran ya conscientes de que su matrimonio estaba tan devastado como ellos mismos lo estaban individualmente por dentro.

Su mujer, que ya sólo lo era por inercia, y no tardaría en dejar de serlo, no estaba allí con ellos, viendo los últimos momentos de la vida que se desmoronaba a ojos vista. El niño era consciente de que se moría, pero el médico les dijo que no pasaría de esa noche, cosa que no podía intuir. Su madre le acostó a oscuras sin apenas hablarle, le besó y se marchó. Si lo hizo sin luz fue para que él no pudiera darse cuenta de que las lágrimas se deslizaban por sus ojos, y si apenas le habló fue para que no apreciara su voz trémula en aquella noche fatal, pensando que la carne de su carne jamás vería un nuevo amanecer.

Swind se quedó allí más tiempo, callado, como si aquello fuera una más de las ilusiones que solía engendrar. Todos esos poderes, esa capacidad de mostrar cosas sorprendentes, para nada. Podía engañar a las leyes de la naturaleza, pero no doblegarlas a su voluntad. Su hijo estaba condenado, y ni los mayores héroes del mundo podrían hacer nada para evitarlo.

Quién le iba a decir que su peor enemigo, el que le motivaría a dejar para siempre de hacer trucos, tanto por diversión como por protección de los débiles, sólo podría verse empleando un potente microscopio. Que nadie sabía cómo derrotarle aunque miles de personas luchaban contra él en gran cantidad de laboratorios.

Estaba librando en silencio su última batalla. Pero no se permitiría perder sin antes haber luchado.

Así, le dijo a su hijo quién era en realidad, qué era lo que hacía por la ciudad además de sus espectáculos nocturnos dos noches por semana. Le confesó por qué siempre desdeñaba con su rostro público lo que hacía con el privado, y le narró muchas de las peleas más épicas, extrañas o sorprendentes que había tenido en toda su trayectoria. Como quien cuenta un cuento, como quien sólo busca distraer al enfermo de su terminal condición.

Cada historia que contaba la adornaba con hologramas, y también con imaginación, con sus propias palabras acompañando la acción del momento. Su hijo escuchaba atento, pero cada vez estaba más cansado, le costaba más y más mantener los ojos abiertos. Hasta que de repente, en la mitad de una de ellas, Swind se dio cuenta de que se había dormido al fin para no despertar jamás.

Apagó los hologramas y no pudo llorar. No lloró. Mucho tiempo después, deseó haberlo hecho. Porque tal vez, entonces, no hubiera puesto fin a su identidad como defensor de la ciudad. Tal vez, hubiera podido usar sus poderes sin temor a que, cada vez que lo hacía, le trajeran a la memoria la muerte de su hijo.

En vez de eso se limitó a usarlos para moverse entre sombras. Abatido, amargado. Castigándose a sí mismo a morar entre tinieblas. Deseando ser tan perverso como aquellos a los que una vez combatió.

Sin entender aún que tal vez esas mismas tinieblas pudieran servir para combatir al enemigo más poderoso: no la muerte, sino su sentimiento de culpa e inutilidad por no poder impedir la muerte de su propio hijo.

\section*{Corrupción de la mente, corrupción del alma}

Ellis Saw llegó a la fábrica principal de Gorgon Enterprises tras haber tomado un taxi deslizador y se bajó frente a su entrada enrejada, fuertemente custodiada por dos vigilantes. Nada más poner un pie sobre aquel pavimento desgastado e inhóspito supo que no era una buena idea lo que estaba perpetrando, pero aun así prosiguió de todos modos con su esquema mental. Ya no había marcha atrás para lo que pretendía llevar a cabo, y por ello sería mejor que empezara a asumir las consecuencias, sin duda irreversibles, de aquel intento de servir a quienes más odiaba en realidad.

Ya sólo por el hecho de hacer aquella entrevista en una suerte de hangar reconvertido a factoría, Saw comprendió que no sería como ninguna de las que hubiera tenido jamás. De hecho, era la primera vez en toda su vida que deseaba de manera inconsciente y con todas sus fuerzas ser rechazado para el puesto de trabajo para el que estaba ofertando.

No tendría tanta suerte, pensó. Su formación en relaciones públicas y administración de empresas era más que notable, y además poseía sobrada experiencia en algunos de los mejores sectores de la ciudad. Una ciudad que, muy a su pesar, estaba comprobando cómo se venía abajo desde sus mismos cimientos podridos.

Entró en la factoría, seguido por uno de los dos vigilantes, y mientras avanzaban por la antesala al bloque más grande del complejo notó que a sus espaldas la verja se cerraba como si fueran las puertas principales de la más inexpugnable prisión nunca ideada por la perversa mente humana. Le hicieron pasar el primero a techo cubierto, y al otro lado, de pie en medio de un pasillo y con postura de impaciencia, esperaba una mujer vestida con una sobria falda y chaqueta a juego que le indicó que le siguiera. Saw observó que había allí suficientes cámaras como para controlar hasta la esquina más recóndita de aquel, al menos después de una primera apreciación, laberíntico lugar.

~---¿Ha venido en taxi? ~---preguntó con tono de altivez la mujer.

~---Así es.

~---Si finalmente es seleccionado para relevarme será la última vez que lo haga ~---se limitó a apuntar en lo que seguían caminando.

Llegaron a una sala habilitada de mala manera para entrevistas. Carecía de ventanas, sus paredes de hormigón eran sucias e irregulares y sólo una débil luz amarillenta de fluorescente alumbraba el lugar, situada justo encima del asiento del entrevistador. Cuando ambos ocuparon sus posiciones Saw notó que la iluminación deficiente generaba sombras inquietantes y siniestras sobre el rostro de su interrogadora.

~---Bien ~---comenzó recitando de memoria, sin necesidad de consultar apunte alguno~---. Ellis Saw, licenciado en ciencias empresariales. Expediente de dos coma nueve en la Universidad de Arnápolis. Ni demasiado callado ni demasiado contestatario, según sus antiguos compañeros. Prácticas de empresa en Virtuatronics, donde firmó por dos años más. Luego decidió trasladarse a trabajar a Kamiyaza en virtud de un acuerdo temporal entre ambas empresas. Abandonó Kamiyaza tres años más tarde. ¿A qué se debió aquella decisión?

Saw empezó a recordar las reglas del juego. Decir medias verdades al mismo tiempo que se pensaba en la verdad, algo que sólo los expertos en microgestos podrían distinguir. El método estanislaski de las entrevistas de trabajo. Porque muy a su pesar su destino era ese trabajo, y ya nada podía ni debía alejarle de tal objetivo.

~---La empresa sufrió un revés económico y decidí que no era adecuado a mis intereses permanecer en ella.

La empresa sufrió un revés económico que yo, con mi otra identidad, provoqué, y decidí que no era adecuado a mis intereses como héroe permanecer en ella una vez la asesté un golpe desde dentro.

~---¿Qué clase de revés?

~---La pérdida de un prototipo que fue sustraído.

La pérdida de un prototipo destinado a fines bélicos y reciclado como arma de superhéroe que fue sustraído por mí.

~---Comprendo. ¿Qué función tenía ese prototipo?

~---Me es imposible decírselo.

No me es imposible decírselo, pero jamás se me ocurriría hacerlo.

~---Ya veo. ¿Por qué tiene interés en trabajar para Ellen Gorgon?

~---Creo que es una empresaria influyente que no tardará en ejercer un gran papel en la ciudad de Ernépolis.

Creo que es una empresaria corrupta y demasiado influyente que no tardará en ejercer un gran papel en la decadencia de la ciudad de Ernépolis.

Tras aquel breve interrogatorio siguió toda una serie de preguntas sobre pormenores técnicos durante las que Saw no tuvo necesidad de encubrir sus pensamientos. Después la mujer le acompañó a la salida, donde las puertas de la prisión se abrieron de nuevo sólo para él.

~---Ya le llamaremos si resulta seleccionado por la señorita Gorgon ~---se limitó a decir la entrevistadora y actual ayudante sin siquiera darle la mano, como si fuera un tísico. Pero Saw sabía que le llamarían. No quería ese puesto, y por eso precisamente le elegirían a él: no mostró nervios y sí aplomo, y además gozaba de una vida laboral y académica brillante y envidiable por otros del sector. Lo malo de no desear algo, pensó, es que puede que lo acabes consiguiendo.

Los días pasaron y como se temía fue aceptado en el puesto. Ya era, de manera oficial, el ayudante personal de Ellen Gorgon.

Ya sólo restaba matarla en el momento adecuado.

Ellis nunca fue un gran héroe. No poderoso, ni memorable. Su único poder era el arma que robó a Kamizaya, una linterna especial diseñada para la guerra en planetas sin sol, capaz de robar luz y a cambio generar oscuridad en su lugar. Como complemento a esa cualidad, tenía además su ojo clínico innato a la hora de juzgar a las personas. De hecho, consideraba a ese \emph{poder} más útil que al primero. Para empezar, gracias a él, estaba convencido de que Gorgon era un monstruo como ningún otro que había hollado antes la ciudad de la eterna Nube. Por ese motivo había tenido que ir tan lejos a la hora de aproximarse a ella, y ya sólo quedaba la parte sencilla de ejecutar. Primero sólo tenía que llevar consigo una pistola, ese objeto terrible que dota de realistas y demoledores superpoderes a cualquier ser humano, que le convierte en un justiciero o asesino letal a los ojos de los demás. Luego, calibrar su linterna especial para que expulsara de una sola vez toda la luz que había acumulado en tantos años, para cegar así a sus guardaespaldas. Luego de eso, disparar. Era sencillo.

¿O no?

A lo largo de las semanas en su nuevo puesto de trabajo, tuvo docenas de ocasiones para ello. Oportunidades que malgastó. Y entonces se dio cuenta de lo estúpido que había sido. Gorgon no sólo le contrató por su currículo, también por sus ideales. Tranquilo, pacífico. Creyente, objetor de conciencia en lejanas guerras coloniales.

Ni para hacer lo que tenía que hacerse había logrado ser un gran héroe. Pero no desfalleció, a pesar de todo. Tal vez algún día, en algún momento, su posición podría ser útil para combatirla. O tal vez ese día no llegara, como sospechó a medida que Gorgon acumulaba más y más poder y fue anunciando su candidatura a la presidencia de la ciudad.

De todos modos, pensaba a menudo Saw, es bonito tener sueños imposibles de vez en cuando, simplemente de cara a no perder un motivo para pensar de manera positiva en el hipotético día del mañana.

\section*{El precio de la justicia}

Charles Razorclaw había conocido la podredumbre humana en tantas vertientes maléficas que ya era poco lo que lograba sorprenderle en ese aspecto. En el banquillo de los acusados había tenido el dudoso honor de comprobar cómo se juzgaba, muchas veces con sorprendente ligereza, a los que habían sido algunos de sus peores enemigos, y también los de otros. Él mismo solía solicitar aquellos casos en calidad de uno más de los múltiples abogados de la acusación del Estado. Por eso no podía resultarle más irónico que lo que estuviera envenenándole la existencia, más que pelear contra aquellos engendros en su identidad alterna, era hacerlo en su identidad estándar. Verles impunes, sonrientes incluso, chulos y contestatarios, o a veces negándose a contestar o aceptar la legitimidad del tribunal que les estaba encausando.

Pero ninguno le emponzoñó el alma tanto como aquel cabrón medio desdentado que tenía ante sus ojos en la sala del juzgado de lo penal, en el Tribunal Superior. Nada para él había representado antes el mal de manera tan pura y perfecta como ese sujeto que estaba en el estrado, sosteniendo su mirada vaga e inocua sobre algunos de los asistentes al juicio, no demasiados dada la repugnancia que el acusado profesaba.

Razorclaw miró a su alrededor, a la sala labrada en madera, vieja y carcomida en muchos de sus relieves de columnas, y luego posó la mirada sobre los asistentes al juicio. Había auténtico odio en sus miradas, y no era para menos. De todos modos, pensó, si hubiera podido mirarse en un espejo estaba seguro de que su caso hubiera sido el más claro de todos. Podía notar cómo tenía los ojos entrecerrados de manera involuntaria y el ceño fruncido con violencia. Cada cierto rato reparaba en que estaba apretando los dientes y se esforzaba por recuperar la posición normal de la mandíbula, pero no pasaba demasiado tiempo hasta que volvía a hacerlo de nuevo. Su cuello era un conglomerado de nudos imposible de deshacer, y sólo hubiera bastado con que lo moviera hacia un lado para que crujieran las vértebras como astillas al quebrarse por efecto de la presión.

Aun con todo, todo eso era una nimiedad comparado con lo que sintió cuando él mismo capturó in situ al criminal. Allí en ese edificio mugriento, destartalado, sucio y desahuciado de los Túneles tuvo sensaciones similares pero acentuadas, pero no fue consciente de ninguna de ellas porque fueron sustituidas por un impulso homicida que a punto estuvo de hacerle perder los estribos.

Entró con seguridad, sin miedo. Era lo que tenía su poder principal, que resultaba imposible sorprenderle por la espalda, además de suponer un eficiente radar con los rastreadores y chips adecuados. Eso, unido a que estaba más que ducho en cuestiones de fuerza y agilidad elemental, sirvió para que cubriera las estancias del bloque a gran velocidad y llegara a la guarida que el criminal estaba empleando, un antiguo piso en el que habían sido apiñados ilegalmente gran cantidad de desarraigados provenientes de otras polis cercanas.

Las sombras eran su aliado. El olor, que dificultaba la concentración de los más impresionables o con estómago flojo, también. No había puerta que supusiera un obstáculo a su camino, pues a todas las habían saltado la cerradura y estaban abiertas de tal modo que se podía ver la estancia subsiguente. Sin emboscadas ni trampas. Demasiada tranquilidad a su paso, razonó.

Encontró al sujeto con los pantalones bajados, aún deleitándose en su éxtasis de sexo y muerte. Ni se fijó en su presencia cuando llegó. No era uno de esos villanos que soltarían un discurso, ni tampoco tenía elaborado un plan maquiavélico y orquestado. Pero tampoco era un criminal cualquiera, un gañán de poca monta. El alcance de su delito estaba a sus pies, inerte, tirado en el suelo como un objeto sin valor. Tres años de edad y ya había conocido toda la maldad que puede albergar el mundo que le rodeaba.

Razorclaw golpeó al violador y asesino con tanta fuerza que le saltó casi todos los dientes a la primera, y antes de desencadenar una batería de ganchos que le resultaría imposible detener bajo aquellas circunstancias, se acercó al niño, recordando que proteger a los débiles era prioritario con respecto a castigar a los criminales. Aunque fuera proteger su recuerdo, su memoria, y permitirles morir con dignidad.

Una fugaz esperanza le recorrió por dentro en ese momento. Él no era médico, tal vez su impresión inicial había sido errónea, tal vez los hados no le habían vuelto la espalda aún a tan inocente e indefensa criatura. Pero no fue así, por desgracia. Con todo el dolor de su alma, Razorclaw no pudo tocar al niño. Sabía que la escena del crimen debía permanecer lo más intacta posible.

Claro que eso no tenía por qué aplicarse a su pederasta homicida.

Razorclaw le miró a la cara. Estaba pletórico, una detestable sensación de felicidad recorría la comisura de sus labios de lado a lado. Varios dientes se le cayeron en el proceso de esbozar una sonrisa, pero aun así no cejó en su empeño de mostrar que por dentro no existía ni el menor atisbo de nada cercano al remordimiento.

En toda su trayectoria en las calles, Razorclaw había tenido impulsos tan violentos como aquella noche, frente a aquel diablo con cuerpo de hombre, ese belcebú salido de las tinieblas que había vejado, torturado, humillado y finalmente exterminado al semejante más inocente e incapaz de hacer daño que la mente humana es capaz de concebir.

Le agarró del cuello y empezó a apretar. Con todas sus fuerzas. No fue capaz de dominar ese instinto que le pedía sufrimiento y dolor a toda costa, que le suplicaba para que esparciera sus tripas por toda la habitación, que abriera nuevos agujeros en las paredes cochambrosas usando su cabeza de instrumento de derribo. Pero hubo algo que le heló la sangre en la venas y le dejó paralizado, incapaz de proseguir. No la compasión, ni la bondad, ni la necesidad de no perder el alma con tal de no castigar a aquel que tenía ante sus ojos.

Estaba disfrutando. Aquel enfermo no sólo gozaba con aquello, se estaba dejando hacer por completo. Tal vez, pensó, hasta esperaba su llegada o la de algún otro para poder vivir un momento así. Le soltó y cayó al suelo de culo, jadeante, sin dejar de mirarle con las pupilas dilatadas. Poco después fue cuando escuchó las sirenas de la policía acercarse. Demasiado tarde, pensó. No sólo por el niño, también por él mismo. Aquello había empezado a ser la gota que podía desbordar el vaso, no por lo que había visto, sino por lo que se temía que estaba aún pronto a presenciar.

No se equivocó una vez el juicio hubo transcurrido. Libre. La intervención de un desconocido que atacó al acusado había introducido una duda razonable. Su abogado fue astuto, y le dijo que acusara al otro. Al que no podía hablar, que no podía subirse a un estrado.

Razorclaw se planteó después que debería haber testificado. Miles de veces. Pero eso hubiera comprometido todos los casos en los que él había estado involucrado tanto en el ministerio de la defensa como en el de la acusación. No podía hacer algo así a todas aquellas personas por un sentimiento de venganza.

No, pensó. Venganza no. Justicia.

¿Pero qué es la justicia en una ciudad como esta?, reflexionaba a menudo. Papel mojado, jueces corruptos o de ideas ultraderechistas, o en el mejor de los casos extraños personajes mediáticos impredecibles como aquel embajador recién llegado que había ascendido al Tribunal Supremo.

Decidió que no volvería a salir a las calles. Para qué, pensó. No hay infraestructura que respalde lo que hago, nadie que apoye mi labor. Es tanto como cazar mariposas para liberarlas después. Por eso descuidó su entrenamiento y equipo, hasta tal punto que los delicados chips del aparato que le permitía tener siempre vigilada su espalda se estropearon y ya sólo funcionaban emitiendo un desagradable pitido, nada útil si se pretendía emplear el sigilo como modus operandi principal.

Haría falta otra justicia, sin duda. No una cualquiera, pero que atemorizara a los criminales. Que les metiera el miedo en el cuerpo. Prevenir, antes que curar.

Ridículo, pensó. Con todos los héroes que estaban desapareciendo, justicia era un concepto que a pasos agigantados se desvanecía de las calles e instituciones de esa eterna, humeante y endemoniada urbe que era Ernépolis~I.

\section*{El dolor de la incertidumbre}

James Sky se miró al espejo de su taquilla en la comisaría de policía, exhausto, incapaz de razonar ni de pensar con claridad. Su hermano. Su hermano había desaparecido, y tal vez estaba por su culpa en grave peligro. Era algo que pensaba que tenía asumido, la posibilidad de que hicieran daño a alguien querido para tratar de llegar hasta él. La muerte de esa persona sería algo terrible de soportar, era consciente de ello, pero al mismo tiempo también estaba convencido de que acabaría resurgiendo de sus cenizas, recuperando el esplendor perdido, precisamente para que a nadie más le pasara lo mismo que le había pasado a él, tener que perder un amigo, un pariente, un o una amante.

Pero nada le había preparado para aquel golpe que flotaba de manera angustiosa en sus sienes, que suponía una tortura gradual, silenciosa, permanente. No se sabía nada de su hermano Paul desde hacía ya más de un mes, demasiado tiempo para que no hubieran hablado de un secuestro. Al mismo tiempo, también, demasiado tiempo para que alguno de los indeseables que infestaban la ciudad no se hubiera hecho eco de ello si es que buscaba vengarse de él por alguna clase de agravio real o imaginario.

No estaba preparado para ello, en absoluto. Nadie lo estaba, y por eso no podía culparse por ello. Pero se culpaba como pocos eran capaces de hacerlo. El dolor de la incertidumbre era infinitamente más insoportable que el de la pérdida, porque era como una herida abierta en su interior que nunca dejaba de sangrar y llagar.

Alguien tenía que saber algo, pensaba de manera constante. Alguien debía acabar por irse de la lengua, jactarse por ello, regodearse en su sufrimiento. Pero nadie aparecía. Y entonces, a los tres meses, decidió que debía esforzarse por hacer memoria y recordar. Recordar a todos aquellos que en algún momento le habían amenazado, que podían conocer quién era él, y cuál era su familia. Una tarea descomunal para la que su memoria era su principal enemigo. Si recordara a todos los que alguna vez habían tratado de intimidarle entonces no le quedaría sitio en la cabeza para poder almacenar nuevas experiencias.

Por eso se dedicó a barrer la basura de la ciudad como quien barre el salón de su casa. Buscó a los sujetos más indeseables, los más peligrosos, los más intimidatorios. Dada la naturaleza de su mayor poder, un visor con el que podía localizar lo invisible, lo lejano, lo ilusorio, muchos de ellos no tenían donde esconderse de él. Con algunos hablaba, con otros negociaba. Los menos colaboradores acababan teniendo una conversación con sus puños. Pero nadie sabía nada del paradero de su hermano.

Hasta que tuvo que empezar a pensar en sus oponentes más escurridizos. Esos que solían estar fuera del alcance de su poder. Y hubo un nombre enigmático que no tardó en venirle a la memoria.

Éxeter.

Poco sabía de aquel villano tuerto y poco hablador que se presentaba siempre a cara descubierta. La naturaleza de sus poderes era también desconocida para él. Se decía de Éxeter que era el mejor si uno quería mantener su cuartel o sus asuntos lejos de las narices de los héroes, y que su fama de callado y discreto bastaba como garantía de su confidencialidad. Sky recordó entonces lo que le dijo la última vez que se encontraron, cuando ahuyentó a otro villano que iba a convertirse en su cliente.

~---Acabaré con tus seres queridos ~---le dijo~---. Nunca les encontrarán.

Y de repente, un sentimiento de terror se apoderó de su mente. Terror por su hermano, por convertirse en el potencial blanco de un chalado como aquel.

Removió cielo y tierra para encontrar a Éxeter en la ciudad. Hasta tal punto le cercó que se corrió la voz de que había instalada una vendetta entre aquellos dos hombres. Éxeter se había convertido por derecho propio en el peor enemigo de Sky. Él no le defraudaría en cuanto a la parte que le tocaba.

Acosó a los criminales recién llegados. Les dijo que Éxeter estaba marcado, que no hicieran negocios con él. Bloqueó su mercado hasta que pasó lo inevitable, y el perseguido se dejó ver a los ojos del perseguidor.

~---Dicen que me buscas. Aquí me tienes ~---declaró en lo profundo de un callejón, escenario de baile en la ciudad de la mayoría de las peleas entre héroes y villanos.

~---¿Dónde está mi hermano? ~---dijo Sky, al borde del desmayo, sin casi haber parado durante semanas para encontrarle.

Éxeter comprendió entonces lo que estaba sucediendo. Alguien había asestado un golpe fatal al corazoncito del héroe. Era su oportunidad de torcer la situación en su provecho, y dado que tenía la intención de cambiar de aires, marcharse con un tanto a su favor en vez de por la puerta pequeña.

~---Si quieres ver a tu hermano tendrás que hacer lo que te diga.

~---¿Dónde le tienes, desgraciado?

Éxeter sonrió. Santurrones. Manipulables con tanta facilidad, sobre todo después de haber sido sometidos a elevadas dosis de cansancio y estrés mental.

~---Tu visor. Ya.

Sky dudó por un momento, y no tardó en darse cuenta de que estaba entre la espada y la pared. La situación que siempre temió había llegado. Se quitó el visor al tiempo que se tapaba los ojos y lo depositó a los pies de Éxeter. Éste no lanzó ningún discurso, ni tampoco se regodeó en su victoria obtenida por medio de la suerte y la astucia. Sólo se limitó a aplastar con el pie el dispositivo y acabar así para siempre con la trayectoria como héroe de James Sky, aun sin siquiera conocer que ése era su nombre real.

~---Para encontrar a tu hermano tendrás que encontrarme de nuevo, ahora sin ayuda de tu aparato. Cuando me localices, sabrás de su destino también.

Dicho lo cual se marchó aprovechando la tremenda situación de vulnerabilidad a la que Sky se había expuesto en ese momento.

Sky se acercó a los restos de su visor, hechos pedazos, y los almacenó como el más preciado de los tesoros, más como fetichismo que por motivos prácticos. Adiós para siempre a su identidad como héroe. La parte más dura comenzaba a partir de ese momento.

Para perseguir a Éxeter tuvo que recurrir a sus recursos como policía, ya lo único que le quedaba para luchar contra el crimen, que no era poco. Empezó a darse cuenta de que si fingía parecer un pringado en los momentos adecuados eso le abría muchas puertas de par en par, y también ganó ciertas dosis de cinismo que exteriorizaba a menudo esbozando una sonrisilla irónica cuando nadie podía verle.

Cuando encontró a Éxeter, sin embargo, ninguna sonrisa se perfilaba en su rostro. En absoluto.

Fue en un callejón muy similar a aquel en que hablaron la vez anterior. Estaba muerto en el suelo, completamente rodeado de gusanos, cucarachas y otros bichos más insólitos. Sky sabía poco de criminología, pero por lo que pudo ver ya había pupas en desarrollo sobre su cuerpo, por lo que su estado de descomposición debía ser muy avanzado. El olor era, además, insoportable.

Con el paso de los años acabó averiguando que el nombre real de Éxeter era Warren Shockman, algo que a esas alturas ya tampoco importaba demasiado. Su hermano no apareció, por supuesto, y Sky se llegó a plantear si Shockman no se habría matado a sí mismo como broma final, indicándole que ése había sido también el destino de Paul. Pero ni eso pudo averiguar, pues para cuando fue a avisar a los otros policías su cadáver había sido retirado, tal vez para exhibirlo ante las mafias como trofeo.

De ese modo Sky decidió que lo mejor que podía hacer era concentrarse en su carrera como policía, algo que por otro lado empezaba a requerir cada vez más su atención. Muchos sujetos poco recomendables estaban empezando a entrar en el cuerpo, y cuando pasó a las órdenes directas del Jefe Brian Wolf el asunto empezó a ir cada vez a peor. Siempre había habido manzanas podridas en las fuerzas del orden, pero a Sky le parecía que era más bien el árbol el que empezaba a estar infectado al completo. Aun con todo no abandonó su puesto de trabajo. Ya pocos polis honrados quedaban como para encima desertar ante aquel panorama. Pero el no cobrar sobornos y no participar en los jueguecitos enfermizos que sus compañeros Fox, Cracker y Warp solían proponer empezó a granjearle más de un problema.

Las cosas tienen que cambiar, pensó. Tal vez esta nueva candidata a presidenta sea lo que Ernépolis necesita. Fue por eso por lo que, como muchos otros ciudadanos, apostó por el cambio en el día de las elecciones.

Desde entonces James Sky nunca dejaría de culparse a sí mismo, una vez más, por haber contribuido, aunque sólo fuera con una mísera papeleta, a poner su granito de arena a la hora de erigir el tiránico reinado de Ellen Gorgon sobre la ciudad.

\section*{El Caído}

De repente las luces de la fábrica se iluminaron.

En las plataformas superiores aparecieron un montón de hombres armados que apuntaban a ambos contendientes. Tanto Reflector como Silenciador pararon y esperaron. Los recién llegados no suponían amenaza para ninguno de los dos.

Ellen Gorgon apareció rodeada de varios de aquellos soldados. Desde donde estaban apenas era poco más que un punto lejano, sin embargo su voz sonaba poderosa en toda la estancia. Parecía, al contrario que en la sala de fiestas, estar en su elemento.

~---Agradezco su colaboración, Reflector. Al parecer la ciudad está a salvo con su presencia.

~---No era necesario que viniera hasta aquí ~---dijo tratando de forzar la voz incluso más de lo habitual. No solía encontrarse con las mismas personas en ambas identidades.

~---Comprenderá que haya hecho lo contrario. Al fin y al cabo tenía que asegurarme de que el plan funcionaba.

Reflector empezó a tener la vaga sensación de que el suelo se hundía bajo sus pies, y que por mucho que supiera volar no podría evitar hundirse a su vez.

~---¿De qué está hablando?

~---Hablo del plan para acabar con usted de una vez por todas\dots\ Capitán Scream.

Aunque el campo de fuerza impedía ver su rostro, sabía que todos los presentes le imaginaban vulnerable y expuesto. Pero aún era un héroe. Aún podía tratar de hacer algo.

~---No sé de qué habla, criatura terrestre. Yo no soy como ustedes.

~---Muy al contrario, Capitán. Es muy humano, aunque físicamente sea más que uno. Por eso hemos podido tenderle esta trampa. Durante mucho tiempo Reflector ha frustrado, sin saberlo, mis intentos por hacerme con esta ciudad, por muchos obstáculos que he puesto en su camino, por muchas veces que se haya enfrentado a mi leal soldado ~---miró a Silenciador~---. De modo que opté por algo distinto. Las elecciones están en juego y no podía dejar a una amenaza como él suelta para discutir mi control, menos aún cuando sea del dominio público.

»Mis sospechas siempre cayeron en alguien que hubiera viajado mucho. Los poderes de Reflector no eran de este mundo, pero bien podían ser de otros. Sin embargo su carácter único lo alejaba de los lugares de habitual comercio y situaba su fuente en algún planeta o bien abandonado, o bien en proceso de formación. Examiné miles de historiales de vuelos espaciales, Capitán Scream. No fue fácil. Había muchos sospechosos, y aunque usted fuera uno de los principales tenía que estar completamente segura. Y lo estuve cuando al fin comprobé que hace muchos años su nave desapareció en una remota región apenas explorada. Encontré un planeta donde había un mineral\dots\ un mineral con efectos parecidos a los poderes de Reflector. Sin embargo fui incapaz de procesarlo. Tal vez si me dice cómo lo hizo le deje vivir.

Reflector sabía que la respuesta no iba a dejar satisfecha a Gorgon. Fue el último habitante del planeta quien, antes de morir, lo hizo para él, llevándose el secreto a la tumba. No podía perder ni un segundo. Era en aquel momento o nunca, mientras Gorgon soltaba su interminable charla acerca de lo magnífica que era y de cómo le había derrotado. Voló lo más rápido que pudo, dejando atrás la mayoría de las balas, rebotando las demás, hasta estar casi a la altura de la que siempre, sin saberlo, había sido su mortal enemiga. Ya casi estaba a su altura y dispuesto a noquearla de un puñetazo cuando, a un metro de distancia, se detuvo. Gorgon llevaba una pistola en su mano alienígena, y con ella apuntaba a Aryn, casi desmayada.

~---Atrás, Scream. Muy lentamente, vuele hacia atrás, o Aryn Life morirá.

Sin otra alternativa, hizo lo que le mandaron. La cosa se ponía cada vez peor.

~---De modo que fue en la fiesta\dots\ la bebida que ella tomó.

~---Todo estaba cuidadosamente calculado, Capitán Scream. ¿Me cree tan estúpida de presentarme aquí expuesta con un montón de soldados armados que no tienen nada que hacer contra usted? No, ellos están aquí sólo para nuestra coartada. Atacaban una de mis factorías, y el causante de hecho era el héroe conocido como Reflector\dots\ ya me deshice de otros héroes antes, y ahora le toca a usted. Descienda y vuelva a su estado normal\dots\ no repetiré las consecuencias de no obedecer.

~---No creerán que ataqué su factoría.

~---Es posible. Pero tampoco me desmentirán.

Incapaz de hacer nada en dicha situación, Reflector hizo lo que le ordenaban. Aterrizó y guardó de nuevo la gema en su bolsillo, cesando la transformación delante de todos los presentes. Aquel momento, comprendió, fue el fin definitivo de Reflector. A partir de aquel instante sólo era John Scream. Sólo un hombre.

~---Suéltela, Gorgon.

~---Como desees.

Gorgon empujó a Aryn fuera de la plataforma, cayendo a toda velocidad a la planta principal. Estaba tan mareada que ni siquiera gritó. Scream sabía que había suficiente altura para que se matara, por lo que trató de correr hacia ella, con todas sus fuerzas acelerar la transformación. Aquel era el momento crucial. Sabía que podía muy fácilmente destruir para siempre la gema por forzarla demasiado, pero la vida de Aryn bien lo merecía.

El campo de fuerza envolvió otra vez su cuerpo, y de correr pasó a volar en cuestión de segundos, elevándose cada vez más. Ya casi podía llegar a su mano, extendiendo la suya desesperado, apenas reparando en los ojos de Aryn, sólo mirando aquella mano que caía como si fuera ella quien estuviera intentando salvarle a él y no al revés\dots

Recibió la descarga justo en aquel momento. Una andanada letal proveniente del arma de Silenciador. Ni el más mínimo ruido. Como una gota de agua desviada por el viento, cayó sin fuerzas de tipo alguno.

Lo último que oyó fue el chasquido del cuello de Aryn al romperse contra el frío suelo de la factoría.

Gorgon bajó, arma en mano, e ignorando el cuerpo de Aryn Life se acercó hacia Scream. Ningún campo de fuerza le protegía. Sólo era John Scream.

~---Está muerto ~---dijo con calma mirando a Silenciador~---. No debías matarlo, idiota. Has agotado el poder de la gema. Ya no me sirve de nada, ni él ni su objeto.

~---¿Qué sugiere que hagamos con él? ~---preguntó lacónicamente Silenciador.

~---Tiradlo a los bajos fondos. A la chica llevadla a los callejones cercanos a la fiesta. Vosotros dos ~---dijo señalando a dos de los guardias~--- vestiros de paisano y fingid que la habéis matado allí en un atraco. Vamos ~---dijo impaciente. Cuando los hombres se fueron Gorgon hizo una llamada~---. Sí, con el Jefe Wolf. Soy yo. Cumpla con su parte. Cuando sus hombres los detengan, ya sabe qué hacer. Sí, mientras huyen. De acuerdo. No me falle.

Ellen Gorgon se dio la vuelta y, satisfecha, miró a sus hombres mientras con la mano alienígena guardaba su arma.

~---Caballeros ~---dijo con voz solemne~--- la era de los héroes ha llegado a su fin.

\endinput
