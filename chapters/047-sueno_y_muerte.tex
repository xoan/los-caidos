Ni siquiera el memento por el compañero perdido pudo ser demasiado largo. No tardaron en llegar nuevos acontecimientos que requirieron su atención, por los que de nuevo no tardaron en mantenerse otra vez alerta. Vigilantes, siempre velando por el bienestar de su mundo oscuro y ceniciento.

\fancyparbreak
Se respiró un indudable alivio en el cuartel de Los Caídos cuando por fin presenciaron el resurgimiento de John Scream como su legítimo y voluntarioso líder, pero aun con todo sabían que su situación personal era, cuanto menos, complicada. Sin una válvula de escape a la que recurrir ante momentos terribles como aquel, Scream era un prisionero de sus propias dotes de liderazgo. Ni siquiera un hogar propio poseía, ¿para qué tenerlo si el tiempo que no pasaba en el Aquerón lo pasaba en Gorgon Enterprises? En ambos poseía rincones apartados en los que descansar, aunque muchas veces durmiera lo justo para que sus neuronas no empezaran a sucumbir una detrás de la otra.

Por otro lado dentro de John Scream, en esa parcela del alma que permanece oculta a los ojos de los observadores externos, su rango de tolerancia estaba cediendo más allá del límite de sus posibilidades. La decepción de Alma Espejo, la persecución de Perséfone, la traición del ejército y aparición de Armor Omega y, finalmente, la muerte de Sam. Demasiadas malas noticias y demasiado seguidas en un mismo intervalo de tiempo, incluso para un sujeto experimentado como él. Por otro lado, muchas de ellas demasiado personales como para simplemente decirse a sí mismo que todo lo que tenía que hacer era mirar hacia delante y nunca, nunca, echar la vista atrás.

Pero las noticias de Sky le habían obligado, de nuevo, a tener que fijar la vista hacia delante y cargar sobre sus hombros con el peso de las batallas pasadas. Asuntos delicados y complicados venían otra vez a situarse en primera línea de la ciudad. Al parecer se había detectado un inusitado aumento de visitantes de los límites del Sistema Solar, concretamente de más allá de Urano. Trabajadores, meros turistas\dots\ el pico de recién llegados era considerable, tanto que había encargado una comisión de investigación para esclarecer a qué podía deberse ese peculiar repunte.

Los primeros datos obtenidos, como reveló Scream a los suyos en el hemiciclo, en la asamblea extraordinaria que convocó nada más marcharse Sky, eran, cuanto menos, intrigantes.

Muchos de los recién llegados que fueron entrevistados, ya fuera de manera oficial ó extraoficial, provenían de la colonia Hidra, que tomaba su nombre de su ubicación, uno de los satélites que orbitaba el cuerpo conocido como Plutón. Al parecer se había producido un radical cambio de gobierno en dicha colonia, aunque en ese sentido apenas quisieron añadir mucho más al respecto, sólo que tenían un nuevo dirigente, que todos ellos se apresuraron a encumbrar a la categoría de poco menos que salvador del pueblo. En consecuencia su situación en términos políticos y diplomáticos era convulsa, y la policía no podía hacer nada por retenerlos pues ante ninguna nueva notificación seguían estando al amparo del antiguo gobierno, con lo que todos sus papeles estaban en regla. Eso, aunque hubiera constancia de que ese gobierno, a efectos prácticos, ya no existía. Peor aún, existía sobre el papel, y mientras eso no cambiara, era en cierto modo un sistema político fantasma.

Scream buscó ciertos datos preliminares sobre Hidra. Era un mundo que gozaba de escaso, por no decir nulo, interés en términos interplanetarios. Abastecedora de Plutón, diminuto tamaño, apenas cien kilómetros de diámetro en promedio. El gobierno anterior era una especie de empresa-estado en la que resultaba sencillo olerse ciertas violaciones flagrantes de derechos humanos y no humanos, pero si ya era difícil investigar esas cosas en el seno de una empresa, más aún de una empresa con identidad política propia. Estaba claro que en semejante lugar el descontento debía de ser más que apreciable para sus habitantes.

Y alguien había utilizado ese descontento para hacerse con el control de la colonia. Alguien cuyas intenciones de legitimarse aún no estaban claras, pero que había puesto sus ojos en la ciudad en términos aún desconocidos.

Scream pensó con calma sobre los datos que poseía. Una suerte de gobernante de nuevo cuño que había llegado al poder, tal vez con relucientes promesas en la mano. Las circunstancias en que lo obtuvo, desconocidas, por lo que no estaba de más considerar que pudieron involucrar violencia y revueltas. Un mundo corporativo que, por lo que los viajeros habían comentado, estaba tratando de reestructurar. Con intereses, además, en Ernépolis.

No tardó en fraguarse una sospecha en su interior. Si lo que pensaba era cierto, eso explicaría además ciertas elecciones que parecían ser en principio arbitrarias. Pero antes de eso, lo mejor sería que se pusieran en marcha cuanto antes. Había un dato que evidenciaba lo importante de aquel asunto, y era que Sky no tardó en recibir como orden expresa del Presidente Scatter cesar toda investigación al respecto de Hidra y sus visitantes. El juego había pasado a esferas más altas. Pero si algo caracterizaba precisamente a Los Caídos era la capacidad para operar en todas las esferas.

Todo miembro de la organización tendría los ojos puestos en cada rincón y esquina de la ciudad, desde los bajos fondos y callejones clásicos para cambiar mercancía hasta los pasillos de los negocios más importantes o las barras de los locales donde los rumores se esparcieran a mayor velocidad.

En particular, dos miembros de Los Caídos obtuvieron información de gran alcance debido a sus situaciones laborales. Información que trajo consigo no poco peligro para sus propias vidas y las de sus compañeros.

\parbreak
La primera historia es una historia acerca de un sueño. De un hombre que tenía altos objetivos, como eran el desarrollo de sus promesas electorales e ideales hasta límites más que insospechados. De metas ambiciosas que podían ponerle entre la espada y la pared.

Ese hombre era el Presidente Scatter, y aquel que obtuvo la información era su ayudante Ellis Saw.

Saw había sido testigo de gran cantidad de propuestas grandilocuentes y descabelladas del que había sido su jefe en funciones desde que resultara elegido por sus conciudadanos. Esas propuestas habían pasado por montar una suerte de concurso público para elegir un defensor de la ciudad, pasando por construir una macroautopista que aún seguía en proceso de finalización o promover el diseño de un módulo espacial con destino al planeta Khorleur. Muchos de esos planes de actuación habían acabado en un sonoro fracaso, pero Saw sabía que si algo caracterizaba a Scatter era su perseverancia, como ocurría con muchos políticos a los que había conocido, incluyendo a Ellen Gorgon en su etapa dictatorial. Por eso estaba seguro de que el Presidente estaba elaborando otra en apariencia descabellada idea, una de esas en las que se lo jugaba todo a doble o nada, tan arriesgada que prefería no consultarla con ninguno de sus asesores, incluyéndole a él, uno de los más cercanos. Pero Saw no tardó en tratar de presionarle, y dado que Scream le dijo que empleara todos los métodos de que disponía, incluso los más radicales, decidió emplear una carta que hasta ese momento no había jugado: chantaje moral.

Se presentó sin llamar en el despacho del Presidente, que estaba en ese momento atendiendo una llamada de su agente de bolsa, y plantó sobre el escritorio su carta de dimisión.

~---Luego te llamo ~---fue todo lo que acertó a decir colgando y bajando el móvil hasta depositarlo a su vez frente al documento que acababa de irrumpir en su planning diario.

~---¿Qué significa esto? ~---preguntó indignado~---. ¿La oposición te ha ofrecido un puesto similar o algo así?

~---De eso nada, señor. Pero no puedo quedarme aquí si no comparte conmigo sus últimas estrategias. Es como si tomara una decisión bursátil sin llamar a su agente, como estaba haciendo en este momento.

~---¿Crees acaso que no lo hago a menudo, Ellis? ~---contestó cortante el Presidente, y trató de moderarse, así como apoderarse de la conversación~---. Pero entiendo tu actitud, por lo que te contaré qué es lo que hemos estado orquestando, un asunto de máxima seguridad para la ciudad. ¿Eso hará que reconsideres este\dots\ desplante impulsivo?

~---Depende de la gravedad del asunto, señor ~---contestó con prudencia Saw, aunque no iba a dejarle mientras pudiera servir como fuente de información para conocer los juegos políticos en los que se decidía el destino de Ernépolis~I.

~---Hace poco se abrió un canal de negociación con un conocido terrorista que amenazó Ernépolis hace ya tiempo. Los acuerdos y negociaciones han sido muy lentos, pero es posible que se llegue a un pacto en breve.

~---¿Se está refiriendo a quien yo creo, señor? Ese hombre es un peligro para todos, nada parece poner freno a sus intentos de castigar a los que ellos considera como criminales de la manera más expeditiva posible.

~---Conozco esa postura, Ellis. Pero ten en cuenta que no estamos en posición de aplicar todas nuestras exigencias. Además de eso, no se le pueden aplicar las leyes de esta ciudad, puesto que no reside en ella en este momento, y ya sabes bien cuán complicados son los asuntos que trascienden atmósferas\dots

~---¿Qué es lo que Hades ha pedido a cambio del cese del terrorismo?

~---Libre circulación de los suyos, reconocimiento de una suerte de embajador en Ernépolis y ser admitido políticamente como colonia independiente. Entre otras cosas.

~---Es un error, señor. No se conformará con eso. No tiene motivos para tener que claudicar. No olvide que él fue quien mató a Alma Espejo ~---recordó Saw, tratando de apelar a sus puntos débiles.

~---¿Crees que acaso no soy consciente de ello? Pero la opinión pública está muy dividida en este asunto. Hay quienes empiezan a considerar sus actos como una necesaria causa mayor. No le faltan seguidores en el Sur, así como en los Túneles. Su actitud de vengador moderno atrae a las masas con facilidad. La ciudad, además, ha visto mucho dolor. Es cierto que su popularidad descendió de manera drástica debido a su involucración en el asunto de Alma Espejo, pero incluso en eso empiezan a circular toda clase de teorías extrañas y contradictorias. El papel que el Caído jugó también en todo ese asunto es una nueva incógnita que añadir. La última vez que él y Hades se enfrentaron eran rivales, ¿y después de eso aliados?

~---Señor, le insisto para que lo piense dos veces.

~---Ya lo he pensado dos y tres veces, Ellis. La decisión está tomada. Sólo queda que contengamos el aliento.

\parbreak
Ellis salió del despacho a toda prisa, no sólo para comunicar a Scream de primera mano los nuevos datos que había conocido, sino también con la necesidad de despejarse y olvidarse de la manera flagrante en que le habían dejado a un lado en decisión tan importante como aquella, en la que incluso era posible que Los Caídos hubieran podido intervenir de alguna manera, aunque sólo fuera en las negociaciones, o claudicando a su vez en beneficio de un bien mayor. Pero habían pasado por encima de él en todos los sentidos. Ni siquiera tenía el Presidente intención de decírselo de manera personalizada, bien podía haberse enterado por los monitores de las calles de no haber sido por el soplo de Sky.

En esa situación estaba, enfadado y contrariado, cuando un hombre joven se plantó en la calle, delante de él. Vestía con una variante extraña de un uniforme espacial, casi una variante de diseño militar, pero claramente customizada para albergar toda clase de cinchos, cinturones y otros aderezos en torso, brazos y piernas. Tenía unos penetrantes ojos saltones, aunque estaba mirando hacia otro lado en el momento en que Saw se topó con él. Sin embargo, al miembro de Los Caídos le quedó claro que estaba esperando a cruzarse en su camino.

Giró la cabeza, apuntando al suelo, y comenzó a hablar. A Saw le resultó desagradable que ni siquiera se dignara a mirarle a la cara.

~---Ellis Saw, ¿verdad? ~---dijo con calma, sin variar la postura del cuello pero cruzando los brazos~---. Considérame tu homólogo en la organización de Hades.

~---¿Quién eres?

~---Mi nombre es Hipnos, y mi cometido es detenerte ~---dijo mirándole fijamente. Sus ojos resultaban aún más desagradables de frente, y además Saw pudo evidenciar que poseía unas tremendas ojeras.

~---Aquí en mitad de la calle no\dots\ podrás\dots

De repente Saw sintió que le fallaban las fuerzas, como si alguien le estuviera extrayendo el alma del cuerpo. Trató de luchar, pero cuanto más se resistía, más acusada era la debilidad que le estaba haciendo perder el conocimiento. Era esa mirada, pensó. Esa mirada que se había clavado como una estaca en mitad de su rostro. Al principio no pasó nada, pero si lo hubiera sabido\dots\ calculó, antes de desmayarse, que apenas cinco segundos era lo que ese desconocido había necesitado para derrotarle.

Cuando perdió al fin la conciencia del todo, Hipnos le cogió justo antes de que se desmayara y le llevó a hombros, como si fuera un amigo borracho al que estuviera acompañando a casa.

~---Da gracias de que tengo órdenes de no dañarte ~---dijo para sí mismo mientras arrastraba con aquel peso muerto que él mismo había generado con el poder de sus ojos.

\parbreak
La segunda historia es una historia acerca de la muerte. De cómo no debe olvidarse que está presente en todo lo que rodea a los seres humanos, y a veces forma una parte esencial de la vida de algunos de ellos. Hombres que se creen genios de la manipulación, perpetradores de planes que involucran a aquellos que les rodean, y no toleran que otros de su misma condición se interpongan en su camino.

Hombres como el Juez Supremo Nitram, antiguo fiscal y embajador, con casi tantos contactos y aliados de alto nivel como años de experiencia llevaba al frente del Tribunal Superior en Ernépolis~I. Hombres que sólo dan la cara si su arrogancia es mayor que su deseo de control, su pretendida visión de justicia. Hombres de fuertes convicciones que se ven a sí mismos como adalides de un bien por el que todo, incluso sus propios principios, puede ser sacrificado llegado el momento adecuado.

Razorclaw no quería tener que verse obligado a obtener información a raíz de uno de los peores enemigos que se habían ganado con el paso del tiempo. No sólo era una cuestión de orgullo, también un asunto de prudencia. Pero Scream había sido tajante al respecto: obtener pistas a costa de lo que fuera. Y eso implicaba tener que tomar decisiones que podían resultar, cuanto menos, arriesgadas y de imprevisibles consecuencias.

~---Pensaba que el Derecho Penal Interplanetario no era su especialidad, letrado ~---dijo Nitram altivo, caminando por los pasillos de los juzgados, esos mismos en los que habló con Starr Miles, y también en los que tuvo su primer encuentro privado con Alma Espejo, otra destacable porción de su pasado a tener en cuenta.

~---Se escuchan rumores, Juez ~---insistió Razorclaw, tratando de apretar las clavijas de su interlocutor, con más prisa de la habitual. Sus aparatos mecánicos no estaban a la vista, y no lo estarían a menos que Razorclaw se mostrara amenazante en algún momento de la conversación~---. Algunos de mis colegas me han dicho que no hace mucho el Presidente en persona vino a hablar con usted, y estoy convencido de que no se desplazaría hasta aquí de no ser porque el asunto a tratar no fuera de la mayor importancia.

~---¿Por qué debería contárselo a usted?

Razorclaw trató de indagar a fondo, rastrear las motivaciones de la oposición del Juez en ese momento. Creía conocerle lo suficientemente bien como para lograrlo.

~---Porque yo lo entenderé, no como esos políticos que creen que la justicia es una herramienta al servicio de sus artimañas electorales.

Nitram se detuvo, como sorprendido por las palabras de su interlocutor, y decidió otorgarle algo más de tiempo que el que le llevaría de los pasillos hacia sus estancias privadas.

~---¿De qué libro de leyes ha sacado eso?

~---De ninguno. Es parte del discurso de Gordon Wave cuando regresó por primera vez del planeta Axcron.

~---Las cosas entonces ya eran muy similares a como son ahora, por lo que veo ~---objetó Nitram, afianzando su posición. Razorclaw pensó que era una cuestión de ahora o nunca, de doble o nada.

~---El Presidente quiere llevar al límite el potencial de las leyes, ¿no es así?

~---Mucho más que eso, en realidad. Quiere alterarlas a su antojo, sin pensar en la verdadera justicia. Me pidió que, llegado el momento, amnistiara a una organización terrorista que amenazaba la ciudad.

Razorclaw pensó cómo la doble moral del Juez podía ser tal que no se incluía a sí mismo en la categoría de amenaza para la ciudad, pero pronto recordó que nadie, nunca, es malo a los ojos de uno mismo.

~---¿Una organización terrorista? Ahora mismo lo primero que me viene a la cabeza son esos disidentes del ejército que atacaron la ciudad no hace demasiado tiempo. O tal vez\dots

El rostro de Razorclaw empalideció cuando comprendió el alcance de la osadía del Presidente Scatter, y no tuvo que decir nada, ni Nitram añadirlo, para que el mensaje estuviera completo. Por otro lado le quedó claro que si el Juez Supremo habló fue porque la invulnerabilidad de su posición le permitía poder despacharse a gusto contra aquellos que opinaba no estaban favoreciendo los intereses urbanos desde su puesto de poder.

~---Ya entiende hasta qué punto el Presidente se cree con potestad para decidir quién debe y no debe estar impune ante las leyes, él cuyo cometido es gobernar, que no comprende ni comprenderá jamás el verdadero significado de la justicia. Y además sugiere ya no sólo a un peligroso criminal, sino a uno que se cree poco menos que un dios, por encima de todos los que le rodean, cuando la realidad es ~---apretó el puño, y Razorclaw pudo ver al enemigo declarado de Los Caídos en ese gesto~--- que no es digno de intentar alcanzar tal estatus divino. Ahora, si me disculpa, debo irme, abogado. No se preocupe por los detalles que no puedo contarle, ni tampoco por mantener el secreto de esta conversación ~---añadió no sin cierta malicia y deseo expreso de que así hiciera~---, al fin y al cabo el Presidente está a punto de revelar a la ciudad su última idea descabellada, si no lo está haciendo en estos momentos.

Dicho lo cual, se limitó a seguir andando por los pasillos del edificio público, dejando a Razorclaw con la urgencia de contactar con el cuartel cuando antes. Salió al patio trasero, donde casi nunca había nadie, con la esperanza de poder sacar el comunicador y así transmitir las noticias de inmediato. Nada más pisar el frío y rocoso jardín artificial rodeado por cuatro paredes, comprobó que estaba solo por completo y tampoco había curiosos atisbando desde las ventanas, y se puso en contacto con el cuartel.

~---Aquí Razorclaw, ¿me recibís?

~---Soy Swart, te escucho.

~---He hablado con el Juez Nitram, y tengo algo importante que comunicar. Por lo que me ha dicho\dots

No acabó la frase. Notó cómo alguien se acercaba desde atrás, alguien que sin duda ya estaba observándole desde hacía tiempo y que tomó la precaución de quedarse al margen cuando estaba entrando en el patio, y entró a tiempo de impedirle hablar con el cuartel, tirando al suelo su comunicador de un golpe para luego pisarlo en el acto con extrema violencia. Razorclaw logró apartarse a tiempo de que le atacaran, aunque su agresor no usó los puños cerrados, sino que mostraba las palmas todo lo extendidas que podía.

El desconocido tenía la misma edad y vestía similar que aquel que había interceptado a Saw salvo por el hecho de llevar una chaqueta sobre el traje y estar menos recargado de accesorios, y si bien su aspecto físico era muy similar, casi como si fueran gemelos pero sin serlo, había optado por lucir una poblada barba y sus ojos no eran tan saltones ni padecían aquellas endémicas ojeras. Miró fijamente a Razorclaw durante un buen rato. Más de cinco segundos, y más de diez, pero Razorclaw seguía en pie, alerta ante su nuevo oponente. Aquel asunto era más serio de lo que se imaginaba, y supuso que no era el único que estaba siendo vigilado de manera individual.

~---Charles Razorclaw, me alegro de conocerte al fin. Tanto he oído hablar de ti, y nunca había tenido el placer de intentar atacarte por la espalda. Veo que tu fama no era exagerada.

~---¿Quién eres? ¿Acaso\dots\ nos conocimos antes? ~---preguntó, intrigado.

~---No, no llegué a tener ese placer. Pero a algunos de vosotros hemos logrado identificaros, aunque esa información ya no valga de mucho ahora que habéis decidido fundiros como las gotas de agua en un mar.

~---Tal vez no seas alguien de los viejos tiempos, pero hablas con la misma pedantería y pomposidad de ellos ~---declaró Razorclaw, insolente, tratando de provocar a su enemigo.

~---No hables así a alguien a quien acabas de conocer, te lo ruego. Menos si está a punto de otorgarte un inmenso honor, un regalo que contarás con envidia a tus aliados. Permíteme hacerte una demostración.

Sacó un pequeño frasco de un bolsillo, que contenía una mariposa gris, la única variedad que existía en Ernépolis. Lo abrió y la cogió de las alas.

~---Ha sido complicado capturarla sin\dots\ pero mejor obsérvala bien.

La mariposa intentaba zafarse de la presa, como Razorclaw pensó que cualquier animal, no sólo un insecto, haría en su lugar, pero aun así algo que no podía explicar le hizo pensar que su manera de moverse era más convulsa de lo esperable en su situación. A los pocos segundos se quedó rígida como si fuera una máquina a la que hubieran quitado la energía, y cayó al suelo como una pluma de pájaro. Ya en el suelo, se quebró igual que una hoja de otoño al ser pisada por un transeúnte cualquiera.

~---Cinco segundos. Cinco segundos me bastan para arrebatar la vida con mis propias manos. Por eso me gané a pulso el sobrenombre de Tánatos. Ahora, déjame mostrarte mi regalo.

Se acercó a toda prisa hacia Razorclaw, más rápido de lo que éste hubiera podido imaginar, y se apartó pálido, con la sangre helada en las venas. ¿Cómo podía pelear contra un enemigo al que no podía tocar? Desesperado, trató de cubrirse las manos con las mangas de su propio traje, algo que no resultaba tan sencillo de efectuar teniendo que moverse al mismo tiempo lejos del alcance de su atacante. En ese momento maldijo su idea de buscar un lugar apartado donde nadie pudiera verle, aunque concluyó que de todos modos algo se habrían sacado de la manga para intentar impedir que avisara a los otros.

Finalmente no tuvo más opción que empezar a contraatacar, pero tenía todas las de perder en ese sentido. Golpe tras golpe su adversario logró esquivarlos, hasta que frenó uno con su propio puño, y en vez de golpear a su vez, como hubiera hecho cualquier otro enemigo, se limitó a plantar su mano sobre el rostro de Razorclaw.

El abogado no sintió dolor en ese momento. Tampoco calor, ni frío. Lo que experimentó era difícil de explicar, como si la vida se le escapara por los poros de la piel. Como si perdiera el aire de los pulmones y la sangre no pasara por su corazón. Pelear era una quimera, una ilusión. Los movimientos espasmódicos de la mariposa, ahora lo entendía, no tenían nada que ver con tratar de liberarse.

Tánatos, por su parte, se limitó a contar. En su cabeza había aprendido a cronometrar a la perfección aquellos segundos y sus fracciones como si fueran horas enteras de tiempo. Cuando llegó el cuarto de ellos soltó la presa, y Razorclaw cayó al suelo, aún vivo pero completamente incapaz ni de mover ni un solo músculo de su cuerpo, que no de escuchar o mirar allá donde sus ojos se habían quedado clavados como dardos en una diana.

~---Ahora conoces de cerca lo que significa el toque de la muerte. Enhorabuena. Mi señor Hades hace que sus más leales lo experimenten, y eso le incluyó a sí mismo y a mi propio hermano, por supuesto. Nuestros enemigos, sin embargo, suelen tener más suerte. Antes de que yo me lleve su ánima mi hermano Hipnos les roba la consciencia. Los desertores y quienes fracasan en su labor le temen más a él que a mí, porque mientras que yo soy la certeza del fin, él es la incertidumbre de no saber si despertarán, algo aún más difícil de asumir. Pero ya basta de hablar ~---terminó cargando a Razorclaw en el hombro, y comprobando que no había nadie en el pasillo, con la intención de ir hacia la solitaria escalera de emergencia~---. Vayámonos de aquí. Cuanto antes acabe mi parte, antes regresaremos. Odio esta ciudad y su acusada carencia de animales y plantas ~---agregó, abriendo la puerta que conducía al tejado del edificio y, por ende, a la libertad.

\endinput
