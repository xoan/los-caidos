Muchas cosas habían cambiado. La ciudad era otra desde la caída de la dictadora. Por supuesto, nuevos problemas surgieron. Problemas graves, de enormes consecuencias. Tan grandes que involucraban mundos enteros al completo.

Pero no eran competencia de ellos. Ellos eran los guardianes de su ciudad, los protectores de su propio mundo de tinieblas.

Y un terrible peligro estaba en proceso de amenazar todo aquello por lo que tanto habían luchado\dots

\fancyparbreak
La lluvia de ceniza caía con la misma habitual parsimonia con que llevaba haciéndolo desde varias semanas atrás. Hacía años que no se recordaba una situación climática tan adversa en Ernépolis~I.

De hecho, el ambiente en la ciudad estaba enrarecido en más sentidos aparte del meramente obvio.

La situación se había puesto tensa en varios de los emplazamientos más cercanos a la Tierra. Independencia, clamaban algunos. Insurrección, decían otros. Daba igual las palabras que se empleara, el resultado era el mismo en ambos casos. Muerte en las calles de mundos a medio formar, que fue sustituida por muerte en cielos oscuros y de pobre atmósfera, y finalmente muerte en el vacío cósmico y silencioso.

La Guerra de las Ocho Colonias no había llegado a Ernépolis~I. Posiblemente nunca lo haría. Al menos las batallas. Pero sus efectos se notaron, y mucho. En más de un sentido, enriqueció a la ciudad aun a su propio pesar. No obstante, la primera de las Ernépolis, y una de las primeras Polis terrestres, había sido concebida en sus inicios como astillero interestelar, y esa era precisamente la función que estaba llevando a cabo desde que las contiendas empezaron a invadir la frontera interestelar del Sistema Solar. Hubo grandes presiones económicas y de los poderes fácticos para que comenzara la producción masiva de prototipos de naves interestelares en las factorías de la periferia. Además de eso, en términos estratégicos y defensivos Ernépolis~I era un emplazamiento perfecto para defender la posición en caso de emergencia y permitir el aterrizaje forzoso de cientos de vehículos monoplaza que hubieran tenido problemas a la hora de enfrentarse a las lentas, pero resistentes, naves de los colonos rebeldes.

A John Scream no le importaba demasiado todo lo relativo a la guerra entre la Tierra y sus colonias desobedientes. Para él tanto un bando como el otro luchaban movidos por un deseo común: dominación. Sabía, además, que había otras posturas, otras alternativas. Pero dichas opiniones nada lograban frente al poder devastador de las armas.

Miró a la Nube, el techo colosal de una urbe poseedora de la eterna oscuridad. Las explosiones de las repetitivas escaramuzas se reflejaban en los cúmulos de polución con una macabra belleza poética que hubiera hecho las delicias de cualquier pintor impresionista. Pero Scream no era pintor, y el arte no estaba entre sus mayores intereses.

Al menos, el arte separado de la ciencia, pensó mirando el prototipo, casi acabado, del diseño de su último vehículo monoplaza.

Desde que se había convertido en el líder de Los Caídos sólo encontraba placer y descanso en el diseño de naves espaciales. Fue por eso que cuando Razorclaw sugirió que compraran Gorgon Enterprises para usarlo como tapadera en el exterior no sólo no se opuso sino que apoyó la idea hasta sus últimas consecuencias.

Una idea perfecta, por otro lado. ¿Quién sospecharía que aquel ser misterioso, que parecía abominar de la antigua presidenta y todo lo que la rodeaba, controlaba la que antaño había sido su propia empresa?

En principio, la idea había sido reducir la producción al mínimo. Sólo como fachada, aparentar ser un negocio normal en el que desarrollar clandestinamente todos los aparatos que permitían mantener el engaño de los múltiples hombres que fingen ser uno solo. Pero el propio Scream acabó por ponerse al frente de un departamento completo de aerodinámica, para fabricar mejores trajes y plantear la posibilidad de diseñar vehículos, llegada la necesidad. Y cuando quiso darse cuenta, estaba proponiendo prototipos, realizando toda clase de bocetos, y creando toda clase de líneas comerciales como la línea \emph{Ares}, de uso doméstico, o la línea \emph{Errante}, para exploraciones científicas.

Scream dejó los planos sobre la mesa y se dio cuenta, con una amarga sonrisa, de que cada vez su rutina diaria se parecía más a la de su antigua enemiga. Y se planteó si, además de haberle robado sus poderes, su identidad y su amada, no le estaba robando también su alma.

~---Si no fuera porque te conozco demasiado bien diría que son pensamientos sombríos los que asoman por su cabeza.

Scream se giró justo a tiempo de ver a Sky entrar en la amplia sala subterránea, proveniente de los oscuros pasadizos que llevaban, tras un largo laberinto, a una de las muchas entradas al Aquerón.

~---Así es, James. De vez en cuando me da por recordar el pasado.

~---Ya sabes que nada bueno puede salir de eso, salvo aprender de los errores cometidos. Y creo que tu aprendizaje ya está completo y ha sido tortuoso.

~---¿Cómo están las cosas entre tus hombres?

~---Inquietos. Muchos de ellos están pensando enrolarse en la guerra y marcharse a librar batallas lejanas. Son hombres de acción, ya lo sabes.

~---¿Y qué hay de tus\dots\ otros hombres?

En ese momento fue Sky el que se permitió una mirada sombría. Pero Scream tuvo que admitirse a sí mismo que no le conocía lo bastante como para poder interpretarla.

~---La guerra ha traído penurias y miseria a muchas partes de la ciudad. Muchas rutas comerciales se han cerrado. A veces tenemos que meter el miedo en el cuerpo a criminales que no han elegido serlo, John. Y eso es algo duro de asumir, por mucho que sea nuestro trabajo.

~---Son tiempos complicados, en efecto ~---contestó Scream mirando hacia el infinito.

~---¿Qué hay del diseño especial que estabas llevando a cabo? ~---dijo Sky señalando los planos que Scream tenía sobre la mesa, sacándole de su ensoñación.

~---He hecho algunos avances al respecto ~---contestó Scream apretando un botón disimulado bajo la mesa. El panel de uno de los hangares se levantó por completo, lentamente, y dejó ver a James Sky un prototipo de nave como no había visto antes. La horizontalidad predominaba de manera contundente, y poseía multitud de detalles y minúsculos cañones repartidos a distintas alturas y profundidades. Su parte delantera albergaba gran cantidad de respiraderos, y además de unas turbinas traseras poseía otro juego delantero y un último par lateral. Todo ello, unido a su color oscuro, provocaba una sensación, cuanto menos, amenazadora.

~---Siempre eché en falta algo de potencia extra en mis viajes a planetas lejanos ~---comentó Scream, orgulloso, palpando aquellos motores adicionales.

~---¿Qué especificaciones tiene?

~---Puede cargarse solarmente además de con combustible tradicional, gracias a estos paneles escondidos ~---dijo señalando a los respiraderos~---. Su color cambia según la intensidad de la luz, variando entre tonos rojizos y purpúreos. Posee varios cañones semiautomáticos y un emisor de energía, para fines de almacenaje y acoplamiento de armas ondulatorias.

~---Es impresionante. Sé que te gusta diseñar sin que nadie vea el resultado, pero esto es titánico.

~---Y esperemos que nunca haya que usarlo ~---añadió con aflicción Scream~---. Pensé en él para defender específicamente el espacio aéreo de Ernépolis~I.

~---Pero podrá salir en órbita, imagino.

~---Imaginas bien.

~---¿Lo has probado?

~---Lo hemos testeado virtualmente. Además, es arriesgado. Si le llegara al alto mando militar la noticia de que tenemos oculto un modelo bélico podríamos tener problemas. Empiezan a no creerse nuestras excusas de que sólo estamos especializados en modelos de transporte.

~---Aunque pudieras no lo probarías, ¿verdad?

~---¿Qué quieres decir?

~---No nos engañemos, John. Tú eras piloto espacial. De los mejores de la galaxia. Ponerte a los mandos de una nave te traería recuerdos de una vida que ya no volverá. Especialmente de esa nave ~---dijo señalando al centro mismo de la oscuridad.

Ninguno de los dos hombres podía mirar a través de la penumbra de la inmensa sala, pero sabían que allí al fondo había una compuerta, y tras esa compuerta, un hangar que albergaba un modelo muy especial. Allí estaba la \emph{Trigger}, la nave de Scream, la misma con la que se estrelló en un planeta perdido tanto tiempo atrás, allí donde consiguió la gema que le convirtió en Reflector. La misma que había surcado el Cosmos, que había dejado atrás púlsares, contrabandistas, cuásares, renegados, agujeros negros y toda clase de peligros interestelares.

~---En efecto, lo admito. No he vuelto a pilotar la \emph{Trigger} desde entonces. Fue una sorpresa para mí, cuando compramos Gorgon Enterprises, descubrir que estaba en su poder, y la conservo sólo para duplicar sus mejores cualidades en los prototipos que sacamos al mercado.

Aunque nunca habrá otra nave como esa en toda la historia de la humanidad, pensó para sí mismo.

~---Sin embargo ~---continuó~---, he pensado muchas veces en destruirla.

~---¿Por qué motivo? No hay peligro, ya nadie recuerda a John Scream y puedes rehacer tu vida como prefieras. Todo tu mundo fue destruido por Ellen Gorgon. Si hay alguien que puede permitirse conservar un vínculo con su origen, ese eres tú.

~---No puedes imaginar todo lo que he pasado con esa nave. Me ha salvado la vida en incontables ocasiones, ha estado a mi lado en muchos momentos cruciales de mi existencia pasada. Aun así, llegará el momento en que deba ser destruida. Hasta he implantado un mecanismo de autodestrucción en su interior.

~---¿Por qué has hecho eso?

~---Recuerda las lecciones de Miles, amigo. Nosotros no somos hombres. Debemos aparentar ser más que eso. Y del mismo modo que desaparecemos en cenizas si nos matan, todo lo que nos rodea debe hacerlo también. Sería peligroso que alguien pudiera relacionar alguna vez a John Scream con Los Caídos. Por eso he implantado el mecanismo.

~---Entonces, John, destrúyela sin más. Desmantélala pieza a pieza.

~---No puedo.

~---¿Por qué? ~---preguntó Sky, intrigado.

~---Aún conserva el aroma de Aryn cada vez que abro la cabina ~---respondió divisando, a través de las reducidas escotillas, el cielo gris y cargado de explosiones escarlata.

\parbreak
Sky no tardó en marcharse del complejo por la puerta principal, en vez de usar los complicados pasadizos subterráneos para regresar al Aquerón. Tenía ganas de dar un paseo, reflexionar. La guerra suele convertir a los hombres en filósofos tan a menudo como los suele convertir también en monstruos.

A pesar de que había cogido el día libre sus pasos no tardaron en llevarle de vuelta a la comisaría. Una vez en las inmediaciones se quedó mirando el callejón donde por primera vez realizó una operación como miembro de Los Caídos, aunque no llevara el atuendo y sólo fuera un actor secundario. Aún recordaba la cara de Fox cuando vio aquella silueta salir de entre las llamas, asustado como si estuviera frente al mismísimo Diablo.

Mucha ceniza había llovido desde entonces, y las cosas habían mejorado para él y para la ciudad. Había ascendido a Jefe de Policía, y el cuerpo había sido depurado. Con una fuerza del orden en la que sí se podía confiar, se empezó a encarcelar a muchos de los criminales que estaban prisioneros en el Aquerón, salvo a los más extraños y peligrosos, y dejó de tener poco a poco la impresión de estar ejerciendo la justicia por su cuenta. A partir de ese momento se convirtió en pieza fundamental de la organización, facilitando la labor del grupo, aunque eso le reportaba de vez en cuando problemas políticos de toda clase, sobre todo con el presidente.

Aun así, todo aquello, sin duda, era infinitamente mejor que tener como compañeros a unos psicópatas que disfrutaban prendiendo fuego a vagabundos.

¿Entonces por qué se sentía como si estuviera atrapado entre dos mundos, sin pertenecer del todo a ninguno de los dos?

Ni nómada ni sedentario. A la deriva entre dos aguas.

Con el ruido de las explosiones de fondo, Sky se detuvo frente a la puerta de la comisaría. Agitó la cabeza y la ceniza cayó de su pelo como nieve suave y oscura, perteneciente a un hipotético mundo monocromo y de colores inversos.

Se disponía a entrar por la puerta principal, directo hacia su despacho, cuando chocó con una mujer que salía justo en ese momento, vestida de traje y con actitud airada. Un montón de papeles se desparramaron por el suelo.

~---Lo lamento, déjeme ayudarla ~---dijo mirando a su aún silenciosa interlocutora. Fue en ese momento cuando vio en el suelo la boquilla de un cigarrillo, junto a un pitillo a medio consumir, y maldijo su suerte. Aquella mujer se trataba de Emma Blades, abogada por cuenta libre y empeñada en empapelar a la policía de Ernépolis~I por incompetente y dejar que una silueta con sombrero les hiciera el trabajo sucio.

~---Vaya, Jefe Sky, me alegro de verle, y no es una frase dicha a la ligera. Salía enfadada puesto que venía a hablar con usted y me habían dicho que estaba\dots\ ilocalizable.

~---De hecho no estoy de servicio en estos momentos, señorita Blades ~---contestó Sky, cortante.

~---Sin embargo se disponía a entrar a la comisaría, a juzgar por su trayectoria. Eso me hace pensar que de todos modos pensaba dedicarse a alguna tarea pendiente, tal vez administrativa. ¿No cree que sea mucho menos aburrido dedicarme unos minutos de su tiempo?

~---No serán unos minutos, y lo sabe bien.

Blades torció el labio ligeramente hacia abajo, en una fracción de segundo apenas perceptible. Si hubiera sido un criminal el autor del gesto, Sky le hubiera inmovilizado en ese mismo momento.

~---Sólo le pido que conteste a unas preguntas. Le invitaré a un café si con eso compenso las molestias que pueda causarle ~---dijo recogiendo la boquilla del suelo, limpiándola y poniendo un nuevo cigarro sobre ella.

~---Supongo que no puedo negarme.

~---Puede, sin duda. Pero no sería muy caballeroso por su parte.

Sky resopló por lo bajo. No tenía mucho interés en hablar con aquella abogada buscaproblemas. Aunque, por otro lado, pensó que al menos había dejado a un lado las tribulaciones que le venían acosando desde que había dejado atrás la factoría.

~---¿Y bien? ¿Qué dice?

~---Conozco un bar a unas cuantas calles de aquí.

~---Espero que no sea un bar de polis ~---comentó Blades dando una calada a su cigarrillo.

~---No tendrá esa suerte. Hoy sólo podrá interrogar a uno de ellos ~---acabó Sky, subiéndose el cuello del abrigo y avanzando con las manos en los bolsillos.

\parbreak
Nada más llegar Blades insistió en que, ya que afuera hacía tan mal tiempo, más que un café mejor podían tomar una copa para entrar en calor. Sky aceptó sin reservas. El alcohol no hacía demasiada merma en su organismo, por no decir ninguna. Era una de las múltiples drogas que había aprendido a resistir durante su entrenamiento como miembro de Los Caídos.

Ser poli en Ernépolis~I desde los tiempos de Ellen Gorgon también le había otorgado una dosis de aguante extra, claro.

Blades no tenía esa misma capacidad de aguante, o al menos eso era lo que parecía a simple vista. Pero Sky se anduvo con ojo. Al fin y al cabo, ella no tenía nada que ocultar, y de todos modos iba a soltar la lengua todo lo que hiciera falta.

Se sorprendió, con todo, cuando empezó a hablar de sí misma, de sus gustos y aficiones, tal vez para ganarse la confianza de él, tal vez no, pero el caso era que había conseguido, sin duda, llamar su atención. Hacía mucho, mucho tiempo, que Sky no tenía con nadie una conversación que pudiera calificar de\dots\ antiprofesional. Simplemente hablar por hablar, por dejar divagar la mente. Había olvidado lo agradable que podía ser eso.

Hasta que Blades empezó a querer tirarle de la lengua y mandó a la mierda aquel momento que hasta ese instante había tenido algo de mágico, incluso en los detalles incómodos y no deseados.

~---Aún me sigue sorprendiendo, Jefe Sky, cómo pudo mantener a salvo su integridad en el periodo en que trabajó para Brian Wolf.

~---Llegué a ser testigo en el proceso que se llevó luego contra su cúpula.

~---Lo recuerdo, y aunque fue inicialmente imputado por ser parte de su fuerza policial, se le soltó al no poder acusársele de ningún cargo reconocible.

Aquello empezaba a resultar incómodo. Él nunca había cometido ningún delito estando a las órdenes del Jefe Wolf, aunque estuvo dispuesto a cargar con la culpa con tal de que no sospecharan que en realidad estaba formando parte al mismo tiempo de otro grupo de vigilantes que sí estaban velando por la ciudad.

~---De hecho, Jefe Sky ~---continuó Blades, torciendo ligeramente la cabeza~--- siempre me ha resultado curioso que el Caído, ya sabe, ese espectro de la gabardina y el sombrero, nunca haya supuesto una especial molestia en su trabajo de mantener en orden la ciudad.

\rquoti Me atrevería a decir, incluso, corríjame si me equivoco, pero me atrevería a decir que lo tolera. ¿No es un poco inapropiado que permita que un justiciero esté haciendo el trabajo por el que a usted y a sus hombres les pagan los ciudadanos?

Sky la miró fijamente. El Caído. Así era como la prensa había empezado a llamar al ser que habían creado, no por casualidad, ni mucho menos. Fue idea del propio Scream acuñar el nombre a través de uno de los miembros que trabajaba en ese sector. Pocas ironías eran más dulces que mostrar la verdad de manera velada a los ojos de todos sin que pudieran llegar a distinguirla.

~---Ese justiciero, como usted lo denomina, es un criminal. Si ataca a otros es porque desea imponer su propia ley en las calles.

Lo cual, de hecho, no se alejaba demasiado de la realidad, con ciertos matices benevolentes.

~---Aun así, he estado compilando declaraciones suyas, ruedas de prensa, y en ninguna ha condenado públicamente a ese criminal, como le acaba de llamar. Sólo se limita a decir que ha huido y poco más. De hecho, pasando el vídeo a cámara lenta, he notado que muchas veces hasta esboza una leve, imperceptible sonrisa. ¿A qué se debe?

~---¿Qué quiere decir?

~---Vamos, no se haga el tonto ~---Blades apuró el vaso de vodka con limón y le dio otra calada al cigarrillo, sosteniendo la boquilla con dos dedos~---. Creo que en realidad\dots

La frase se quedó a medias en el momento en que un militar se acercó a la mesa de ambos. Apenas se tenía en pie y bastaron un par de frases para que quedara claro que toda la sangre estaba en ese momento en una sola parte del cuerpo.

~---Maldita sea ~---dijo Blades levantándose, apagando el cigarrillo en el cenicero con desprecio~---. ¿Te importaría largarte?

~---Te he dicho que si querías que yo te\dots

~---Escuchado perfectamente tu romántica proposición, y no me interesa. Ahora esfúmate. Estaba en medio de una conversación más interesante.

Fue en ese momento cuando Sky se dio cuenta de que Blades no estaba borracha, sino fingiendo. Debía de saberse todos los trucos sucios y unos cuantos más. Aun así, no pudo evitar echarla un cable.

Al fin y al cabo, era un caballero.

~---Que te he dicho que\dots

~---La señorita te ha dicho que no le interesa ~---intercedió. Blades se sorprendió de la reacción de su compañero de mesa.

~---¿Y quién lo dice?

~---Lo dice esto ~---dijo enseñando su placa, única en toda la ciudad, aunque el soldado, obviamente, no reparó en ese detalle.

Hubo un momento de silencio. El soldado se quedó callado, como si no supiera si darle un esforzado puñetazo o emitir alguna clase de insulto soez y poco original.

~---¿Y bien? ¿Qué va a ser, cabo? ~---insistió Sky.

El soldado se marchó por donde había venido con cara de pocos amigos, se volvió a juntar con los suyos y siguió berreando y molestando a los de las mesas contiguas.

~---Pensaba que no estaba de servicio ~---dijo Blades con cierto tono de agradecimiento sincero.

~---Y yo pensaba que no caería tan bajo como para fingir estar como una cuba y así sonsacarme información.

Si hubo algún amago de momento emotivo, se arruinó por completo después de esa lapidaria declaración.

~---La culpa es suya. Obviamente oculta algo, hechos importantes que aclararían muchos puntos oscuros en el proceso que sirvió para cesar del cargo a Brian Wolf y muchos de sus agentes.

~---¿Qué es lo que busca, la verdad?

~---Sólo eso ~---declaró Blades, echándose para atrás y cruzándose de brazos.

~---La verdad es que el cuerpo de policía estaba podrido hasta los mismos cimientos, esos cimientos cayeron, y de los rescoldos salimos algunos agentes íntegros. Tuve la suerte de ser ascendido, y eso es todo.

~---Es todo. Claro. Solucionado. Muchas gracias, es la información que llevaba tanto tiempo esperando ~---dijo con tono de amarga ironía~---. Le diré lo que pienso, y que antes no he podido decir. Lo que creo es que en realidad usted\dots

Pero Sky ni siquiera escuchó terminar la frase. Porque por encima de la música de fondo, del ruido amortiguado de otras mesas, escuchó algo que le resultaba vagamente familiar. Un sonido ajeno a ese entorno. Un pitido lejano, que llegó incluso a pensar que estaba dentro de su cabeza pues nadie más lo notaba.

Pero no tardó en sonar cada vez más alto, y varias personas más se dieron cuenta de su presencia. Justo en ese momento se dio cuenta de qué era ese ruido y dónde lo había escuchado.

Concretamente en Gorgon Enterprises, cuando John Scream probaba las turbinas de los prototipos a máxima potencia.

Se levantó como un resorte y sacó de nuevo la placa.

~---¡Policía! ¡Desalojen la sala de manera ordenada!

Sky sabía que nadie haría caso de la última parte de la segunda frase, por lo que indicó a Blades que saliera con el resto de la gente en lo que él se aseguraba que no quedaba nadie.

Todo el mundo se había marchado, personal incluido. Ya había cumplido con su deber. Ya era un héroe por encima de lo que pudiera pasar.

Era el momento de salir cagando leches de aquel lugar.

Corrió como una auténtica exhalación, pero dado que venía del fondo del local, el impacto de la nave le pilló aún dentro, cuando estaba a punto de alcanzar la salida con los dedos. La onda expansiva del choque le lanzó contra la puerta y después cayó de bruces al suelo.

Luchando por no perder el conocimiento, James Sky, Jefe de Policía de Ernépolis~I y miembro activo de Los Caídos, pudo distinguir de manera borrosa la enorme nave que se había estrellado contra la parte baja del edificio en el que se encontraba, y las llamas que empezaban a cubrirlo todo.

Después de eso vio la silueta.

Más de dos metros de altura, llena de pequeños picos y aristas puntiagudas. Una visual única y blanca, refulgiendo contra el naciente incendio, estrías verticales en la boca. Quieta, de pie, con los puños cerrados, grandes como mazas.

De repente recordó aquel primer día en el callejón, y se planteó si lo que sentía en ese preciso instante sería el mismo tipo de miedo que ellos provocaban en sus enemigos.

Después de eso, se desmayó al fin.

\endinput
