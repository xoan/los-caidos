\begin{prev}
    Todo ha terminado para John Scream, más conocido como Reflector. Su novia ha sido asesinada, su identidad secreta desvelada, y sus poderes perdidos para siempre. Con él, termina una era en la podrida ciudad de Ernépolis\dots
\end{prev}

\noindent{}El fin. De todo lo que alguna vez fue su vida, su aliento. Esperanzas desaparecidas. Aquel que debió ser un defensor, indefenso. Y sin embargo, tal vez un nuevo comienzo estaba esperando para darse a conocer\dots\\

\noindent{}Silenciosamente, un vehículo deslizante llegó a la zona sur de Ernépolis~I, una de las más deprimidas de la ciudad, con la intención de librarse de una carga que debía desaparecer. Se detuvo frente a un callejón mientras su también silencioso ocupante descendía y abría la compuerta trasera. Miró al cielo y maldijo por lo bajo, pues se preparaba una lluvia de ceniza. Deseoso de que no le atrapara en plena faena, cogió el cuerpo y lo llevó al fondo del callejón oscuro. Con un poco de suerte, pensó, nunca lo encontrarían, aunque sabía que Gorgon no se sentiría satisfecha con eso. Ella querría que lo encontraran. Él no. Era la muerte que le correspondía a su enemigo. En nada heroica, en nada épica. En un callejón. Perdido, abandonado de todos. Sin identidad. Sin poderes, sin nada que acreditase lo que había hecho por los demás. Anónimo. Despreciado, incluso rechazado.

Le miró un momento fugazmente. Aquel era el gran Reflector, el que había frustrado sus planes siempre, incluso antes de que tuviera que convertirse en soldado de Ellen Gorgon. Un muñeco roto. Del mismo modo que tarde o temprano él acabaría en otro callejón oscuro, igualmente olvidado, sus grandes peleas polvo y ceniza. Una parte de él se alegraba de haber acabado con su enemigo. Otra deseaba el retorno de la danza sin fin. Reconcilió ambas partes y subió al vehículo, pero antes, dedicó un momento a mirar su arma. Recordó cuando tuvo que matar a aquel que la construyó para él. Para que fuera única, irrepetible. Lo mismo pasaba con la marioneta sin hilos que acababa de dejar en el callejón. Se marchó reflexionando que ya habría otros Reflectores en el mundo.

O tal vez no.

Finalmente la ceniza empezó a caer. Con lentitud, como si estuviera siendo esparcida desde los balcones, con más fuerza al final. Sin embargo parecía que los desagües funcionaban bien. Aquella noche el nivel no subiría de un centímetro. Cuando se atascaban, la capa de ceniza podía ser de diez. Un motivo por el que la mayoría de los ciudadanos que podían permitírselo no vivían a nivel de calle.

La gema, que aún permanecía en el bolsillo de John Scream, brilló. Con un fulgor que no había tenido nunca antes y que no volvería a tener. En un último esfuerzo reunió todo su poder para, con mucho tiempo y paciencia, curar las heridas mortales de su portador. Sin testigos, sin nadie que pudiera apreciar el milagro, realizó su tarea y su luz se extinguió definitivamente para no volver a brillar jamás. Reflector había dado su vida al salvar a John Scream, la última víctima inocente.

Scream movió un dedo. Con mucha dificultad, como si no fuera suyo. Apenas un palmo. Nada perceptible ni por una rata. Estaba vivo. No debía estarlo, pero lo estaba. No era el final. No al menos de momento. Estuvo mucho, muchísimo tiempo tumbado sin poder recuperar la posición vertical. La gema le había salvado con su último aliento. No lo había visto, pero lo sabía. Se acabó.

Sus ideales, sus creencias, sueños rotos. Derrotado. El fin de Reflector. El fin de John Scream tal y como se le conocía. Debía huir. Esconderse de sus enemigos. Ya nada podía hacer contra ellos más que observar desde la retaguardia. Los héroes habían perdido. No sólo él, sino también los otros que Gorgon decía haber aniquilado. La creía capaz de ello. Parecía tener mucho tiempo, mucha paciencia, y sobre todo recursos y voluntad para utilizarlos. Empezó a sospechar que el atentado que le costó la mano en realidad nunca fue tal.

Como un símbolo de la vida que perdió, contempló la gema apagada y sin brillo. Los rescoldos de un fuego que no volvería a arder. Se incorporó y corrió a refugiarse de aquella mugre que caía del cielo.

A partir de entonces las calles se convirtieron en su hogar. Sabía que no podía volver a su anterior vida. Aun sin Aryn, aun sin Reflector, no podía volver a ser John Scream. No podía volver a pilotar. Si es que quería sobrevivir, tenía que convertirse en alguien completamente distinto de quien fue. Un vagabundo era una opción tan buena como cualquier otra. Salvo la propia Gorgon, dudaba que algún otro de los que habían presenciado su amarga derrota recordara su rostro. Por si acaso se dejó crecer barba, aunque suponía que su aspecto sucio y su ropa hecha jirones bastarían para hacer el resto.

Él, que había surcado los cielos, se convirtió de repente en otra de las motas imperceptibles que desde arriba apenas alcanzaba a divisar. En cierto modo lo prefirió así. Desaparecer. No era tan malo, y la opinión pública siempre le recordaría como un héroe. Los titulares así los remarcaron. \textsc{¿Dónde está Reflector?} No se asoció la desaparición de Reflector con la de John Scream, al fin y al cabo Reflector era una figura de gran relevancia, y John Scream sólo sería una reseña al final de la última holopágina, y eso si alguien llegaba a echarle realmente de menos, dado que al ser piloto no le eran desconocidas las prolongadas ausencias.

Ni siquiera se le relacionó con la muerte de Aryn Life. Scream esperaba malevolencia por parte de su enemiga, la ironía de cargar con el crimen a aquel que intentó salvarla, pero no fue así. Fue lista. No quiso llamar la atención. Con el tiempo todo se olvidó, la gente empezó a dejar de preguntarse dónde estaba su héroe, y el asunto fue dejado de lado. Simplemente pasó, como las rachas de viento.

Pero la ciudad necesitaba gente como Reflector. John Scream lo sabía. Con su ausencia, Ellen Gorgon se convirtió en la dueña del crimen organizado. Usando sus influencias controló media ciudad, y una vez lo consiguió ganó las elecciones, convirtiéndose también en dueña de la otra mitad. La corrupción inundó Ernépolis~I. Pocos eran los policías que desafiaban las normas, y la mayor parte de ellos acababa formando parte de las naves espaciales al ser echados sus cuerpos en los tanques de fundición de las factorías de Gorgon. Algunos héroes surgieron, pero tan pronto como aparecían volvían a desaparecer, bien chantajeados, bien eliminados por el eficiente brazo de Gorgon y nuevo colaborador de la ley, el reformado Silenciador, pues su deuda con la ley había sido pagada con las influencias de Gorgon.

Los jueces, la ley, las factorías, el gobierno. El ascenso de Gorgon era imparable. Sólo había que mirar al cielo. La Nube era mucho más densa y oscura. Lo que en su día fue gris pero de colores vivos se convirtió en negrura, una ciudad donde los extranjeros no querían parar, donde la noche duraba veinticuatro horas. Una ciudad que cada vez exportaba más, y por tanto donde nadie quería meter la narices. Una ciudad que empezó a prescindir del trámite de las elecciones.

Retazos de todas estas noticias llegaban a los oídos de John Scream, que no hacía más que seguir mirando en qué se había convertido la ciudad que trató de defender con todos sus esfuerzos. La justicia, la venganza, el poder perdido, todo aquello bullía por sus venas, entremezclándose, clamando por salir. Pero John Scream no lo dejaba salir. Sólo era un hombre, pensaba. Nada podía hacer. Sólo observar\dots{} y lamentarse en silencio.

Y así pasaron los años. Y Ernépolis~I se hizo más grande, más poderosa, más rica, pero no para todos por igual. De cara al mundo era una gran exportadora. De cara a sí misma era un vertedero de hombres. Un reino de polución y ceniza. Donde no hacía falta escarbar para encontrar la podredumbre de la raza humana.

Donde sin embargo era necesario hacerlo si se quería encontrar a los que en su día fueron un ejemplo para los demás.\\

\noindent{}Una noche, tiempo después, Scream estaba en un callejón, en plena lluvia gris, buscando comida, cuando vio al fondo del mismo, recortada por la luz de los vehículos deslizantes, una silueta de un hombre con gabardina y sombrero. No parecía humano, y sin embargo nada indicaba lo contrario. Apartó la mirada un momento y desapareció. Nada. Como si nunca hubiera estado. Pero estaba. No habían pasado tantos años como para haberse atrofiado su instinto. A veces deseaba que fuera así, pero no había visto, al menos aún, su deseo cumplido.

\emph{Tú eres él.}

La voz se propagó por las reverberantes paredes del callejón. Scream se olvidó de la comida y trató de escudriñar sin éxito el lugar. La oscuridad y las sombras eran aliadas de aquel que le acosaba.

---Descúbrete ---dijo sin más.

\emph{No trabajo para ella, si es eso lo que te preocupa. Sé por lo que has pasado, John. Sé lo que es dejarse llevar por los demonios. Pero tú los has dominado. Contenido, al menos.}

---¿Quién eres y cómo sabes mi nombre?

La silueta apareció como si no se hubiera movido y se acercó a Scream lentamente.

---Sé tu nombre porque te he estado observando mucho tiempo\dots{} Reflector.

Scream se acercó a la silueta y vio a su interlocutor. Bajo la gabardina se ocultaba un rostro anciano pero vigoroso, el semblante de alguien que parecía haber acumulado gran experiencia en batallas pasadas. Scream se sintió como si estuviera frente a una versión distorsionada de sí mismo de haber sido otras las circunstancias.

---Reflector murió hace años. Sólo queda John Scream.

---Lo sé, John. Del mismo modo que en mi caso sólo queda Starr Miles.

Scream le miró y se fijó en él. Como si de repente una parte del pasado volviera para atormentarle.

---Tú\dots{} yo te conocí. Tú fuiste como yo. Tú tenías otra identidad. Pero no soy capaz de identificar quién eras.

---Eso ya no es importante, John. Como tú, yo también perdí esa identidad. Siempre supiste que tus poderes sólo te traerían desdicha. Quizás incluso intuiste un trágico final, para culminar con la muerte, solo, perdido y sin nada a lo que aferrarte. Pero no imaginaste que hubiera un después. Los comics que leías cuando eras pequeño no hablaban de eso. Esa es la infortunada existencia que nos tocó vivir. Tal vez hubiera sido mejor que hubiéramos muerto, como le pasó a otros muchos héroes. Pero somos supervivientes, y tenemos una obligación. No sólo para con nosotros mismos, también con la ciudad que nos prometimos en silencio defender.

---Pero sólo somos hombres. Nada queda ya de lo que fuimos.

---Un hombre ---sentenció Miles--- puede ser suficiente. Fuiste derrotado por alguien que no tenía poder alguno, que no era más que humana. Los poderes no son nada sin voluntad, John. La voluntad es más fuerte que todos los rayos, todos los puñetazos, todas las armas del mundo.

---Es muy bonito todo tu discurso, Starr Miles. Pero no sé qué podemos hacer ahora, salvo fundar superhéroes anónimos y hacer un par de asambleas al mes ---dijo Scream con chispeante amargura.

---Es mucho lo que podemos hacer aún. Hace años yo pensé como tú. Yo tuve mi propia Ellen Gorgon, y tampoco salí bien parado. Como tú no vi esperanza alguna, pero pronto comprendí que debía hacer algo. Juré que mientras viviera protegería mi ciudad. Estoy en puertas de cumplir ese juramento. Ha sido duro, John. No es tan fácil como cuando se tienen asombrosos poderes, pero es posible marcar la diferencia. Hacen falta muchos años de paciencia, y nuestras hazañas no serán el capítulo de ningún libro de historia. Nadie nos lo agradecerá. Es posible que incluso nos teman. Pero no por eso debemos dejar de intentarlo.

---¿A qué te refieres?

---Me refiero a los Caídos.

John Scream miró de reojo a su interlocutor. Starr Miles captó su incredulidad.

---Te preguntas si estoy loco. Si sólo soy un viejo charlatán que ya ha visto pasar sus momentos de gloria y ahora se aferra a un pasado que es como la ceniza que cubre el suelo ahora. Sólo polvo.

---Digamos que me intriga lo que tengas que decirme.

---Unirnos, John. A lo largo de los años he hablado con más gente como tú. Aquellos que fueron héroes, que defendieron ésta y otras ciudades, y que luego desaparecieron en extrañas circunstancias. En mi vida civil yo era detective. De ese modo me aproveché de aquello que siempre nos empeñamos en dejar de lado en nuestras actividades como protectores para encontrar a otros como nosotros. Ha sido, y sigue siendo, una búsqueda agotadora. Sabemos escondernos bien. Pero ha dado sus frutos. Aunque sólo hombres somos, juntos seremos mucho más que eso. Los años han pasado, y ha llegado el momento de surgir. De enfrentarnos con Ellen Gorgon y destronarla.

---Ellen Gorgon ---dijo Scream con furia---. La mataría de poder hacerlo.

Starr Miles se acercó hacia él y le dio un potente puñetazo. Scream tuvo que usar ambas manos para apoyarse contra el suelo polvoriento.

---No eres un asesino, así que no finjas comportarte como tal; por lo menos no a mí. Sé lo que hicieron. Sé que mataron a la persona que más querías en el mundo. Siempre lo hacen, y nuestro deber es ser mejor que ellos. Puede ocurrir que, por defensa propia, acaben muertos. Es algo que todos los que hemos sido héroes hemos pensado alguna vez, si no nos ha ocurrido de hecho. Pero matarles deliberadamente es un deliberado asesinato.

John Scream se levantó. Contra lo previsto por Miles, nada de furia había en sus ojos. Sabía que tenía razón. Por mucho que odiara reconocerlo, estaba en lo cierto.

---¿Y cuál es tu plan? ¿Ser un ejército? ¿Derrotarles por superioridad numérica? No creo que haya tantos héroes caídos como para eso.

---No tiene sentido atacarles cara a cara. Nuestra derrota sería ineludible. Tenemos que ser conscientes de nuestros medios. Si nos disparan sangramos, si nos golpean caemos. Como has caído ahora. Pero podemos levantarnos, y volver más fuertes de lo que nos fuimos. Todos nosotros somos muy experimentados, y eso nos da una gran sabiduría conjunta. Nuestros conocimientos añadidos, de hecho, nos otorgan un gran potencial científico. Por otro lado estamos en todos los estratos de la sociedad, los mismos que Ellen Gorgon cree gobernar.

---¿De cuántas personas estamos hablando?

---Debo deducir que te unes al proyecto ---dijo Miles con satisfacción.

---No puedo negar que la idea me seduce.

---Me alegra oír eso. Sin embargo, como tuvo que hacer Platón con sus discípulos, tendremos que trabajar duro. Primero debes hacerte a la idea de que ya no eres, ni volverás a ser jamás, Reflector. Después, te convertirás\dots{} en otra cosa.

---La primera parte será más fácil de lo que crees ---dijo Scream tras pensarlo un momento.

Starr Miles rió por lo bajo.

---Eso, amigo mío, es lo que creen todos al principio.

\begin{next}
    ¡El entrenamiento! ¡John Scream, convertido en un nuevo miembro de los Caídos! Pero, ¿quiénes son? ¿Qué es lo que desean? Y lo más importante\dots{} ¿lograrán su objetivo?
\end{next}

\endinput
