\begin{prev}
    Tras la derrota de Armor, James Sky decide abandonar los Caídos para centrarse en su trabajo como Jefe de Policía. Al mismo tiempo, la Guerra de las Ocho Colonias toca a su fin y se reabren las rutas comerciales. Es el momento de viajar\dots
\end{prev}

\noindent{}Porque no siempre todo lo importante sucede en el mismo lugar, y no siempre los protagonistas están en el foco de los hechos principales. Hay otras historias, a veces ajenas, que saltan a la primera línea de los acontecimientos, y resultan ser tan cruciales como aquellas en las que uno había enfocado su atención en un primer vistazo\dots\\

\noindent{}John Scream aún no podía creerse lo que estaba a punto de hacer, más aún, lo que los suyos le habían convencido para que hiciera. Desde que se había convertido en el líder de Los Caídos no había tenido un solo momento para plantearse siquiera algo similar, y la vida en las calles le había hecho olvidar que tal momento pudiera existir.

Vacaciones. Por llamarlas de alguna manera, claro. La idea era que en nombre de Gorgon Enterprises visitara una colonia lejana para hablar de negocios acerca de un material que fabricaban en ella. Todos en la organización, en especial Razorclaw y Saw, le habían dicho que era la persona más indicada para ir hasta allí, ya que podría evaluar con todo lujo de detalles hasta qué punto aquel material era tan revolucionario como se rumoreaba fuera de Ernépolis~I.

Pero Scream no era tonto, y no se le pasaba por alto que aquella era una orquestada excusa para que tratara de olvidarse, aunque fuera por un escaso lapso de tiempo, de todo lo concerniente a coordinar un ejército de guerreros ninja que se dedicaban a moverse entre las sombras y a jugar a las películas de género negro.

Por otro lado era más que cierto que razón no les faltaba. La ciudad había levantado la mayor parte de los bloqueos aéreos, pero algunas restricciones de la guerra todavía eran patentes, lo que le había tenido supervisando operaciones nocturnas de escuadrones durante días enteros. Además de ello la derrota del temible y letal Armor aún estaba fresca en el recuerdo reciente y Sky ya no era oficialmente parte del grupo, por lo que había que reforzar nuevos lazos con las fuerzas del orden por vías de actuación distintas, sin comprometer la misión del Jefe de Policía de hacer prevalecer las leyes y la justicia de manera independiente y eficaz.

Realmente necesitaba un descanso. Pero un cierto instinto le decía que no iba a disfrutar demasiado del mismo pues no tardaría en encontrarse con problemas, tal vez en el viaje, tal vez a su regreso a la ciudad, tal vez en ambos casos.

O tal vez, más que un instinto, era el pesimismo lo que marcaba su línea de pensamiento. Pero también era cierto que el pesimismo era lo que le había mantenido con vida tantos años en las calles, al estar cada día preparado ante la idea de que el siguiente podría ser el último de su existencia.

La Colonia elegida para su exilio de ocio forzado era el pequeño mundo conocido como SKF. Se trataba de un peñasco mineralizado que apenas poseía población útil y no tenía interés en las rutas turísticas principales. Y eso, irónicamente, era lo que más atraía a John Scream, antiguo piloto estelar que había visto gran cantidad de mundo masificados y llenos de viajeros tratando de conocer lo que todo el mundo les había dicho que tenían que visitar antes de abandonar el planeta de turno donde estuvieran en ese momento. No deseaba ir a un lugar que estuviera marcado como imprescindible en todo viaje de rigor. Quería aislamiento, soledad y, con suerte, quietud y abandono voluntario. Desentenderse de todo, y poco más que eso. No eran muchas exigencias las que le pedía a la vida en ese momento. Simplemente que todo el mundo se olvidara de él aunque fuera relativamente hablando.

Desaparecer de las mentes de todos, ser expulsado del tiempo.

Claro, tendría que ver si cerraba el trato relativo al material que iba a negociar en nombre de Gorgon Enterprises, pero esperaba que no resultara un asunto laborioso ni de difícil decisión. Al fin y al cabo la experiencia de años en órbita le había ayudado a identificar qué era lo que mejor podía ayudar a la mejora de una nave.

Al menos cuando el material no era tan extraño que podía llegar a afectar al mismo flujo del espacio y el tiempo.

El nombre oficial era CT3, y Cronocorp era la empresa que lo utilizaba con mayor asiduidad para crear sus famosos y revolucionarios unicronos, un chisme del que se había hablado mucho en muchas partes pero que no había hecho apenas irrupción en Ernépolis~I, donde la tecnología estaba un poco chapada a la antigua, y parecía más bien como una mala copia de la que se vaticinaba en las arcaicas novelas de ciencia ficción del siglo veinte.

El material en cuestión creaba campos de tiempo de manera análoga, pero mucho más peligrosa y sutil, a los campos gravitatorios generados por una masa o los eléctricos por una carga. Empleado en los unicronos, de aspecto similar a antiguos relojes de pulsera, servía para realizar ni más ni menos que viajes en el tiempo, aunque con rangos de lejanía más que minimizados, de modo que las versiones comerciales de un unicrono permitían un rango del orden de minutos, ya fuera hacia el futuro o hacia el pasado. Ese detalle, unido a la famosa teoría reformada de la relatividad, que decía que el tiempo es una línea inalterable y rígida que encierra en ella todos los potenciales viajes que se realicen, permitió su comercialización en masa. No había peligro de que nadie alterara el flujo de la historia con ellos, sencillamente, porque sólo existía una historia, y alterada o no, era la única que existiría jamás.

Pero a Scream no le interesaban demasiado todos aquellos datos que, por otro lado, en gran medida se hallaban ya recogidos y estudiados a fondo en la evaluación del Inspector Científico Gubernamental de turno, y que popularmente se conocieron con el nombre de \emph{El Informe Cronocorp}. Lo que él deseaba conocer era las implicaciones que la materia prima de los unicronos, el CT3, podía tener en la manufacturación de naves espaciales, tales como saltos temporales en viajes espaciales para llegar al punto de destino al mismo tiempo o incluso antes de la partida, la anulación de las consecuencias relativistas de viajes a altas velocidades, o incluso maniobras de evasión de objetos peligrosos en la inmensidad del siempre misterioso Cosmos.

La idea era, en caso de mostrar interés por el proyecto, emplearlo para la línea \emph{Errante} de las naves y, ante una aceptación comercial positiva, ampliarlo a la línea \emph{Ares}, pero antes de nada Scream pediría ir a las canteras a ver el material en estado natural con sus propios ojos. Era lo menos que podía solicitar si pretendían cerrar un acuerdo con el gobierno de la zona.

Lo que le llevó a otro posible problema, al mismo tiempo que depositó sus pensamientos en línea directa con la realidad del momento que le rodeaba.

La nave en la que viajaba acababa de aterrizar en el único núcleo urbano de toda la Colonia, lo que era patente en el hecho de que se le asociaba con el mismo nombre que a ésta. La ciudad estaba llena de elevados edificios con tejados horizontales y gigantescas cornisas destinadas a circulación de vehículos deslizantes e incluso, en muchos casos, al tráfico peatonal, pues tampoco estaban exentas de locales comerciales situados a cientos de metros del suelo. La más alta de las torres era la que pertenecía a la sede del gobierno, aunque Scream no creía que tuviera la necesidad de visitarla para comenzar la negociación. Lo más probable era que un representante local fuera a recibirle a la torre en la que se iba a hospedar, y allí darían las primeras pinceladas de cara a un posible preacuerdo. Un proceso tan tedioso que muchas veces, sin duda, echaba de menos vagar por las calles cazando criminales mientras lo llevaba a cabo.

La nave en la que viajaba aterrizó en la planta vigésima de su torre y las compuertas se abrieron para dejar salir con calma a los escasos pasajeros. Scream llegó a plantearse trasladarse en una nave monoplaza y pilotar en persona, pero la batalla final con Armor había dejado un poso desagradable en su memoria al respecto de ponerse a los mandos de una nave, y tendría que pasar más tiempo hasta que quisiera plantearse poner en marcha de nuevo las turbinas de ningún cacharro similar.

Una vez entró en la torre y avisó de su llegada no tardó en llegar a su habitación y, dado que no llevaba consigo más equipaje que una bolsa de mano, lo dejó descuidadamente sobre la cama y bajó a la recepción del hotel, de corte clásico como la antesala de un casino.

No había problema alguno, por otro lado, con respecto a la seguridad de su maleta. Lo importante y que pudiera comprometer a la organización iba con él a todas partes. En el mejor de los casos, para no tener que ser usado durante su corta ausencia de Ernépolis~I.

Se sentó a pedir una bebida local y no tardó en verse acompañado por otro pez fuera del agua que rondaba por allí, un tipo con traje negro y que llevaba un sombrero borsalino marrón, bastante de moda en las calles de las Polis terrestres. Tras quitarse el sombrero y solicitar educadamente asiento pidió una bebida que a Scream no le sonaba pero que, tras escuchar su composición, no tardó en solicitar también.

~---El capitán John Scream, si no me equivoco ~---se limitó a decir el recién llegado.

~---¿Es usted el representante del gobierno? ~---preguntó Scream.

~---En realidad no. Permítame que me presente, Ten Scream ~---dijo quitándose el sombrero~---. Soy Marlowe Winston, tal vez haya escuchado hablar de mí.

Scream no tardó en notar las primeras huellas de su viaje al extranjero en cuanto escuchó que se dirigían a él con el tratamiento Ten, empleado en distintas partes del universo pero no demasiado usual en Ernépolis~I.

~---No es así, por desgracia. ¿A qué se dedica?

~---Soy lo que en su ciudad llamarían un sabueso, un hombre de tiempos pasados, si lo quiere ver de ese modo ~---comenzó encendiendo un cigarrillo~---. En esta era de cualidades sorprendentes y criaturas excepcionales mi arma sigue siendo la que ha estado presente a lo largo de toda la historia de la humanidad: la lógica.

~---Celebro escuchar algo así. El mundo que nos rodea está muy falto de lógica en estos días, así como de sentido común. ¿Trabaja por libre?

~---Como todo investigador se pueden contratar mis servicios, míos y los de mis compañeros, en realidad. Juntos somos los Esclarecedores, tal vez ese nombre comercial que usamos ya le suene más.

~---Sí, en efecto. Muchos escándalos han sido destapados gracias a su intervención. Gente como ustedes sería muy útil en Ernépolis~I.

~---El Universo es vasto, pero aun así no sería inusual que coincidiéramos allí algún día. Dígame, ¿qué le trae por esta Colonia perdida?

~---No más que negocios. Eso, y la necesidad de desconectar, aunque sea temporalmente, del trabajo.

~---¿En serio? No parece un hombre que delegue responsabilidades muy a menudo, permítame decírselo. De hecho, si por mi impresión personal fuera, diría que parece como si cargara usted con el peso de una carga insoportable sobre sus hombros.

~---¿Por qué cree eso?

~---Intuición, Ten Scream. Es mi única arma, al fin y al cabo, y por eso la tengo bien entrenada. No se me escapa que estuvo involucrado en un incidente militar extraño y peligroso, algo concerniente a una servoarmadura robada.

~---Así es, en efecto. Trataron de llevarse una de nuestras naves con ella.

~---Tal vez pueda usted aclarar algunas sospechas que tuve al respecto del caso. Una de ellas era la identidad del ladrón de la armadura. Al principio pensé que se trataba de una mujer llamada Eileen Drift, pero luego desistí en la idea.

~---¿Y quién cree que fue?

~---¿Puedo confiar en usted?

~---Adelante.

~---Creo que nadie. Opino que la armadura, por sorprendente que pueda sonar, cobró vida propia.

Scream se lo quedó mirando callado, tratando de aparentar sorpresa. No había pensado cómo reaccionar de encontrarse en una situación similar.

~---Hubo algunas declaraciones extrañas al respecto, si me permite exponer mis argumentos ~---continuó Winston~---. No sólo de soldados, también de ciudadanos. Aparte, el comportamiento del sujeto parecía evidenciar una suerte de pauta, cuanto menos, mecánica y maquiavélica. Parecía como si actuara sin pensar en consecuencias a largo plazo de sus acciones, sólo en el instante reciente. Eso me hizo pensar que el portador de la servoarmadura tenía, cuanto menos, un escaso o nulo interés en el mundo de los humanos, lo que me hizo deducir que o bien era un alienígena, o bien una posible inteligencia programada. Descarté la posibilidad del alienígena concluyendo que, en ese caso y dada la forma de la armadura, sería más o menos antropomórfico, y por tanto la posibilidad de que tuviera un desconocimiento total de nuestras costumbres y viviera al margen de ellas era muy reducida. Me decanté por la opción de una máquina, pues. ¿Dígame, estoy en lo cierto?

~---Creo que es improbable ~---comentó Scream, tratando de distanciarse de la deducción~---. Ya habrá leído que junto a los restos de la armadura se encontró un esqueleto humano.

~---Sí, pero sorprendentemente el esqueleto estaba, digamos, \emph{intacto}, mientras que la armadura fue volada en pedazos. Creo que el sistema se sobrecargó y empezó a funcionar de manera autónoma con alguien dentro. O tal vez ya lo hacía antes y alguien tuvo la desgracia de quedarse encerrado en su interior.

~---Es sorprendente lo que dice, sin duda ~---comentó Scream, incapaz de alejarse de la evidencia.

~---Lo cierto es que la ciudad de Ernépolis es poco menos que fascinante para mí. Podría estar con usted horas hablando de las teorías que he elaborado sobre la identidad del ser que dicen que vigila sus calles. Uno a uno he ido descartando multitud de posibilidades: no se trata de un hombre, ni tampoco de un alien. De más está decir que no es un falso rumor, ni un invento perpetrado por las autoridades locales. ¿Sabe lo que me sorprende, de hecho? Su pasado. ¿Dónde estaba antes de darse a conocer? ¿Quién era? ¿Quién le enseñó todo lo que sabe, si es que sólo fue una persona quien lo hizo? Tengo ciertas sospechas más concretas, pero sin corroborarlas del todo preferiría no decir nada al respecto, aunque le emplazo a que tengamos algún día esta conversación en el futuro.

~---¿Y qué es lo que le ha traído hasta aquí a usted y sus compañeros, señor Winston? ~---preguntó Scream, tratando de cambiar de tema, pues cualquier comentario podía adquirir tintes peligrosos en presencia de aquel hombre.

~---Venimos a investigar un rumor, la existencia de un hombre que tiene en jaque a las autoridades de la Colonia. Nos ha contratado personalmente Carl Krok, uno de los tres máximos dirigentes de Cronocorp. Al parecer está entorpeciendo la extracción de mineral de la cantera.

~---Algo he escuchado. Un rebelde local, por lo que dicen, al que se le tiene cierta simpatía.

~---¿Usted cree que es sólo eso? Yo diría más, en realidad. Creo que le mueve la venganza, y tal vez un instinto protector. De qué, me es imposible deducirlo con los pocos datos de los que dispongo.

No tardó finalmente en llegar el representante local del gobierno, al que Scream no reconoció pero sí identificó al momento pues se dirigía hacia él como acción prioritaria a tomar.

~---Ten Scream, bienvenido ~---Scream fue a realizar un ademán de estrechar la mano, pero recordó que no era costumbre en aquella colonia, donde todos se conocían a todos~---. Disculpe el retraso, problemas burocráticos.

~---Les dejo, caballeros ~---fue la escueta despedida de Winston, pero el funcionario le detuvo.

~---De hecho, Ten Winston, también me envían para hablar con usted. Usted, Ten Longhealth y Ten Scummer regresarán con Ten Scream hoy mismo a las canteras de CT3. Espero que eso no entorpezca los planes respectivos de cada uno de ustedes.

~---Por mi parte en absoluto ~---se limitó a decir Scream, aun sabiendo que tal vez sería expuesto a más preguntas incómodas, aunque estaba casi convencido de que Marlowe Winston había deducido, si no la verdad, al menos sí gran parte acerca del Caído. El nombre de los Esclarecedores no había sido elegido precisamente al azar a la hora de referirse a él y sus dos compañeros.

~---Nosotros tampoco tendremos problemas ~---comentó Winston con sencillez.

~---Perfecto. Si les parece un transporte les recogerá en una hora, tiempo local, para llevarles hasta la cantera. La nave les dejará allí y pasará a recogerles unas horas después.

~---¿Nadie nos guiará? ~---preguntó Scream.

~---Pueden moverse a sus anchas, no se preocupen. Sólo recuerden no cruzar más allá de las compuertas negras. Es peligroso debido a la fluctuación del CT3 en los campos de tiempo.

~---¿La ausencia de guía se debe a alguna clase de superstición? ~---preguntó inquisitivamente Winston, poniéndose de nuevo su borsalino. Scream tuvo muy claro que aquel detective iba directo al grano, aunque con preguntas llenas de subterfugios.

~---En absoluto, no se preocupen por nada. La nave les recogerá en este mismo nivel, en la plataforma exterior. Espero que tengan una buena estancia en SKF ~---terminó tras despedirse y alejarse.

~---¿Supersticiones?

~---Fantasmas. Un sujeto que se mueve en las sombras. Aunque imagino que nada de esto le resultará extraño a usted ~---terminó marchándose y ajustándose el sombrero mientras se alejaba.

Scream no dejó de pensar, en lo que regresaba a su habitación, hasta qué punto era conveniente tener a un hombre como aquel como potencial compañero a su lado.

De modo que aquella colonia tenía sus secretos. Y en cierto modo, tenía curiosidad por averiguarlos, separar el trigo de la paja y saber qué había pasado allí en realidad. Había escuchado rumores, habladurías sobre un rebelde solitario que había derrocado al anterior gobernador, Jason Hook. Al parecer, un tirano que dominó con mano de hierro toda la región.

No sabía por qué, pero la historia le resultaba en cierto modo familiar.

De repente se detuvo en mitad del pasillo, oscurecido y bañado por una luz de penumbra. Un aviso en su cabeza, una alerta de naturaleza animal le advirtió que algo extraño rondaba a su alrededor. Como si las leyes de la naturaleza acabaran de ser violadas ante sus mismos ojos y se hubiera abierto una puerta a otra dimensión que sin embargo no era capaz de localizar.

Se preguntó si esa sería la sensación que tendrían los rateros de Ernépolis~I cuando él y los suyos entraban en acción.

Un hombre apareció al fondo, vestido con un traje de extracción de minerales. El traje estaba ajado y su rostro demacrado. No trataba de ocultarlo ni nada parecido, se mostraba sin necesidad de parapetarse en sombra alguna, aunque Scream no pudo comprender cómo no le había visto antes. En la mano tenía un energolátigo, objeto típico en la minería, y estaba ligeramente encorvado. Parecía más una cosa que una persona.

~---Hola, John Scream ~---dijo con una voz extrañamente cavernosa.

Scream se lo quedó mirando fijamente. En aquella colonia perdida todos parecían conocerle mejor aún que en su propia ciudad.

~---¿Quién es usted?

~---A su debido tiempo. Dentro de poco tendremos más cosas que contarnos. Sólo estoy aquí para anticipar ese encuentro.

~---¿Qué pretende, asustarme? ¿Disuadirme?

~---No tengo por qué hacerlo. No habrá ningún acuerdo entre tu compañía y la colonia. No confiarás en ella, ni tampoco en el uso del material. Tampoco concluirás que pueda servirte para mejorar el traje oficial de Los Caídos.

~---No sé de qué me habla ~---fue la escueta respuesta de Scream.

~---No hace falta que finjas conmigo. Yo lo sé todo. Sé lo de Starr Miles, y cómo te reclutó y te enseñó lo que sabes. Sé lo que hizo Ellen Gorgon, y también Silenciador. Sé lo de Aryn Life y que en este momento lamentas que James Sky se haya apartado de vuestro lado, aunque siga con vosotros en cierto modo.

~---¿Cómo sabes todo eso? ~---no pudo finalmente evitar decir Scream.

~---Tú mismo me lo has dicho. Muchas veces, en muchos lugares distintos. Podría hablarte de lo que vendrá, también. De todos los enemigos y aliados que tendrás que conocer o enfrentar. Conociste ya a Armor aunque los nombres de Nitram, Hades, Alma Espejo o Dobleseis, entre muchos otros, no te dicen nada ahora. Pero te lo dirán. Podría hablarte de él, también. Desearía hacerlo. Ojalá pudiera hacerlo. Pero no quiero influir en las cosas tal como deberían ser. Olvidarás esta lista de nombres, de hecho, porque así debe ocurrir.

\rquoti{}Recuerda, sin embargo, y no olvides. Ellos lo saben. Ellos están de tu lado y del mío. No les hagas nada, ni te expongas por mí. No delatarán el valle.

Después de aquello se dio la vuelta y se alejó hasta doblar la esquina. Scream no trató ni de perseguirle, pues ya suponía que alguien que había preparado semejante entrada y discurso no iba a dejarse localizar con facilidad.

Aunque había partes de lo que había dicho que, por algún motivo relacionado quizás con el cansancio, Scream era totalmente incapaz de recordar, como si las tuviera agazapadas en la punta de la lengua.\\

\noindent{}Un rato después, y sin nadie con quien hablar de lo sucedido, Scream llegó con antelación al punto de encuentro y esperó junto a la nave hasta que los otros tripulantes se presentaron. Junto a Winston iban otros dos sujetos trajeados, uno de ellos ancho de hombros y llevando un sombrero panamá, y el otro delgado y de mirada ágil, con un fedora gris.

~---Ten Scream, le presento a Spade Longhealth y Corey Scummer ~---dijo Winston señalando alternativamente al corpulento y al delgaducho. Las habilidades del primero resultaban evidentes a simple vista, y las del segundo no eran difíciles de deducir en cuanto uno se fijaba en el láser Taurus 51 que llevaba en la sobaquera.

La nave despegó elevándose hacia el cielo crepuscular en llamas, en nada parecido al decadente paisaje que la Nube tenía acostumbrado a Scream. No obstante, en cuanto hubieron salido del radio urbano, empezaron a volar rasante, sólo varios metros por encima de la montaña de estalagmitas que circundaban como cuchillos de piedra por todas partes en SKF.

En mitad del trayecto Scream creyó ver un ave volar a lo lejos, pero no tardó en convencerse de que parecía demasiado grande para tratarse de un simple pájaro o criatura similar.

~---Es un criminal que opera por la zona ~---anunció Scummer~---. Dicen que era un matón del anterior gobernador, pero se volvió loco por motivos no del todo claros. Si no fuera porque vamos con el tiempo justo, no me importaría probar puntería con él.

~---En realidad se cree con bastante certeza que el hombre al que buscamos le hizo eso, y estoy convencido de que le conocía.

~---¿Por qué está tan seguro?

~---La mente de ese criminal está quebrada. No fue castigo lo que se le aplicó, sino tortura, como nunca antes he conocido. Pero aun con todo, no he logrado encontrar una sola pista sobre quién pudo haber sido, por mucho que he investigado a todos los sospechosos. Muchos testigos afirmaban haber escuchado que un hombre había desaparecido en la cantera, un extractor, pero nunca se produjo desaparición alguna ni nadie denunció la ausencia de un ser querido.

La nave se posó en la entrada a la cantera y una vez allí dejó a los cuatro hombres, avisando de su regreso posterior y acordando una hora concreta. Una vez allí se encontraron solos frente a la gran cueva. Los restos de una enorme compuerta deslizante les recordaron que no siempre todo el mundo fue bienvenido en aquel lugar.

Recorrieron el primer gran pasillo, que no tardó en empequeñecerse hasta que regresó a proporciones humanas, y les llevó a una caverna de nuevo más amplia, llena de agujeros por los que se filtraba la luz exterior. No había emanaciones de gases, ni apenas ausencia de luz. Aquel debería ser un destino idílico para la mayoría de los extractores en comparación con las minas de procesamiento de otros materiales.

Ninguno de los presentes se sorprendió por la presencia del CT3 en vetas expuestas y vivas, puesto que aquella sección de mina aún estaba en proceso de ser horadada, y los unicronos, además de ser aún caros, requerían de muy pocas cantidades de aquel material. La producción, sin embargo, podía intensificarse si se empezaba a emplear también en el diseño de vehículos de toda clase, empezando por los espaciales.

~---No me gusta este lugar ~---comentó Longhealth de repente, cruzando los brazos. Scream pensó que no le hubiera gustado estar atrapado en una presa efectuada por aquellos dos pistones de puro músculo.

~---¿Alguna teoría? ~---comentó Scummer, vigilando las entradas y dirigiéndose a Winston~---. Pero qué digo, siempre tienes una teoría.

~---Ya la tenía en realidad antes, a partir de las visitas a vetas secundarias que efectuamos, pero la pensé descabellada.

~---Estoy ansioso por escucharla, teniendo en cuenta que no hace mucho sugirió que la servoarmadura que atacó mi ciudad tenía vida propia ~---argumentó Scream con ironía de doble sentido.

~---El punto esencial para mí es la advertencia de que tengamos cuidado con los campos de tiempo en esta cantera. Las otras canteras que nosotros hemos visto eran más pequeñas, pero no mucho más que ésta. Y si en verdad querían que averiguáramos algo productivo, ¿por qué no llevarnos aquí desde el principio? No creo que el rebelde sea tal. Dudo mucho que le mueva un sentimiento político o de boicot empresarial. Creo, de hecho, que protege esta cantera, y que esas puertas negras son más producto del miedo que de la pura precaución.

~---Hay una manera de averiguarlo ~---sugirió Scream.

Sin que nadie dijera nada, empezaron a caminar hasta toparse con una de ellas, no muy lejos de su camino. Tras explorar un rato más vieron que las otras no quedaban muy lejos.

~---Una mina pequeña, demasiado ~---siguió razonando Winston~---. Spade, ¿puedes hacernos los honores?

Longhealth se acercó a la compuerta y la agarró por la parte inferior, aprovechando resquicios porosos por donde podía introducir sus gordos dedos. Haciendo alarde de una fuerza cuanto menos descomunal, pero no sin gran esfuerzo, levantó la compuerta con tanta brutalidad que ésta fue incapaz de regresar de nuevo a la posición original.

~---Veo que se toman lo del reparto de tareas muy en serio ~---dijo Scream entrando el primero.

Empezaron a andar por el lugar y se adentraron tanto que perdieron por completo la noción del tiempo que llevaban ahí dentro. De más está decir que no sintieron que les estuviera pasando nada extraño en ningún sentido, y por tanto no era descabellado concluir que la advertencia del peligro era una pantomima.

Al menos, el peligro de los campos temporales.

De nuevo Scream volvió a ver a aquel extractor con aspecto de haberse quedado encerrado mucho tiempo aislado del resto de la humanidad. Hizo un rápido gesto a los otros para que se apartaran, pero fue demasiado tarde para él. Un rápido golpe de energolátigo derrumbó parte del techo y le dejó al otro lado, junto con su atacante. Cuando quiso darse la vuelta, estaba solo de nuevo.

De modo que se había quedado aislado, y casi seguro le estaban vigilando. Preparó los aparatos que había llevado a la colonia, entre los que estaban las armas aturdidoras y los hologramas, y siguió andando.

No pudo creer lo que vio cuando lo tuvo frente a sí.

Ante él se ubicaba una ciudad construida enteramente con CT3. Un conglomerado de edificios de formas irreales que se sucedían alternativamente alrededor de una construcción más grande en forma de pirámide. Tuvo el deseo irresistible de bajar hasta allí, pero le habían advertido de los peligros de la cercanía a excesivas cantidades del material. Sin embargo, otros peligros de los que le habían advertido eran completamente ficticios.

Estaba decidido a bajar cuando escuchó de nuevo aquella voz de ultratumba.

~---No lo hagas. No será bueno para tu mente.

Scream se giró y ahí estaba de nuevo, mirándole fijamente. Su rostro tenía un algo extraño, y era como si su cuerpo emanara un aura invisible que erizara sus cabellos igual que si de repente, tal como le pasaba a los gatos y otras criaturas, pudiera captar los poderes invisibles del mundo.

~---¿Quién eres?

~---Una vez fui un hombre como tú. Ahora sólo soy Outcast.

~---Tú me has traído hasta aquí, ¿verdad? ¿Para qué?

~---Para mostrarte lo que quieren ocultar, esta acrópolis silenciosa a la que yo llamo el Valle de la Eternidad. Desean derribarla sacando el material de la mina para crear sus máquinas y venderlas en todo el Universo.

~---De modo que quieres que no firme ningún acuerdo.

~---Sé que no lo vas a hacer.

~---¿Cómo puedes estar tan seguro?

~---Tú defiendes tu ciudad, y yo defiendo la mía ~---contestó Outcast con voz profunda, más aún teniendo en cuenta su tono habitual hasta ese momento.

~---Algo te pasó con ese material, ¿verdad? Algo que te dio esas cualidades.

~---En cierto modo soy un privilegiado. No envejezco, y me puedo mover por el tiempo y, por tanto, por el espacio, sin dejar huella alguna en el mismo. He conocido a millones de personas en millones de lugares y épocas distintas. Te he conocido a ti, y a Sky, y también a Razorclaw y Saw, incluso a gente que no conoces aún. Son muchas las cosas que podría decirte, pero que no debo hacer.

\rquoti{}Este poder tiene también su maldición, por otro lado. Nadie sabe quién soy, porque mi pasado fue literalmente borrado del tiempo. Dicen que el tiempo es una línea indeleble. Yo soy la prueba de que, aunque sea muy extraño, algo así puede ocurrir.

De repente escucharon ruidos a lo lejos. Se trataba de los Esclarecedores, que habían encontrado otro camino.

~---No pueden verme, pero recuerda lo que te dije. No es necesario que te expongas por mí. Bastará con que te diga algo antes de que me vaya.

Estuvieron hablando un poco más, y después de aquello Outcast se alejó hacia la ciudad, algo que podría haber enloquecido a cualquier otro ser humano, y no tardó mucho en perderse en su interior fosilizado. Los Esclarecedores no tardaron en llegar a su lado.

~---¿Se encuentra bien? ~---preguntó Winston.

~---Perfectamente.

~---Alguien ha estado aquí recientemente ~---comentó Scummer poniéndose en posición de disparar. Examinó el rastro y vio que salía de la ciudad, y haciendo gala de una vista prodigiosa vio a Outcast a lo lejos, donde probablemente él no alcanzaba a verle.

Sacó el arma y apuntó hacia el guardián de la ciudad. Scream se limitó a hablar, con total tranquilidad.

~---Él me dijo que usted le vería. Pero me dijo que no me preocupara. Que sólo mencionara la historia de su padre y el pájaro.

Scummer siguió quieto, pero ya no sólo era porque estaba apuntando.

~---Dice que cuando era pequeño su padre quería enseñarle a tirar y le obligaba a disparar pájaros. Mató cientos hasta mejorar su puntería. Desde entonces nunca ha vuelto a atacar a un solo animal, y no lo volverá a hacer de nuevo.

Scummer bajó el arma, completamente anonadado.

~---De usted ~---prosiguió Scream mirando a Longhealth~--- dice que no debe preocuparse por no haber podido salvarle. Que no fue su culpa no poder arrastrar su cuerpo fuera de la casa antes de que se propagara el incendio. Toda la fuerza que ganó con los años y todo lo que ha ayudado a otros con ella lo compensa con creces.

~---¿Y de mí qué dice? ~---preguntó Winston, poniendo la mano sobre el enorme hombro de su corpulento amigo~---. Tengo interés por saberlo.

~---Dice que no necesitaré explicarle nada porque lo comprende ya casi todo, y que yo no necesitaré hacer nada más al respecto.

~---¿Algo más?

~---Me ha dicho que no me preocupe de usted, pues lo que tiene de genio deductivo lo tiene de discreto.

~---¿Por qué le habrá dicho eso? ~---preguntó Winston como mirando hacia otro lado.

~---La verdad, si he de serle franco, no tengo ni idea ~---respondió Scream mirando en la misma dirección que él.\\

\noindent{}Tal y como Outcast vaticinó, los Esclarecedores no dijeron una sola palabra al respecto de lo sucedido en el interior de la cantera de CT3. Dejaron el caso abierto y añadieron unos cuantos rumores extra a los ya existentes, como que el fantasma de esas canteras sabía cosas del pasado que nadie más podía saber. Para eso no fue necesario mentir, sólo dejar hablar un rato a Longhealth y Scummer, que pasaron todos los detectores imaginables de Cronocorp.

Por su parte, Scream se alegró por un lado de no tener que recurrir a sus poderes, y también de no haber tenido que firmar el contrato para la utilización de CT3 por parte de Gorgon Enterprises. No sólo por ayudar así a la preservación de aquella increíble ciudad de origen desconocido. También porque temblaba de sólo pensar en tener que reformar los planos de todos los modelos de la línea \emph{Errante} para la inclusión de aquellas más que dudosas mejoras, pensó en lo que su nave de regreso le alejaba de la colonia de SKF, no sin antes haber pasado un par de días allí en la más absoluta paz y tranquilidad.

\begin{next}
    Si pensabais que Armor era peligroso, ¡esperad a conocer la nueva amenaza que se cierne sobre Ernépolis~I! Y todo ello, al mismo tiempo que la Nube desciende a la altura de las calles. ¡Una nueva saga comienza!
\end{next}

\endinput
