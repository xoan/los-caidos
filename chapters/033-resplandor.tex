Cuanto más sabían sobre el recién llegado menos creían entenderle. A veces parecía que ocultara una estrategia maestra tremendamente estudiada. Otras que sólo le moviera el afán improvisador.

Tal vez la realidad estuviera ubicada en el promedio de ambas opciones. Pero siempre sería un buen momento para intentar salir de dudas al respecto de ello.

\fancyparbreak
Sam Grove consideró un milagro a efectos prácticos que Scream no se enterara de la escapada que él y Shockman habían hecho por cuenta propia. No se hizo mención alguna en la prensa sobre ello, seguramente debido a las influencias de Nitram, y ni siquiera Saw se enteró, pues Alma Espejo tampoco pareció airear el asunto en ningún entorno, ya fuera oficial o privado, sea lo que fuera la parte de su vida que pudiera considerarse privada.

Sí, habían salido con buen pie de aquella situación. De una pieza, como se diría. Sin tener que dar explicaciones a nadie.

Era por eso que seguía sin entender qué extraño mecanismo de su cerebro le llevó a tener que confesárselo a Scream con todo lujo de detalles.

Culpabilidad. Sentimiento de haber metido la pata, tal vez. Incapacidad de fingir una vez le diera al líder de Los Caídos la nueva información que habían averiguado. Además de todo eso, si volvían a encontrarse con Nitram, lo que no sería en absoluto improbable, no creía que éste permaneciera callado respecto al encuentro inmediatamente anterior.

La había cagado pero bien. Ése fue el pensamiento recurrente de Grove a medida que se lo contaba todo a Scream, metidos los dos en su pequeño pseudodespacho del Aquerón, su lugar de recogimiento y ensimismamiento personal.

Scream no dijo una sola palabra en todo el tiempo que Grove habló. Sólo se limitó a mirarle. Fijamente, sin apenas parpadear. Como si fuera otro más de los criminales de la ciudad a los que meter miedo, o tal vez era una apreciación subjetiva de Grove, más preocupado cuanto más cerca estaba de llegar al final de la historia.

Una vez terminó, Scream comenzó a hablar. Aunque era su tono habitual, Grove lo percibió como si llevara puesto el modulador de voz.

~---De modo que desobedecisteis una orden.

~---Sí, señor.

~---Y por poco os matan, de no ser por Alma Espejo, que por otro lado estuvo a punto de descubriros.

~---Sí\dots\ sí, señor ~---Grove estaba cada vez más y más atribulado.

~---Entiendes que esto no puede quedar sin castigo.

~---Lo entiendo, señor.

~---Shockman limpiará todas las celdas sucias y pasará la revista de hoy.

~---¿Y yo, señor?

~---¿Qué quieres decir?

~---¿Cuál será mi castigo?

~---Tu castigo será decirle a Shockman que has confesado por ambos, ordenarle que limpie las celdas y vigilar que cumple con su cometido y no trata de encasquetárselo a otro.

Grove se quedó pálido por completo, como si acabaran de darle una mala noticia que no esperara tener que escuchar.

~---Señor\dots\ ¿no podría limpiar yo en su lugar?

~---No ~---contestó Scream escuetamente~---. Eso es todo, Sam. Retírate.

Grove se marchó cerrando tras de sí, al mismo tiempo aliviado y preocupado. No había sido tan terrible como esperaba, aunque no tardó en darse cuenta de que en realidad el peor trago estaba aún por llegar. Pero sí, tenía que admitir que Scream sabía bien cómo imponer castigos acordes con la personalidad de cada uno de sus subordinados.

Una vez estuvo solo Scream agarró el comunicador de su traje, cuidadosamente colocado sobre la mesa, y lo conectó con calma pero sin demorarse más de lo necesario.

~---¿Charles? Tenemos que hablar. Hemos tenido un pequeño motín a bordo, pero que se ha saldado con la obtención de información que puede resultar crucial para saber más de nuestro nuevo héroe invitado a la ciudad. No, ya les he amonestado yo mismo. Tranquilo, el escuadrón de Grove no saldrá hoy. A la vista de los nuevos datos, cuantos menos seamos hoy en las calles mejor. Sí, lo son. No, no creo que él esté llevando a Grove por el mal camino, son demasiado distintos para eso. Sí, sí, yo también era así. Y mira dónde estoy ahora. Por eso les castigo, para que no acaben como yo. De acuerdo, te espero.

Scream se recostó sobre la mesa y examinó los informes que tenían de Alma Espejo de reojo, en lo que trataba de extraer las conclusiones inmediatas de los nuevos acontecimientos. Alma Espejo había doblegado a Nitram. Con él se sumaba a la lista otro enemigo de Los Caídos que tenía más o menos bajo control, y no uno cualquiera sino de los más peligrosos, astuto en las sombras y letal cuando decidía mostrarse a la luz.

Una vez más, el nuevo héroe de Ernépolis~I había triunfado donde ellos fracasaron.

¿Qué hubiera pensado Starr Miles de todo aquello? Alma Espejo era un héroe como los de antaño, eso estaba claro. Pero no, se corrigió a sí mismo, eso no es del todo verdad. El poder de Alma Espejo supera con mucho al que ellos tuvieron en su momento. Alma Espejo se hubiera desayunado a un tipo como Silenciador\dots\ o quizás no, eso ya no habría manera de averiguarlo jamás. ¿Qué hubiera pasado de haber conocido a Ellen Gorgon? Más aún, ¿conoció a Ellen Gorgon? ¿Dónde había estado todo este tiempo?

Muchas incógnitas, pocas respuestas. Al menos de momento. Sin embargo, gracias a la rebeldía de Grove y Shockman, que le pudo salir muy cara a la organización, sabían que Nitram y él se conocían de antes. No sólo eso, Nitram parecía respetarle. Por lo que Grove contó su actitud resultó incluso sorprendente, como si de repente creyera que aquel cruel magistrado sin alma, de hecho, albergara en su interior algo similar a afecto por algún ser humano.

Siempre estaba la posibilidad de preguntar a Nitram de nuevo, pero eso podía conllevar enormes riesgos añadidos. No era igual que ir a indagar a Felicity Hound, con quien existía al menos ciertos puntos ideológicos comunes. El Juez era uno de sus peores enemigos, y todo dato que supiera sobre ellos era veneno que podía acabar infectando el corazón de la organización.

Tendrían que recurrir a otros métodos. Pero antes de eso, antes de dar un solo paso por cuenta propia, necesitaba consultarlo con uno de sus mejores y más fieles compañeros de armas.

Razorclaw entró al despacho y se sentó con tranquilidad, aunque no hacía falta más que echarle un vistazo rápido para darse cuenta de que se pasaba la mayor parte de su tiempo corriendo de un lado para otro del Aquerón.

~---Bien, John. Te escucho.

Scream le contó de manera sintetizada pero concisa todo lo que Grove le notificó, con pausas ocasionales para que Razorclaw asimilara la información. Él era, además en cierto modo, parte interesada en todo aquel asunto, pues de los miembros de Los Caídos era aquel cuya vida personal le ponía más rápidamente en ruta directa a hablar con Nitram, gracias a su profesión. Eso también implicaba que, aunque fuera a un nivel superficial y distante, conocía mejor que ninguno de ellos la compleja personalidad del Juez.

~---Nitram no es persona que se deje impresionar fácilmente ~---comentó Razorclaw juntando las manos en posición de ojiva. Por un momento Scream se sintió como si le estuviera escuchando decir un alegato~---. De hecho, la llegada de Alma Espejo en días anteriores no dejó huella alguna en él hasta que le reconoció.

~---El problema es que no podemos indagar demasiado en esa conexión o corremos el riesgo de que se entere, lo que podría ponerle en línea directa hacia nosotros. Tenemos que apañárnoslas con poco más que lo que tenemos.

~---¿Qué tal si hablamos con Distorsión y los suyos? Parecen ser muy eficientes en su trabajo.

~---No creo que sea buena idea. Hicieron huir al Juez en el asunto de los cazarrecompensas, y a buen seguro ha dispuesto la eventualidad de que ellos pudieran cruzarse de nuevo en su camino algún día. No, tiene que ser alguien que apenas necesite acumular más datos que los que tenemos para elaborar una discreta y sencilla investigación. Alguien que trabaje con el poder de la mente a máxima potencia.

~---Pero algunos los mejores detectives de la ciudad están ya trabajando para nosotros y están saturados por el exceso de posibles líneas de actuación abiertas.

~---Sí, eso es cierto ~---comentó Scream, que de repente se incorporó hacia el escritorio~---. Pero hay alguien a quien podemos recurrir y que no creo que tenga problemas en echarnos una mano en este asunto. Alguien a quien conocí en circunstancias cuanto menos peculiares.

~---¿Es bueno? ~---preguntó Razorclaw intrigado.

~---Es tan bueno que creo que ni él mismo es consciente del alcance de su talento ~---se limitó a decir Scream en lo que se dirigía hacia el hemiciclo, sin dar más explicaciones.

\parbreak
Razorclaw no fue detrás de Scream cuando éste salió del despacho. Comprendía que, aunque fuera alguien de fiar, era mejor mantener las formas con respecto al contacto en el que su jefe había pensado.

No era que no supiera qué era lo que Scream se traía entre manos. Cuando se conocieron ya dejó bastante claro que no pensaba que el Capitán John Scream, piloto retirado y diseñador de naves para Gorgon Enterprises, fuera alguien que viviera de manera única y exclusiva para su trabajo corporativo. Los límites de su conocimiento eran algo que escapaba a la deducción de Scream, pero una cosa tenía clara. Eran sin duda más vastos de lo que pudiera tan siquiera imaginar.

La pantalla del hemiciclo alteró su imagen monocorde y en ella apareció el rostro de la persona con que deseaba hablar. Ni tan siquiera para una conferencia a larga distancia se quitaba su borsalino marrón.

~---Celebro verle de nuevo, Ten Scream ~---dijo Marlowe Winston ajustándose la corbata, de vivos tonos rojos.

~---Lo mismo digo ~---fue la respuesta esforzada de Scream, que dudó si emplear a su vez también el tratamiento Ten, pero decidió finalmente no hacerlo~---. Espero que tenga un momento para hablar conmigo de un asunto de gran importancia.

~---Siempre es un placer hablar con usted. Cada vez que lo hago saco fascinantes conclusiones nuevas sobre la situación de Ernépolis~I. Por ejemplo, ahora mismo no puedo evitar fijarme en el lugar desde el que me está hablando. Una sala de reuniones, amplia pero sorprendentemente sombría. No es mucho lo que alcanzo a ver, pero su diseño me sugiere que está bajo tierra, y también que es un secreto bastante bien guardado. La ausencia de gente a su alrededor me hace sospechar también que tal vez su puesto empresarial sea más importante de lo que en un principio pueda uno imaginarse examinando los datos que de usted pueden encontrarse disponibles.

~---¿Qué hay de sus compañeros, no están con usted en este momento?

~---Digamos que mi labor en el caso que nos ocupa, una vez he llegado a las conclusiones indicadas, ha pasado a un segundo plano. Ahora es una cuestión de fuerza bruta e intimidación bien empleada, y en esos aspectos ellos me superan con creces. Somos un buen equipo, se puede decir. Pero dígame, ¿desea nuestros servicios para la empresa, tal vez? ¿O se trata de un asunto más\dots\ personal?

~---Pongamos que es personal, como ha sugerido ~---aclaró Scream, sintiendo de nuevo esa sensación de fugaz incomodidad por sentirse transparente ante un hombre capaz de penetrar mentes con la palabra~---. Necesito información sobre un nuevo habitante que ha llegado a Ernépolis~I.

~---¿Sería mucho aventurar por mi parte, Ten Scream, si sugiriera que se trata del nuevo héroe que parece trabajar para defender la ciudad?

~---Pongamos que se trata de la persona que ha mencionado ~---prosiguió Scream, para nada sorprendido de la deducción. Se preguntó sin Marlowe Winston tendría esposa e hijos, y en ese caso cómo se las arreglaban para hacerle regalos el día de su cumpleaños~---. Debo imaginar que ya ha centrado su atención en él de una ú otra manera.

~---Sólo como mero ejercicio intelectual, no por encargo de nadie. El problema es que la falta de información que hay sobre el sujeto en cuestión es alarmante, tanto que me hizo pensar en alguna clase de vida problemática y caracterizada por una tremenda inestabilidad en términos afectivos y personales.

Scream pensó que incluso aunque Winston hubiera dado en el clavo dicha información no le ayudaba mucho a catalogar a Alma Espejo en clase alguna de categoría. Nada podía deducir de ella que le sirviera para identificarle a él o a sus intenciones. Eran muchas las personas que a lo largo de los años había conocido que, ante las mismas circunstancias vitales, habían decidido transitar por senderos del porvenir completamente divergentes.

~---De todos modos debo asumir que si recurre a mi ayuda ~---prosiguió Winston con tranquilidad, rascándose la barbilla con el índice y el pulgar~--- es porque puede ofrecerme nuevos datos que me ayuden en mis indagaciones y pesquisas.

~---Supone bien ~---agregó Scream sin dejar de mirarle fijamente~---. Intentaré darle los mayores detalles que me sea posible, pero esencialmente creemos que el individuo tuvo alguna clase de vínculo en el pasado con un miembro importante del Tribunal Superior.

~---Ya veo. Imagino que se debe tratar del Juez Supremo Nitram.

Scream pensó si no habría alguna posibilidad de convencer a Winston y al resto de los Esclarecedores para que formaran parte de la organización. Pero no tardó en asumir que más que un sentimiento de justicia poética para los tres detectives aquello era una forma de vida en sí misma.

~---Pongamos que se trata de ese hombre ~---dijo Scream de nuevo, jugando una vez más a aquel juego que empezaba a adquirir peculiares tintes de bizarrismo~---. ¿Pero por qué cree que se trata de él?

~---Una cierta sospecha sobre nuestro individuo que usted mismo fue fraguando en mi cabeza con sus comentarios. Pero no se preocupe, haré todo lo que pueda y se lo explicaré con calma cuando obtenga algún resultado.

A Scream no se le pasó por alto que hablaba no como si fuera a intentarlo, sino como si estuviera seguro de conseguirlo pero lo único que no acababa de tener del todo claro era el momento exacto en que iba a lograrlo.

~---Perfecto, entonces. Sobre sus honorarios\dots

~---No se preocupe por eso, Ten Scream. Considérelo una retribución por habernos ayudado cuando nos encontramos en SKF.

~---Como le parezca, entonces. Pero me sentiré en deuda con usted si no acepta que le pague.

~---Tengo la certeza de que el hecho de que esté dispuesto a ayudar a mí o a mis colegas en un hipotético futuro es mil veces más valioso que cualquier tarifa que tuviera a bien sugerir.

~---En cuanto a una vía de comunicación para hablar conmigo\dots

~---Póngase en contacto conmigo dentro de cinco horas, Ten Scream.

~---¿Tan poco tiempo?

~---Cuando trabajo a contrarreloj y bajo presión cada hora de mi tiempo es tan productiva como días enteros ~---explicó Winston justo antes de cortar la transmisión.

\parbreak
A miles de millones de kilómetros de distancia de Ernépolis~I, en un lejano satélite, un grupo de cuatro guardianes vigilaba la entrada a una enorme prisión. Se encontraban al aire libre, y aunque la atmósfera circundante no era tan espesa como para que el cielo dejara de ser negro y coronado por una infinidad de estrellas, tampoco era tan escasa como para tener que llevar equipos de supervivencia en condiciones climáticas adversas. La temperatura ambiente no era especialmente desagradable, tampoco. Podría compararse con una noche fría en alguna ciudad no especialmente contaminada del lejano y añorado planeta Tierra.

La labor que desempeñaban era, como poco, monótona y aburrida, lo que podía considerarse todo un privilegio en el entorno en el que se movían. Debían limitarse a vigilar unas enormes compuertas que estaban a sus espaldas, robustas y pesadas, y no abrirlas jamás, bajo ningún concepto, pasara lo que pasara. Ésa era la única nota de color en su labor, el único resquicio de emoción en lo relativo a su cometido. Al parecer las compuertas estaban en una zona famosa por su tendencia desconocida para crear espejismos visuales, y muchas veces los vigilantes fueron testigos de toda clase de extrañas visiones, como animales que no eran autóctonos del lugar o lluvia en un lugar en el que no había ni siquiera posibilidad de que se condensaran nubes sobre sus cabezas.

~---No me gusta esta labor ~---decía siempre uno de ellos, el que curiosamente más tiempo llevaba desempeñándola y no había pedido ser trasladado a alguna clase de tarea distinta~---. No puede ser bueno lo que está al otro lado de esas compuertas.

~---Eso es lo de menos. Si fallamos en nuestra labor casi será mejor para nosotros que nos mate lo que sea que escondan ~---se limitó a agregar su compañero.

De repente, del extremo contrario, vino uno de los guardianes de la otra pareja, más alejados de su posición, corriendo y jadeante.

~---Están abiertas ~---dijo, entre preocupado y asustado~---. Las compuertas están abiertas.

Al principio no podían creer lo que estaba diciendo su compañero de armas. No hacía ni diez segundos que habían dejado de mirarlas. Pero en efecto, levantaron la vista y las encontraron abiertas ligeramente, apenas medio metro, pero suficiente para que por allí pudieran salir todos los demonios del infierno si se apretujaban lo suficiente.

Los cuatro guardias se acercaron a la compuerta y la tocaron con sus propias manos. Era menos pesada de lo que nunca habían imaginado. De hecho apenas un empujón entre dos de ellos bastó para cerrarla de nuevo, pues corría sobre un mecanismo de engranajes muy nuevo y, al parecer, lubricado con esmero. Ignoraban qué podían haber pasado, pero trataron de hacer como si no hubieran detectado nada anómalo, con la esperanza de que todo siguiera tal y como estaba.

No fue así, por supuesto. No tardaron en enterarse de que habían fracasado en su labor, y lo que fuera que estuviera al otro lado de las compuertas había escapado sin que lo vieran en ningún momento. Por un momento pensaron en huir, marcharse de allí antes de que él llegara. Pero luego pensaron que no tenía sentido hacer algo así, y además, no había sido su culpa que aquello pasara.

Cuando él llegó en persona a inspeccionar qué era lo que había pasado comprendieron demasiado tarde que sería sorprendente que salieran de allí con vida. Llegó en una nave y descendió, acompañado de algunos de sus generales más leales. Se acercó a las compuertas y las examinó con su fría y gélida mirada, que ponía los pelos de punta a la mayoría de sus subordinados.

~---Como puede ver las cerramos en cuanto pudimos ~---explicó uno de ellos, tratando de justificarse.

Su líder se giró hacia él, como quien se gira hacia un mosquito que interrumpe sus pensamientos, y le agarró del cuello, levantándole en vilo.

~---No suelo enfurecerme a menudo, y no me gusta dar muestras innecesarias de tiranía y mano dura entre mis subalternos, pero este fracaso no puede pasar sin un castigo ~---se limitó a explicar lanzándole al suelo con violencia.

~---Pero señor\dots\ las compuertas\dots

~---Las compuertas están abiertas, soldado.

El vigilante miró hacia las compuertas. No lo entendía. Estaban cerradas. Cerradas a cal y canto.

~---No hicisteis caso a las instrucciones, ¿no es así? Se os dijo que no os acercárais a ellas bajo ninguna circunstancia, ¿o no?

~---Sí, señor, pero...

~---No. No más excusas. Habéis desobedecido una orden directa mía. Y ya sabéis lo que eso significa.

Hizo un gesto hacia los dos generales que le acompañaban, y éstos se acercaron a los desdichados vigilantes. Aunque al principio hubo protestas y súplicas agónicas, no tardaron en ser rápidamente silenciadas, sin que, por extraño que pueda parecer, ningún arma de clase alguna mediara en el conflicto.

Su líder y comandante se limitó a mirar las compuertas abiertas, ignorando tanto los cuatro cadáveres desperdigados alrededor de ellas como a sus dos ejecutores, igual que si el Universo ya no existiera a su alrededor. No solía ser tan expeditivo ni cruel, pero aquel asunto era de crucial importancia y podía hacerle perder los estribos más de lo que jamás lograría ninguna otra circunstancia. Por un lado porque era un entorpecimiento a sus planes de permanecer en letargo un tiempo más, y por otro porque constituía su mayor punto débil, una tragedia de carácter personal.

~---Puedo imaginar cuál es el lugar al que has ido. Y puede que tu decisión sea fatal para muchos, entre ellos yo mismo ~---se limitó a declamar mirando al firmamento, sus ojos rojos refulgiendo en dirección a la impoluta bóveda celeste.
