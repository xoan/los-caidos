\begin{prev}
    Felicity Hound revela a John Scream que la armadura de Armor es más que un simple prototipo militar robado, pues ha sido infectada con un virus informático. Eso la convierte en completamente impredecible y, además, letal para los que la rodean, por lo que los Caídos estarán dispuestos a toda clase de tratos con tal de detenerla cuanto antes\dots
\end{prev}

\noindent{}El momento de las respuestas había llegado. De saber más sobre su enemigo, sobre sus motivaciones. Aunque ese conocimiento tan largamente buscado trajera aparejadas terribles revelaciones\dots\\

\noindent{}Como parte del plan de aproximación de Los Caídos hacia Armor, el que se había convertido por derecho propio en su enemigo más peligroso e incontrolable hasta la fecha, James Sky se dirigió directo hacia la base militar en la que se había asentado el Coronel Straxus. No le gustaba lidiar con aquel sujeto que decía mucho y ocultaba aún más, pero sabía que era su misión ineludible. Del mismo modo que Scream se movía entre recovecos de dudosa legalidad, él hacía equilibrios, dado su puesto de Jefe de Policía, a lo largo de una delgada línea entre el deber y su responsabilidad moral y personal. Una línea que resultaba complicada de trazar, y más aún de recorrer una vez había sido delimitada.

A pesar de la insistencia de sus hombres prefirió ir solo, conduciendo él mismo el deslizador patrulla. No estaba seguro de qué clase de argumentos tendría que emplear para convencer a Straxus, y por ese motivo prefería gozar de libertad de palabra y actos.

Además, si el plan salía bien, era mejor que ninguno de los suyos estuviera con él en el momento crucial, y mucho menos armado.

Aparcó cerca de la entrada vallada del complejo, y nada más salir del vehículo un par de cámaras y otros tantos láseres le apuntaron como si fueran animales callejeros y acabara de silbar para llamar su atención. Se limitó a quedarse quieto, esperando alguna clase de reacción hospitalaria por parte del servicio de seguridad, hasta que un par de soldados, armados de igual modo que sus compañeros caídos, aparecieron al otro lado, le cachearon y tras quitarle su arma reglamentaria le indicaron que pasara. No tuvieron que caminar mucho hasta que llegaron al despacho del Coronel, lleno de papeles y toda clase de bocetos y diseños que Sky no dudó en identificar como pertenecientes a la extraviada servoarmadura.

Sky se sentó al otro lado de la amplia mesa. Straxus apenas reparó en su presencia. Desde las ventanas del fondo la Nube ofrecía un crepúsculo a medias ceniciento, a medias rojo sangre, como si quisiera recordar que era la muerte en los cielos la que le ofrecía ese aspecto.

El Coronel no levantó la vista de los papeles, pero comenzó a hablar.

---Es como si se hubiera esfumado ---dijo con tono de in\-cre\-du\-li\-dad---. Como si no tuviera interés en mostrarse.

---Se refiere a la servoarmadura, ¿verdad?

El Coronel levantó la vista de su legajo de papeles viejos y maltratados por multitud de manchas de café de madrugada y miró a Sky con ojos nuevos.

---Veo que ha estado investigando por su cuenta, Jefe Sky.

---Tengo la misión de conocer todo lo que pasa en la ciudad, Coronel, pueda o no hacer algo para impedirlo. Y los rumores vuelan. Por mucho camuflaje óptico y sensorial que tuviera su artefacto ya hay quien le ha visto deambular como un Frankenstein desbocado por las calles de Ernépolis~I.

---Pero es que el Proyecto Armor no es un Frankenstein, ni mucho menos. Ojalá, y discúlpeme por ser tan crudo, ojalá fuera una implacable máquina de matar desquiciada. Contra esa clase de amenazas sabría bien lo que hacer. Sacaría los tanques, equiparía a todos mis hombres con pesado y contundente armamento. Pero está jugando al escondite con nosotros y no tengo ni idea de qué es lo que pretende con esa actitud.

---Tal vez no pretenda nada. Tal vez se haya vuelto loco.

Sky dudaba mucho de esa afirmación, pero lo que quería era provocar a Straxus, de una manera muy distinta a como hubiera hecho Scream de estar en su lugar. Él tenía que jugar a ser el elemento comprensivo, el simbolismo de la fuerza del orden.

Una máscara que ya no tenía claro hasta qué punto era parte de su piel.

---Entonces es una locura selectiva la suya, sin duda. Porque no ha atacado más que cuando no ha tenido más remedio, como si pretendiera pasar desapercibido el mayor tiempo posible. Como si mis hombres fueran más un incordio que el objetivo real.

---Creo que es hora de que seamos claros, Coronel. De que me diga a qué nos estamos enfrentando.

Straxus se levantó y se acercó hacia una de las ventanas, con los brazos a la espalda, tratando de afianzarse en su postura de confianza. Sky no tuvo que buscar su funda para darse cuenta de que, como siempre, estaba armado. Aquel hombre, pensó en un instante de lucidez, debía dormir con el arma bajo la almohada, o poco menos.

---Es una servoarmadura, en efecto. Capaz de resistir incluso los misiles de una lanzadera de combate. Posee una presión de una tonelada por centímetro cuadrado en cada uno de los dedos de sus guanteletes. Puede literalmente aplastar cráneos como quien aplasta una pajarita de papel.

\rquoti{}Además de eso, está especializada en combate cuerpo a cuerpo y en manipulación de la electricidad. Puede acceder por contacto a numerosos sistemas para utilizarlos en su provecho y posee un camuflaje óptico y energético de última generación.

---¿Qué hay de posibles puntos débiles?

---Necesita grandes cantidades de suministro, pero hasta ahora ese detalle no ha parecido detenerle. Creemos que, de algún modo, ha sido perfeccionada.

---¿Perfeccionada?

---Su autonomía nos tiene poco menos que sorprendidos, y quién sabe qué más cosas pueden haber cambiado en su diseño.

---¿Qué sugiere, entonces?

---¿Sugerir? Esto no es un juego de prueba y error, Jefe Sky. Hemos creado una armada de un solo miembro, una perfecta máquina de matar que incluso ha sufrido notables mejoras.

---¿Qué hay de su portador?

---Tenemos nuestros sospechosos, pero no podemos hablar al respecto de ello.

De repente todas las luces se apagaron y el despacho se quedó a oscuras salvo por la escasa luz artificial que provenía del exterior de la base. Se activaron los sistemas de emergencia y gran cantidad de soldados empezaron a moverse a lo largo de los pasillos.

---¿Todo bien, señor? ---preguntó uno de los hombres abriendo de repente.

---¿Qué ha ocurrido, soldado?

---Parece que alguien se ha infiltrado en la base, señor.

---Infórmenme de cualquier novedad.

---Así lo haré, señor ---dijo el soldado cerrando tras de sí.

---No sé lo que ha ocurrido, Jefe Sky, pero tenga por seguro que\dots

Fue en ese momento cuando Straxus reparó en el pie que asomaba del otro lado de la mesa, y vio a Sky inconsciente en el suelo.

---¡Soldado! ---gritó sacando el arma de su funda.

\emph{No le oirá, Coronel. Pero no se preocupe. No vengo a pelear, sino a hablar.}

---¿Por qué debería fiarme de tu palabra? ---replicó el Coronel buscando un objetivo al que apuntar.

\emph{Porque los dos vamos tras el mismo objetivo. Yo tampoco quiero que una bomba de relojería campe a sus anchas por las calles de mi ciudad.}

Straxus trató de llegar a la puerta, pero una sombra se interpuso en su camino. Antes de que pudiera tan siquiera plantearse apuntar, le arrebató el arma y se colocó al instante lejos de su alcance.

---¿Cómo haces eso? No puedes ser un hombre.

\emph{Soy mucho más que eso, Coronel Martin Straxus. En estos momentos creo que usted y yo podemos entendernos mejor de lo que piensa.}

---¿Qué quieres decir?

\emph{Usted dice que Armor, como nuestro enemigo ha dado en denominarse a sí mismo, necesita grandes cantidades de energía para moverse por su cuenta.}

---Sí, así es ---contestó Straxus, inquieto---. El Proyecto Armor en principio estaba diseñado sólo para incursiones puntuales.

\emph{¿Qué clase de energía?}

---Eléctrica, principalmente.

\emph{Siga ese rastro, entonces. Busque trazas de energía residual en el ambiente, o picos desproporcionados en la escala de algún sector de la ciudad.}

---¿Qué crees, que nacimos ayer? Sabemos lo que buscar, pero no hemos encontrado ningún indicio como los que indicas.

\emph{¿Qué hay de su escuadrón, el que se enfrentó a Armor?}

---¿Cómo sabes eso?

\emph{Conteste, Coronel.}

---No les hemos encontrado. Creemos que los mantiene como rehenes.

\emph{¿No poseían alguna clase de dispositivo rastreador?}

---Todos han sido inutilizados.

La sombra calló. Por un momento parecía que ya no estuviera allí.

---¿Cómo pretendes encontrarle?

\emph{Hasta ahora, Coronel, ha estado buscando la armadura. Ha llegado el momento de cambiar la directriz de búsqueda, y buscar a la persona debajo de la armadura.}

---¿Por qué debería confiar en ti?

\emph{Porque no le queda más remedio, Coronel. Precisamente por eso.}\\

\noindent{}Una vez se restableció el suministro eléctrico en la base militar y Sky recuperó la consciencia, lo primero que hizo fue enfilar hacia el Aquerón, donde esperaba que le pusieran al día de la conversación. Era lo malo de ser noqueado por tu mejor amigo para mantener un perfecto y orquestado engaño.

John Scream estaba reunido con varios de los miembros de Los Caídos, en concreto aquellos que, mientras fueron héroes, habían basado su lucha contra el crimen en términos de investigación en vez de meramente físicos. En aquel momento, más que nunca, su capacidad deductiva era sin duda necesaria.

Una vez se personó ante Scream éste dio por finalizada la reunión y puso a Sky al corriente de la nueva información obtenida.

---No es mucho lo que ha dicho Straxus, pero creo que no hay mucho más que podamos averiguar por su parte. En todo caso, aunque supiera los puntos débiles de Armor, no se los diría a nadie, y menos a un proscrito que se esconde en las sombras.

---¿Qué haremos si averiguamos algo? ¿Iremos por nuestra cuenta?

---Es tentador, pero no nos conviene en este caso. Nuestros dos puntos fuertes, el sigilo y el temor, son inútiles contra Armor. Por otro lado el problema del ejército es que no conoce este terreno, y confía única y exclusivamente en el uso de la fuerza para atajar el problema. Aparte, jamás confiarán en nosotros para dirigir un asalto, entre otras cosas porque ignoran que exista un \emph{nosotros}.

---En ese sentido mis hombres podrían ser útiles ---apuntó Sky.

---Sabes que habrá bajas si mandas allí a tus policías.

---Ellos serían los primeros que querrían estar allí, sin importar la gravedad de aquello a lo que se enfrentasen. Además, ahora mismo yo soy el enlace perfecto entre Straxus y tú. Si le digo que he obtenido información nueva, no se le ocurrirá ponerla en duda.

---Pero puede que sí sospeche que tú eres parte de la organización.

---Straxus ni siquiera sabe que hay una organización.

---Puede llegar a sospecharlo. Ya cuando le acechamos se sorprendió de que un solo hombre pudiera hacer todo eso. Es científico. Podría atar cabos.

---Tendremos que arriesgarnos ---sentenció Sky.

Pero antes de eso, tendrían que jugar a los detectives y seguir la línea de investigación que el propio Scream había sugerido al Coronel. Si era cierto que Armor necesitaba grandes cantidades de combustible para mantenerse operativo, entonces sólo cabían dos posibilidades: o bien conseguía combustible de alguna manera clandestina, o bien alguien lo conseguía por él. Y salvo aquel primer encuentro donde asesinó a un escuadrón de la organización en su práctica totalidad, no habían vuelto a verle. Scream sospechaba que el camuflaje que poseía, si bien era efectivo, requería un enorme sacrificio de recursos, motivo por el que no estaba paseándose por la ciudad todo lo que hubiera deseado.

Por otro lado había una posibilidad, aunque remota, plausible y difícil de comprobar. ¿Y si aquel o aquella que estaba tras la servoarmadura se había quedado sin pilas para su juguete nuevo? En ese caso, no hubiera sido extraño que hubiera tratado de deshacerse de él, o peor aún, de venderlo.

Sin embargo, Scream no olvidaba la información que Felicity Hound había compartido con él, y cuyo dato más importante era que la servoarmadura había sido infectada con un virus. Un importante detalle que no podía pasar por alto e ignoraba en qué sentido podía estar alterando las reglas del juego.

Las horas pasaron fugaces, como si fueran minutos. Numerosos informes llegaban de todos los escuadrones al Aquerón, pero ninguno aportaba datos que pudieran resultar de alguna utilidad. El equipo de investigación seguía trabajando, pero no obtenía resultados visibles. Había demasiados lugares sospechosos, demasiados posibles escondites para aquel armamento humanoide, y Scream sabía que investigarlos uno por uno sólo serviría para sufrir la baja de alguno más de sus escuadrones.

El cansancio empezaba a hacer mella en sus facultades y las de su grupo, aunque era consciente de que Armor estaba también contra las cuerdas. Podía creerse invencible, pero esconderse era parte crucial de su estrategia. Y si bien tenían la clara sospecha de que Eileen Drift se encontraba debajo de aquella piel de metal, tampoco ella había dado señales de vida, ni sabían en qué estado podía encontrarse, tal vez con el cerebro quemado por el virus, o enajenada por los poderes de un traje de combate más allá de toda imaginación, puede que incluso mejorado por ella misma, ya que había formado parte del equipo que lo diseñó.

¿Y qué pasaba con esos rehenes\dots{} para qué los quería? ¿Acaso temía que lo acorralaran? Llegado el momento podían resultar más entorpecedores que útiles, si pretendía pasar desapercibido.

---Eso es ---dijo Scream en voz alta---. Swipe, busca denuncias de desapariciones desde el día del accidente.

---Al momento ---contestó Swipe, tecleando en la base de datos del cuartel.

---¿Qué se te ha ocurrido? ---preguntó Sky, intrigado.

---Ignoro para qué puede haberse llevado Armor a aquellos soldados, pero ¿y si no han sido los únicos? ¿Y si necesita más rehenes? Es una pista débil, pero podría funcionar.

---Ha habido un total de doce desapariciones que se tenga constancia. Cuatro son de criminales que nosotros mismos hemos capturado y encerrado en las celdas del Aquerón, a la espera de mandarlos a disposición de las autoridades.

---Para que no tarden en salir de nuevo en la mayoría de los casos ---añadió Sky con amargura.

---De las otras ocho, cinco han sido a lo largo de los Túneles, y tres en el distrito bajo del acceso noroeste a la ciudad.

---Olvida las de los Túneles ---comentó Sky de nuevo---. Esa cifra encaja con las estadísticas.

---Pero no con las del acceso noroeste ---acabó Scream, mirando a la pantalla---. Además, las desapariciones se han producido linealmente, casi se podría trazar un segmento perfecto que las uniera.

---¿Qué sugieres, John? ¿Qué iba secuestrando gente a medida que avanzaba?

---Sólo hay una manera de averiguarlo. Nosotros tenemos una sospecha. Ellos tienen la capacidad de confirmarla con sus detectores energéticos.

---Llegó el momento de movernos, entonces ---supuso Sky, pensativo.

---Así es. Contacta con Straxus y dile que has detectado actividad sospechosa en la zona. Llegó el momento de sellar el pacto largamente postergado.\\

\noindent{}James Sky avanzó con sus hombres a lo largo de las mugrientas y sombrías calles aledañas a la autopista deslizante de Arnápolis~VI. Al ser aquella una zona llena de tramos altos de asfalto eclipsados por casas bajas, resultaba ser un laberinto en miniatura ideal para toda clase de trapicheos de dudosa legalidad.

Aquella noche, sin embargo, los yonquis y estraperlistas fueron rápidamente desalojados de las inmediaciones del lugar y multitud de policías y soldados empezaron a registrar, uno por uno, todos los amplios y abandonados almacenes que fueron encontrando en su camino, sin aparente resultado.

Sky dio la enésima patada a un edificio sombrío de tres plantas, sólo para encontrar que estaba vacío como el estómago de un lobo hambriento en mitad del desierto.

---Aquí Sky, despejado.

---Despejado ---fueron contestando uno por uno todos sus hombres.

---Sky a Straxus, nada.

---Siga buscando ---contestó Straxus al momento---. Las lecturas son claras pero poco precisas. La servoarmadura está por esta zona.

Aun encontrándose cerca del trofeo, Sky era consciente de que la búsqueda bien podía acabar siendo infructuosa. No era fácil buscar una armadura como aquella, especialmente diseñada para misiones de comando y asalto de un solo hombre contra todas las fuerzas del enemigo, apiñadas y parapetadas en alguna lejana y perdida fortaleza colonial.

---Jefe ---escuchó decir a uno de sus hombres, un poco por delante de él, asegurando la zona---. Creo que tengo algo.

---Atención, todos listos, posible localización del objetivo en nuestras coordenadas.

Sky se acercó a su subalterno y éste señaló un pasadizo angosto y estrecho formado por escaleras de piedra descendentes. Mal asunto, pensó en cuanto lo vio. Abajo era probable que estuvieran en clara desventaja.

---Atención, busquen túneles, trampillas, o lo que sea que se le parezca ---advirtió por el comunicador---. Tenemos que cortarle la salida. Una vez abajo no usen comunicadores, repito, no usen comunicadores. Nuestro objetivo puede pinchar nuestras señales. Revisad la zona en busca de otros posibles accesos, me quedaré aquí asegurando la posición ---dijo a su equipo.

Nada más estuvo solo, Sky hizo un gesto al tejado y una sombra apenas perceptible se deslizó hasta tocar con suavidad el suelo.

---Es una locura, John. Te matará.

\emph{Tengo que arriesgarme. Si logro llegar antes que los militares tal vez pueda atrapar al portador de la armadura antes de que haga uso de ella. Sé que somos un equipo, pero si nos desplazáramos en grupo ahí abajo seríamos fácilmente detectables, además de movernos más lentamente de lo que podría hacerlo por libre.}

---Ten cuidado ---dijo dejándole paso, justo antes de que llegaran sus hombres.

---Han encontrado más accesos y los soldaditos de plomo están descendiendo. ¿Todo en orden, jefe?

---Todo en orden. A partir de aquí es cosa de ellos. Nosotros nos limitaremos a asegurar la zona.

---A la orden, jefe.

Sky miró a un lado y otro, sin apartarse nunca de la entrada que hace un momento acababan de descubrir.

---Me hago viejo para jugar a ser agente doble, John.

\emph{Lo sé, amigo, lo sé.}\\

\noindent{}A medida que John Scream descendía por el estrecho acceso notó un fuerte olor a cerrado que invadía la escasa atmósfera respirable del lugar. Se planteó si no tendría que usar la mascarilla para poder avanzar, aunque no tardó en darse cuenta de que el hedor, más que peligroso, resultaba molesto, cercano a lo insoportable, pero en todo caso, cuando fue primero un héroe, y después un vagabundo en las mugrientas calles de Ernépolis~I, había tenido que padecer efluvios similares.

A medida que avanzó, sin embargo, el olor empezó a volverse más y más fétido y difícil de ignorar, y una terrible sospecha empezó a cruzar por su mente. Pero aun con todo, continuó avanzando, con la esperanza de que los militares no tuvieran un estómago tan entrenado como el suyo y algún que otro indispuesto retrasara la comitiva de bienvenida, otorgándole unos valiosos segundos de ventaja.

Las primeras bifurcaciones no tardaron en llegar, y algunas de ellas eran bastante angostas, aunque en todo caso de dimensiones más que suficientes para que pasara a través de ellas, agachada eso sí, una servoarmadura de alrededor de dos metros y medio de altura. No lo tuvo muy difícil a la hora de elegir el camino más adecuado en cada cruce que fue encontrándose.

En concreto, no tuvo más que ir siguiendo la ruta que irradiara el olor más penetrante.

De un rápido vistazo Scream notó que muchos de esos pasadizos eran de creación reciente, seguramente abiertos por aquella apisonadora humana en días anteriores. No porque poseyera una gran capacidad de deducción, sino porque siempre estuvo presente en todas las ampliaciones de túneles que se hicieron en el Aquerón desde que Miles murió y él tomó el mando.

Eso quería decir que había estado invirtiendo grandes cantidades de tiempo y esfuerzo en la elaboración de esos pasadizos, lo que dejaba en el aire dos preguntas. La primera era de dónde había sacado la energía necesaria para llevar a cabo semejante tarea, y la segunda con qué finalidad la estaba efectuando.

Ese comportamiento confirmaba sus peores sospechas. Armor podía ser una fuerza devastadora, pero no empleaba sus capacidades arbitrariamente. Tenía un plan, una directriz bien marcada, que obviamente tenía que ver con la mente que se ocultaba tras aquel siniestro yelmo de blanquecino e insondable fulgor.

Empezó a oler a quemado, y no tardó en concluir que estaba muy cerca de llegar al final del túnel. Sin embargo, antes de hacerlo, llegó a un ensanchamiento similar a una gruta en el que fue testigo de un espectáculo absolutamente dantesco.

A un lado yacían un montón de cadáveres, en avanzado estado de descomposición. Entre huesos que sobresalían de los jirones de piel y ropas podridas podían contarse como poco seis o siete personas, algunos de los cuales eran indudablemente soldados, y muy probablemente los desaparecidos. Scream se planteó si no los habría explotado para ampliar el túnel y luego prescindido de ellos una vez consideró que ya no podían servirle de mano de obra debido a la fatiga extrema. Era una posibilidad, pero en aquel momento no era la fuente principal de su incertidumbre.

Ese honor estaba reservado a la armadura que estaba apoyada contra la pared del fondo de la cueva, junto a un nuevo estrechamiento del túnel.

Estaba vacía, de hecho la manera que tenía de abrirse era más que inquietante, pues parecía como si hubiera eclosionado, igual que si una vaina hubiera reventado, o alguien hubiera sido abierto en canal. La armadura, lejos de estar formada por varias partes, era una sola pieza abierta por la mitad. Era un diseño muy inteligente, sin duda. La perdición de las armaduras medievales había sido las junturas por donde se podían ensartar flechas y otra clase de proyectiles, un error que al parecer no se repetía en esa ultramoderna versión.

Scream estaba a punto de continuar por el túnel, en busca del dueño de la armadura, cuando escuchó llegar a los militares. Decidió quedarse para evaluar su reacción ante el escenario, y así saber hasta qué punto esperaban encontrar algo parecido.

En cuanto dos de los soldados entraron uno de ellos, al mirar la pila de restos humanos, no pudo evitar vomitar. El otro, sin embargo, mantuvo la templanza suficiente como para apuntar a la carcasa vacía que yacía al fondo.

---Suficiente, soldado ---ordenó Straxus, evidentemente sa\-tis\-fe\-cho---. Parece que hemos llegado a tiempo. La traidora puede huir, pero sin la servoarmadura en su poder ya sólo será un problema secundario.

\emph{Cuidado, Coronel} ---dijo Scream saliendo de las sombras---. \emph{Algo no encaja en todo esto.}

Los soldados apuntaron a Scream como si fuera un pato de feria, pero a la orden de Straxus bajaron las armas.

---Tranquilos, chicos. De no ser por él no estaríamos aquí.

Scream no dijo nada.

---Suponía que fuiste tú quien filtró la información a la policía. Te gusta usar a los demás como si fueran marionetas.

\emph{Sólo hice lo que consideré más rápido para ponerle sobre aviso.}

---¿Crees que soy estúpido? Nos necesitabas, por eso avisaste. Tal vez incluso planees quedarte con la servoarmadura.

\emph{No es de mi estilo} ---añadió Scream tajante, volviéndose hacia los cadáveres en lo que Straxus hacía lo propio hacia la armadura. Con semejante gesto quedó claro para los demás presentes cuál era la prioridad de cada uno de esos dos hombres.

\emph{Ahora que tiene su juguete, Coronel, supongo que se largará de mi ciudad.}

---Aún es pronto para decirnos adiós, en realidad. No hasta que Eileen Drift haya aparecido y sea llevada ante una corte marcial.

Scream siguió mirando los cadáveres, e hizo un descubrimiento que acabó por desconcertarle del todo.

\emph{Hay ropa de mujer aquí, Coronel.}

---¿Qué dice?

\emph{¿Qué ropa llevaba puesta Drift cuando el accidente?}

---¿Por qué cree que\dots{}?

\emph{Esa mujer robó su preciada armadura, Coronel. Apuesto a que podría decirme cuántos lunares tiene en la cara si se lo preguntara.}

---Llevaba una falda de cuadros, una blusa azul y una bata blanca.

Sin pensarlo ni un segundo, Scream metió las manos entre los restos humanos y encontró jirones de una prenda larga y cuarteada que, sin duda, pudo haber sido una bata de laboratorio. Sin embargo, estaba notablemente chamuscada.

\emph{Eileen Drift está muerta, Coronel Straxus. Y no sé qué quiere decir eso.}

---¿Entonces, quién\dots{}?

Scream ya lo estaba empezando a sospechar. Pero era una sospecha tan descabellada que hasta que no vio cómo los enormes brazos se movían por sí solos y atrapaban al Coronel, arrastrándolo hacia el interior de la armadura abierta como una crisálida, no tuvo la completa noción de que habían caído en una trampa. Una sospecha, en realidad, que contestaba a dos de las preguntas que se había estado haciendo a lo largo de esos días.

La primera respuesta era que nadie, en realidad estaba manejando la armadura. Nadie vivo, al menos.

La segunda respuesta era que el cuerpo humano era una perfecta pila voltaica llena de gran cantidad de impulsos eléctricos.

La servoarmadura se cerró sobre sí misma aprisionando al Coronel igual que una planta carnívora atrapa una mosca. Los pliegues se retrajeron de una manera tan orgánica, tan endiabladamente asimétrica, que muchos de los soldados se quedaron paralizados, incapaces de reaccionar.

Después de eso el visor blanco brilló con plena intensidad y Armor estuvo de nuevo recargado por completo.

Los soldados le apuntaron y comenzaron a disparar a discreción, pero el monstruo interpuso su escudo de energía entre ellos y la lluvia horizontal. Scream era consciente de que el ser se estaba conteniendo, y esa actitud le asaltó por dentro de dudas. Armor tenía un plan, un esquema trazado, y no podía permitirse un gasto extra de energía para deshacerse de unos cuantos soldados.

Aun así, tenía que seguir apelando a su condición de ser vivo, aunque no fuera un ser humano.

Y todos los seres vivos, de una u otra manera, desean expresarse, comunicarse.

\emph{De modo que todo era un disfraz. Eileen Drift nunca controló la armadura. La armadura se controlaba a sí misma.}

<<No te gastes, humano>> contestó Armor con aquella voz metálica, antinatural. <<La científica fue muy útil para mí. De no ser por ella nunca hubiera logrado salir por mi propio pie del accidente>>.

Los soldados se quedaron sin balas. Armor les disparó una descarga eléctrica que les lanzó al suelo, donde se empezaron a convulsionar entre estertores incontrolables. Scream fue consciente de que era un ataque de baja intensidad, sólo destinado a detenerles temporalmente. Reservaba su arsenal al milímetro.

Scream trató de apelar a su previsible sentimiento de superioridad.

\emph{Podrías arrasar todo lo que te rodea, y en vez de eso has elegido esconderte como un gusano} ---apuntó con desprecio---. \emph{Juntos podríamos dominar esta ciudad.}

<<Pobre criatura de carne. Tu ciudad es todo lo que te importa, de más maneras de las que quieres reconocerte. No tengo necesidad de unirme con nadie, y ya me he entretenido suficientemente contigo>>.

Se movió hacia la salida del fondo y una vez allí apuntó al techo, con la clara intención de sepultarse a sí mismo al otro lado.

Scream sintió el peso de la derrota hasta tal punto que no dudó en proclamar una indiscutible osadía.

\emph{Si tan listo eres, ¿por qué no me matas aquí mismo? Sabes que soy una amenaza para ti.}

<<Tus palabras no están exentas de sabiduría. Pero sé quién eres, John Scream, y si no te hago daño, es porque aún podrías serme útil>>.

Después de aquella declaración de intenciones el ser disparó contra el techo, poniendo entre ambos una extensa barrera impenetrable de roca. Para cuando los demás soldados se atrevieron a entrar, Scream ya se había ocultado en las sombras.

El único lugar en el que podía reflexionar en silencio sobre la amargura del fracaso.

\begin{next}
    ¿Cuáles son los planes de Armor? ¿Qué clase de ataque planea contra los humanos? ¿Y por qué John Scream podría ser útil para él? ¡No te pierdas la impactante conclusión y el enfrentamiento definitivo contra esta máquina de destrucción masiva!
\end{next}

\endinput
