Detrás de la competición había muchos intereses ocultos por gran parte de los observadores. Había quienes no confiaban en el resultado del mismo y también quien sólo lo usaba de trampolín hacia la libertad. Había quien lo vigilaba escondido, al margen, permaneciendo ajeno a todo el proceso.

Y finalmente, había quien lo había convertido en una pieza crucial de toda su estratagema\dots

\fancyparbreak
~---De modo que dices que acabó con la vida de tu hermano ~---comentó Scream, de pie en el hemiciclo de la sala de reuniones del Aquerón, mirando la pantalla donde se estaban mostrando todos los datos que habían logrado encontrar del sujeto mencionado.

Junto a él estaba James Sky, en la que era una de las primeras veces que regresaba a su antiguo hogar desde que lo dejara para centrarse de manera exclusiva en su cargo oficial. Miraba al rostro de Shockman, a su ojo vacío e inerte, y un sentimiento de profundo desagrado le recorría por dentro.

~---Me amenazó con que acabaría con mis seres queridos, que nunca les encontrarían. Y así fue, en efecto. Éxeter era astuto. De ese modo, no podía acusarle oficialmente de nada, y si lo hacía sólo tendría que hablar y decir quién era yo en realidad. Luego de eso fue cuando apareció muerto, y no mucho más tarde llegó la era de Ellen Gorgon.

~---¿Cómo ha podido sobrevivir?

~---No lo sé. A diferencia de otros enemigos a los que nos pudimos enfrentar en nuestras\dots\ vidas pasadas, Shockman no hacía alarde de sus poderes. Los usaba de manera indirecta. Siempre estaba detrás de los planes de muchos otros de mis enemigos, conspirando, en la sombra. En realidad, creo que no le movían las mismas ansias megalomaníacas que a sus compañeros de fechorías.

~---En todo caso tampoco podemos acusarle de nada concreto. Sigue sin haber cometido ningún delito que podamos probar. Pero si se ha dejado caer de nuevo por la ciudad y, más aún, presentado al torneo, será porque espera que pueda serle útil de alguna manera que no somos capaces de deducir de momento.

~---Aun así, si su idea consistía en ser el vencedor, creo que sus planes van por mal camino. El Presidente ya me ha notificado que ha tomado una decisión con respecto al ganador del concurso, pues era él quien tenía el veredicto final al respecto. Y me sorprendería mucho que le eligiera para representar el emblema de la ciudad.

~---Quiere marcarse el tanto hasta el final ~---comentó Scream, apagando la pantalla~---. Saw nos ha dicho que habrá una suerte de ceremonia oficial donde le presentará a la población y además le harán entrega del guantelete.

~---¿Por quién apuestas tú, John?

~---Tú primero, James ~---dijo con tono de burla, usando también su nombre de pila deliberadamente.

~---Barro para casa y apuesto por el policía, por supuesto. ¿Qué hay de ti?

~---Vamos, soy una mente perversa, ya me conoces. Los engranajes de mi cerebro buscan la peor opción. El policía no tiene carisma, los dos sujetos de las colonias, incluyendo a Shockman, son extranjeros en una ciudad no demasiado hospitalaria, y el alienígena, tres cuartos de lo mismo. Además, ¿crees que los militares dejarán su preciado equipo en manos de un civil? Apuesto sin duda por el soldado.

~---¿Y cómo están las apuestas?

~---¿Las apuestas?

~---Vamos, no me digas que no hay alguna clase de porra interna entre los demás.

~---No tengo ni idea, pero dado que soy el jefe, debería saberlo. Eh, Sam ~---dijo Scream viendo que Grove pasaba justo en ese momento, de camino a reunirse con su escuadrón en la sala de entrenamiento~---, ¿habéis hecho alguna clase de apuesta sobre quién ganará el torneo?

~---¿Cuál de las dos, señor? ~---preguntó Grove deteniéndose en seco, con cierto tonillo jocoso.

~---¿Dos apuestas? ¿A qué te refieres?

~---Bueno, sobre quién ganará el torneo la mayoría ha apostado por Shockman, pues a la vista de los nuevos datos creemos que alguna clase de plan se estará trayendo entre manos. Y sobre quién cree usted que lo ganará\dots\ la cosa está muy igualada, pero yo he apostado a que opina que el soldado. Nunca le ha caído demasiado bien el ejército. ¡No me defraude, señor!

Justo después de eso Grove siguió caminando hasta llegar el pasillo más próximo. Scream no tuvo que girarse para comprobar si Sky esbozaba una de sus sonrisas irónicas. Sabía que lo estaría haciendo.

~---Veo que las cosas por aquí están algo más\dots\ distendidas ~---comentó caminando lentamente hacia la última fila de asientos del hemiciclo.

\parbreak
Pocas veces Ellis Saw se había sentido con las manos tan atadas como con aquel asunto del torneo, ni siquiera cuando trabajaba para Ellen Gorgon y hacía de espía para Los Caídos. Era testigo de todos los acontecimientos, sin duda, y podía informar de ellos casi al instante de que se produjeran. Pero era como si fuera un espectador impotente que sólo estuviera contemplando el devenir de los acontecimientos.

El Presidente Scatter no había dicho a nadie a quién había elegido para proteger la ciudad de Ernépolis, ni siquiera a él, su ayudante desde el comienzo de su mandato. Se estaba tomando aquel asunto bastante a pecho y pretendía convertirlo en el estandarte de su política, sin lugar a dudas.

Se planteó la posibilidad de advertir al Presidente de que Warren Shockman era el villano conocido en el pasado como Éxeter, pero sin pruebas de ello lo único que conseguirían sería levantar las sospechas sobre ellos mismos. Además, aun así, no había ninguna garantía de que Shockman hubiera urdido alguna estrategia astuta, algo relacionado con redención, o arrepentimiento, o que el propio Presidente lo considerara como la ocasión perfecta para demostrar que la ciudad de Ernépolis estaba dispuesta a perdonar y dar segundas oportunidades.

Pero el problema consistía en que Warren Shockman era un asesino. Había matado al hermano de Sky, y quién sabía a cuántos más. Y además lo había hecho de modo que nadie pudiera jamás demostrarlo, sin dejar pruebas de ello. Su identidad criminal y civil avanzaban por rutas paralelas.

Sin embargo Sky logró conocer quién era en realidad, y tal vez esa fue su verdadera fatalidad. Tal vez, incluso, fue por ese motivo por lo que acabó con la vida de su hermano, ojo por ojo, tú me tienes en la estacada y ahora te tengo yo a ti. Aun así, estaba claro que se dejó cegar por la venganza, y más le hubiera valido largarse sin haber dejado una huella final de sus actos.

En primer lugar, porque tendría un enemigo menos a esas alturas. Ahora que estaba de vuelta en la ciudad, Sky no le quitaría el ojo de encima, y tal vez podría usar ese elemento en su contra.

Saw también estaba preocupado por Sky. Como él, también había sido un héroe, había sufrido pérdidas, y sabía lo que el odio podía hacerle plantearse. Sky vigilaría a Shockman, sí, pero ellos tendrían que vigilar a Sky.

Por fin el día de notificar al ganador llegó puntual y sin demora. Se eligió un evento al aire libre para tan magno acontecimiento, lo que sin duda traería aparejados problemas adicionales para Los Caídos, que ya no sólo tendrían que vigilar el transcurso de la ceremonia, también asegurarse de que no les tomaban a ellos mismos por terroristas dispuestos a sabotear la misma.

El lugar elegido fue el mismo estadio donde The Jammers actuaron en directo tiempo atrás ya, y cuya instalación lumínica había sido reparada y verificada varias veces con motivo del acontecimiento. Había aforo para decenas de miles de personas y la seguridad era máxima, aunque si bien era difícil perpetrar un crimen silencioso y salir indemne del mismo no era tan complejo llevarse a mucha gente por delante si uno no tenía pensado huir o sobrevivir después de haber cometido el atentado.

El evidente problema de que habría muchos ojos y cámaras en gran cantidad de esquinas no hizo sino empeorar la planificación de Los Caídos para dejarse ver en caso de que hubiera problemas. Fue por eso que se acordó que en el peor de los casos sólo el escuadrón de Saw entraría en potencial combate, ya que así él mismo podría dirigirles desde su posición privilegiada, y los demás ofrecerían apoyo, infiltrados entre los civiles.

Al mismo tiempo Sky estaría también junto al Presidente, recibiendo al ganador cuando éste fuera proclamado. Como era de esperar muchos de sus hombres estaban también situados entre el público, y sólo esperaba que no hubiera necesidad alguna de que tuvieran que mostrar sus respectivas placas.

Ninguno de los aspirantes sabía quién era el ganador tampoco. Les habían colocado estratégicamente entre el público, cerca de las primeras filas, para que el elegido se levantara sin problemas a recibir los aplausos que para él estaban destinados.

Shockman esperaba que el plan hubiera salido como tenía en mente. Si resultaba ser el ganador, bastaría con valerse de su nuevo puesto para llevar a cabo una guerra contra aquellos que pretendían silenciarle. En caso contrario aprovecharía para tratar de dirigir al vencedor contra ellos. A lo largo de los tests y las pruebas no había escatimado en sugerir veladamente en más de una ocasión que ciertos sectores eran peligrosos y debían ser pacificados antes que ningún otro lugar de la ciudad. Sea como fuere, de una ú otra manera tenía que salir con ventaja de todo aquello.

Había otros espectadores contemplando la función. En los asientos más alejados John Scream, vestido de civil, se limitaba a observar el escenario con ojos inquisitivos, preocupado por lo que pudiera pasar. Era una de las primeras veces que no se encargaba de manera directa de la contienda, pero al mismo tiempo, bien podía ocurrir que nada raro fuera a pasar. Habría un nuevo cabeza de turco en Ernépolis, todo el mundo podría verle, recogería el guantelete como símbolo de su estatus y complemento a sus habilidades, y todos a casa a dormir tranquilos.

~---¿Todo bien, Charles?

~---Sin novedad.

~---¿Matt?

~---Roger, jefe.

~---¿Sam?

~---Nada nuevo en el horizonte, señor.

~---¿Jim?

~---Nada.

Desde el asunto de los cazarrecompensas el número de escuadrones se había reducido drásticamente hasta ser seis en total, de modo que se coordinaran mejor a pesar de su menor tamaño. Aun con todo había siempre miembros en reserva que llevaban a cabo tareas no directamente relacionadas con el combate, pero igualmente necesarias para mantener el engaño maestro y global que eran Los Caídos.

Nadie reportaba nada. No tenía que preguntar siquiera a Saw, bastaba con verle sobre el escenario, a punto de que comenzara toda la pantomima. El Presidente, de hecho, ya había subido al estrado a soltar un discurso complaciente típico de los políticos, lleno de frases lentas, entrecortadas y subliminales y gestos calculados como la coreografía de un musical.

Nada fuera de lo esperable.

Nada interesante que reportar.

Pero aun así sabía que algo estaba a punto de pasar.

En el exterior, el sujeto del traje negro escuchó los primeros aplausos apoyado contra una descascarillada pared, brazos y piernas cruzados. Giró la cabeza cuando empezó el discurso del Presidente, y se planteó si de verdad el momento había llegado o era sólo una falsa alarma.

Sky no dejaba de mirar a Shockman y eso era algo que no le pasaba desapercibido a Saw, situado junto a él. Le atravesaba con la mirada como si pudiera desintegrarle con la misma. Era tal el odio que le recorría por dentro que Saw empezó a plantearse si no tendría que vigilarle más de cerca pasara lo que pasase a partir de aquella noche.

~---\dots\ es por eso, ciudadanos de Ernépolis, que necesitamos que los héroes vuelvan. No porque estemos perdidos sin ellos, sino porque ellos son parte de nosotros. Desde que se fueron la sombra del crimen pulula por la ciudad, y es por eso que su presencia aquí es más necesaria que nunca. Ahora, es necesario establecer unas reglas. No pueden actuar por libre, sino que deben hacerlo en consenso con las fuerzas del orden y la justicia.

Otra de las críticas a la idea del Presidente Scatter. No buscaba un héroe, sino un superpolicía. Un héroe era libre, independiente. No tenía que rendirle cuentas a nada ni a nadie. Un héroe, generalmente, solía tener problemas con los poderes fácticos. El mero hecho de que trabajara con ellos hacía que ya la gente, por definición, desconfiara de él.

~---Y es por eso que, como muestra de ese vínculo, se le hará entrega a ese héroe de este símbolo, este guante diseñado expresamente para él, para que sea su placa, su estandarte y su arma en los momentos de necesidad.

Era el momento de Saw para entrar en escena. Sacó el estuche que contenía el guantelete, un cubo de unos treinta centímetros de lado, y lo colocó sobre una mesilla dispuesta de modo que las cámaras pudieran enfocar con claridad hacia su superficie. Lo abrió y las pantallas del estadio pudieron mostrar la imagen aumentada a todos los presentes, del mismo modo que en su momento mostraron a Distorsión y su banda desgranando sus canciones más conocidas.

Los cinco aspirantes, a los que nada se les había dicho tampoco de aquel guantelete, lo miraron con especial curiosidad. El rostro de algunos estaba surcado por el deseo y el de otros por la intriga, pero el de Shockman reflejaba una profunda preocupación.

Principalmente, porque pudo notar, gracias al dispositivo de su bolsillo izquierdo, cómo su rata, que estaba escondida en el derecho, le transmitía el miedo que la presencia de ese guante le estaba produciendo.

Algo iba mal. Muy mal. Shockman ya había conocido en otras ocasiones el miedo de los animales que controlaba, no sólo insectos o ratas, también perros y, sobre todo, gatos. Temían gran cantidad de cosas, desde rayos hasta deslizadores, pasando por otros animales más grandes o los propios humanos en según qué circunstancias. Pero el miedo que a la rata le inspiraba aquel chisme del ejército bordeaba la frontera de lo irracional, incluso en un ser no especialmente inteligente como aquel.

Podía callarse y dejarlo pasar. Pero nada bueno podía suceder si lo dejaba correr de esa manera, no sólo a los presentes, a él mismo en caso de que cerrara la boca.

Pero todo el mundo podría verle. Las cámaras le enfocarían, y sería más vulnerable que nunca, porque no estaría subiendo a la palestra como el vencedor. Podía no ganar y además permanecer a la vista de todos. Perderlo todo en un momento.

Qué demonios, pensó. Total, como si fuera a hacer lo más conveniente.

~---¡Aléjese de eso! ~---dijo gritando en voz alta, levantándose de su asiento cercano~---. ¡Es peligroso!

Bastó un segundo para que una mezcla de miedo y desconcierto se apoderara de todos los presentes. No es que la gente se pusiera a correr histérica ni nada parecido, pero como poco el temor a no saber qué era lo que estaba pasando comenzó a circular por todos los asientos, aunque no todos lo vivieron de la misma manera. Sky no tardó en dar una orden a todos sus agentes para que se pusieran en guardia, y preparó el arma para disparar a Shockman en caso necesario. Scream avisó por su parte a todos los jefes de escuadrón.          Lo que ocurrió, sin embargo, fue algo que difícilmente ninguno de los presentes pudo haber imaginado.

El guantelete se expandió y estiró, como si sólo se hubiera replegado sobre sí mismo, y tras abrirse en canal se agarró a la mano de Saw, que cayó de rodillas, incapaz de quitárselo. Al mismo tiempo, placas y paneles empezaron a recorrer el brazo con lentitud, y Saw levantó la vista.

Miró a Sky, pero con una mirada que no parecía la suya propia.

<<Hola, jefe Sky>> dijo con una voz completamente metálica e inhumana.

Scream se levantó de su asiento como accionado por un resorte. No. No podía ser. Era él. Había vuelto.

~---Armor ~---dijo Sky, sacando el arma reglamentaria de su funda.

El pánico empezó a cundir entre los asientos, y un grupo de guardaespaldas salió para llevarse de allí al Presidente Scatter, pero éste quiso quedarse, desconcertado por todo lo que estaba presenciando.

<<No exactamente. Sólo soy un resto de él, un remanente del virus que logró infectar un prototipo que el ejército tenía bajo custodia. Pero mi objetivo es encontrar a la matriz original y restaurarla de nuevo>>. 

~---De modo que no has muerto\dots

<<No puedo morir, jefe Sky, porque nunca he estado vivo>> contestó con perversa malevolencia, manejando a Saw como si fuera una marioneta.

~---¡Sam, Matt, Jim, sacad a la gente de aquí cuanto antes! ~---ordenó Scream por el comunicador a gritos, aprovechando la confusión. La situación era peor de lo que pensaba. Armor, o lo que fuera, no sólo tenía en el pasado poder suficiente para arrasar un estadio entero sin problemas, poseía también la voluntad necesaria para hacerlo. La única esperanza de Los Caídos era que su punto débil, el elevado gasto de energía eléctrica, estuviera de nuevo de su parte.

Saw se incorporó, en lo que los circuitos reptaban hasta llegar a su hombro, y cerró el puño con fuerza, apuntando al Presidente Scatter.

<<Ahora déjeme marchar, o acabaré con él>> fue su único comentario.

Estaba débil. De eso no cabía ninguna duda. Y había una manera sencilla pero terrible de detenerlo. Sin embargo, Sky no podía hacer algo así. Si ordenaba que dispararan a Saw, seguramente no les quedaría más remedio que matarle para cortar el suministro de energía de Armor con su huésped. Crudo y directo.

Había además un polvorín extra que nadie sabía cómo manejar en ese momento, y consistía en la reacción de los cinco aspirantes. Para empezar dos de ellos, el policía y el soldado, ya estaban encarados frente a él, dispuestos a atacarle.

Scream miraba desde lejos la escena, pero sabía lo que estaba a punto de pasar. Se comunicó con Sky a toda velocidad.

~---¡No dejes que se acerquen a él! ¡Eso es lo que quiere!

~---¡Alejaos! ~---gritó Sky, transmitiendo lo que Scream acababa de contarle~---. ¡Busca un nuevo huésped!

Saw miró a Sky con furia contenida y apuntó hacia su posición. A duras penas el Jefe de Policía de Ernépolis logró esquivar la descarga, que le lanzó despedido varios metros y le estrelló contra un conglomerado de sillas caídas debido a la huida precipitada de los espectadores.

<<Calla, humano. Mi intención era, de hecho, hacerme con el control del vencedor lejos de miradas curiosas, pero tú>> señaló a Shockman <<has arruinado el elemento sorpresa. Por ese motivo, ahora iré a por ti>>.


Bajó las escaleras de la tarima y se acercó poco a poco hacia la posición de los cinco aspirantes. A medida que se acercaba, los otros cuatro que no eran Shockman se echaron hacia atrás. Sabían que tendrían que enfrentarse contra cosas así de haber resultado vencedores, pero dado que no era a ellos a quienes estaba amenazando, y tampoco exactamente a la población de la ciudad, su cobardía interna salió a flote.

Shockman no corrió. Sabía cómo funcionaba la mente de los villanos, aunque fueran monstruos sin alma como aquel. Correr era lo que esperaría que hiciera. Trató de pensar qué hacer para detener a aquel zombi andante. Podía intentar lanzar una plaga de cucarachas contra él, pero eso tendría el doble inconveniente de que tendría que acabar con el huésped y, peor aún, mostraría sus cartas ocultas. Sin embargo, aquella podía ser fácilmente su última mano en la gran partida de la existencia universal.

Preparó la modulación del dispositivo cuando, de repente, vio que su oponente se detuvo en seco y miró al cielo, y, por supuesto, no pudo evitar hacer lo mismo.

Le costó un poco, pero al fin pudo ver que una figura negra estaba descendiendo de los cielos, flotando con lentitud, y comprendió que su enemigo estaba percibiendo algo que estaba vetado a sus sentidos.

No tardó en averiguar de qué se trataba cuando la figura voladora, con un gesto de ambas manos, destrozó el traje y un brillo como nunca antes se había visto en las calles de Ernépolis~I le obligó a taparse los ojos.

Aquella silueta era humana e irradiaba luz. Luz pura, igual que un sol en movimiento, como una fuerza cósmica e imparable. Su descenso suave, pausado, con los pies estirados y los brazos en cruz, era como un advenimiento divino que se mostraba a los ojos de todos los presentes.

Al descender del todo disminuyó ligeramente su intensidad, sin duda de manera consciente, y habló con una voz que debió ser similar al tono de los instrumentos que tiraron abajo la muralla de Jericó.

~---Soy Alma Espejo, monstruo de metal. Ríndete, porque nada puedes contra mi poder.

Saw, cuyo cuerpo ya estaba cubierto en un cuarenta por ciento por el artefacto infectado con el virus, se volvió y miró al recién llegado con unos ojos que no presagiaban nada bueno.

<<Eres lo que necesito>> se limitó a comentar, y se acercó a Alma Espejo, que no hizo nada por alejarse de él. Sólo le miró como si no pudiera entender los actos de su enemigo, y fueran insignificantes e inútiles bajo su punto de vista.

El artefacto pasó de Saw al brazo de Alma Espejo, y mientras le recorría y cubría por completo, su brillo desaparecía, tornándose en opaco metal. Apenas unos segundos pasaron hasta que le tapó por completo, igual que si un eclipse acabara de acontecer.

<<Este poder es\dots\ increíble>> dijo Armor, y por primera vez la sorpresa hizo mella en aquella mente artificial. <<Nada podría detenerme. ¡Nada!>>

Hizo un ademán de moverse hacia Shockman, pero ni siquiera pudo dar el primer paso. El Presidente Scatter aún miraba toda la escena, visiblemente impresionado. Lo mismo hacían los policías, los miembros de Los Caídos y el público que aún no había escapado, y cuyo estado general había pasado del pánico a la curiosidad.

<<¿Qué está pasando?>> dijo, incapaz de entender al nuevo ser que había fagocitado.

El metal comenzó a brillar y a volverse incandescente, como si fuera a entrar en ebullición. La silueta de Alma Espejo se elevó hacia los cielos, y cuando estuvo a más de diez metros del suelo comenzó a irradiar. Muchos fueron los testigos que narraron la extraña sensación que aquella luz virgen producía en contraste con la negrura superior de la eterna Nube.

El ser volador tensó todos sus músculos, dobló las extremidades, y cuando parecía que el metal no podía ponerse más rojo, Alma Espejo estiró brazos y piernas y desintegró la vaina de metal que le aprisionaba, haciendo que llovieran pedacitos no más grandes que limaduras de hierro.

Después de eso descendió como si estuviera retornando a sus dominios.

~---¿Todo el mundo está bien? ~---comentó, y su tono de voz, aunque solemne, también disminuyó en intensidad, al igual que su brillo.

El Presidente Scatter fue el primero en acercarse a aquel fascinante extraño al tiempo que sus guardaespaldas no le perdían de vista, aunque realmente se estaban planteando qué podían realmente hacer si el sujeto brillante suponía una amenaza. Al mismo tiempo Shockman se acercó hacia donde había caído Sky y le ayudó a levantarse. No por amabilidad, pero al fin y al cabo apreciaba el valor sincero, incluso entre aquellos a los que consideraba sus rivales.

Sky se lo agradeció empujándole a un lado.

~---No quiero tu ayuda, basura ~---fue su único comentario.

Shockman comprendió. Entendió que aquel tipo sabía quién era él en realidad. Sea como fuere, no tenía escapatoria. No podía hacer nada ahí, delante de todo el mundo, por lo que optó por al menos saber quién le había descubierto.

~---¿Cómo sabes quién soy? ~---preguntó mirando de reojo, en lo que todo el mundo empezaba a congregarse en torno a Alma Espejo.

~---Tú mataste a mi hermano ~---fue la cruda explicación de Sky.

Shockman calló, y Sky se dio cuenta de que le estaba mirando con ojos nuevos. Como si acabara de juntar piezas sueltas del puzzle en su cabeza.

~---Tú eres él ~---dijo si más~---. Eres\dots

~---Ya no existe. Pero tú sí\dots\ Éxeter.

~---Ya veo. Amenacé con matar a tus seres queridos. Pero la realidad es que no les toqué un solo pelo. Ni siquiera sabía quién eras. Nunca he matado a nadie\dots\ que no se lo estuviera buscando.

~---¿Qué hay de mi hermano, entonces? Paul Sky.

~---Yo no acabé con él. Tal vez otro lo hiciera.

~---No te creo.

~---Seguro que tampoco imaginaste que volverías a verme vivo, ¿verdad? Igual que yo a ti.

Sky vio de reojo que Saw se estaba levantando de nuevo y comprobó que parecía encontrarse bien. Habría que hacerle algunas pruebas para asegurarse, pensó, pero en todo caso ya podía darse por afortunado pues era el primer huésped de Armor, o de una parte de él, que sobrevivía al contacto con éste.

Cuando se dio la vuelta comprobó que Shockman había desaparecido por completo. Entre la confusión y el ir y venir de gente sería complicado dar con él de nuevo. Al menos, pensó, había una posibilidad de que su hermano siguiera vivo, si creía las palabras de su antiguo enemigo.

Desde la lejanía, Scream observó con gesto reflexivo toda la escena. La entrada de Alma Espejo había sido un hecho completamente inesperado, y resultaba evidente a quién le ofrecería el Presidente Scatter ser el nuevo defensor de la ciudad. Pero nada sabían de ese forastero, si es que era de fuera de la ciudad, y tampoco conocían sus intenciones y motivaciones.

Además de eso, le intrigó el comportamiento de Warren Shockman. Bien podía haber callado, haberse mantenido al margen al notar problemas inminentes. Pero en vez de eso prefirió advertir del peligro, y tampoco lo rehuyó como los otros aspirantes cuando lo tuvo frente a frente. Además parecía haber hablado con Sky, y éste parecía, sin duda, más tranquilo después de haber conversado con él.

Un problema había llegado a su término, pero en su lugar nuevas incógnitas se abrían a sus pies.

\endinput
