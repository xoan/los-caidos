\noindent
Mucho tiempo después, sólo recordó la ceniza en su boca, dispersa por todo el suelo, mientras su mano trataba de aferrar la mano de ella. Tanto que había perdido\dots\ tanto por ganar.

\parbreak\noindent
Existen muchas cosas que formaron parte de los recuerdos de John Scream por aquel entonces. El día que se estrelló en un planeta perdido sin esperanza de volver a remontar el vuelo. Cuando conoció al ser que creó la gema para él. La noche que conoció a Aryn. Su primera pelea contra Silenciador. El terrible instante en que éste, por orden de Gorgon, le disparó por la espalda y le impidió salvar a Aryn.

Pero ningún recuerdo fue tan vivo como el día que Starr Miles le mostró el Aquerón, el cuartel general de Los Caídos.

Había miles de entradas por toda la ciudad. Miles de pasadizos escondidos en todas partes. Ocultos, impenetrables, bien vigilados, para que si alguno de ellos era fortuitamente descubierto no se supiera mucho más al respecto de dónde llevaba. Simplemente, si eso ocurría, se procedía a inutilizar dicha entrada y se cerraba para sustituirla por una nueva. La que ellos utilizaron estaba cerca del lugar donde Scream había vivido la mayor parte de su tiempo desde que estaba en la calle. Miles se aproximó a un segmento de pared en un callejón oscuro. No parecía tener nada de particular, y durante el rato que esperaron, por mucho que Scream lo observó, nada vio que le hubiera hecho sospechar que tras ese muro se escondiera nada distinto de un almacén de ratas. Al poco el arcaico ladrillo se deslizó y les dejó paso a una oscura sala.

~---Una de las primeras cosas que aprenderás será a reconocer esta clase de accesos ~---afirmó Miles.

La puerta se cerró tras ellos y una tenue luz iluminó el pasillo. Scream se sintió más como si estuviera entrando en una base perteneciente a alguno de sus viejos enemigos que como conociendo el que supuestamente iba a ser su nuevo hogar. Había por todas partes metros y metros de pasillos intrincados en los que se hubiera perdido con gran facilidad.

~---Disculpa si tardamos, John, pero mi intención era enseñarte el módulo principal.

John Scream se preguntó si la espera valdría la pena. No lo dudó cuando llegaron al fin.

Una enorme sala de más de diez pisos de altura se ofrecía a sus ojos, escondida de la ciudad bajo sus cimientos. En ella pudo ver múltiples departamentos acristalados donde toda clase de personas se estaban entrenando. Mientras andaba con Miles no pudo fijarse más que en detalles aislados por aquí y por allá: hombres luchando entre sí con los puños y otras armas nobles, salas donde poner a prueba el sigilo, laboratorio químico donde vagamente pudo distinguir cómo hacían pruebas de combustión de tela, departamento electrónico, donde parecían hacer pruebas de modulación de sonido. Al cabo de un rato se pararon frente a una compuerta.

~---Entra ~---le dijo Miles. Una luz de alarma se encendió en la cabeza de Scream.

~---¿Por qué no entras tú\dots?

El golpe le dejó inconsciente al momento.

Cuando despertó se encontraba desnudo en la misma habitación donde le habían obligado a entrar. No había nada más que él. Golpeó la compuerta. Nadie respondió. Hacía frío. Mucho frío. Pero Scream no protestó. No iba a dejarse humillar.

No mucho después llegó comida a través de un panel. La tomó hambriento. Otra vez el silencio, la nada. Al cabo de un tiempo sintió la terrible necesidad de vaciar sus intestinos. Aguantó cuanto pudo, pero al fin tuvo que rendirse a la evidencia. La humillación era completa. Una especie de brazo robot surgió de la pared y le devolvió parte de la dignidad perdida limpiando la habitación hasta dejarla de nuevo como al principio. De nuevo silencio. El sueño comenzó a invadirle y dedujo que era hora de dormir. Sin embargo no durmió. Estuvo tres días sin hacerlo hasta que el sueño le derrotó.

Al cabo de una semana le dieron ropa. Pasó un día entero sólo tratando de volver a entrar en calor. Estaba enfermo, pero sobreviviría. Se dedicó a entrenarse dentro de su celda, no sólo para ocupar el tiempo. Su propio instinto latente le obligaba a ello.

Finalmente, al cabo de otra semana, dos hombres entraron en la jaula. Le miraron de reojo y comenzaron a atacarle. Golpes cuidadosos, calculados, destinados a cercarle. Scream les observó. No estaban acostumbrados a atacar cara a cara. Se aprovecharía de ello. Trató de agarrar a uno por la muñeca para inmovilizarle, pero éste se soltó con un rápido movimiento de pinza y con la mano opuesta hizo aquello que Scream no había conseguido. Al mismo tiempo el otro le agarró del brazo libre y se lo colocó a la espalda. Estaba inutilizado.

O no.

De un cabezazo que le dolió tanto o más que a la víctima, Scream derribó a uno de sus atacantes mientras se zafó de la presa del otro. Se disponía a realizar otro ataque cuando la compuerta se abrió de nuevo. Era Starr Miles.

John Scream se olvidó de sus atacantes y se lanzó al cuello del viejo.

~---Valiente pero impulsivo ~---dijo Miles agarrando de la barbilla a Scream con fuerza. Scream sabía que si apretaba le haría puré~---. No pierdas el sentido, John. Eso será lo que ellos pretenden. Si no te pueden derrotar, entonces tratarán de volverte loco.

Soltó a Scream y éste cayó al suelo con estrépito.

~---Me siento orgulloso, John ~---dijo indicando a los dos hombres que les dejaran solos~---. Era una prueba muy dura ésta a la que te has sometido. Pensabas que lo habías perdido todo. Ahora sí. Para nosotros ya no tienes dignidad. Has caído lo más bajo posible.

Miles le sonrió.

~---Igual que esos hombres. Igual que yo.

Los siguientes días algunos de los Caídos le enseñaron el arte de la defensa personal. Scream podía ver en todos ellos el halo de los que han sido héroes alguna vez. Ocasionalmente el propio Starr Miles le adiestraba en persona, y fiel a sus ideas de adoctrinamiento le daba lecciones verbales mientras peleaban, consejos que eran asimilados por Scream fácilmente a pesar de tener que estar concentrado en otra cosa mientras los escuchaba.

~---Nosotros no combatimos, John. Esto es sólo una medida preventiva. Podíamos combatir cuando teníamos poderes que respaldaban esa clase de actos. Ahora nuestro poder es otro.

~---¿En qué consiste nuestro poder ahora?

~---En la astucia, John. Somos como los magos que efectúan sus trucos en los burdeles del este de Ernépolis~I. Tenemos que jugar con la percepción del enemigo. Engañarle. Estos tiempos que nos han tocado vivir favorecen la estrategia. La Nube es más oscura que nunca, la ceniza cubre las calles. Callejones, sombras\dots\ todo a nuestra disposición. Son armas tan poderosas como las que antes teníamos.

~---¿Y en qué consistirá el engaño?

~---Sólo somos hombres, John. Pero para ellos seremos más que eso.

Cuando su entrenamiento de defensa personal terminó, Scream pasó a la siguiente fase: el silencio.

~---Recuerda, John ~---decía Miles observando a Scream en el laberinto donde debía poner a prueba su sigilo~--- que todo es una cuestión de paciencia. Para ser una sombra debes comportarte como tal. No te desplaces, deslízate. Aprovecha los ruidos de la ciudad, son parte de tus movimientos. No seas rítmico. Y cuando llegue el momento de presentarte, juega con los claroscuros. Que no vean tu rostro. Calla cuando esperan que hables, habla cuando esperan que calles. Desconciértales, pero no les subestimes.

Un sensor tocó a Scream y disparó las alarmas del laberinto. Al momento tuvo que ponerse a cubierto, pues sabía que los disparos efectuados por las defensas del laberinto, aunque no eran letales, eran extremadamente dolorosos. A duras penas lo consiguió.

~---Sobre todo, John, piensa que si engañas al enemigo, si le haces temerte, éste puede hasta a llegar a dudar si dispararte o no por puro miedo. Ese poder es mayor que cualquier arma que puedas empuñar.

Una vez el entrenamiento básico estuvo completado, por fin Miles comenzó con Scream la etapa clave para ser uno de los Caídos. La comprensión de lo que estaban haciendo. Le dio una gabardina y un sombrero a medida y Scream se los puso. No parecían tener nada de especial.

Hasta que Miles dejó la habitación en penumbra.

La combinación de luz y oscuridad ensombreció su rostro, convirtiéndolo en indetectable para alguien que estaba lejos de él. Se miró en un espejo y su imagen le resultó perturbadora. Anónima, pero al mismo tiempo dotada de personalidad.

~---Esta es la primera parte del engaño, John. Todos nosotros vestimos igual. El complejo cometido, como ya sabes, es hacer creer que sólo somos uno. Por eso la ropa, no sólo por su aspecto ambiguo y protector. Cuando estés fuera no tendrás identidad, aunque pertenezcas a los Caídos. Eso es lo que somos. Muchos hombres que actúan como uno que no existe. Sin nombre. Sin pasado. Como una plaga de hormigas. Imparables.

Se acercó a Scream y le dio un par de lentillas y una pastilla.

~---Las lentillas son para que todos tengamos el mismo color negro de ojos. Los ojos son cruciales. Deben expresar todo lo que nunca haríamos. Si los criminales ven odio en los ojos, ellos sentirán temor. Si por el contrario ven temor, sentirán odio. La debilidad de los demás es su fortaleza. Nosotros les haremos sentir lo que provocan.

~---¿Y la pastilla? ~---preguntó Scream.

~---Algunos de nosotros, desgraciadamente, moriremos. Esa pastilla fue creada para hacer desaparecer nuestros cuerpos en caso de que eso ocurra. Tenemos que darles la sensación de que la criatura a la que se enfrentan es inmortal, ajena a las leyes de la naturaleza. Trágatela.

Así lo hizo Scream.

~---Bien ~---dijo Miles satisfecho pero sin ocultar la dureza de sus palabras.

~---¿Qué hay de la ropa?

~---Llevó mucho tiempo, pero diseñamos un tejido que puede arder espontáneamente cuando nosotros queramos. Puede ser accionado por un compañero oculto en las cercanías, y en caso de emergencia podemos hacerlo nosotros desde la base, previa advertencia para evitar posibles accidentes. Del mismo modo todos los aparatos que puedas llevar serían consumidos por las llamas para quedar sólo una capa de ceniza.

~---Muy apropiado ~---comentó Scream~---. Ceniza no falta en Ernépolis~I.

~---Y lo más importante al respecto, John. Tardarás mucho, pero aprenderás a no exteriorizar el dolor. Nunca, jamás, debemos gritar. Sobre todo al morir. Eso es lo que ellos esperan. Librarse de la pesadilla atravesándola, disparándola, lanzándola agua hirviendo. El éxito de la idea depende de detalles cruciales como éste.

~---Lo recordaré.

~---Sí ~---comentó Miles~---, no me cabe duda que lo harás.

A partir de ese momento Scream comenzó el entrenamiento con otros miembros del grupo. Resultaba muy duro para él, que siempre había trabajado solo, coordinarse con otros ya no en ataques sino en movimientos. Al final comprendió que era como si estuviera en un gran escenario y fuera un actor más sobre él. Mientras los focos no le iluminaran no sería el protagonista, pero tenía que aprender a entrar en escena cuando fuera su turno. Cuando la sincronización era correcta, cosa que cada vez sucedía más a menudo, los resultados eran espectaculares.

~---Perfecto, John. Aprendes deprisa. Te nombraré director de un escuadrón.

Los escuadrones solían ser de cinco miembros, los cuales se conocían a la perfección. Todos ellos se movían de idéntica manera para hacer el engaño lo más creíble posible. Scream no tardó en simpatizar con ellos: James Sky, Charles Razorclaw, Frank Raid y Ellis Saw. Sabía que ya se conocieron en el pasado, pero el entrenamiento de Starr Miles resultó tan eficaz que ninguno de ellos fue capaz de imaginar qué fueron los otros en la era dorada de los héroes.

Nadie solía hablar de su etapa anterior. No había ninguna regla que lo prohibiera, pero parecía una especie de acuerdo silencioso entre los miembros de los Caídos. Sin embargo las pocas veces que el tema surgía y alguien se atrevía a contar su secreto, Scream se sentía sorprendido por las tan diversas formas en que habían acabado la carrera heroica de aquellos que hablaban. Sin embargo el tema recurrente siempre era el mismo: Starr Miles.

~---He oído que él no fue un héroe sino un villano, y que se arrepintió de su conducta ~---comentaba Razorclaw.

~---Eso no es así ~---contradecía Saw~---. Dicen que era el Hacedor.

~---Yo también he oído eso ~---terciaba Sky~---. Y si no es él, ¿de dónde sacó todo esto?

Pero Scream callaba. Sabía que la paciencia de un hombre bien podía construir todo lo que les rodeaba. Su escuadrón subestimaba no a su líder, sino su capacidad conjunta. Sólo sumando los cuarteles secretos que muchos de ellos tuvieron en su momento el resultado era una fortaleza casi inexpugnable.

A lo que Scream al fin comenzó a comprender la delicada tarea que se habían impuesto. Debían parecer ya no como sus enemigos, peores aún que ellos. La policía les buscaría, pero la policía estaba corrupta. El Jefe Wolf, por orden de Gorgon, no cejaría hasta detener al extraño de la gabardina y sombrero. Pero ellos no estaban indefensos. Sky era policía. Razorclaw era abogado. Raid era profesor universitario. Saw trabajaba para Gorgon en persona. Tenían acceso a todos los niveles. Sin embargo él estaba fuera de la sociedad, y sabía que por eso era el jefe de su escuadrón. Tenía más tiempo para dirigirles, para perfeccionar su entrenamiento personal. Para asimilar el objetivo.

~---No deben creernos héroes ~---repetía Miles una y otra vez~---. Ellos no tienen escrúpulos a la hora de amenazar a los que nos importan. Con todo el dolor de nuestro corazón, tenemos que dar la sensación de ser peores criminales que ellos. Jamás deben sospechar que nos preocupan los inocentes, y de ese modo no los usarán para amenazarnos. No nos chantajearán con vidas ajenas, como te ocurrió a ti y a muchos de los que están aquí ahora. Hay veces que saldremos al exterior sólo para provocar miedo y pánico entre los transeúntes, llegando incluso a fingir que matamos alguno de ellos. Seremos erráticos. No sabrán qué queremos, si justicia, venganza, o controlar la ciudad.

La última fase introdujo los aparatos especiales. Moduladores de voz para que todas sonaran iguales. Hologramas de múltiples tipos, desde distorsionadores de imagen para aumentar las siluetas y las sombras hasta fieles reproductores de imagen real. Armas aturdidoras que parecían letales. Anuladores de fotones para provocar y manejar la oscuridad. Una tecnología sin igual extremadamente difícil de manejar con éxito.

~---Dentro de poco nos pareceremos a los Cazadores de Nocturnos de Talópolis~X ~---murmuró Raid para sí cogiendo los aparatos.

~---Vamos, tenemos que practicar. Hoy emboscada ~---dijo Scream en voz alta.

Con unos reflejos asombrosos, los miembros del escuadrón se pusieron en sus puestos. Todo funcionaba a la perfección entre ellos. Pero Scream no estaba satisfecho.

~---Tenemos que salir al exterior ~---dijo a Miles en cuanto pudo~---. Tenemos que conocer la ciudad.

~---Y así se hará ~---respondió Miles~---. El momento ha llegado. Poco a poco al principio, con más intensidad después, vamos a mostrarnos a Ernépolis~I.

Scream le miró fijamente y por primera vez notó un atisbo de temor en sus ojos. Nunca una palabra de Miles lo corroboró, pero supo que el entrenamiento, al fin, había finalizado.

\endinput
