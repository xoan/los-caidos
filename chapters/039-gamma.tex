Un primer asalto, un primer objetivo encubierto. Preocupación ante la idea de que algo más que lo que se veía a simple vista estuviera gestándose en el seno de la ciudad.

Preocupación también de que la ciudad pudiera convertirse en un improvisado campo de batalla de escaramuzas militares, lugar de pruebas de armamento, o quizás incluso algo más difícil de predecir que tales posibilidades\dots

\fancyparbreak
Los paramilitares no volvieron a atacar después de aquel intento en una de las plataformas de la nueva autopista. No se redobló la seguridad pues ya estaba en límites máximos, pero sí que patrulló la policía por la zona más de lo habitual para calmar a los vecinos, y la empresa a cuyo edificio se accedía mediante la pasarela amenazada aumentó los turnos de sus guardas de seguridad interna, ante el temor que ellos pudieran ser el objetivo central de aquel rápido y efímero golpe.

No se aireó tampoco el hecho de que Batería hubiera desaparecido. A la opinión pública no le importaba tal hecho en absoluto, para ellos sólo era un cazarrecompensas más de muchos, otro mercenario a sueldo al que nadie echaría especialmente de menos, salvo quizás sus propios compañeros. Lo irónico de todo aquel asunto, por otro lado, era que había sido capturado cuando, más que dedicarse a buscar delincuentes y fugados, estaba jugándose el pellejo por la prosperidad de la ciudad. Y una buena paga aparejada con ello, claro.

El resto de sus compañeros de Fortaleza no dejaron de buscarle en los días posteriores, pero al mismo tiempo tampoco podían desatender la labor para la que habían sido contratados. Todos sabían los riesgos a los que se sometían en esa profesión, por lo que aunque hacían todo lo que podían por encontrar a su socio de brazo biónico entendían también que siempre cabía la posibilidad de que no volvieran a verle de una sola pieza.

Scream era también consciente de que algo raro se había cocido en torno a la desaparición de Batería. Al ejército poco le importaba un cazarrecompensas más o menos, pero montar tal operativo por parte de los disidentes para no más que hacerse, de manera lo más subrepticia posible, con uno de ellos, era cuanto menos para ser tenido en cuenta. Al menos que él supiera, Filo Omega no había puesto demasiado empeño en buscarle. Algo coherente con su posición y su manera de ser, por otro lado. Era una guerrera más que una estratega. Y si su preocupación por los suyos era limitada y plagada de frialdad, más aún por uno de aquellos buscadores de fama, dinero, gloria o adrenalina.

Pero no podía dejar de pensar en aquello, y ocupaba la mayor parte de su tiempo, incluso el que tenía que dedicar a otras tareas no relacionadas con Los Caídos pero de igual importancia en ese momento, como la relacionada con el inminente módulo espacial que sería lanzado en breve al espacio, otro de los proyectos destinados a revitalizar la ciudad.

La nave había sido parcialmente diseñada por Gorgon Enterprises, y aunque él no había participado en su diseño y se había delegado a un equipo independiente, sí que se estaba encargando de la configuración del navegador de vuelo. El planeta al que iba a ser enviado, llamado Khorleur, estaba aún en proceso de formación pero ya poseía una colonia asentada. Había resultado ser un astillero perfecto para el tratamiento de metales y control de maquinaria, pero debido a sus altas temperaturas y el intento de evitar la presencia humana en la mayor parte posible, casi todas las factorías estaban en proceso de automatización. Eso incluía las naves, y ese era el propósito del diseño que tenían entre manos, que el módulo realizara un viaje autónomo sólo de ida con el fin de ser recibido allí por los escasos técnicos residentes, que lo reconfigurarían con las coordenadas de vuelta para así dar por inaugurado el tráfico aéreo entre la Tierra y Khorleur.

Era un proyecto muy interesante que requería, además, todo el talento y concentración que Scream había acumulado durante aquellos años en materia aeroespacial. Y precisamente por eso, era incapaz de dar de sí todo lo que tenía que ofrecer para su consecución exitosa.

Frustrado, subió a lo alto del edificio para despejarse y cuál fue su sorpresa cuando, nada más cruzar la desvencijada puerta que llevaba a la azotea, se encontró allí con un anciano de bigote blanco y cuidado que, vestido con traje, gabardina y sombrero, y apoyado en un bastón, estaba observando el horizonte, pensativo. Scream no dijo nada, esperando que el desconocido justificara su presencia allí arriba.

~---John Scream ~---dijo sin variar su posición, dando su perfil derecho al líder de Los Caídos~---. Supongo que ya había llegado el momento de conocernos. Mi nombre es Caronte, y ya imaginará para quien trabajo una vez me he presentado.

Scream dio un paso adelante, sin dejarse amedrentar. No era la primera ni la segunda vez que trataban de desconcertarle de esa manera.

~---¿Qué es, un esbirro de Hades? ¿El sustituto de Perséfone, tal vez?

~---No es esa mi función, en realidad. Digamos que a efectos clandestinos, puede considerarme su\dots\ embajador en Ernépolis~I.

No le gustó nada esa revelación a Scream. Eso significaba que Hades tenía planes en marcha, y tan sofisticados como para haber llegado a tales extremos de meticulosidad.

~---¿Y bien, qué es lo que Hades desea tan solemnemente comunicar? ~---declaró Scream, con cierta sorna.

~---Comprendo su reticencia, la última vez que se encontró con mi señor fue bajo circunstancias extrañas para ambos. Sólo desea remarcar que él no está detrás de nada de lo sucedido. Eso es todo.

~---No ~---replicó Scream, severo~---. No lo es. Sabe más, pero no me lo va a decir.

~---Él confía en que se valdrán por sí mismos para frenar la crisis inminente que se avecina. No desea intervenir pues podría resultar poco conveniente a corto plazo en su esquema de actuación.

~---Así que se cree el protector de la ciudad, pero al mismo tiempo quiere que le hagamos el trabajo sucio. Muy noble por su parte, sin duda.

~---No, John Scream ~---replicó Caronte muy serio~---. Es mi señor el que está dispuesto a realizar el trabajo sucio que ustedes no desean hacer.

~---Dígale a su señor que nunca seremos aliados. Da igual lo que piense, ha cruzado una línea de no retorno desde hace mucho tiempo.

~---Lamento oír eso, sin duda. Sólo espero que su organización, algún día, no tenga que lamentarlo también ~---dijo, y se desvaneció ante los ojos de Scream. Otro invisible, pensó. Aunque aquel parecía usarlo más en términos teatrales, diplomáticos y, seguramente, de espionaje de los asuntos que estaban al tanto en la ciudad. Grande debía ser su influencia, pensó, si ya sabía más del asunto de los paramilitares de lo que ellos habían averiguado, que era más bien nada.

Se colocó justo donde su interlocutor había estado y miró hacia donde él había mirado. El horizonte negro y siniestro de edificios, coronados por la eterna Nube, siempre presente, siempre imposible de olvidar, trajo a su memoria que había muchos que, más que tratar de defender aquella ciudad y reparar sus múltiples y terribles defectos, preferían tratar de derribarla a partir de sus cimientos para luego construir su propio imperio naciendo de los escombros y las cenizas.

\parbreak
Repulsor no estaba contento desde el día que fueron atacados y Batería se esfumó delante de sus narices y las de sus compañeros. No es que él fuera el jefe de manera oficial ni nada parecido, pero en materia de actuaciones y movimientos tenía siempre la primera palabra y sabía que había sido su culpa que le capturaran, o al menos así lo había asumido. Había estado casi a su lado en algunos momentos, demonios. Y al igual que a John Scream le estaba pasando, su atención principal se había desviado por completo hacia ese problema al que no encontraba fácil solución.

El primer desagrado por su parte vino de la casi total negativa de Filo Omega a destinar parte de sus hombres a buscar a Batería. Decía que la seguridad de la autopista era lo más importante, y no apartaría efectivos para buscar a un desaparecido en combate, menos si era un cazarrecompensas.

Repulsor, a su pesar, tuvo que reconocerse que su manera de pensar era lógica, salvo por un importante detalle: Batería no era un sujeto normal, sino alguien que podía cargar máquinas, instrumentos, artefactos, lo que fuera necesario. Era probable que le estuvieran utilizando para que, donde quiera que tuvieran instalada su base de operaciones, los picos de energía no delataran su posición. Pero aun así, Filo Omega no daría su brazo a torcer. Al fin y al cabo, si eso era cierto, bastaba con seguir investigando tal como estaban haciendo. Llegar a los paramilitares, y de ese modo también hasta el secuestrado.

Repulsor empezó a comprender que, si usaban la vida de Batería como garantía de seguridad ante una mujer que debía tener una capacidad de negociación nula, la vida de su socio ~---no llegaría a llamarle amigo~--- podía valer tanto como la ceniza que impregnaba todas las calles de la ciudad.

Fue por ese motivo que decidió que, ya que tendrían que montárselo por su cuenta para detenerle, pediría ayuda a un viejo conocido que sabía estaba también en el planeta. Se comunicó con él y, como esperaba, no lo dudó a la hora de virar el rumbo de su nave en dirección a Ernépolis de manera temporal.

Además, su socio multibrazos estaba en ese momento enfrascado en una cacería más larga y complicada de lo normal, solo y a su aire en un terreno en el que no podía recibir ninguna clase de apoyo aéreo, por lo que no pondría objeciones a su ausencia temporal.

\parbreak
Repulsor, Barrera y Silencio no recordaban la nave de Dobleseis, la Snake Eyes, tan mugrienta y abollada cuando la visitaron la última vez, en plena vorágine de recompensas cruzadas, cuando por todos ellos se podía sacar tajada y no era conveniente dejar la espalda demasiado al descubierto. También era cierto que había pasado ya tiempo suficiente como para que gran cantidad de hechos fortuitos y no tan fortuitos la hubieran reducido a semejante estado.

~---Tomad asiento, damas ~---ofreció Códec dejando las herramientas de trabajo y tumbándose a lo largo del sofá, pitillo en mano, en lo que los demás se acoplaban allá donde había un hueco libre o no demasiado atestado de trastos inútiles como componentes inservibles o restos de embalaje.

~---Parece que no tenéis mucho tiempo de hacer limpieza por aquí ~---observó Repulsor.

~---Hemos tenido varios trabajitos seguidos, incluyendo algunas paradas inesperadas y fallidas, una de ellas recientemente aquí ~---contestó Códec, encendiendo el cigarrillo~---. Esta ciudad siempre acaba atrayendo a todas las malas hierbas.

~---Supongo que eso lo dirás por nosotros, viejo amigo ~---bromeó Repulsor, aprovechando el pitillo encendido de Códec para encenderse él también otro.

~---Pero pasemos al meollo de la cuestión, entonces. Dices que Batería ha sido secuestrado. ¿No tendrían algo concreto contra él? Siempre ha sido un gañán con facilidad para caer gordo a todo al que se dirigía.

~---Sé que no es el tipo más carismático del mundo, pero creo que esto va más allá de lo personal.

~---Se lo llevaron esos paramilitares ~---añadió Barrera, sentado con los brazos apoyados en los muslos, encorvado todo lo grande que era. Silencio asintió.

~---Ya veo. Así que a lo mejor tenemos que montar nuestra propia guerrilla.

~---Eso ya lo somos nosotros sin que nos eches una mano. Lo que necesitamos es cobertura, un cuartel itinerante. Poder rastrear la ciudad desde el aire, realizar un escáner de energía y temperatura.

~---Porque crees que si se están aprovechando de los poderes de Batería no tardaríamos en encontrar aunque fuera una pista ~---Códec dio una calada al cigarrillo~---. Bien, de acuerdo. Mientras 6-6 no se entere, estará bien. Yo soy el que lleva las cuentas, pero adivinad quién es el más tacaño en recursos de los dos ~---dijo apagándose el pitillo en la mano como si nada~---. En lo único en lo que no escatima es en dados. Los tiene al por mayor por todas la puñeteras salas. Configuraré los parámetros, entonces.

Los tres cazarrecompensas se levantaron y aprovecharon la ausencia de su anfitrión para dar una vuelta por el interior de la nave. Aunque echaron un vistazo a la sala de motores y las estancias públicas, no tardaron en recalar en la armería, el lugar al que en realidad estaban deseando ir desde el principio, más por instinto de cazador que otra cosa.

Una vez allí comprobaron que estaba tan sucia y mugrosa como el resto de la nave, uno de los inconvenientes de estar enfrascados en un safari, como solían llamar en la jerga a aquella búsqueda de objetivos encadenados uno detrás de otro. Se preparaba todo el armamento y la planificación como si de un tour de tratara, y no se pensaba en otra cosa hasta que se había llevado a cabo. Debido a ello Dobleseis no echaría en falta a su compañero momentáneamente. Aparte de eso, a juzgar por los huecos vacíos, llevaba encima suficientes armas como para enfrentarse él solito a todo un regimiento, si no era eso lo que estaba haciendo.

Silencio no tardó en fijarse en una sábana que cubría al fondo lo que parecía una especie de gran trípode montado, o algo similar, y Barrera no tardó en notarlo también. Repulsor se giró en la misma dirección que ellos, pero no expresó sorpresa alguna.

~---¿Qué es? ~---preguntó Barrera, intrigado, sabedor de que obtendría una respuesta.

~---Una reliquia de los viejos tiempos. Un cañón que solíamos usar en batallas espaciales, cuando éramos socios, y que un buen día desmontamos para dejar sitio a un propulsor extra. Estuvimos realizando en su momento experimentos para tratar de dotarlo de más potencia de fuego. Ignoro si Códec ha avanzado algo al respecto.

~---No, no lo ha hecho ~---contestó Códec entrando en la armería~---. Ya está calibrado el escáner de energía de la nave, por otro lado, aunque la señal debe ser muy clara para que logre encontrar algo, me temo. En cuanto a esta belleza ~---dio un golpe a la sábana que resonó a metal robusto y pesado~---, nunca volvimos a usarlo porque básicamente no hubo necesidad de emplear tan demoledora potencia de fuego. Un disparo de esto volaría cualquier puerta de seguridad. Además de eso, su estructura interna era tan resistente que podríamos usar plasma líquido como proyectil y no sufriría apenas desperfectos internos después del disparo. Si queréis provocar un hongo de tamaño modesto, esta es vuestra arma.

Barrera se acercó al artefacto y trató de levantarlo. A pesar de su fuerza, no lo movió ni medio milímetro.

~---Buen intento, chaval ~---objetó Códec~---, pero hacen falta tres personas para manejarla y una grúa para moverla. O eso, o acoplarla a la nave de nuevo, claro.

De repente el escáner empezó a zumbar como loco, y los cuatro mercenarios se miraron extrañados.

~---¿Tan pronto? ~---comentó Repulsor.

~---La señal debe ser muy clara, entonces ~---dijo Códec encendiendo el monitor y mostrando la pantalla con el mapa de Ernépolis~I a vista de pájaro~---. Está casi sobre nuestras cabezas y son varias señales, muy erráticas. Aterrizaré en el tejado libre más próximo.

~---¿Señales erráticas? ¿Qué podrá ser? ~---trató de imaginar Barrera.

~---Tengo la sospecha de que si ponemos las noticias no tardaremos en averiguarlo ~---comentó Repulsor, reconfigurando el monitor para mostrarlas. En efecto, no tardaron en ver imágenes de la zona y unos chismes metálicos, algo más pequeños que un hombre, que volaban por todas partes como diablos enloquecidos, a tanta velocidad que resultaba difícil ver qué eran exactamente, aunque disparaban descargas en todas direcciones, como si hubieran perdido el control por completo.

~---Ahí tienes tu señal ~---comentó Repulsor mirando a Códec~---. ¿Sales con nosotros?

~---Trataré de cubriros con la nave, aunque será difícil disparar a una de esas cosas sin provocar daños colaterales.

Se acercó a la cabina de control y les abrió la compuerta para que salieran al exterior. Casi al mismo tiempo, mientras estaban bajando por la rampa de descenso, notaron un ruido por encima de sus cabezas. Silencio apuntó instintivamente, pero no tardó en bajar el arma, seguro de que no era una amenaza lo que estaba sobre ellos en ese momento.

El espía se dejó caer frente a ellos, con lentitud, como si la gravedad no le afectara. Después de eso les miró con sus ojos fríos y carentes de emoción.

\emph{Bienvenidos a la fiesta. Elijan a su pareja} ~---dijo señalando con la mano a aquellas cosas robóticas, en lo que el viento hacía ondear su gabardina y el ala de su sombrero.

\parbreak
Debido a que estaban patrullando casualmente por aquella zona, fue Grove y su escuadrón los primeros que se percataron de aquellas máquinas extrañas que habían aparecido y estaban provocando el caos a su alrededor, con movimientos tan impredecibles que podía decirse que, o eran aleatorios, o estaban al borde de la locura cibernética. Algunas de ellas se estrellaban contra las paredes, otras se llevaban por delante ventanas y escaparates de tiendas. No habían pasado ni cinco minutos desde que aparecieran y ya habían causado gran cantidad de destrozos.

Fue gracias al escuadrón de Grove que aún no había sido herido nadie, ya que lograron atraer la atención de aquellos artefactos, que los consideraron blancos móviles más atractivos que los ciudadanos que pisaban las calles en esos momentos. Por fortuna, debido a lo imprevisible de sus movimientos, nadie pensó que fuera extraño que muchas de ellas estuvieran disparando hacia lugares donde aparentemente, para la percepción humana, no había blanco alguno al que apuntar.

Grove era el único que permanecía visible, aunque frente a aquel enemigo no había diferencia alguna en ese sentido. Viendo a algunos de los civiles, además, no estaba seguro de que en ese momento no estuvieran más del lado de aquellas máquinas endiabladas que del suyo propio.

Supuso que los militares no tardarían en aparecer, pero ver llegar la nave de Dobleseis y, más aún, descubrir que tenía dentro a los miembros de Fortaleza, era la mejor noticia que en ese momento podía esperar. Les hubiera abrazado de haber podido, pero trató de parecer lo más sereno e imperturbable posible.

~---¿Cuántos son? ~---preguntó Repulsor, poniendo en marcha sus guanteletes, contento de saber que no tenía necesidad alguna de reprimirse a la hora de golpear y disparar.

\emph{Ocho en total. Eso nos deja dos para cada uno. Sugiero que vayamos de uno en uno y luego nos concentremos en la otra mitad.}

Más que nada, pensó Grove, porque en realidad eran ocho peleando, y no cinco. Pero eso no podía decírselo sin delatar la manera de operar de Los Caídos.

~---A la carga, entonces ~---dijo Barrera, agitando la vara en el aire a gran velocidad.

Grove trató de detenerles, de establecer un plan de ataque común, unas pautas estratégicas. Pero se dio cuenta de que en momentos como ese era donde podía verse más claramente que aquellos sujetos habían estado mucho tiempo trabajando en solitario.

Además de eso, Grove no pondría la mano en el fuego, pero hubiera jurado que les había notado felices antes de entrar en batalla. Como si realmente llevaran tiempo esperando algo así. Poder lanzarse a la refriega sin pensar en nada más que en repartir palos a diestro y siniestro.

Pensó si Scream les habría logrado contener, frenar antes de lanzarse a dar leña como fieras enfurecidas, no muy distintos de sus oponentes en ese sentido. Pero finalmente se planteó que eso no importaba demasiado y ya era demasiado tarde para siquiera pararse a pensarlo.

La contienda empezó poniéndose a favor de los recién llegados. Aquellas máquinas carecían por completo de astucia o capacidad de alterar su forma de pelear. Eran minions del tres al cuarto, poderosos, eso sí, pero en absoluto rivales para sujetos como ellos, maestros de la defensa, el sigilo y el combate cuerpo a cuerpo. Apenas unos minutos bastaron para convertir aquellos amasijos chiflados en chatarra.

O más bien, para reconvertirlos en chatarra. Porque cuando Repulsor agarró a uno de ellos de modo que no se descuajeringara demasiado, se dieron cuenta de que eso era lo que eran en realidad: un amasijo de cables, hierros y piezas mal atornilladas.

~---¿Qué demonios es esto? ~---preguntó Barrera, intrigado.

~---Son restos de componentes ~---explicó Repulsor~---. Ese es una pieza de un motor, aquel otro parece parte de un tanque de asalto.

\emph{Entonces ni siquiera son robots en el sentido de la palabra} ~---terminó Grove~---. \emph{Sólo un montón de piezas sobrantes y sin arreglo.}

~---Todo esto son trastos bélicos ~---aclaró Repulsor~---. Esto tiene que ver casi seguro con lo de los paramilitares que nos atacaron.

~---Tal vez prefieran mandar artefactos para pelear con nosotros ~---sugirió Barrera.

~---Puede ser, pero\dots

Repulsor se quedó mirando fijamente detrás de su compañero. Docenas de esos chismes empezaron a surgir de las esquinas, sin intención de ser demasiado sutiles, sólo de abrumar por mera superioridad numérica. Aunque llevar la cuenta no era imposible, desde luego tampoco era recomendable a la vista de la lluvia de disparos que estaría a punto de caerles encima.

Por fortuna para ellos la caballería acababa de llegar también, y los militares aparecieron en sus vehículos todoterreno, con Filo Omega montada en el primer convoy. Bajó a toda prisa, desenvainó su espada y se preparó para unirse a la pelea. Su reacción ante la batalla no era como la que había visto en Repulsor y los suyos. Sin duda disfrutaba con aquello pero no mostraba placer alguno, sólo la rigidez de quien no ha conocido otra cosa más a la que dedicar su vida por completo.

~---No quiero bajas ~---se limitó a decir sin emoción, como si siempre hiciera ese comentario, fuera cual fuese el nivel de amenaza del bando rival.

El subsiguiente tiroteo fue de tal envergadura que Grove no tardó en comprender que aquella situación sobrepasaba por completo a él y sus muchachos. Eso era poco menos que una guerra en terreno abierto, que al menos no duró demasiado pues la puntería del enemigo dejaba mucho que desear. Aun así, con todo, hubo varias bajas y numerosos heridos. Ni Repulsor, ni Silencio ni Barrera resultaron dañados a pesar de ser cogidos por sorpresa, pues éste último cubrió la retirada de todos y lograron parapetarse bajo la adecuada barricada que proporcionaba un vehículo deslizante mal aparcado y a cuyo dueño más le valía tener un buen seguro a todo riesgo.

Filo Omega se acercó a los restos metálicos de aquellas psicopáticas máquinas de matar y los observó con desprecio. De un tajo limpio, como había hecho varias veces a lo largo del combate, partió uno de ellos en dos mitades como quien abre una nuez con las manos. Por dentro no era muy distinto que por fuera.

~---Esto puede tener que ver con nuestro compañero desaparecido ~---sugirió Repulsor, acercándose a su posición.

~---¿Tenía esta cualidad acaso?

~---No que tengamos noticia, pero quién sabe de qué medios disponen esos insurrectos para torturarle o manipular su poder.

~---Parece que han salido del subsuelo ~---dijo Filo Omega mirando alrededor, y haciendo notar que la destrucción se circunscribía sólo a su alrededor~---. Rastrearemos los niveles inferiores de la ciudad.

Después de eso se marchó, sin dirigir la palabra a ningún otro de los presentes, ni siquiera sus propios soldados. La prensa no tardó en tratar de adentrarse en el epicentro de la contienda, pero los militares les cortaron el paso, al menos hasta que se llevaran de allí a los soldados abatidos.

\emph{Repulsor, Barrera y Silencio regresaron a la base del edificio donde la nave había aterrizado, y subieron las escaleras exteriores de emergencia hasta llegar al tejado. Una vez allí, justo antes de entrar en la nave de nuevo, una voz les detuvo.}

\emph{Es absurdo eso que dice, como buscar una aguja en un pajar.}

~---¿Tienes alguna idea mejor, acaso? ~---objetó Repulsor.

\emph{Dame tiempo. Trataré, además, de buscar refuerzos adecuados a nuestra situación. Pero necesito que seáis mi enlace con los militares. Que todo lo que sugiera lo hagáis pasar por una idea vuestra.}

~---Si eso sirve para hallar a nuestro compañero, cuenta con ello ~---proclamó Barrera.

Silencio entró en la nave como si aquello no fuera con él. Repulsor, por otro lado, no dijo nada. Sólo se giró y miró fugazmente a sus dos interlocutores. Sacó un cigarrillo del bolsillo, se lo puso en la boca e hizo ademán de encenderlo, pero finalmente lo tiró al suelo y lo pisó con desgana.

~---Jóvenes ~---protestó caminando, a paso decidido, hacia el interior de la nave de su antiguo socio y amigo.

\parbreak
Por otro lado, en alguna parte, alguien reflexionaba con los suyos sobre los resultados de un experimento que acababa de poner en marcha.

~---Son inútiles y muy vulnerables, sin duda ~---dijo de manera pensativa, sin adornar el comentario~---, pero lo importante es que podrá haber más de donde salieron éstos. Diría que la prueba ha sido todo un éxito.

~---¿Es seguro que nos fiemos de él? ~---preguntó un subordinado, preocupado.

~---Lo es. No puede hacernos nada, y nos necesitamos mutuamente. Juntos mejor que separados, sin duda ~---concluyó, mirando al fondo de la sala. Dio unos cuantos pasos y se plantó junto a Batería, al borde del agotamiento más absoluto, casi incapaz de hablar o pensar~---. Sólo espero que no forcemos a nuestro amigo común demasiado. Tenemos aún grandes planes para él ~---terminó, percibiendo a su prisionero más como un objeto que como un ser humano.
