Al filo de la navaja. Nunca habían estado tan cerca de la derrota definitiva. Pero al mismo tiempo eso otorgaba nuevas fuerzas, el renacer de un espíritu que por mucho tiempo había permanecido aletargado.

\fancyparbreak
La sombra siniestra de la gabardina y el sombrero llegó al callejón donde no mucho tiempo atrás se había producido una encarnizada pelea entre dos rivales concienzudos. Lo primero que buscó con la mirada era lo que más temía encontrar. Un cuerpo. El cadáver de quien había sido su jefe, su amigo.

La sombra siniestra tenía nombre y apellidos humanos. Respondía ante los hombres como Charles Razorclaw. Pero si llegaba a descubrir que el cuerpo de John Scream estaba en aquel callejón repugnante y destartalado sólo habría un concepto humano ante el cual respondería.

Venganza.

No encontró restos humanos de ninguna clase, pero sí las huellas que demostraban que hubo una feroz contienda en aquel escenario. Peculiares proyectiles hechos añicos. Impactos de láser en las ya de por sí maltratadas y desconchadas paredes. Imposible saber si eran de naturaleza letal. Distribuidos con precisión envidiable, propia de maquinaria de relojería. Parecía como si el pistolero hubiera nacido en aquel mismo lugar y lo conociera desde la más tierna infancia, o como sea que hubiera sido la suya.

Llegó a una esquina del callejón, más chamuscada que el resto de las zonas afectadas por las descargas. No había visto un rastro como ese en la vida. Quizá si alguien del equipo científico hubiera estado presente le podría haber explicado que era la traza dejada por una red electrificada como pocas en el mundo. Si hubiera sido el individuo adecuado, el indicado para la ocasión, podría haber sugerido, sólo sugerido, el arma causante de tal rescoldo. Y en su fuero interno estaría seguro de ello, pero aun así sólo seguiría sugiriéndolo, pensando que la prudencia era la base de la ciencia.

Fue entonces cuando la sombra siniestra, que cada vez parecía más una sombra y menos un hombre, vio los dos dados en el suelo, marcando un seis doble. Una firma. Una impronta, algo que rastrear, que verificar. Junto a la información que había recibido de Sam Grove, suficiente como para verificar las sospechas.

Se giró y miró a su alrededor. Allí ocultos, invisibles, otras doce sombras siniestras le observaban sin decir una sola palabra. Era tremendo el riesgo que estaban corriendo. Si John Scream les hubiera visto comportarse así se hubiera enfadado mucho, y lo peor es que habría usado los argumentos más lógicos del mundo para justificarlo.

Pero Scream no estaba allí, y situaciones desesperadas requerían medidas desesperadas. Además, en aquel momento era más que la razón lo que dominaba a aquellos sigilosos justicieros. Por dentro de ellos bullía una mezcla extraña de miedo, odio y valentía, un cóctel de emociones que corría el riesgo de volverse inestable en cualquier momento.

\emph{Se lo ha llevado. Vivo o muerto, no hay manera de saberlo. No hay sangre, al menos.}

\emph{¿Qué es lo que haremos?} ~---contestaron desde la penumbra del callejón. Resultaba perturbador que un ser tan atemorizante hablara y su propia voz le contestara, como si llevara a cabo una imposible conversación consigo mismo.

\emph{Volver al Aquerón. Cuanto antes. Si él está muerto, entonces ese tal Dobleseis, más que seis brazos, va a necesitar seis piernas para llegar allá donde no podamos alcanzarle.}

La legión de sombras se movió en perfecto silencio y se dirigieron a un pasadizo situado a apenas una manzana de allí. Todas menos una, que se paró frente a Razorclaw.

~---Lo lamento, señor ~---dijo quitándose el modulador de voz~---. Nunca debí abandonarle.

\emph{Seguiste una orden, Sam. Dudaste, pero la cumpliste. Ahora te ordeno que dejes de pensar en ello y vuelvas con los demás. Vamos a necesitarte en plena forma tanto como a cualquier otro.}

~---De acuerdo ~---fue la escueta respuesta de Grove antes de alejarse.

Razorclaw le miró irse, pensando que tampoco es que le hubiera dado un gran consejo, ni sus palabras hubieran sido una lección memorable que el chico recordaría hasta el fin de sus días. Él no era un líder. Era un comandante, la mano derecha, el apoyo necesario, pero no un líder. Y no estaba preparado para serlo.

Fue entonces cuando escuchó llegar a la policía y se preparó para seguir expandiendo los límites de la búsqueda.

Eran dos, patrulleros ordinarios. No les había visto en la vida, pero tampoco podía esperar conocer a todos los hombres del cuerpo de la ley, aunque sí que había coincidido con muchos en los juzgados. Parecían jóvenes y voluntariosos. En tiempos peores hubieran sido miembros de Los Caídos. En aquellos tiempos podían permitirse el lujo de pertenecer a un tipo de justicia que reportaría muchas más ventajas a su situación personal y profesional. Y les otorgaría también más horas de sueño, además.

Se dejó ver. Con solemnidad, pero sin pretender resultar una amenaza. A veces llevar a cabo el engaño con gente así era mucho más difícil que con el más repugnante de los criminales de las calles de Ernépolis.

\emph{Llegáis tarde} ~---fue su único comentario.

~---La ausencia de radios\dots\ nos ralentiza ~---dijo uno de ellos, como si no supiera cómo reaccionar y hubiera contestado sólo por mera inercia. El otro no habló y se limitó a mirarle, sorprendido. Debía estar pensando que los rumores sobre el Caído no eran en absoluto exagerados.

\emph{Ha habido una pelea. Uno de los que ha estado aquí, un cazarrecompensas, podría haber matado a un hombre. Esos dados} ~---señaló al suelo~--- \emph{son su firma. Si es sospechoso de asesinato, la policía puede ir a por él.}

~---Por lo que sabemos\dots\ tú podrías haberle matado. Si esos tipos están en la ciudad es porque han puesto precio a tu cabeza.

\emph{Escucha, novato. Dile a tu jefe que un cazarrecompensas llamado Dobleseis podría ser un asesino. Seguro que él lo entenderá mejor que tú. No en vano, por algo es Jefe de Policía.}

~---¿Pero quién es\dots?

Era tarde. La sombra ya no estaba allí, y nunca antes había lamentado tener que marcharse sin dar más explicaciones. No podía contarles a aquellos polis qué era exactamente lo que había sucedido, pero esperaba que Sky entendiera que algo grave había pasado y se pondría en contacto con él. No tenía tiempo para ir a visitarle y decírselo en persona.

Antes de eso, tenía a todo un ejército de tinieblas que comandar en una guerra contra un solo hombre, o lo que quiera que fuese aquel mercenario multibrazos.

\parbreak
Nada más llegar al Aquerón comprobó que todos los miembros estaban allí esperándole, ya reunidos en el hemiciclo del módulo central. Esperaban órdenes con tremenda impaciencia. De hecho era más que evidente que muchos de ellos se estaban conteniendo para no salir a hacer de llanero solitario por los bajos fondos de la ciudad.

Razorclaw se quedó mirando a todos aquellos voluntariosos defensores y se planteó cómo hacía Scream para tomar la voz cantante y sonar tan decidido, tan enérgico. No tardó en darse cuenta de que la única manera de saberlo era por medio del método de prueba y error.

Más que nada, porque podía ocurrir que no volviera a tener la ocasión de preguntárselo.

Miró a Saw, esperando a escucharle hablar, sin duda el responsable de haberles reunido y calmado, a la espera de su inmediato regreso. Entre otros rostros se fijó también en Swind, con los puños apretados, y Grove, ya recuperado el aplomo que nunca debería haber perdido. Tuvo también un momento para dedicarlo a los miembros de su propio escuadrón.

~---Estamos en una situación delicada ~---comenzó con calma, intentando entrar en el ritmo del discurso~---. Por un lado debemos y queremos buscar a John Scream con todos los medios que estén a nuestro alcance. Pero por otro no podemos traicionar el principio de la organización y lanzarnos a las calles sin un mínimo de prudencia y cuidado. Las comunicaciones siguen interceptadas, otro grave problema en nuestra contra. Aquel que encuentre a John Scream estará solo o, a lo sumo, en pareja, y deberá tomar una decisión. Si se marcha, puede pedir refuerzos, pero también podría perder la pista de John. Si se queda, podría ser capturado o morir, y no sabríamos nada de su descubrimiento.

\rquoti Solos cubriréis más terreno, pero seréis también muy vulnerables. No conviene que os mostréis salvo para recabar información. Cada uno de vosotros tendréis sólo cinco minutos para dejaros ver a los ojos de vuestros informadores, fuentes, contactos, lo que sea. Así evitaremos coincidencias temporales masivas y prolongadas. Subdividiremos el calendario habitual en unidades más estrictas y pequeñas.

\rquoti Algunos, los menos experimentados, pasaréis apuros. En ese caso destruid todos los aparatos importantes. El laboratorio científico, además, está lleno de prototipos fallidos de partes importantes de nuestro equipo. Nada más acabar se le dará a los voluntarios uno de ellos, y en caso de ser atrapados, fingirán ser imitadores del Caído. De ese modo evitaremos que los medios sumen dos y dos. No debéis preocuparos por vuestra integridad, en cuanto el Jefe Sky os vea os reconocerá al instante y podréis contarle lo que está pasando. Pero aquellos de vosotros que seáis capturados, ya nunca más podréis salir a las calles de nuevo. Podéis trabajar aquí dentro, aunque entiendo que para muchos de vosotros eso será a todas luces insuficiente.

\rquoti Creo que ya he dicho todo lo que tenía que decir. Ahora, que los voluntarios se pongan en pie.

Como era de esperar no hubo uno solo de los miembros de escuadrones del hemiciclo que no se levantara de su sitio. Razorclaw estaba orgulloso de ello, y sabía que Scream lo estaría también.

Pero la aflicción recorría su mente. Porque sabía que muchos serían capturados, o peor aún, se verían obligados a pelear contra algún cazarrecompensas. Y si bien no todos eran tan peligrosos como Dobleseis, tampoco ellos eran guerreros tan notables como el líder de Los Caídos.

Gran parte se vería obligada a dejar las filas. Otros podrían llegar hasta a inmolarse, arder sin vacilaciones para proteger el secreto de la organización con la perfección más absoluta.

Y tal vez, sólo tal vez, la búsqueda podría ser totalmente en vano.

\parbreak
Cuando John Scream recuperó el conocimiento no tardó mucho en darse cuenta de que estaba encerrado en el interior de uno de los muchos almacenes abandonados que circulaban por la zona oeste de Ernépolis. Podía incluso estar a pocos metros de Gorgon Enterprises, o de alguna otra entrada segura y rápida hacia el cuartel general. Bajo circunstancias favorables lo único que tenía que hacer era levantarse, esconderse entre las sombras del almacén, esperar su ocasión para tomar por sorpresa al enemigo y salir hacia el pasadizo más próximo, directo al Aquerón. Sólo había un problema al respecto.

Las circunstancias no eran ni mucho menos favorables.

Dobleseis estaba justo frente a él, de pie, cuatro de los seis brazos cruzados, los otros dos cerca de las cartucheras de la cintura. Podía haberse pasado así horas o llevar sólo minutos, para Scream era imposible discernir cuál era la situación real. Pero lo importante, sin duda, era la imagen que quería dar, la de no haberle quitado ojo en ningún momento. Cada movimiento estaba también más que elegido en la actitud del cazarrecompensas.

Le miró fijamente, sin apartar la vista. Comprendió que su casco reflectante era una eficaz manera de distanciarse emocionalmente de la presa. Sin rostro visible no hay compasión ni tampoco incertidumbre. Ni siquiera odio o rabia.

Scream llevaba aún todo el atuendo, pero el equipo, en su mayor parte, no estaba. Lo único que conservaba era el modulador de voz, supuso que debido a que no podía ser de ninguna utilidad a la hora de rebelarse contra su captor.

\emph{¿Dónde estamos?} ~---preguntó sin tratar de fingir misterio o altivez.

~---Eso no importa. Lo único que me interesa es que es aquí donde he acordado la cita con quien puso precio por tu cabeza.

\emph{Podría pagarte más de lo que él te paga} ~---dijo Scream intentando ganar tiempo, tratando de pensar qué era lo que podía hacer para distraer a su interlocutor.

~---Tengo una fama que mantener, John Scream.

\emph{Veo que no sólo eres rápido con las pistolas, también con la información.}

~---Yo no, en realidad. Pero basta de hablar de mí.

\emph{Muy bien, hablemos de tus armas, entonces. ¿Qué hacías con un dispositivo tan sofisticado como ese en tu poder?} ~---preguntó Scream mirando la cartuchera donde estaba el antiguo revólver de Silenciador.

~---Eres observador, sin duda. Sabes apreciar los detalles importantes. Pero yo también, y me he dado cuenta de que ya antes de que te disparara con él te sorprendió verlo entre mis manos.

No dijo nada más, y Scream empezó a darse cuenta de que la próxima batalla sería con las palabras, más que con las armas.

\emph{No sabes de dónde es, ¿verdad? Desconoces quién fue su anterior dueño.}

~---Pensé que podrías ilustrarme al respecto.

\emph{Lamento decepcionarte. ¿Cómo llegó a tus manos? ¿La robaste, acaso?}

~---Yo no soy un ladrón, Capitán Scream. Soy un profesional, y éste arma fue el pago por uno de mis trabajos en el pasado. Un experto en balística me la dio; según él, la encontró entre los restos del trabajo de uno de sus rivales. El caso es que el rastro de ese arma se pierde entre la maraña de manos inexpertas que lo han tenido entre manos.

\emph{Dices que no eres un ladrón, pero tienes mis anuladores de fotones.}

~---Mi socio los está estudiando. Son un juguete ingenioso, sin duda. Pero no es mi intención quedármelos, ni mucho menos. Los adaptaré a mi estilo, y una vez lo haya hecho los destruiremos. De todos modos no te servirán de mucho una vez te hayan metido en una celda y tirado la llave.

\emph{Sea quien sea quien me quiere, no me quiere vivo. Si ha dicho lo contrario se trata de un engaño para no incurrir en un delito, al menos de cara a la opinión pública. Pero a ti te venderá, Dobleseis. Te lanzará a los perros para borrar su propio rastro.}

~---Deja que resuelva mis propios problemas ~---dijo, y acto seguido se giró de repente~---. Ya han venido a por ti.

Sin perder de vista a su prisionero, el cazarrecompensas amplió su ángulo de visión y desenfundó un par de lanzarrayos. El resto de las manos se mantuvieron en posición de alerta también. Resultaba fascinante cómo podía efectuar actitudes aparentemente disjuntas con los distintos juegos de brazos.

Un grupo de varios sujetos empezaron a distinguirse al fondo. La oscuridad impedía que se les viera con claridad, pero Dobleseis comprendió que algo no marchaba tal y como había esperado. Tal vez fuera por su manera de caminar, enérgica y decidida, por lo que entendió que eran compañeros de su misma profesión. Desenfundó el resto de sus armas y mantuvo la posición sin perder de vista sus objetivos, cuatro en total.

~---Mira lo que trae la polución consigo ~---dijo Dobleseis mirando de reojo hacia Scream, que había hecho un ligero amago de levantarse que el cazarrecompensas no impidió~---. Si buscáis quitarme la presa será mejor que os lo penséis dos veces.

Scream echó un vistazo a los recién llegados. Gracias a los informes que había recibido de días anteriores y a las bases de datos de aduanas que le había facilitado Sky, no tardó en reconocerles: uno de ellos, anciano y de mirada sagaz, respondía al nombre de Silencio y era un maestro de la infiltración. A su lado, alto, fuerte y con una larga y extraña vara metálica llena de circuitos electrónicos en su punta y parte media, estaba otro que era conocido como Barrera, capaz de generar toda clase de escudos físicos y energéticos.

Batería era el tercero, y resultaba fácil distinguirle por su brazo biónico, con la capacidad de amplificar la potencia de máquinas hasta límites insospechados. Y el último de ellos, con una servomochila a la espalda y cables que terminaban en sus guanteletes, estaba Repulsor, alguien que, como Los Caídos, no solía tener que pelear para lograr sus objetivos, debido a la fama que le precedía.

Aun con todo, Scream sabía que por separado no eran rival para Dobleseis. Sin embargo, se planteó qué les había llevado a juntarse para capturarle, teniendo en cuenta que esos tipos eran, antes que ninguna otra cosa, sujetos que nunca antes habían trabajado en equipo.

Se pararon a una decena de metros de Dobleseis, y fue Repulsor el que tomó la palabra en primer lugar.

~---Hola, 6-6. En algún momento teníamos que reencontrarnos.

Dobleseis no respondió. No hacía falta mirar indagar el rostro de detrás de su casco para notar que estaba irritado por el cambio inesperado de los acontecimientos.

~---La última vez lo dejé en tablas, Repulsor. No me gusta matar si no saco beneficio de ello. Pero en este momento tú y esos monos de feria estáis a punto de conseguir captar toda mi atención.

~---Qué ironía, 6-6. Resulta que eres tú quien ha captado la nuestra, ¿o estabas tan ocupado que no te has enterado de quién es la nueva presa del día?

Dobleseis no dijo nada. No hacía falta darse cuenta de que con el problema de las comunicaciones las noticias en Ernépolis no llegaban tan rápido como uno pudiera desear.

~---Tú, Dobleseis. Han retirado la recompensa sobre esos jirones que están a tu espalda ~---dijo señalando hacia la sombría presencia de Scream con uno de sus guanteletes. Y la recompensa por ti es, además, notablemente cuantiosa.

\emph{¿Cuándo fue eso?} ~---preguntó Scream de repente.

~---Nadie te ha dado vela en este entierro, fantasma ~---se limitó a contestar Barrera. Los otros no dijeron nada.

\emph{Escucha, cretino} ~---se limitó a decir Scream, irguiéndose de nuevo y dejando que su silueta acompañara la contundencia de sus palabras~---. \emph{Ha sido recientemente, ¿verdad? Hace unas horas, quizás.}

~---¿Y qué si ha sido así? ~---preguntó de nuevo Repulsor.

\emph{Casualmente van detrás de Dobleseis justo después de que me capturara y tratara de comunicarse con quien ha puesto el dinero para que me atraparan.}

Dobleseis escuchaba a Scream al mismo tiempo que no dejaba de mirar a los cuatro recién llegados. Silencio y Batería llevaban un par de lanzarrayos, y los otros dos iban equipados con las armas de sus propios trastos de combate, por un lado la vara y por otro los guanteletes. De vez en cuando los dedos de todos ellos dejaban escapar un ligero tic nervioso. Allí había un polvorín en potencia.

Dobleseis empezó a reírse.

No era una risa escandalosa, ni tampoco hueca. Se reía de verdad, por una situación que le había hecho auténtica gracia.

~---Curioso ~---repitió para sí mismo, sin variar la postura de sus seis brazos ni un solo segundo~---. Decís venir para capturarme, pero en realidad creo que aquí todo el mundo tiene algo que ocultar.

Scream les miró fijamente, y por fin comprendió lo que Dobleseis quería decir.

\emph{Por todos. Hay una recompensa por todos y cada uno de vosotros.}

~---Por eso se interceptaron las comunicaciones antes de que llegáramos ~---continuó Dobleseis~---. Para que no pudiéramos hacer equipo entre nosotros. Y esta sombra fugitiva ~---giró ligeramente la cabeza hacia Scream~--- era sólo un señuelo.

Los cazarrecompensas se miraron entre ellos. Todos sabían que se había puesto precio a la cabeza del otro, pero todos fingieron no saberlo. Estaban jugando a un mismo juego. Unirse temporalmente contra el más peligroso de ellos, y una vez fuera de la partida, arreglar sus problemas de manera personal.

Pero ahora que las cartas estaban al descubierto muchos de ellos empezaron a dudar. Tal vez podían hacer equipo\dots\ o tal vez no. Si confiaban en el otro, éste podía atacar por la espalda en el momento apropiado y cobrar la recompensa.

Además, habría más demanda y menos competencia para satisfacerla.

~---¿A cuánto asciende mi recompensa? ~---preguntó Dobleseis.

~---Tres millones ~---respondió sin titubear Repulsor.

~---¿Y las vuestras?

Ninguno habló en un principio. Finalmente, fue Batería el que tomó la iniciativa.

~---Tres cuartos de millón por cada uno ~---aclaró.

~---De modo que la recompensa por mí vale tanto como por todos vosotros juntos ~--- agregó Dobleseis con tono de sorna~---. ¿Cuánto por mi socio?

~---No hay recompensa por él ~---volvió a contestar Batería.

~---Eso no le va a gustar, no.

\emph{Escucha bien, Dobleseis} ~---continuó Scream, tratando de hacerle razonar~---. \emph{Esto es una trampa. Alguien ha orquestado todo esto para que os matéis entre vosotros. No puedes seguirle el juego.}

~---¿El juego? ~---replicó Dobleseis, volviéndose poco a poco de modo que tenía a todos al alcance inmediato de la mirada~---. Sí, tienes razón. Esto es un juego. Pero para mí la vida entera es un juego. Es cuestión de suerte, de azar. A veces sacas los dos seises, a veces no. ¿Y sabes qué?

Seis dedos se cerraron ligeramente sobre sus respectivos gatillos. Todos los presentes tensaron brazos y piernas, a la espera de reaccionar ante el inminente estallido de violencia.

~---Me gusta este juego ~---acabó justo antes de empezar a disparar.
