\begin{prev}
    Tras escapar Armor usando la nave prototipo de Gorgon Enterprises, con destino a la Luna, John Scream se lanzó a por él a bordo de la Trigger, su viejo vehículo espacial. El enfrentamiento titánico entre ambos acabó, no sin un tremendo esfuerzo, con la destrucción de Armor. Los Caídos, una vez más, habían triunfado. Pero toda victoria tiene sus consecuencias.
\end{prev}

\noindent
Una vez más, las aguas volvieron a su cauce. La normalidad regresó a las oscuras calles de Ernépolis~I. Era el momento de reflexionar sobre lo sucedido, realizar un informe de daños, y evaluar los resultados del mismo. No sólo a escala de la organización, también a nivel personal.

\bigskip\noindent
James Sky miró al fondo del local, donde John Scream estaba charlando animadamente con Emma Blades, su pareja por aquella noche. Aquel detalle, que podría haber provocado recelo en otros, no lo hizo en Sky. Confiaba en su amigo, y sabía que no trataría de hacerle ninguna jugarreta al respecto.

Era consciente, además, de que la presión a la que se veía sometido le dejaba pocas posibilidades de disfrutar sin reservas una noche, y aquella era una de ellas. No sin grandes esfuerzos, Sky le había convencido para que delegara responsabilidades por una vez y fueran juntos a relajarse después de días tan llenos de tramas, intrigas y conflictos armados en la sombra. Había, de hecho, muchos otros miembros de Los Caídos en la sala, por lo que era una buena ocasión para todos de disfrutar un merecido descanso.

La guerra, además, entró en fase de declive. Ya habían surgido los primeros tratados, en los que la Tierra otorgaba numerosas libertades a algunos de los insurrectos a cambio de ciertas concesiones en términos de comercio y relaciones diplomáticas. Otros mundos, sin embargo, serían reconquistados sin piedad, entre ellos el hogar natal de Felicity Hound, Scorpon. De ese modo la Guerra de las Ocho Colonias parecía estar próxima a su fin, aunque seguramente no tardarían en llegar conflictos posteriores.

Sky apuró su vaso y echó un vistazo a la calle. Seguía cayendo ceniza. Algunos climas no cambian nunca, pensó, y no sólo pensaba en el tiempo. A pesar de la muerte de Gorgon aún había muchos rincones oscuros en la ciudad, mucha corrupción y decadencia, y monstruos por todas partes esperando para surgir de las sombras.

En efecto aún había mucho que hacer. Y a ese respecto se alegró de que al menos la locura de todos esos días le hubiera ayudado a comprender al fin dónde estaba su sitio.

~---Se te ve silencioso, amigo ~---comentó Scream regresando a su lado, en lo que Blades iba a pedir una copa~---. Armor al fin ha desaparecido, y el ejército ha declarado públicamente que ha destruido todos los prototipos y pruebas de laboratorio que experimentaban con tecnología de servoarmaduras, incluyendo sus restos. Todo ha terminado, pero aun así estás sombrío. Cualquiera diría que nos hemos cambiado los puestos.

~---He llegado a una determinación, John.

Scream no dijo nada, sólo le dejó continuar. No quería interrumpirle en ese momento, cuando lo mejor que podía hacer era escuchar.

~---Dejo la organización, John. Creo que no te coge por sorpresa.

~---No, no lo hace ~---respondió Scream de manera escueta.

~---No sé qué es lo que pensarás, pero considero que es lo mejor. Seguiré colaborando con vosotros, por supuesto, pero como el Jefe de Policía James Sky. Estaba empezando a resultar complicado para mí ser al mismo tiempo el dirigente de las fuerzas del orden de la ciudad y un alto cargo en una organización clandestina que opera al margen de la ley.

~---Lo entiendo, y me parece lógico. A medida que la ciudad se va recobrando de la etapa dictatorial que sufrió bajo el yugo de Gorgon las instituciones deben recuperar su esplendor perdido. Sin embargo lamento perderte como compañero, que no como aliado.

~---Me consta. Pero somos amigos, y eso no cambiará nunca. ¿Cómo te sientes tú, por otro lado? Sé que fue duro para ti volver a pilotar, más aún tratándose de esa nave, y que Felicity Hound te recordaba en cierta manera a Aryn Life, ese amor ya perdido.

~---Era algo que tenía que suceder tarde o temprano. Al fin y al cabo, mi pasado llevaba tiempo muerto.

~---Es Reflector el que murió, no John Scream. Nunca olvides eso. Una cosa es que finjas ser una criatura desarraigada de las sombras, y otra muy distinta es que acabes convirtiéndote en una de verdad.

~---Las cosas funcionan así de momento para mí. Supongo que soy un actor del método Stanislavski, para interpretar un papel debo sentirlo por completo.

~---¿Qué ocurre por aquí, chicos? ~---comentó de repente Blades apareciendo con una copa en una mano y la boquilla en la otra~---. Esto parece un funeral. Vamos, divirtámonos un poco. James, espero que no tardes en sacarme a bailar o se lo pediré a John.

~---¿A quién, a Mister Aburrido? ~---replicó Sky con una de sus clásicas sonrisas irónicas~---. Ahora mismo voy, no tardo nada.

Blades se marchó y se limitó a quedarse junto a la barra, esperando a su pareja de baile en lo que bebía a sorbos lentos de la copa.

~---Está un poco loca y tiene algo de apego por acumular toda clase de información poco recomendable, pero en el fondo es buena chica ~---bromó Sky mirándola desde la distancia.

~---Cuídala bien, James. Es inquieta y despierta. Es difícil encontrar a alguien así en estos tiempos que corren. Seguramente ya sospeche algo, también.

~---¿El qué, que soy un enmascarado que salta de tejado en tejado? ¿Que lo eres tú, tal vez? ¿O que formamos parte de una única organización? Ninguna de esas tres cosas es ya del todo cierta.

~---Ni del todo falsa ~---puntualizó Scream.

Por un momento los dos amigos se quedaron sin decir nada, sólo reflexionando para sus adentros. Era mucho lo que habían pasado juntos, y ya nada podría borrar eso. Al fin y al cabo ambos, al igual que sus otros compañeros, habían fingido ser una sola clase de persona, y eso crea, sin duda, vínculos especiales entre los hombres.

\bigskip\noindent
Lejos de allí, en el laboratorio experimental del ejército en la Luna, un equipo de científicos había estudiado los restos que quedaban del ser que había sido previamente conocido como Armor. Maquillando de manera adecuada sus palabras habían declarado a los medios que la servoarmadura había sido robada y no hubo más remedio que destruirla una vez recuperada, pues su producción en cadena conllevaría una escalada militar sin precedentes, además de efectuar ese gesto de desarme como señal de buena voluntad para con las colonias que habían acordado firmar una tregua.

Existían ciertos matices en aquella afirmación, claro.

Había sido recuperada, sin duda. No de una pieza precisamente, debido a aquella potente explosión en la que había sido alcanzada de lleno, pero esencialmente todo lo importante estaba ahí. También había sido destruida, por supuesto. No por ellos, pero ya no parecía poder funcionar nunca más.

Procederían a su desmantelamiento definitivo, pero antes tenían que recuperar todo el trabajo perdido. El mecanismo de ensamblaje de las piezas, la aleación de la coraza, el proceso de conversión de electricidad. Todo eso podía ser reciclado en provecho de la humanidad, por supuesto. Lo bueno de las guerras era que el progreso científico avanzaba cuatro o cinco veces más rápido que en cualquier época de paz y prosperidad.

Había que tener cuidado, claro. Aquel material era peligroso y convenía tratarlo con sumo cuidado. No ponerlo cerca de otros aparatos eléctricos, ni en contacto directo con máquinas ni redes de información de tipo alguno, especialmente la pieza correspondiente al yelmo. Quién sabía qué capacidades había podido desarrollar desde que cobró autonomía propia.

Incluso algún que otro miembro del equipo estaba empezando a dejarse llevar por supersticiones sin fundamento. Mismamente, uno de ellos creyó haber visto cómo el visor del yelmo brillaba ligeramente, con una luz fugaz, débil, como de una linterna que se apaga. Estudiaron el fenómeno, escanearon el componente, buscaron restos de energía residual, pero no hallaron nada. Estaba apagado por completo.

Concluyeron que sólo era una falsa alarma, seguramente un destello reflejado sobre la lisa y bruñida superficie del cristal reflectante y blindado.

\begin{next}
    Tras los doce primeros números, ¡tenemos un capítulo especial de larga extensión! En esta historia autoconclusiva, John Scream viajará a una lejana colonia, donde conocerá nuevos y apasionantes aliados. ¡No te lo pierdas!
\end{next}

\endinput
