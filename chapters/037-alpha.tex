Tras la irrupción de lo personal en el esquema general era el momento de volver a centrar la mirada en la ciudad. Tiempos difíciles se vislumbraban en el horizonte, y volvían a entrar en juego mecanismos de control que por largo tiempo habían permanecido al margen de la situación\dots

\fancyparbreak
Casi al mismo tiempo que la luz regresaba a las calles de Ernépolis se desvanecía en el interior de sus habitantes. El fin de Alma Espejo fue más duro de sobrellevar de lo que había parecido en un principio. El paso de las semanas no logró aliviar la ausencia que suponía no vislumbrar su fulgor en lo alto del cielo, o ver las noticias en los inmensos monitores y descubrir que ya no volvería a aparecer jamás.

Hubo rumores, también. Rumores de que no fue un héroe todo lo grande que debía haber sido. Rumores de que había perdido el norte, de que él mismo fue el causante de su perdición. Pero no se trataron de chismes propagados de manera malintencionada por alguien concreto, o enemigos que quisieran difamar a los que ya no estaban para defenderse. Simplemente se trataba de un conjunto de opiniones más del crisol que envolvía a aquella emblemática figura.

Porque Alma Espejo fue más que emblemático para la ciudad, sin duda. Reflector también lo fue, y otros como él. Pero un buen día Reflector desapareció sin dejar rastro. Nunca más se supo de él. Como si nunca hubiera existido.

Alma Espejo, sin embargo, cayó a la vista de todos, acrecentando aún más, si cabe, la tragedia de su pérdida. Ya no había un modelo que seguir, un ejemplo al que regresar. Otros héroes que pudieran llegar a la ciudad habían sido advertidos con el mensaje más cruel que uno pudiera imaginar. Éste será vuestro destino. Abandonad toda esperanza, los que en esta ciudad entráis.

Hubo momentos de mucha tensión, como cuando Perséfone llegó a la ciudad. Hubo enfrentamientos de amigo con amigo, aliado con aliado, a todos los niveles sociales existentes. Se impuso la mano dura en los podridos callejones de Ernépolis I. Ya que el gran protector no estaba alguien tendría que realizar el trabajo sucio a las malas, ya que a las buenas no había funcionado.

Eso no sólo incluía a Los Caídos, también a la policía, al menos en un principio. El odio y rencor no se desvaneció fácilmente. No fue suficiente cazar y ajusticiar criminales más por clamor de venganza que por crímenes cometidos, como fue el caso de Perséfone. Más de un policía decidió tomarse la justicia por su mano, y el Jefe Sky tuvo muchos altercados en esos días, dentro y fuera de sus filas. Incidentes que le trajeron a la memoria tiempos terribles, en los que el peligro podía encontrarse tanto en el delincuente al que pretendías detener como en el compañero con el que estabas llevando a cabo el arresto.

El crimen en la ciudad sufrió un brutal repunte, además. Una escalada de violencia paralela al odio que manaba de cada persona, de cada esquina, de cada rincón de un lugar en el que cada vez parecía que la ley era más una impostura y unas reglas distintas marcaban el juego real.

Y en medio de toda aquella vorágine, de aquel maremagnum que parecía amenazar con devorarles, Los Caídos habían afianzado su lugar como defensores no deseados de un mundo que cada vez parecía odiarles más.

Starr Miles dijo que tenían que parecer tan amenazantes como los enemigos a los que pretendían combatir. Nunca habían estado tan cerca de lograr algo parecido a los ojos de la opinión pública, sin duda.

Ya después de la primera incursión de Hades dieron la imagen de ser un ser desalmado e incomprensible, caprichoso como sólo una criatura que no se mueve por designios humanos puede ser. Como si su único propósito fuera esparcir el miedo a lo largo de la ciudad. Pero después del incidente de Alma Espejo, más que nunca se afianzó la idea de no ser el Caído mejor que Hades, de querer destruir y arrasar con las esperanzas de todos los habitantes, sin excepción. Bien sabida era la animadversión de Alma Espejo hacia ese ser que se movía a sus anchas por las calles que tanto había jurado proteger.

Y sin embargo la ironía era que nadie lamentaba más la marcha del ente luminoso que aquellos que habían sido, en cierto modo, los encargados de plantarle cara en el momento final. Para Los Caídos él no sólo era la esperanza de la ciudad, era la de ellos también. La oportunidad de sentir que ya no eran necesarios, que otros podrían tomar el relevo. Que la era de los héroes podría repetirse sin que volviera a sucumbir bajo condiciones gemelas a su abrupta finalización.

Pero era una quimera, un imposible. Los nobles y los honrados se corrompen o mueren. Sólo queda sitio para aquellos que esconden su verdadera condición en un entorno donde la mezquindad, la crueldad y el horror campan impunes a sus anchas.

En un principio se planteó si no sería adecuado decir la verdad. Inmunizar a la población, hacerles comprender que su modelo de héroe ya no era tal, que estaba ya muerto antes de extinguirse su brillo sin remedio.

Pero optaron por no hacer algo así. Ya era duro perder una esperanza de normalidad, pero pensar que esa esperanza ya estaba perdida era más de lo que la gente necesitaba saber. Al menos así podrían pensar que algún día regresaría un inédito Alma Espejo, un nuevo rayo de luz en la penetrante oscuridad.

Además de un símbolo Los Caídos también habían perdido un aliado. Ni más ni menos que el hijo de su fundador, de hecho. Junto a él podían haber desentrañado muchas incógnitas referentes al pasado de su creador. Pero ya era tarde para eso, y como Scream bien sabía, no cabe sitio en el presente para rellenarlo con más tribulaciones de las necesarias sobre el pasado.

No habían sido días fáciles para el líder de Los Caídos, por otro lado. En Álex Miles había visto la oportunidad de enseñar a un pupilo y llevarle por el camino correcto. No porque no lo hubiera hecho antes con otros como Sam Grove o, a su manera, Warren Shockman. Pero ese caso era sin duda especial. Suponía para él la compensación por todo lo que había aprendido de su maestro, que aniquiló todo deseo de venganza en su interior para recordarle que dentro de él había cualidades mucho más importantes que el odio crudo y salvaje por la muerte de su amada y la pérdida de su vida y sus poderes.

Pero eso ya no podría ser, y no lo sería jamás. Había fracasado donde su mentor triunfó, y había sufrido el castigo cuando Perséfone llegó a la ciudad y se encontró a sí mismo enfrentado a los que siempre habían estado a su lado.

No hubo consecuencias por este último incidente. Todos habían sufrido lo bastante para comprender a largo plazo las acciones de Scream, sin duda el que más había perdido de todos desde que dejó de ser héroe, incapaz de conservar ni siquiera su trabajo civil o su hogar. Pero aquello no hizo sino afianzar la su creencia de que no había para él oportunidad de compaginar más vida que la que la organización le pudiera ofrecer. Cualquier otra cosa, cualquier otra actitud o aparición pública, sólo sería una mentira, una trampa a los ojos de los demás.

Por siempre en compañía de los suyos. Pero por siempre solo ante sí mismo y su futuro, pensó.

Fue por eso que en un principio, cuando supo que habría un acto público en homenaje a Alma Espejo, razonó que lo mejor sería no asistir al mismo. Limitarse a estar encerrado en Gorgon Enterprises con sus planos, o en el Aquerón con los esquemas semanales de actuación de los escuadrones. Trataron de convencerle Razorclaw, Saw, Swind y Swart, sin éxito. Ni siquiera Grove lo logró. Sólo una persona fue capaz de hacerle salir, a su manera, de ese letargo que se había impuesto.

~---Un ojo menos, una rata mutilada en el bolsillo, y empezarás a recordarme a alguien que conozco bien ~---dijo Shockman de manera ocasional, sin siquiera quedarse a esperar réplica por parte de su teórico superior.

Fue así como Scream decidió asistir al acto, que se intuía iba a ser completamente multitudinario, y en efecto así fue. Se celebró en la céntrica Plaza Wave, donde se mostraría un monumento que había sido elevado en honor al héroe perdido. El propio Presidente Scatter ordenaría que se retirara la cubierta que lo ocultaba, y sería una sorpresa para todos los presentes, pues había sido diseñado y erigido en el más absoluto de los secretos.

La plaza estaba tan llena que parecía que se hubiera convocado una multitudinaria manifestación. El tráfico terrestre había sido cortado y el aéreo restringido. El evento estaba siendo mostrado a través de todos los monitores de la ciudad, y los medios de prensa lo cubrían desde lo alto de edificios cercanos. En todos los edificios de los alrededores la gente miraba por las ventanas, incluso en los oficiales.

El ejército y la policía vigilaban también la zona, ante el temor a que pudiera haber un sangriento atentado. Los Caídos estaban tanto infiltrados entre el público como escondidos en tejados y puestos elevados, aunque su maniobrabilidad ante tantísima gente era más que limitada.

Scream estaba a nivel de calle, entre las primeras filas de la multitud. A su izquierda estaban James Sky y Emma Blades, y más al fondo Ellis Saw permanecía, como de costumbre, junto al Presidente Scatter, en primera línea del acontecimiento. Otros miembros como Razorclaw y Grove estaban también mezclados entre el público, mientras que los restantes o bien vigilaban las salidas, o bien coordinaban los puestos elevados. De más estaba decir que Shockman no estaba por allí y había salido a ajustar cuentas por libre con una de las pandillas de la zona opuesta de la ciudad.

Había mucha aflicción en los rostros de la gente, como Scream pudo comprobar. No era fácil vivir un día como ese, en el que se echaba por tierra todo un futuro que ya no tenía sentido considerar. Curiosamente fue Scream de repente el que se sintió ajeno a todo aquello, viviendo como había vivido tantas crisis anteriores, tantos descensos y ascensos personales.

Sky le miraba con atención. Scream le conocía lo bastante para saber que se estaba preguntando por el misterioso lugar al que estaban transportándole sus pensamientos.

~---Te veo distante, John. Cansado.

Scream sabía que también se refería a lo sucedido con Alma Espejo. El incidente ya había sido aclarado entre ambos, y no hacía falta más explicaciones al respecto.

~---Es como si todo saliera al revés de como habíamos planeado. Como si un ser superior nos indicara que de poco vale lo que hagamos, las cosas sucederán como tienen que suceder nos guste o no.

~---Vaya, qué filosófico se ha vuelto de repente nuestro indolente piloto ~---agregó Blades de repente, molesta por el síndrome de abstinencia al no poder fumar entre la multitud.

~---Creo que ya me deberías conocer lo suficiente para saber que soy menos despreocupado de lo que parezco ser ~---dijo Scream cortante.

~---Vaya, lo siento John, qué humor tenemos. Cualquiera diría que el mono lo tienes tú.

Sobredosis más bien, pensó Scream. Sobredosis de tinieblas, de oscuridad a su alrededor. Pero suponía que se le acabaría pasando, tarde o temprano.

Miró a su alrededor, a los rostros que apuntaban al suelo por todas partes. La muerte de sueños, de ilusiones por todas partes. Nadie lloraba, ni los niños siquiera. Era lo que tenía aquella ciudad, que endurecía el corazón de los que habían tenido la desgracia de nacer y vivir en ella.

Miró al cielo, a la eterna Nube que era testigo mudo de sus batallas. Siempre vigilante, siempre alerta. Sin que nadie fuera capaz de discutir su hegemonía. Amenazaba con ceniza, pero tal vez les diera un efímero momento de respiro.

Miró a los edificios aledaños, también, y su mirada se detuvo más de lo habitual en los ventanales del Tribunal Superior, donde alguien a quien conocía muy bien, aunque él no creyera conocerle a su vez, observaba el réquiem que se había instalado a los pies de su terreno de actuación. No era posible para Scream vislumbrar su rostro desde tanta distancia, pero se distinguían perfectamente los seis discos que siempre solían orbitar a su alrededor.

El Presidente dio un sentido discurso perfectamente elaborado por sus asesores y validado, seguramente, por el propio Saw. Unas palabras emotivas y políticamente bien escogidas que tratarían de levantar el ánimo de una ciudad cuya autoestima poblacional estaba en ese momento en mínimos históricos.

Al fin llegó el momento de descorrer la cubierta, pesada y rugosa para soportar las inclemencias del tiempo de Ernépolis, y mostrar el monumento que había llevado a tanta gente hasta allí, siendo algunos incluso de las zonas más empobrecidas de la urbe. Un sistema de cables permitió que de manera casi limpia se fuera desvelando poco a poco y sin interrupciones extra la escultura, cuyo tamaño era de casi cinco metros de altura.

Como casi todo el mundo esperaba la escultura no era una representación concreta de Alma Espejo, tal y como había sido en vida. Eso hubiera resultado demasiado doloroso de mostrar ante tantos miles de personas desoladas. Era una escultura abstracta, con forma de flecha que apuntaba al cielo, rodeada a su vez de otras flechas brillantes mucho menos largas, de apenas dos metros de altura y que apuntaban a los lados, haciendo al mismo tiempo de verja protectora de la principal. Toda la escultura era blanca y, como era de esperarse, refulgía con luz propia, ejerciendo también de foco de la plaza y coincidiendo con la reforma de las instalaciones de luz por toda la ciudad.

~---Este monumento ~---comenzó el Presidente Scatter~--- está aquí no sólo para honrar a Alma Espejo, también a todos los que han dado en el pasado su vida por defender esta ciudad. Héroes anónimos, policías, bomberos o sencillos ciudadanos que se vieron envueltos en tragedias inesperadas, es para honrarles a todos ellos y que no se pierda su memoria en el recuerdo.

»Está aquí también como símbolo de que somos una población unida, y que nadie podrá acabar con eso por mucho que se esfuerce en intentarlo. Y si este monumento fuera alguna vez atacado, o derribado, lo reconstruiremos, y si fuera destruido, construiremos otro mejor y más esplendoroso. Porque Ernépolis es una ciudad que ha renacido muchas veces de sus cenizas, un elemento que conocemos todos muy bien pues es parte esencial de nuestro día a día.

Renacer de las cenizas, pensó Scream. En efecto, lo habían hecho muchas veces, aunque en ocasiones no era tan fácil como el Presidente quería hacer creer. Solía implicar sacrificio, dolor extremo y la pérdida de valores que ya nunca se volverían a recuperar.

Nada perturbó la solemnidad de aquella ceremonia, que llegó incluso a ser retransmitida a las lejanas colonias. No hubo altercados, incidentes ni disturbios callejeros, como en un principio pudo temerse. Todo transcurrió con relativa normalidad. La gente regresó poco a poco a sus casas, la multitud se dispersó, volvió el movimiento de deslizadores, el monumento permaneció. Estaban pasando página a un capítulo escrito en un folio emponzoñado.

Tanto Scream como Sky comprobaron que sus respectivos operativos, clandestino uno y oficial el otro, transmitían las mismas buenas y simétricas noticias de que todo estaba en orden al término de la ceremonia. Después de eso, siendo ambos ya prescindibles para el resto del día, al menos hasta que alguien les requiriera de urgencia, lo que seguro no tardaría mucho en pasar, fueron juntos a tomar una copa en un bar que el mismo Sky sugirió. Blades tenía que irse a ultimar los detalles de un caso en el que estaba trabajando, por lo que podrían charlar tranquilamente sin ninguna clase de subterfugios ni dobles sentidos.

No se sentaron, sino que se limitaron a apoyarse en la barra, observando ambos sus copas tibias y con un par de cubitos tintineantes, como si al fondo de ellas se alojara alguna clase de respuesta. El bar lo había elegido Sky, otro de esos bares de polis que solía frecuentar a menudo, y en el que al menos sabía que tendrían trato preferente y podría tener vigilados a sus chicos, al menos a los que salían del servicio en esos momentos. Scream se sorprendió a sí mismo dándose cuenta de que no sabía en qué consistía la vida privada ni de la octava parte de los miembros de Los Caídos, aunque ellos sí conocían la suya: inexistente, para ser exactos.

Se miraron de repente durante un par de segundos, como los dos viejos amigos que eran, y que habían pasado por muchas buenas y malas situaciones en el pasado lejano y no tan lejano. Aun con todo, era más lo que les unía que lo que les dividía.

~---Debe haber sido duro también para ti ~---comenzó Sky, tratando de sacar adelante el tema tabú.

~---Sí, lo es ~---se limitó a decir, apurando su copa.

~---No hemos vuelto a saber nada de Hades, aunque ya estarás también al tanto de eso, supongo.

~---Sí. Sí, claro ~---Scream no apartaba la mirada del vaso.

~---John. Mírame, John.

Scream no se giró.

~---Tú no tienes la culpa, lo entiendes, ¿verdad? Hiciste lo que pudiste, no supo asumir la responsabilidad de su poder.

~---Yo hice lo mismo que él. Me dejé llevar por las emociones, con Aryn y con Perséfone.

~---Pero te has sobrepuesto a esas dos ocasiones. Esa es la diferencia que hay. Además, dudo mucho que hubieras actuado distinto de tener una segunda oportunidad. Si hubieras intentado atacar a Gorgon en vez de rendirte, Aryn hubiera muerto igualmente. En cuanto a Perséfone\dots\ sé que te tendría de nuevo corriendo por las calles y volviendo locos a mis chicos ~---dijo esbozando su clásica sonrisa irónica.

Scream se quedó callado de repente, mirando a un punto concreto del local, donde una mujer vestida de soldado estaba de pie, bebiendo en soledad.

~---Su cara\dots\ me suena su cara ~---mencionó en murmullos~---. ¿Quién es?

~---No se te escapa una, por lo que veo. Imagino que sigues estudiando cientos y cientos de fichas, como cuando Miles nos aleccionaba.

Scream siguió en silencio, tratando de recordar a cualquier precio.

~---¿Ves el bulto a su espalda? Si te digo que es una espada, ¿entonces sabrías identificarla?

~---¿Se trata de Ángela Mason? ¿La veterana de la Guerra de las Ocho Colonias?

~---La misma. Valerosa y temperamental, pero también fría y despiadada. Entre los soldados la llamaban Filo Omega, ya que ese emblema está grabado en el filo de su arma de combate.

Scream la miró con algo más de detenimiento. A juzgar por la funda que llevaba su espada era tan ancha como la palma de una mano, aunque supuso que tendría utilidades de tipo cibernético que la volverían más útil en esos tiempos que una simple arma blanca de corta distancia. Vestía como un soldado raso, aunque supuso que tenía rango más alto que uno, y del cuello llevaba un colgante de tipo electrónico con lo que parecía una \textsc{a} mayúscula lacrada en el mismo. Sus datos personales por si caía en combate, dedujo.

~---¿No era como la imaginabas?~---terminó Sky~---. De hecho, pensé que ya la conocías, que tal vez habíais coincidido en alguna ocasión cuando eras piloto.

~---No, nunca la conocí, aunque escuché hablar de ella. Ya era muy joven cuando entró en el ejército.

~---Joven y ambiciosa ~---apuntó Sky.

~---¿Por qué está aquí?

~---Aún no puedo decírtelo. Es confidencial.

Scream le miró, a punto de imitar una de las sonrisas burlonas de su amigo.

~---El ejército la ha destinado para que proteja ciertos proyectos civiles que se van a gestar. No sé cuáles son, pero se hará público en breve. Imagino que estarán en parte destinados a devolver la confianza en la ciudad. Tengo entendido que económicamente aún seguimos siendo débiles, aunque la industria aeroespacial de momento no nos falla.

~---Sí, eso espero desde un punto de vista personal.

~---Yo también.

~---¿Tú también? ~---preguntó Scream, intrigado.

~---Sí. Como algún día pierdas lo único que haces por placer, diseñar naves, entonces serás el líder clandestino más amargado e insoportable que habré conocido jamás ~---aclaró terminando su copa de un solo trago.

\endinput
