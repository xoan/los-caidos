Una vez más, la amargura de perder a un compañero. Alguien ajeno al dolor del pasado, a la carga de la derrota y el peso de la culpabilidad. Un momento duro, terrible. No sólo para ellos, también para las víctimas colaterales de los crímenes de sus enemigos.

\fancyparbreak
Pocos lugares en toda la historia de la humanidad resultan tan deprimentes como el cementerio de Ernépolis~I, situado, como no podía ser menos, en la zona sur de la ciudad. En una urbe en la que el valor que se le otorgaba a la ceniza era escaso, por no decir nulo, al estar en todas partes, y en el que hacían falta cuantas más plantas mejor para compartir los efectos de la polución de la Nube, no resultaba extraño que la mayor parte de los ciudadanos prefiriera el entierro a la incineración. Al menos, solían razonar, era una manera de pasar a formar parte de la ciudad que no involucraba, una vez más, aquella asquerosa sustancia que dificultaba el avance, se colaba por la ropa y el calzado y se volvía aún más desagradable cuando se mezclaba con la lluvia de agua sucia y fétida proveniente del contaminado firmamento.

Pero Ernépolis era una ciudad con un nivel bastante elevado de mortalidad, no sólo por la criminalidad reinante, también debido a toda clase de enfermedades derivadas de insuficiencias respiratorias y extrañas alergias. Era por ello que el cementerio estaba empezando a convertirse en toda una ciudad dentro de la ciudad, una necrópolis macabra a la que cada vez resultaba más difícil poner freno alguno.

Fue allí donde, a falta de un cuerpo, se instaló la tumba simbólica de Sam Grove, civil anónimo que había sido tomado como rehén y finalmente asesinado por el Caído. Estaba en la parte más septentrional del plano, en una zona relativamente despejada y con muchos nichos aún al descubierto. El cementerio era, de hecho, un pequeño laberinto que, como si fuera una especie de réplica de la ciudad en la que estaba asentado, poseía gran cantidad de callejones sin salida, debido a que errores de diseño motivaron que hubiera que empezar a recuperar espacio incluso a raíz de cortar sus avenidas practicables. A pesar de que algunos árboles especialmente resistentes y acostumbrados a las más adversas condiciones crecían por todo el lugar, como paredes naturales que separaban unas zonas de otras, estéticamente hablando estaban tan negros, retorcidos y pelados que no ayudaban mucho a suavizar la dureza del entorno. No eran como los árboles que uno pudiera encontrarse en un frío bosque tormentoso de invierno; aquellas plantas eran realmente un producto único del lugar, con nudos imposibles producidos por el esfuerzo del tronco para desarrollarse en la dirección donde pudiera obtener la máxima luminosidad, incluso aunque ésta proviniera de una farola. Eran auténticos supervivientes en el sentido más literal de la palabra, además de una muestra de hasta qué punto la naturaleza podía ser capaz de adaptarse a los más hostiles entornos imaginables.

El pavimento era en su mayor parte rocoso en las zonas transitables, aunque siempre el exceso de aguaceniza acababa por provocar la filtración superior de cantidades ingentes de tierra putrefacta que los cuidadores solían aplanar tan pronto como localizaban una nueva protuberancia. Las tumbas eran clásicas, labradas en mármol, tomando las más variadas configuraciones y disposiciones según la tendencia o moda del siglo en el que fueron levantadas, tales como disposición en cuadrícula, fila de uno, colmena o incluso distribuidas a lo largo de columnas hexagonales.

La lápida de Grove se erigía solitaria, cubierta en su mayor parte por tierra y con una losa en la parte superior, dispuesta en vertical y con una estética gótica. Sólo el nombre indicaba la identidad de su dueño; en Ernépolis no solían ponerse fechas de clase alguna. Sí había, sin embargo, una suerte de epitafio labrado un poco por debajo de la mitad de la losa:

\begin{verse}
    \begin{em}
        Beneath every church of dust\\
        Beneath every soundless wind\\
        Beneath every path to void\\
        I abandoned the corruption for your holy sight\\
    \end{em}
\end{verse}

Estrofa que no correspondía a religión alguna sino a la letra de la canción Black Lips de The Jammers, una de las favoritas de Grove.

La responsable de que se escribiera ese epitafio, tal como alguna vez Grove había comentado en broma que hiciera, estaba de pie frente a la tumba, quieta, tan inmóvil como muchas de las estatuas marmóreas que había a su alrededor, muchas descuidadas, olvidadas y torcidas por los corrimientos de tierras, como una señal más de la decadencia de aquel sitio de descanso eterno y, por extensión, de la ciudad al completo. Estaba cayendo aguaceniza en grandes cantidades pero ella no reaccionaba, como si hubiera perdido toda sensibilidad. Había tenido que ofrecer su mejor cara frente a amigos, como Roy, y familiares, como los padres de Grove, que viajaron desde Talópolis~VII para asistir al funeral. Ella no hacía más que escuchar la misma frase de boca de su padre: “le ha matado la ciudad. Esta ciudad es la muerte personificada”. En cierto modo, eso era lo que le había pasado.

Tras la exequias oficiales había estado allí, sola, casi todos los días, como en ese mismo momento. Todo había sido rápido, repentino, fugaz. Sam asesinado, el culpable impune, ni siquiera un cadáver que enterrar, aunque al menos supo que cuando fue atravesado estaba inconsciente y, por tanto, no sufrió en absoluto.

Aún no podía creérselo. Un día le llevaba a ver a sus músicos favoritos, y al día siguiente la única música que podía ofrecer era la del silencio imperturbable. Algo estaba mal, terriblemente mal en todo aquello. ¿Por qué a él? ¿Por qué a ambos? Se sintió horrorosamente sola y se arrodilló en las piedras puntiagudas, sin importarle las heridas, ya que llevaba unos pantalones negros no especialmente resistentes ni protectores. Se había pintado los labios y las uñas de azul, y se prometió a sí misma que así sería para siempre, para expresar la gelidez que atravesaba su alma. Que muy probablemente esa boca no encontraría otra boca, ni esos dedos rozarían otros dedos.

Pero no estaba sola. Alguien la observaba. Llevaba haciéndolo varios días, de hecho. Y finalmente, harto de comportarse como un intruso en vidas ajenas, dio un paso vago pero directo para mostrarse a los ojos de los demás. Al principio ella no se dio cuenta de su presencia, le confundió con una estatua más. Pero luego su mente la hizo discernir alucinación de realidad. Era él. No podía ser. Solía verle a menudo, en sus pesadillas. Pero no en las horas de vigilia, ante sus ojos despiertos y llorosos.

Bien pudo haber sido incapaz de reaccionar. Pero su observador habló primero, privándola del derecho a permanecer en silencio.

\emph{Ellie Wing} ~---fue todo lo que acertó a decir.

Ella, de repente, estalló. Se levantó y caminó hacia él, casi corriendo, furiosa, incapaz de controlarse.

~---¡Tú le mataste! ~---gritó, con los puños por delante, dispuesta a golpearle con todas sus fuerzas~---. ¡Tú mataste a quien hacía que me levantara por las mañanas!

El objeto de sus iras no la detuvo. Se dejó golpear una vez, y otra, golpes sordos en su tórax que le hubieran hecho bastante daño de no ser alguien más que acostumbrado al dolor, que había hecho de él su mejor y único amigo sincero.

Finalmente la agarró de las muñecas, sin apenas fuerza, como si no deseara hacerlo, como si tuviera la esperanza de que cada golpe pudiera hundirle en la tierra húmeda y plastosa, y así desvanecerse de la vista de todo y todos. Ellie siguió forcejeando, tratando de zafarse, las lágrimas deslizándose a los lados, como la inocencia que se pierde en el camino para no recuperarla jamás.

Terminó por cansarse y dejó de luchar. Justo en ese momento fue cuando la soltaron y cayó al suelo de nuevo. Te he fallado Sam, se dijo a sí misma. Los demonios siguen vivos sobre la superficie de la Tierra, y tú ya no estás entre nosotros.

\emph{Levanta} ~---dijo la sombra, casi como una orden.

Ellie se incorporó pero no se alejó de aquel que le hablaba. Le temía, pero era mayor el desprecio que le producía su mera presencia.

\emph{Eres valiente, teniendo en cuenta lo sucedido. Déjame hablar contigo.}

~---Yo no hablo con monstruos psicópatas y asesinos.

\emph{Puede que yo sea un monstruo, pero yo no maté al chico.}

~---No te creo.

La sombra se quitó el sombrero y, acto seguido, apagó todos los dispositivos que le conferían un aspecto amenazador. Poco a poco dejó de ser una leyenda urbana para convertirse sólo en un hombre torturado y con un pasado tenebroso que la miraba con su único ojo sano.

~---¿Haría esto el asesino de tu novio? ~---dijo Shockman con su tono de voz habitual.

~---Sigo sin creerte. No veo por qué no estés tratando de torturarme o engañarme.

~---Escúchame, mocosa. Nunca, en toda mi vida, me he humillado hasta semejantes niveles. A mí nadie me importa una mierda, ¿entiendes? ¡Nadie!

De repente, al escuchar esas duras palabras, Ellie empezó a tener una extraña sensación. Como si empezara en verdad a creer al portador de las mismas, porque no eran hermosas, ni trataban de adornar la realidad u ocultarla con mentiras piadosas.

~---Entonces… si nadie te importa, ¿por qué estás aquí? ¿Para torturarme?

~---Yo no me regodeo en el sufrimiento de los débiles. Es ruin, mezquino y propio de guerreros indignos de considerarse tales.

Ellie miró a su alrededor, al tenebroso cementerio escenario de aquella extraña conversación. Cientos de lugares donde ocultarse, donde pasar desapercibido.

~---Has venido aquí más veces ~---concluyó al fin.

~---Algunas ~---aclaró Shockman. En realidad, todos los días desde el funeral.

~---¿Quién eres tú? ¿Quién mató a Sam?

~---Me temo que eso aún no puedo decírtelo. No estás preparada para escucharlo.

~---¡Dímelo! ~---gritó Ellie.

~---Pronto ~---prometió Shockman~---. Ahora, vive tu pena. Recuérdale ahora que puedes. No quieras cambiarla demasiado pronto por rabia hacia el mundo que te rodea, porque una vez efectuado ese paso la marcha atrás es una maniobra peligrosa y arriesgada.

Después de eso Shockman se colocó el sombrero de nuevo, ajustó los dispositivos y se marchó sin realizar ninguna salida espectacular, sólo desapareciendo por las calles del cementerio. El entorno ya abarcaba suficiente protagonismo por sí solo como para añadir una dimensión tenebrosa y siniestra a su alejar lento y silencioso.

\parbreak
Cuando Shockman regresó al cuartel general lo primero que verificó fue si había nuevas noticias de John Scream, pero, al igual que antes de marcharse, seguía medio recluido en su despacho sin recibir a nadie ni tampoco dirigir la palabra a ninguno de los suyos. Sólo se limitaba a planificar el esquema de actuación de cada día, en el que él mismo no estaba incluido, hacerlo público subiéndolo a la pantalla del hemiciclo y proseguir con múltiples tareas tanto relacionadas con la organización como con el diseño de planos para Gorgon Enterprises. Razorclaw y Saw nunca le habían visto así antes, ni siquiera poco después de que dejara las calles, con el fantasma de la mendicidad y la derrota aún a sus espaldas. Los demás no hacía tanto que le conocían, pero estaban preocupados de igual manera por su hermética condición.

Shockman era el único al que Scream no le producía ninguna compasión en absoluto. En su opinión se estaba comportando como un cobarde y un débil, y los muertos no iban a resucitar por mucho que actuara como uno más de ellos, invisible, silencioso e imperturbable a los ojos de los demás. Por eso no pudo ocultar su expresión de desagrado cuando Swart le preguntó dónde había estado, con plena desconfianza.

~---Eso no es de tu incumbencia ~---contestó dejando que la rata saliera de su bolsillo para subirse a una silla cercana, en una de tantas mesas llenas de artefactos recién utilizados.

~---Sí lo es si llevas el traje puesto cuando estás fuera ~---se limitó a decir Swart cruzándose de brazos e impidiéndole el paso.

~---¿Quién se ha muerto y te ha nombrado el jefe ahora? ~---contestó Shockman con muy poca delicadeza.

Swart se encaró y estuvo listo para lanzarle un puñetazo a aquel, en su opinión, estúpido insolente, pero Matt Swind se interpuso entre los dos y trató de hacer de intermediario.

~---Vamos Jim, déjale. Está claro que él también está muy afectado por la muerte de Sam…

En ese momento, para sorpresa de Swart y los demás presentes, fue Swind el que recibió un sonoro puñetazo por parte de Shockman. Swart se quedó tan atónito que no fue capaz de cerrarle el paso de manera consciente ni un segundo más.

~---Yo no me siento afectado, ¿te enteras? ~---replicó con furia, casi escupiendo las palabras~---. Ese chaval no significaba nada para mí. Nada.

Se alejó a pasos largos a través del pasillo, sin que nadie hiciera ademán de ponerse en su camino. Algunos aún podían escucharle murmurar a lo largo de su prolongado avance solitario.

~---Nada ~---se repetía más para sí mismo que para los que pudieran escucharle en ese momento.

\parbreak
Al mismo tiempo, en otro lugar del Aquerón, alguien ajeno a la infraestructura de Los Caídos pero que fue parte de ella en otro tiempo hacía acto de aparición usando una de las entradas más cercanas a su propio cuartel general y regresando al que tiempo atrás había sido también un hogar para él. Como John Scream, Charles Razorclaw, Ellis Saw y los otros, James Sky también fue un héroe en el pasado. También creyó perderlo todo cuando se dio cuenta de dónde estaban los límites de su potencial, pero experimentó un renacer gracias a Starr Miles, aquel hombre que les juntó a todos para planificar la revancha contra un mundo y una ciudad que les habían dado la espalda. Sky decidió, llegado el momento, que podía resultar más útil como Jefe de Policía de Ernépolis que como uno más de los miembros de Los Caídos.

Sin embargo, por lo que había podido ver con sus propios ojos, Scream estaba creando una coraza a su alrededor. Apenas hablaba con nadie, ni siquiera con él mismo. Un par de días atrás él y Blades trataron de sacarle a dar una vuelta por la ciudad, de que se olvidara de todo aquello, que se diera la oportunidad de perdonarse a sí mismo aunque fuera sólo por un momento. Pero fue inútil. Se negó en redondo, limitándose a poner como excusa que en su tiempo al margen de la organización tenía demasiados diseños que supervisar, y tomó el pasillo que conducía de manera directa con Gorgon Enterprises, sin ninguna prisa por llegar a su destino, evidenciando que lo único que buscaba era regodearse en su propia soledad.

Sin embargo Sky no estaba allí sólo para intentar levantar el ánimo a su amigo. De hecho, eso era lo que más le preocupaba al respecto de su visita. Tenía novedades que contarle, noticias que podían hacerle salir de su letargo. Pero al mismo tiempo no quería comunicárselas, pues sólo serían un parche a su condición actual, algo con lo que olvidar lo sucedido para que resurgiera en su cabeza tiempo después. Era muy consciente de todo ello, sí… pero también de que no tenía elección real. Se trataba de la ciudad que ambos, desde distintos lados de la ley, habían jurado proteger, y eso estaba por encima de todos sus problemas y necesidades personales.

Llamó a la puerta del despacho, pero nadie contestó. Abrió la puerta con calma y se encontró a Scream encorvado sobre marañas y marañas de documentos relacionados con la organización, en su mayor parte informes rutinarios de incursiones de escuadrones por toda la ciudad. Se sentó frente a él, pero Scream no le saludó ni hizo ademán alguno de levantar la cabeza. Sky se dio cuenta de que era como si se impidiera a sí mismo sentirse triste o mostrar signo alguno de debilidad.

~---John ~---dijo con calma, rompiendo el incómodo silencio.

~---Hola, James ~---dijo sin dejar lo que estaba haciendo~---. Me coges ocupado, como ves.

~---He estado pensando, John. Se supone que soy tu mejor amigo, pero nunca he podido entender de verdad el dolor de tus pérdidas. Nunca conocí a Aryn Life, y tampoco tuve apenas trato con Sam Grove, pues le reclutaste cuando yo ya me había marchado.

~---No es necesario que hagas esto, de verdad ~---se limitó a decir Scream, sin cambiar la posición.

~---No, pero quiero hacerlo. Nunca he podido ser un apoyo para ti en ningún sentido. ¿Qué clase de amigo soy yo, entonces? Es patético por mi parte, y no merezco tal apelativo.

~---Sabes que eso no es cierto.

~---Exactamente, John. Lo mismo ocurre con las cosas que te dices a ti mismo en tu cabeza.

Scream se detuvo y poco a poco alzó el rostro. Pero qué bien le conocía el muy cabrón, pensó. Aun así, le dejó continuar.

~---Tú no tienes la culpa de la muerte de Sam Grove. Ya te lo dije en su momento. Hiciste todo lo que pudiste, te exprimiste más allá de los límites de lo concebible. A veces los criminales ganan, y lo único que podemos hacer es asumirlo y luchar de cara al día siguiente.

~---Yo no he dejado de luchar, James.

~---Luchar no sólo implicar tratar de vencer a nuestros enemigos. También se trata de ganar al desaliento que crece en nuestro propio interior. Y eso sí que lo he experimentado, como Razorclaw, Saw y los demás. Todos dejamos de ser héroes en el pasado. Cada uno por distintos motivos, pero conocimos la amargura de la derrota. Por eso estamos más preparados para afrontar esta clase de pérdidas. Pero eso no quiere decir que tengas que construir una armadura que la aflicción no pueda atravesar.

~---No puedo permitirme el lujo de abatirme. Tengo que dirigir la organización.

~---Pero es que tienes que dar rienda suelta a tu tristeza. Aceptarla, y después de eso mirar hacia delante. Has sufrido muchas decepciones, lo sé. Todos esos enemigos, sus constantes intentos por acabar con todo aquello que defendemos. El fracaso que supuso no poder hacer de Alma Espejo un héroe digno de esta ciudad. La desgarradora sensación que debe producir ver cómo a alguien joven, en quien habías depositado esperanzas de futuro, le es segada la vida justo en el mejor momento de la misma.

~---Le vi morir, James. Delante de mis ojos. Le mató porque yo estaba allí, para que lo viera.

~---Tú no le mataste, John. Fue ese ángel caído, ese demonio que logró engañarnos a todos, que frustró todas nuestras esperanzas. Deposita tu odio sobre él si quieres. Pero deja de culparte a ti mismo.

~---Te agradezco tus palabras. También sé que no has venido sólo por ellas, o de lo contrario no habrías bajado aquí a contármelo. Me hubieras dicho que subiera a la superficie, que respirara el aire de la ciudad.

~---Un aire muy sano, ya lo sabes ~---añadió Sky con una de sus clásicas sonrisas.

~---Necesitas al John Scream clandestino, entonces. ¿Qué es lo que ocurre?

~---No te va a gustar saberlo ~---contestó cerrando del todo la puerta del despacho.

Al cabo de un rato Sky salió de allí, calmado, caminando al tiempo que estaba absorto en sus pensamientos. Razorclaw fue el primero que se le acercó para hablar.

~---¿Qué te ha dicho? Apenas habla con ninguno de nosotros.

~---No es el mismo de siempre, pero espero que no tarde en recuperarse. Por desgracia, un nuevo problema se ha presentado, y por una vez han sido mis hombres los que han notado antes que algo anda mal. Pero ahora debo dejarte, me reclaman en comisaría. Saluda a Ellis de mi parte, y a los demás también.

Mientras Sky se marchaba por uno de los pasillos, se cruzó con Shockman un momento. Un instante fugaz en el que se quedaron mirándose el uno al otro, en el que por una vez, ambos hombres pensaron que, irónicamente, la vida a menudo daba muchas más vueltas de las que uno pudiera imaginar.

Shockman siguió caminando, sombrío, absorto en sus pensamientos, buscando una esquina lo más oscura posible en lo que parapetarse. Una vez encontró el lugar perfecto, y su rata subió por la gabardina hasta posarse en su mano, comprobó cómo, por fin, el ausente despertaba de su largo letargo.

~---Reunión urgente en el hemiciclo ~---dijo Scream, otra vez lleno de energías~---. Un nuevo peligro amenaza nuestra ciudad.
