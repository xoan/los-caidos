Crisis. Descontento. Población encrespada y enfadada por una situación cada vez más precaria y complicada. Era un trasfondo difícil para ellos, en el que incluso no sabían muy bien cómo debían comportarse.

Pero en realidad se trataba del caldo de cultivo de un problema que llevaba mucho tiempo gestándose. Un problema mayor, que sí les concernía en todos sus aspectos y consecuencias.

\fancyparbreak
La posguerra no fue una época fácil de digerir para Ernépolis~I. Paradójicamente, el fin de las batallas trajo consigo un declive económico considerable, al frenar la actividad de astillero temporal que tenía la ciudad. Gran parte del bloque comercial fue también levantado, pero en numerosos casos ya no había economía con la que comerciar, pues o bien la guerra la había devastado, o bien los tratados de paz les habían impuesto férreas normas de exportación e importación controlada.

Una guerra colonial había acabado en Ernépolis, pero al mismo tiempo otra amenazaba con comenzar.

Una guerra civil, más concretamente.

Motivada por falta de recursos, desamparo, necesidad de supervivencia en todos los aspectos. La ciudad rugía como un animal furioso y sus células eran los ciudadanos. Ciudadanos que empezaban a estar cada vez más hartos de la corrupción que aún existía, y que se llevaba el dinero como el agua que escapa de un grifo goteante. Hartos de la incapacidad del gobierno, inútil para frenar la avalancha que se venía encima. Hartos del crimen, de las bandas, de la escoria que pisaba de vez en cuando las sombrías calles cubiertas de ceniza.

Ciudadanos que, en muchos casos, habían empezado a olvidar que hubo un tiempo en que las cosas habían sido mucho, mucho peores.

Pero aun así el descontento avanzaba imparable. Y como todo, había quien pensaba en sacar provecho de ello. Poder, liderazgo, dinero, siempre había un buen motivo para mover masas en favor de intereses propios.

John Scream lo sabía muy bien, y James Sky también. Por eso ambos miraban desde la ventana de la casa de Sky, no muy alejada del centro de la ciudad. Scream tenía un whisky seco en las manos, mientras que Sky prefería mirar con las manos desnudas, escondidas y cerradas en los bolsillos. Cada uno tenía su manera única y personal de mantenerlas ocupadas.

El apartamento de Sky estaba en un edificio no especialmente lujoso, pero tampoco mugriento como muchos de los de la periferia. Situado a la altura de un cuarto piso, desde él podía verse a la perfección la calle y las personas que la infestaban cada vez con mayor profusión.

Ambos sabían que en breve no se podría vislumbrar nada desde ese bastión improvisado, y entonces seguramente empezarían los verdaderos problemas para ambos.

~---¿Cómo le está yendo a la policía? ~---preguntó Scream rompiendo el silencio, consciente de que era mejor sacar el tema cuanto antes, para evitar peligrosas divagaciones interiores.

~---No damos abasto. Hay disturbios por toda la ciudad, y donde no los hay se debe a que los ciudadanos no se atreven a manifestarse. Los robos se han triplicado desde la última semana, y también es un momento propicio para los ajustes de cuentas y las muertes en extrañas circunstancias. No hay mejor coartada que una multitud desbocada.

~---Son como un ejército de civiles en peregrinación hacia el enemigo ~---apuntó Scream.

~---Son más difíciles de manejar, de hecho. Ahí abajo, John, la desesperación campa a sus anchas, y muchos están empezando a hacerse eco de ella con sus actos. Hemos tenido que parar numerosos saqueos ocasionales, así como intentos de venganza social contra cadenas y franquicias varias.

~---¿Qué hay de tus efectivos?

~---Insuficientes. Los agentes están doblando turnos. La mayoría se niega también a tomar descansos reglamentarios. Aprecio su entrega y espíritu decidido, pero como inevitable resultado el cansancio está haciendo mella en ellos.

~---¿Y en ti? ¿Cuánto tiempo llevas sin dormir?

~---Creo que no te ganaría si decidiéramos competir en este momento ~---contestó Sky esbozando, por primera vez en semanas, una de sus clásicas muecas irónicas.

Scream dejó pasar un intervalo de silencio. De reflexión apagada, sólo justificada por la necesidad del momento, de sentir, aunque fuera brevemente, unos pocos segundos de paz al margen de complicadas decisiones.

~---Hacemos lo que podemos por ayudarte, ya lo imaginarás. Todos en el Aquerón están entregados en cuerpo y alma a minimizar el impacto del caos social que estamos viviendo. Hay gente infiltrada en casi toda manifestación, en todo motín, intentando parar la barbarie antes de que comience.

~---Lo sé y te lo agradezco. Imagino que no estará siendo fácil para vosotros, tampoco.

~---No, no lo está siendo. Muchos de Los Caídos dudan de su capacidad para afrontar esta situación. Quiero decir, ellos fueron héroes, se metieron en esto porque querían proteger a los ciudadanos de los criminales. Pero ahora están protegiendo a los ciudadanos de sí mismos. Y eso sin mencionar que muchos de ellos sienten un descontento similar a la población con respecto a la manera en que está evolucionando el panorama económico en Ernépolis~I.

~---Yo también lo comparto. Las huelgas que el sindicato de policía empezó para solicitar más efectivos fueron en parte comenzadas por comentarios críticos míos para con la gestión del Presidente Scatter. ¿Qué hay de ti, John?

~---Creo que esta situación pasará. La gente saldrá adelante. Esta ciudad ha vivido situaciones terribles, y esta crisis social no es la peor de ellas.

~---¿Entonces qué es lo que te preocupa tan mortalmente?

Scream se detuvo un momento a suspirar en voz alta. Apuró el vaso y continuó.

~---El Descenso de la Nube.

Sky le miró intrigado, y comprendió que la visita de Scream, a pesar de llevar tanto tiempo sin verle, no respondía únicamente a motivos de cortesía y amistad.

~---Creo que te sientes como el portador de malas noticias.

~---Aún no, James, aún no. Pero temo que no tardarán en llegar. Todo por culpa de esa cosa que flota sobre nuestras cabezas.

Ambos miraron a la Nube a través de la ventana, espesa como si pudieran palparla sin más que estirar la mano hacia ella. Las primeras advertencias fueron realizadas por expertos climáticos en los últimos meses de la Guerra de las Ocho Colonias. Las continuadas explosiones más allá de las capas altas atmosféricas podían producir serias perturbaciones, tal vez alterar temporalmente el flujo de movimiento de la Nube. Scream estudió los datos bajo su perspectiva de diseñador de naves, y desde el punto de vista de la aeronáutica los informes recogidos eran clarificadores, aunque no mucha gente hiciera caso de ellos al principio. La Nube sufriría un desplazamiento momentáneo.

Un rayo de esperanza le hizo pensar que tal vez se apartaría de la bóveda celeste, que concedería a Ernépolis~I la capacidad de disfrutar de la luz solar. Pero tras efectuar cálculos y predicciones utilizando mecánica de fluidos a bajas viscosidades, llegó a la peor conclusión posible: la Nube descendería hasta estar más o menos a nivel de calle.

Predicciones mejores y más precisas que la suya no tardaron en estimar que tal fenómeno, de una rareza singular, duraría cinco días en los que la ciudad se vería inundada de una espesa capa de niebla, hasta que afortunadamente la diferencia de densidad devolvería de nuevo a la Nube a su legítimo lugar en el firmamento, haciendo que los que estuvieran en la ciudad en aquel momento vivieran una experiencia única que contar a generaciones futuras, como si hubieran sido testigos de un eclipse o la explosión lejana de una supernova.

Y aquel día se iniciaba la cuenta atrás. Ya empezaba a notarse cómo la Nube estaba cada vez más baja, cubriendo los pisos superiores de las torres de la ciudad, hundiéndolas, más que nunca, en un mundo de tenebrosa espesura.

~---Tal vez esa Nube sea un respiro, John. Al acabar el día se cerrarán todos los accesos a la ciudad, y muchas personas ya se están marchando, sobre todo a Talópolis~VII y Ernépolis~II. Habrá muchos que no querrán estar a la intemperie, del mismo modo que ocurre cuando se convocan manifestaciones en días que cae aguaceniza. Tal vez el clima sea en este instante nuestro mejor aliado.

~---Ojalá fuera tan optimista como tú, James. Porque donde tú ves una tregua temporal yo veo la ocasión perfecta para toda clase de problemas que no podemos ni imaginar. No sólo eso, el Descenso ha sido un fenómeno anunciado y esperado.

~---¿Hay algo concreto que te preocupe, John?

~---La incertidumbre. El no saber qué es lo que puede pasar en estos cinco días. Pero sin duda, los que nos quedamos no los olvidaremos. Cinco días de una niebla que no nos dejará ver ni la otra acera de la calle, aislados del exterior. Atrapados en esta ciudad en la que empiezo a creer que puede pasar cualquier cosa.

~---Reforzaré la presencia policial, eso no lo dudes. Pero espero, sinceramente, que te equivoques.

~---Yo también lo espero, James. De todo corazón.

\parbreak
Cuando Scream salió de la casa de Sky buscó el pasadizo más cercano hacia el Aquerón y se introdujo en cuanto tuvo claro que nadie estaba espiándole. Tampoco es que ocurriera nada grave por ello, pues los pasillos que llevaban al centro del cuartel estaban astutamente diseñados y mejor vigilados, pero todo esfuerzo por no levantar sospechas era sin duda recomendable.

Era obvio que en alguna ocasión alguien había estado en el lugar inadecuado en el momento inapropiado. Que los rumores, las leyendas urbanas, empezaban a aflorar. Por fortuna la desinformación era tan amplia y numerosa que era imposible separar el trigo de la paja y se empezó a decir que aquella criatura de sombras nacida de la Nube vivía en las profundidades de la ciudad, donde residían sus dominios. Algo que, en cierto modo, era absolutamente cierto.

Más preocupante era para Scream el hecho de que gran parte de la población estaba empezando a considerar los actos del Caído como puramente heroicos. Ese no era el objetivo, la misión. Tenían que provocar miedo, incluso a aquellos a los que deseaban proteger. Empezaba a temer que surgieran imitadores, emuladores de sus actos. Peor aún, que atrajeran nuevos enemigos y amenazas a una ciudad que ya estaba demasiado saturada de criminales de poca monta.

Se dio cuenta de que la salida que acababa de emplear sería, seguramente, la misma que Sky utilizó muchas veces antes, cuando aún era un miembro de la organización, para dirigirse al Aquerón. Su ausencia se notaba, era palpable, sin duda. Aunque bien cierto era también que la policía estaba empezando a manejar muchas situaciones de las que antes no lograba encargarse, lo que suponía un alivio de carga para él y los suyos, más preparados para las situaciones donde serían indudablemente necesarios.

Situaciones que estaban a punto de llegar. No porque lo temía, sino porque lo sabía.

No quiso decir nada a Sky, pero no tenía ninguna sospecha, sino una terrible certeza, de que algo iba a ocurrir. Bastantes problemas tenían ya sus hombres con la población civil como para además preocuparle con datos que, aun conociéndolos, escaparían al alcance de su competencia. No, tendrían que ser ellos los que sacaran la basura en aquella ocasión.

A medida que se aproximaba al módulo central de la base pensó en otro de los rumores que se venían escuchando, concretamente uno que decía que, con el Descenso, aumentaría el dominio e influencia del Caído, que gustaba de proclamar a los cuatro vientos que la ciudad era de su propiedad. Se preguntó qué pensarían los que creían en tales patrañas si supieran que quienes fingían ser tal criatura ficticia sentían en el fondo la misma incertidumbre que aquellos a los que pretendían atemorizar.

Cuando Scream llegó al módulo principal notó que la actividad era, como siempre, frenética. Se estaban siguiendo los pasos de todos los efectivos que se encontraban en la calle, y se procesaban los informes que llegaban del exterior para decidir el curso de acción más adecuado a cada uno de manera personalizada.

Era, en realidad, una labor a la que no estaban acostumbrados, pues Los Caídos patrullaban la ciudad en una pauta que, salvo por amenazas puntuales en sitios concretos, tenía mucho de ronda estandarizada, de vigilancia medida, calculada y calibrada. Pero tener que controlar toda una ciudad y sospechar de cada individuo, de cada rostro huraño y taciturno, era una tarea que les quedaba, sin duda, grande en todos los sentidos.

Razorclaw y Saw coordinaban los grupos y las comunicaciones que provenían del exterior. Los que habían sido compañeros de escuadrón de Scream junto al propio Sky y el fallecido Raid habían ascendido en la escala de mando, y aunque se habían convertido en directores de sus respectivos escuadrones clave, en ese momento lo mejor que podían hacer era comandar a aquellos que ya estaban en el exterior, tratando de detener los focos de violencia tan tenaz pero inútilmente como el niño que trata de cavar una agujero en la costa antes de que el agua lo inunde por completo.

~---Lamento haber tardado en llegar ~---fue lo primero que proclamó Scream.

~---No tienes nada que disculpar, en toda esta sala eres sin duda quien más horas de sueño reclama ~---argumentó Razorclaw aprovechando un momento de calma en el que no recibía nuevas transmisiones.

~---¿Cómo están las cosas? ~---inquirió Scream, preocupado. Se pasó la mano por la frente, como si el cansancio fuera igual que una capa de sudor que pudiera desprenderse con un sencillo gesto de mano.

~---Varias manifestaciones cerca del distrito financiero han estallado en revuelta, pero los alborotadores están empezando a dispersarse.

~---¿Algún equipo está usando la imagen del Caído en este momento?

~---No, sólo están entre los manifestantes. La aglomeración de gente hace muy difícil salir con el equipo completo.

~---¿Qué hay de nuestros huéspedes?

~---Nada. No han soltado prenda.

~---Si no me equivoco, cuando se les encontró, lo hicieron hombres de paisano, ¿no es así?

~---Así es, en todas las operaciones.

~---Voy a salir hacia el foco de los disturbios. No estaré mucho fuera.

~---John\dots\ ~---inquirió Razorclaw de repente, dejando pendiente de contestar una llamada~---, descansa. Lo necesitas. Los demás tenemos vidas alternas que nos hacen olvidar esta carga aunque sea temporalmente, pero en tu caso la dedicación es completa.

~---No te preocupes por mí, Charles. Cuando era Capitán pasé también por momentos en los que no pude permitirme ni un solo segundo de descanso. Estaré bien dentro de cinco días, cuando todo esto haya acabado. Hasta entonces me apañaré con sueños ocasionales de media hora, cosas así.

~---De acuerdo, como creas conveniente. ¿Quieres salir con un grupo? Ahora mismo sólo permanecen en reserva los más novatos, pero sabiendo que están en tus manos les enviaría hasta los confines del Universo.

~---No será necesario, mantenles listos por si surgiera alguna clase de imprevisto.

~---¿Y qué es lo que harás tú?

~---Voy a atrapar gamusinos ~---dijo Scream de camino a la sala de los trajes.

\parbreak
Nada más regresar a los angostos y oscuros pasillos del Aquerón, un laberinto de laberintos, sutil y al mismo tiempo ramificado con fractal eficacia, Scream tomó la ruta que le llevaría lo más cerca posible de la zona de la ciudad que quería visitar pero sin desembocar directamente en ella. Prefería flanquearla desde el exterior, rodearla, analizar con calma la geografía y el ecosistema humano que en aquel momento se desarrollaba en su interior.

Iba de caza, y como un cazador tenía que comportarse.

Nada más salir al exterior notó que la Nube había bajado casi por completo, dando así comienzo de manera oficial a los cinco días que duraría el Descenso. Hasta ese momento Los Caídos no habían podido patrullar apenas por las decadentes calles de la ciudad, debido a que la muchedumbre se agolpaba en casi cualquier avenida amplia que mereciera la pena ser mencionada, y de ahí se ramificaban a accesos menores, incluyendo callejuelas.

Pero la niebla ya estaba a la altura de los hombres, y con ella, nacía una cobertura perfecta. Los Caídos volvían a dominar la ciudad.

El problema ahora estaba en averiguar hasta qué punto la ciudad podría dominarles a ellos.

Se aproximó de tejado en tejado hasta llegar a una zona donde los manifestantes parecían estar dispersándose. Ya no era un líder, ni siquiera era un hombre. Se había convertido en una sombra que se deslizaba furtiva, viscosa, como si estuviera pegada a las paredes.

Analizó comportamientos, buscó patrones de conducta, gestos. Señales hostiles. Descartó a los que iban en parejas y tríos, luego a los grupos de mayor tamaño. Tal vez alguno de ellos pudo ser un instigador, un alborotador. La presión de los compañeros y la imposición de las ideas de la masa podía convertir a una persona más apagada que una cerilla consumida en dinamita a punto de estallar. Pero no buscaba esa clase de amenaza, sino otra más oculta, más desapercibida. De naturaleza similar a la suya.

Finalmente localizó a uno de ellos. Iba solo, manos en los bolsillos. Andaba encorvado pero con pasos firmes. Mirada entornada, rigidez muscular. Un lobo rodeado de ingenuos corderos.

Le siguió hasta que estuvo alejado del resto de la muchedumbre, dejando atrás los deslizadores quemados y el ruido de las alarmas de tiendas. Cuando llegó a una calle estrecha la niebla le tragó. Podría haber supuesto un problema para un sabueso poco experimentado, pero no para alguien que poseía, no sólo un equipo para detectar presencia humana, sino el talento necesario para explotar al máximo tales habilidades.

La sombra reptó por las paredes, acercándose más a su inconsciente presa.

Cuando estuvo a la altura suficiente, comenzó el baile.

\emph{Eres nuevo en mi ciudad} ~---dijo apuntándole con el arma y enfatizando el \emph{mi}, con un toque de evidente y sincero desprecio.

El perseguido arqueó los hombros y sacó la mano de los bolsillos. Scream supuso que mostraría un arma, alguna clase de pistola, pero en vez de eso se limitó a quedarse quieto.

Estaba bien entrenado, pero tenía miedo. Después de tanto tiempo aprendiendo a provocarlo, era casi como si pudiera olerlo.

\emph{No saldrás vivo de aquí, soldadito} ~---prosiguió con el teatro.

~---Él dice que no nos matarás.

\emph{¿Quién es él?} ~---la sombra empezó a hacerse más grande a ojos vista, hasta alcanzar proporciones inhumanas. El sujeto se arrodilló y se echó a temblar.

~---No puedo decirlo. No puedo.

\emph{¿Es que acaso no me temes?} ~---fue la respuesta de Scream, saliendo de las sombras para colocarse frente a su pusilánime presa.

Sacando fuerzas de flaqueza, el sicario le miró a sus ojos negros, de pupila infinita. Era la primera vez en muchísimo tiempo que un matón del tres al cuarto tenía el valor de soportar su mirada.

El motivo estaba en la elocuente respuesta que estaba a punto de decir.

~---Te temo, pero le temo más a él.

Scream estaba furioso, más que decepcionado. Sea quien fuera el artífice del entrenamiento de ese tipo, lo había doblegado usando el miedo. La misma arma que era en teoría terreno exclusivo de ellos.

\emph{Me temerás} ~---dijo con solemnidad, y dejó pasar un periodo de silencio de varios segundos.

Después de eso, le golpeó.

Un golpe innecesario a todas luces, pues para llevar a aquella tembladera viviente al Aquerón no hacía falta ni dejarle inconsciente, dado que no abriría los ojos en todo el trayecto. Pero Scream estaba furioso por sentirse impotente, con las manos atadas, y tenía que desahogarse de alguna manera.

Cuando regresó al cuartel con su prisionero, Razorclaw corrió a recibirle.

~---¿Capturaste a otro?

~---Así es. Esta vez con el teatro completo.

~---¿Habló? ¿Dijo algo?

~---Nada. No sé a quién nos enfrentamos, pero está infiltrando hombres en toda la ciudad, seguramente con funciones diferenciadas, desde meros observadores, como debía ser el caso de éste, a señuelos o instigadores de las masas. Están tratando de minarnos, de distraernos y dispersarnos.

~---¿Qué hacemos con él?

~---Yo mismo lo llevaré con los demás, descuida. En cuanto le haya dejado, iré a relevaros.

~---No es necesario ~---comentó Saw desde su puesto.

~---Insisto ~---zanjó Scream de manera tajante. Llevó al sujeto a los calabozos y le metió en una de las celdas. No podían ponerlo en manos de la policía, pues en realidad no había cometido delito alguno, pero sabía que tampoco podía dejarlo suelto para que informara a su amo y señor.

Scream se sintió como si estuviera tratando de agarrar humo con las manos. Una metáfora muy adecuada, teniendo en cuenta la situación del exterior.

Se alejó de las celdas sin reparar más en ellas.

A sus espaldas, decenas de infiltrados como el que acababa de capturar observaban, aterrados, a la silueta de la gabardina y el sombrero mientras se marchaba, deslizándose sobre el suelo mugriento como si no necesitara de piernas para avanzar.

\endinput
