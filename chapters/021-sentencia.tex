Debían resistir ante las adversidades, por muy numerosas que éstas fueran. Fugitivos, escondidos, sin apenas aliados ni defensores. Obligados a fingir ser aquello que más odiaban, a parecer monstruos a los ojos de los hombres.

Divididos. Incomprendidos. Su labor, eternamente cuestionada.

\fancyparbreak
Como consecuencia de la acción de Distorsión y su grupo, The Jammers, la comunicación de onda corta estaba completamente inutilizada en Ernépolis~I. Podría durar semanas, tal y como dijo el líder del grupo. Semanas interminables en las que quién sabía lo que podía pasar en la ciudad.

Eso no era, sin embargo, lo que más preocupaba a Los Caídos. No, su líder, John Scream, estaba pensando en jugadas posteriores a la inmediata, la imposibilidad de enviar escuadrones de cinco sujetos a patrullar las calles.

Él pensaba en las causas, los motivos concretos para hacer algo así. Quién contrató a aquellos críos para que interceptaran las señales y por qué.

En qué sentido ellos podían suponer, en sentido figurado, una interferencia para los planes de su desconocido oponente.

Siempre cabía la opción de que hubiera sido Hades el que estuviera detrás de todo. Medios no le faltaban, y tampoco intención. Además, aquella estrategia era perfecta de cara a su filosofía, inutilizar a Los Caídos pero al mismo tiempo sin destruirlos, al menos mientras en su retorcida mente les siguiera considerando rivales más que enemigos.

Los días pasaron, sin embargo, y nadie se hizo eco de la acción. Nadie proclamó haber sido el causante intelectual de tal acto de terrorismo silencioso. Y a medida que el tiempo transcurría Scream fue ignorando a Hades como posible causante. No al menos teniendo en cuenta sus ansias grandilocuentes de mostrar cuán lejos llegaba el alcance de su poder.

Otros enemigos fueron descartados con igual rapidez, y Scream comenzó a darse cuenta de que su lista de archinémesis empezaba a volverse cada vez más preocupantemente extensa, al tiempo que, en una relación de proporcionalidad inversa, y sobre todo después de la destrucción de la sede de QI, su lista de potenciales aliados se volvía más y más pequeña.

En todo caso Scream era consciente de que aquel hecho podía ponerles a prueba de maneras que nunca había imaginado que podían suceder. Sin comunicación no había coordinación, y sin coordinación no había equipo. Volvían a ser un grupo de sujetos experimentados con más buena voluntad que otra cosa.

Razorclaw y Ellis fueron inicialmente conservadores al respecto y recomendaron a Scream que cesara toda actividad que implicara sacar al exterior a los escuadrones. Podían vigilar en la sombra, infiltrarse, enviar informes al cuartel, aunque fuera de viva voz o por medios telemáticos no instantáneos. Pero teniendo en cuenta que incluso las líneas de comunicación telefónicas experimentaban averías casi permanentes era más bien escaso lo que podían hacer permaneciendo relegados al estado de meros observadores.

Los primeros días Scream optó por hacerles caso. Ordenar que se replegaran los escuadrones. Hibernar, dejar crecer la leyenda. Pero su ausencia de las calles no pasó desapercibida, y los problemas de comunicación que las fuerzas oficiales del orden estaban también padeciendo aumentaron la densidad de delitos de carácter individual, como robos, ajustes de cuentas o violaciones. Las mafias locales, por otro lado, que siempre habían funcionado por medio del eficaz boca a boca, vieron su oportunidad de sellar acuerdos y tratos que habían estado paralizados durante mucho tiempo por riesgo a sufrir emboscadas o redadas.

Aun con todo Scream tenía una muy mala sensación al respecto del aumento de la criminalidad, y no era el único en Los Caídos cuyo instinto estaba despertando a un peligro aún durmiente. Muchos forasteros estaban empezando a recalar en Ernépolis~I; nómadas del espacio, pistoleros de gatillo fácil, escoria cósmica que prefería moverse en solitario por callejones abandonados y tugurios de dudosa reputación. Había una relación causa efecto entre el aparente caos organizativo que estaba asediando la ciudad y su llegada, sin duda. Una tierra donde la ley no era todo lo rápida que se podía desear resultaba perfecta para que recalara toda aquella basura humana y no tan humana. Pero Scream tenía la sensación de que había algo más, algo que aún no podía intuir pero que flotaba latente en el ambiente, a la espera de mostrarse cuando menos pudiera esperarse.

Fue por eso que decidió que era mejor reanudar las incursiones al exterior, pero sólo en grupos de dos, y sólo los directores de escuadrones. Salvo algunas honrosas excepciones, los demás miembros de la organización no tenían la experiencia suficiente para coordinarse en pareja con un compañero con el que no podían hablar a distancia. Eso trajo consigo el problema aparejado de que cubrían mucho menos área de la habitual, pero al menos resultaba mejor opción que esconderse y dejar pasar la tormenta sin importar la destrucción que podía arrastrar consigo.

La situación en el exterior resultaba también muy peligrosa. No sólo era más fácil resultar herido y quedarse sin cobertura, también podían ser superados en número con enorme sencillez. Además de todos aquellos problemas de carácter letal estaba otro que afectaba a la filosofía misma de la organización, y era la posibilidad de ser descubiertos. Al fallar las transmisiones a corta distancia no había posibilidad de destruir los aparatos y ropas de un compañero caído, dejando atrás pistas que podían resultar a la larga fatales para la subsistencia de los ideales del grupo.

Las primeras incursiones fueron tímidas y calculadas. No podía haber errores. Los habría, pero tenían que mantener una política de tolerancia cero al respecto. La improvisación estaría a la orden del día, complejos esquemas largamente pensados y elaborados resultaban inútiles ante la mayor parte de las situaciones. Estaban como en el pasado. Juntos, pero esencialmente solos en el exterior.

Y por si todo aquello fuera poco, estaban a punto de vivir un duro golpe contra sus esfuerzos. Uno que venía de allá donde menos podían haberlo imaginado.

\parbreak
Emma Blades llegó al Tribunal Superior de Ernépolis~I, situado al Este del Distrito Financiero, en una zona bastante comercial y concurrida. Destacaba como el edificio más emblemático de la ya de por sí emblemática Plaza Wave, llamada así en honor al famoso piloto estelar Gordon Wave, descubridor de algunos de los paraísos siderales más hermosos conocidos por el hombre\dots\ y también de unos cuantos infiernos.

Era por ello que no dejaba de ser irónico el hecho de que en dicha plaza se fuera a juzgar a un ser que, posiblemente, no estaría entre los hombres de no haber sido por el afán explorador de tan insigne personaje.

Blades iba acompañada por Sky, que aunque asistía para estar con ella también lo hacía movido por la necesidad profesional. Pues si bien él mismo fue quien procedió al arresto del acusado en su momento, a veces pensaba que había transcurrido un millón de años desde entonces. Había solicitado al abogado de la acusación que no le llamara a declarar a menos que fuera absolutamente necesario, pues los problemas de comunicaciones hacían que no diera abasto en esos días. Pero pronto le dejaron claro que su presencia sería más que necesaria entre los testigos.

~---Puede que perdamos este juicio, y eso es algo que a ti no te gustaría ver ni a mí tampoco, pero que puede enfurecer lo impensable a John, que fue testigo directo de sus crímenes ~---le comentó off the record en cuanto tuvieron la ocasión de quedar para ensayar su declaración.

~---Lástima que no puedas llamarle ~---fue el único comentario de Sky ante tan, por desgracia, acertada observación.

El juzgado estaba muy concurrido, pues era una de las pocas veces que se sentaba un alienígena en el banquillo de los acusados, una de las escasas en que sus crímenes resultaban tan aterradores, y sin duda la primera en que el acusado era el último ser vivo de su propia especie.

Blades y Sky entraron en la sala y tomaron asiento en la cuarta fila, cerca de la mesa de la acusación. Sky pensó que sería la primera vez que vería a Razorclaw implantar justicia\dots\ fuera de las calles, claro.

Blades, por su parte, había coincidido con él en varias ocasiones. Nunca habían sido rivales enconados, pero tampoco eran amigos indiscutibles hasta el final de los tiempos. Tanto mejor así, pensaba Sky a menudo, o tendría que dar un montón de explicaciones a su novia que siempre había ensayado pero que esperaba no tener que utilizar.

Miró al banquillo, labrado en madera noble pero, para aquella ocasión, protegido por un material traslúcido, de porosidad anormalmente reducida, inferior a la del más compacto cemento. No era para menos, teniendo en cuenta que aquel que estaba dentro podía atravesar la práctica totalidad de los materiales conocidos. El cubículo en el que estaba encerrado apenas tenía respiraderos, pero tampoco eran demasiados necesarios, pues la especie a la que pertenecía el acusado, los Axcronianos, tomaba aire apenas una vez cada diez minutos.

Sky miró atentamente al acusado y éste le reconoció. Guiñó sus siete ojos, distribuidos a lo largo del rostro, de manera simultánea, algo que Sky aprendió que expresaba descontento entre los de su especie, y pudo percibir en ellos la malevolencia en su más perverso estado natural.

Krexon. Cuánto odiaba a aquel cabrón de piel violácea.

Pero su atención se desvió pronto del alienígena para centrarse en la del hombre que desde el elevado estrado presidía la sala. Gozaba ya de cierta edad, pero parecía tan vitalista como en su más tierna juventud. Sus ojos brillaban con un fulgor indescriptible, y aunque eran los de un ser humano Sky pudo percibir también en ellos que había alguna clase de matiz que no le acababa de resultar de fiar. Eso también podía deberse, pensó, a que desde que entró en el cuerpo había presenciado cómo muchos de aquellos hombres de leyes habían puesto en duda tanto pruebas cruciales como declaraciones de la más indiscutible solidez.

En todo caso, pensó, saliera como saliese el litigio, al fin había conocido al Juez Supremo Nitram, famoso en todo Ernépolis~I por su arrolladora personalidad y sus extravagantes procedimientos y conocimientos.

Su mirada pasó después a Razorclaw, visiblemente preocupado, y no era para menos. Por lo que le había contado de Nitram, era un juez muy chapado a la antigua y de difícil clasificación. Fue claramente neutral frente a Ellen Gorgon durante su ascenso presidencial, y con el hundimiento de su imperio logró permanecer al margen de toda conexión ilegal que pusiera en peligro su carrera. No en vano su puesto era de un poder inimaginable, pero que se empleaba para propósitos muy distintos de la mera dominación electoral.

Aun así no faltaban los rumores que decían que el Juez Nitram era de los que en el fondo echaba de menos el reinado de Ellen Gorgon, pero que astutamente sabía ocultar aquella parte de su personal e íntimo descontento.

Razorclaw comenzó a hablar al juez y a la sala. No había jurado popular en aquella ocasión, pero sus palabras, sin duda, estaban destinadas a provocar una reacción múltiple en todos los presentes, ya fuera directa o indirecta.

~---Hoy estamos juzgando a alguien que no es como nosotros. Un ser que viene de estrellas lejanas que seguramente ninguno de nosotros conoceremos jamás. Un ser cuya especie fue descubierta ni más ni menos que por el explorador Gordon Wave, al que todos conocemos y admiramos. Más aún, un ser cuyo planeta ya no existe, pues en el año 635 de la etapa tercera del Universo, según cómputo de pilotaje espacial, un meteorito de más de cien kilómetros de diámetro chocó con el planeta Axcron, hogar del acusado, y exterminó toda vida que sobre él se encontraba, todo ello a pesar de ser un mundo más que acostumbrado a las colisiones de meteoritos, y que entre otras cosas motivó la capacidad de su especie dominante para poder volverse intangible y sobrevivir a tan duras condiciones.

\rquoti Estamos, pues, ante un ser sin duda desdichado. Un paria, exiliado de su propio mundo ya antes incluso de que se convirtiera en polvo cósmico, y quiero que se tenga presente este detalle. Un ser que, sin duda, ha sufrido lo inimaginable a lo largo de su existencia. Imagínense estar en su lugar.

\rquoti Ahora piensen por un momento que ustedes fueran Krexon. Imaginen que fueran el único ser humano en todo el Universo. Tendrían que verse forzados a sobrevivir en un mundo hostil, desconocido para ustedes. Aprender otros idiomas, otras costumbres. Pero, díganme, ¿matarían a aquellos que trataran de ayudarles? ¿Respetarían la ley de los lugares en los que irían a parar? Teniendo en cuenta los niveles de universalización a los que hemos llegado lo harían, sin duda. De ello depende la coexistencia pacífica de nuestra especie con muchas otras conocidas.

\rquoti Él no tomó esa decisión. Krexon se convirtió en un criminal por propia voluntad, sin que nadie le forzara a ello. Ya era un desarraigado en su propio mundo y no tardó en serlo en el nuestro. Extorsión, secuestro, tortura y asesinato. No son delitos menores, ni mucho menos. Son delitos que atentan contra los derechos universales más básicos y profundos. Atentan contra los mismísimos cimientos de nuestro sistema penal, hechos tan básicos que incluso los niños comprenden el alcance de su horror. Piensen en ello, por favor.

Razorclaw se sentó y la abogada de la defensa, Iranagi Noriko, se levantó para dar su propio discurso. Aunque aún quedaban varios testigos aquellos alegatos se producían en todas las sesiones, y muchos de ellos estaban siendo grabados y recogidos para la posterior decisión del juez. Blades nunca había coincidido contra Iranagi, pero sabía que era una buena abogada que no se dejaba amedrentar fácilmente, por escasas que fueran las expectativas a su favor.

~---El señor Razorclaw ha hablado con gran sabiduría. He visto también que ha estudiado con detenimiento la historia de los Axcronianos. Lo que me pregunto es si habrá estudiado también su cultura. Analicémosla por un momento. Los Axcronianos, en términos de creencias, son poco menos que animistas, por decir algo. Admiran los acontecimientos cósmicos del mismo modo que nosotros admirábamos el rayo o el fuego, y no personifican tales admiraciones. Consideran otras religiones como absurdas, y hay exoteólogos que los han llegado a llamar animales inteligentes.

\rquoti Viven, perdón, vivían bajo tierra, en clanes separados entre sí varios kilómetros de distancia. Eran comunales y sin líder alguno, aunque se imponía la ley del más fuerte. Sus casas, por cierto carentes de puertas y ventanas ya que los muros no suponían barreras para ellos, eran un claro ejemplo de completo desinterés por lo cultural. Ni adornos ni elementos pictóricos, decorativos ni emotivos, o en un grado casi inapreciable. Su estadio de desarrollo artístico es ocho, un nivel bajísimo comparado con el nuestro, aunque su grado de conocimiento científico es dos, bastante superior. Pero aun así su ciencia, por llamarla algo, era básicamente química y unos pocos conocimientos intuitivos de matemáticas y física.

\rquoti Ignoro qué imagen pueden estar formándose de este ser, pero piensen por un momento que un ser humano de un pasado lejanísimo, casi prehistórico, despertara en nuestro mundo y tuviera libertad para caminar por donde él quisiera. Imaginen, sólo imaginen, la sensación de amenaza constante que podría sentir ante un mundo desconocido como el nuestro, y su previsible reacción. ¿Habría que encarcelar a alguien así? Yo no lo creo. Habría que comprenderle, estudiarle. Más aún teniendo en cuenta que se ha prestado voluntario para toda clase de estudios de naturaleza bioquímica y psicológica, y que supone nuestro único eslabón para entender a una especie ya desaparecida. Por todo ello, si lo que la justicia pretende es redimir, y no castigar, considero que Krexon, este ser desdichado, como ha dicho mi colega, debería ser puesto en libertad.

Iranagi volvió a su sitio y los testigos comenzaron a sucederse, no tardando en llegar el momento en que Razorclaw llamó a Sky al estrado. Ya curtido en esas situaciones, Sky prestó juramento y se preparó para las preguntas, cortas pero directas.

~---Dígame, Jefe Sky, ¿de qué conoce al acusado?

~---Secuestraba civiles para usarlos como rehenes. Su intención era introducirse en el mundo del crimen organizado de Ernépolis, y los usaba como seguro mientras establecía contactos.

Por consejo de Razorclaw, Sky no mencionó nada relativo a Ellen Gorgon y sus tentáculos mafiosos.

~---¿Estuvo presente en su detención?

~---Sí.

~---¿Estaba cometiendo algún delito cuando fue capturado?

~---Secuestro.

~---¿De quién?

~---Una niña de ocho años.

~---¿Qué pensaba hacer con ella?

~---Esa pregunta es ambigua ~---protestó Iranagi.

~---Reformule la pregunta, señor Razorclaw ~---solicitó Nitram.

~---¿Qué le sucedió a los otros secuestrados?

~---Murieron todos de inanición. Los dejaba morir lentamente.

~---¿Por qué?

~---Porque tardaba más en alimentarles que en buscar otro rehén.

~---¿Cuántos fueron?

~---Doce. Uno por semana.

~---Gracias, Jefe Sky.

No tardó en llegar el turno de preguntas de Iranagi.

~---Dígame, Jefe Sky, mientras Krexon estuvo bajo su custodia, ¿qué comía?

~---Nada.

~---¿Nada? ¿Entonces cómo se alimentaba?

~---Respira gases nocivos.

~---¿Cómo los que podrían producirse en capas altas de la Nube, por ejemplo?

~---Sí, pero\dots

~---Responda, ¿es así?

~---Sí ~---se resignó a contestar Sky.

~---¿No es cierto que puede comer también ceniza, Jefe Sky?

~---Así es.

~---Dígame, Jefe Sky, ¿qué fue lo que se encontró en la boca de todas las víctimas de Krexon?

~---Restos de ceniza.

~---¿Diría usted que son sólo resultado de haber apoyado la cabeza contra el suelo?

~---No, pero\dots

~---Limítese a contestar.

~---No.

~---De modo que las víctimas murieron de inanición, pero tenían restos de ceniza en la boca. Dígame, Jefe Sky, ¿alguna vez ha visto un poblado de Axcron?

~---No, pero sí sé ver a un mentiroso manipulador, y él lo es.

~---Tal vez quiera usted condenarle sin necesidad de juicio, entonces.

~---Jamás ~---fue la dura respuesta de Sky.

Finalmente, llegó la pregunta que Razorclaw estaba temiendo.

~---¿Tuvieron que pelear contra Krexon para detenerle?

~---No.

~---¿Cómo estaba?

~---Atado e inconsciente.

~---¿Lo hizo aquella sombra que pulula por las calles de Ernépolis?

~---Protesto, está orientando la respuesta ~---reclamó Razorclaw.

~---Cambio la pregunta. ¿Quién cree que lo hizo?

Hubo un silencio. El momento de decidir si mentir o decir la verdad había llegado.

~---¿Quién, Jefe Sky?

~---Un justiciero ~---fue su escueta respuesta.

~---No haré más preguntas.

Después de la declaración de Sky hubo otras, y el propio acusado fue interrogado por ambas partes, pero sus respuestas fueron, en el mejor de los casos, cortas y poco explicativas, lo que apoyaba la tesis de Iranagi de sentirse ajeno al incomprensible mundo que le rodeaba.

~---A veces no logro entender ni las preguntas que me está haciendo ~---fue una de sus frases que se le quedó marcada a Sky a fuego vivo.

El Juez Nitram salió a deliberar, pero no tardó demasiado. Razorclaw concluyó que posiblemente su decisión ya había sido tomada desde antes que comenzara el juicio.

~---Como muchos de los presentes saben, he pasado muchos años viajando por el Universo en calidad de embajador. Es por ello que me he encontrado con las especies y situaciones más insólitas imaginables. Este ser no es la más extraña de ellas. Sus actos no son en absoluto justificables, pero sí son comprensibles si tenemos en cuenta que se siente como el último cuerdo en un mundo de locos. Es por ello que declaro que sea puesto en libertad, pero deberá someterse a un régimen especial que le permitirá estudiar, apreciar y comprender la cultura humana, así como discernir nuestros conceptos más intuitivos relativos a la justicia, la ley y la moral. Se levanta la sesión.

Sky ya se temía el resultado, pero no pudo evitar un sentimiento de furia inundarle por dentro. El mismo precisamente que habían temido que podía desatarse en Scream.

~---Si hubiéramos llegado un poco más tarde ~---confesó a Blades~--- no habríamos encontrado a esa niña, y estaría muerta.

Blades no pudo por más que asentir con la cabeza, aun sin suponer que Sky, en realidad, no se estaba refiriendo a él y sus policías.

Al mismo tiempo Razorclaw se acercó hacia el juez mientras éste se disponía a abandonar la sala y la celda que rodeaba a Krexon era desmantelada.

~---Juez Nitram\dots\ dijo tocándole en el hombro. Al instante, una decena de discos flotantes surgieron de las mangas del juez y se interpusieron entre él y Razorclaw, en posición vertical y pululando con suavidad, como si fueran insectos vivos.

~---Supongo que no sabía lo que ocurría si se me toca bruscamente ~---explicó con calma.

~---Algo había escuchado, juez. Los fabricaron los Gilock para que fueran sus guardaespaldas personales, si no me falla la memoria.

~---En realidad ofrecieron el modelo para todos los magistrados, pero debido a desavenencias en el acuerdo se paró la producción y me ofrecieron los únicos prototipos construidos, como regalo por haber sido su embajador en su mundo durante muchos años. Su idea era que con el aumento de rango judicial se poseyeran más de estos discos personales. Usaron como argumento su avanzada tecnología de movimiento aéreo, especialmente pensaba para crear escudos móviles contra la constante lluvia de meteoritos que asola también su planeta.

~---De uno a diez, supongo. El número más alto reservado para el Juez Supremo.

~---Así es. ¿Qué desea, letrado?

~---No quiero cuestionar su decisión, pero espero que no tengamos que ver otra vez a Krexon sobre un estrado en breve.

~---Aunque así fuera, señor Razorclaw, eso no quita validez a mi decisión ni mis argumentos. Es tanto como decir que si un criminal reincide a pesar de obtener la libertad por buena conducta, esta libertad nunca debería ser usada en beneficio de los presos.

~---Entre nosotros, juez. Sé que sus discos impiden que se pinchen sus conversaciones, de modo que puede ser franco. ¿No podría ser que sus circunstancias personales le hicieran inadecuado para juzgar este crimen y hubiera sido más acorde con el sistema judicial pedir ser recusado?

~---Muy al contrario, letrado. Mis conocimientos me colocaban en una posición privilegiada para comprenderlo en toda su extensión. Y ahora, si me disculpa\dots\

Se marchó de la sala y sus discos le siguieron a muy corta distancia. Algunos curiosos observaron la situación, y la interpretaron como un conflicto entre abogado de la acusación y juez, pues esos discos solían saltar en las acaloradas y filosóficas discusiones que su dueño solía mantener a menudo. En realidad nunca habían tenido que protegerle de ninguna incidencia, al menos que se supiera, pero sí que contribuían, sin duda, a otorgarle un halo más que atemorizante.

Razorclaw regresó con Sky, justo a tiempo de ver cómo Krexon se marchaba de la sala, usando la puerta como todo el mundo para no llamar demasiado la atención.

~---Por una vez, me alegro de que las comunicaciones estén en mal estado ~---argumentó Razorclaw acercándose hacia Sky y Blades.

~---¿Por qué dices eso? ~---preguntaron casi al unísono.

~---Al menos no veremos en los grandes monitores de las calles el rostro de un asesino desalmado ~---terminó en lo que recogía, con evidente desencanto, los papeles que tenía preparados para el juicio.
