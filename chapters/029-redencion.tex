Cambiar no es fácil. Nunca lo ha sido. Ni siquiera lo es plantearse el cambio en sí mismo, ese primer paso tan terrible que puede que no se llegue a dar jamás.

Un avance que puede alterar la vida de las personas de manera tan radical que ya nunca vuelva a ser la misma\dots

\fancyparbreak
Hubo gran revuelo en Ernépolis tras la aparición de Alma Espejo en la escena global de la compleja ciudad. Unos lo señalaron de manera instantánea como un salvador, un héroe sin reservas. Otros, por el lado contrario, lo consideraron una amenaza, un peligro potencial al delicado equilibro de la ciudad.

John Scream ya había experimentado esa situación antes. Dos veces, de hecho. Ambas de primera mano, además.

La más reciente fue con Los Caídos, claro. Con esa organización oscura, tenebrosa, que se movía por medio de subterfugios y planes ocultos, que había llegado a tal extremo de sacrificio que cada vez que alguno de ellos daba su vida por la ciudad, o la intentaban salvar con éxito, a nadie realmente le importaba que así hubieran hecho.

La mayoría de la gente de la calle les tenía miedo. Pensaban que sus peores enemigos eran aliados. Que tal vez ellos liberaron a Armor, o confabularon un pacto secreto con Hades, y sólo lucharon contra ellos después que el monstruo de metal se rebelara contra su yugo, o que trataran de traicionar al tirano invisible en el último momento.

La ciudad estaba amenazada por la sombra de los monstruos y los malvados, y el mundo pensaba que ellos eran uno más del grupo, tal vez, incluso, el peor de todos.

Pero aquellas dudas, aquellas incertidumbres que nunca se disipaban, no entraron por primera vez en la vida de Scream cuando pasó a formar parte del legado de Starr Miles. No, ni mucho menos, ya las había vivido antes, de manera más parecida a lo que estaba ocurriendo tras el fallido torneo.

Porque Scream había sido un héroe. Scream había destacado en los cielos. No de manera tan espectacular como había hecho Alma Espejo, en sentido literal incluso, pero sin duda también tuvo su momento.

Porque para qué negarlo, Alma Espejo tenía un carisma al que era imposible sustraerse. Alma Espejo era un hombre, pero era un hombre que volaba, que poseía una voz que provocaba respecto en los aliados y temor en los enemigos. Era un hombre que tenía la voluntad de resistirse al control un engendro tecnológico. Un hombre que brillaba como una supernova, eclipsando cualquier otro vago destello a su alrededor. Llenando de luz hasta los callejones más oscuros de la ciudad.

Alma Espejo era todo lo que ellos alguna vez habían deseado ser de manera individual para la ciudad de Ernépolis~I.

Más aún, era un símbolo. Un estandarte, una bandera. Alguien a quien admirar, la esperanza de los oprimidos, de las víctimas, el temor de los asesinos.

Pero por primera vez en mucho tiempo Scream estuvo de acuerdo con las opiniones venenosas que sobre el recién llegado se estaban vertiendo.

Le consideraban un peligro para la ciudad. Él también le consideraba un peligro, pero no para la ciudad, sino para él mismo. Apenas sabían nada de él, era un newcomer, un recién llegado al corazón mismo de la podredumbre y la corrupción. Ése no era su mundo, o tal vez había decidido destacar por encima de él. No se fundía con el ambiente, sino que lo usaba para contrastar, para mostrarse como se mostraría el Sol de no existir la Nube.

Alma Espejo sería un gran héroe, una gran esperanza, pero era un novato. Y los novatos, Scream lo sabía bien, solían cometer errores. No era algo reprochable por sí mismo, claro. El pequeño problema estribaba en que las meteduras de pata de los héroes primerizos podían costar la vida a otras personas, si no a sí mismos.

Scream fue también crítico con su propia opinión. Pensó si tal vez los celos no estarían mediatizando sus palabras. En apenas unas semanas el nuevo forastero, el desconocido más absoluto, se había granjeado la popularidad de una ciudad que estaba hambrienta de emblemas a los que venerar. No por eso estaba exento de detractores, pero nadie se atrevería a considerarle un criminal, ni mucho menos. Tampoco tenía reparos en ocultar toda pregunta que quisieran hacerle, y así en los primeros días circularon muchos rumores contradictorios sobre su origen. Pero el Presidente Scatter se apresuró a recabar y censurar toda la información, usando como argumento que cuanto menos se supiera acerca de él tanto más fácil le sería realizar su labor sin verse presionado. Dio su garantía personal de que estaba identificado y con sus datos en orden, y eso bastó para acallar al resto de los que trataban de desenterrar su procedencia.

Como medida especial, dado que era necesario incluirle en bases de datos de la ciudad, se le otorgó de manera oficial el apellido Shine, uno que al parecer el mismo héroe sugirió pues había usado en su momento en el pasado. Su apellido real fue mantenido en secreto, tanto que ni siquiera Saw llegó a saberlo, siendo sólo terreno exclusivo del propio Presidente y quizás nadie más. Su nombre de pila tampoco se consideró que fuera de dominio o interés público.

Scream recapituló. Tenemos nuevo héroe en la ciudad. Uno que se está empleando a conciencia en las calles y los bajos fondos. Su nombre, Alma Espejo. Edad, desconocida, aunque joven. Nombre real, desconocido, figura en las bases de datos con el apellido Shine, usado en el pasado, aunque en el Aquerón no se encontró nada después de investigarlo.

Scream y los suyos tenían las manos atadas. Al menos, pensó, de momento no se habían cruzado con él en su camino.

Por otro lado, tuvo que admitirse que su ayuda indirecta fue inestimable para que vieran liberada su carga de trabajo para mantener en discreto orden las calles de la ciudad. De ese modo podían dedicarse a otras tareas, otros asuntos que estaban pendientes.

Y había uno, de hecho, que llamaba la atención de Scream de manera imperiosa. Alguien que había aparecido recientemente en el gran teatro en que se representaba el destino último de la ciudad.

Nunca fue un héroe, pero tampoco fue un villano. Y por lo que se contaba en las calles, estaba en graves problemas.

Sin embargo Scream veía el asunto como algo en cierto modo personal. No porque hubiera tenido una vida paralela, pero sí que se identificaba en ciertos momentos con su situación. Él también había sentido la rabia, la amargura. Había vivido largos años en la calle, con todo arrebatado, y comprendía cómo puede cambiar a un hombre el veneno del odio.

Porque no veía oscuridad genuina en ese paria perseguido y misántropo. Veía más fachada que genuina ansia de provocar dolor en los demás. Y también inseguridad, incluso. Y un punto de rebeldía.

No lo pensó más. Los últimos informes decían que se le había visto cerca del acceso noroeste de la ciudad, tal vez tratando de buscar cobijo en algún almacén abandonado. Ese asunto lo trataría en persona, mejor. Solo.

Se enfundó el traje y se preparó para la salida al exterior.

Tareas de reclutamiento. No muy habituales, pero sin duda de sus favoritas.

\parbreak
Warren Shockman miró desde la ventana del único despacho del almacén, en la primera planta, y volvió a ver a aquellos matones equipados con potentes lanzarrayos. Había perdido la cuenta de cuántos de aquellos sujetos rondaban por la zona, pero llegó a la conclusión de que podían ser fácilmente una veintena. En los viejos tiempos una veintena de matones ya suponían para alguien como él una cantidad de enemigos considerable. Por eso en la era moderna en la que estaban, con sus facultades y aparatos mermados, si salía al exterior a luchar contra ellos lo más fácil sería que acabara frito en el suelo en unos cuantos minutos.

Sin embargo fue consciente de que no podría jugar al gato y al ratón mucho más tiempo, y nunca mejor dicho. Había disuadido a los matones de que entraran a explorar el almacén por medio de cucarachas, arañas y otras criaturas desagradables. Por mucho que fueran tiradores y soldados entrenados poseían aprensión como cualquier otro ser humano, y además la presencia de numerosos insectos les hizo pensar a los menos impresionables que aquel lugar no había sido visitado en mucho tiempo por nadie.

Pero el engaño no duraría para siempre. Su intervención en el torneo había salido terriblemente mal. No había ganado y para colmo de males el ganador, aquel ser de fulgor indescriptible, era alguien completamente ajeno a él, con quien no tenía contacto alguno y no podía, por tanto, servirle de barricada contra sus enemigos. Estaba, de hecho, dedicándose a atosigar a los criminales y matones del sur de la ciudad, por lo que difícilmente aparecería por la zona para librarle de sus dos docenas de perseguidores, que además estaban estrechando cada vez más el cerco a su alrededor.

Pero eso no era lo que más enfurecía a Shockman, no. Ni mucho menos. Lo que le enfurecía, crispaba su cuerpo hasta la última célula, era tener que esconderse como un animal asustado. De repente miró a la rata, que estaba sobre la mesa de la habitación, mordisqueando indiferente un trozo de cristal. Un símil inapropiado, pensó.

La impaciencia e inactividad torturaba su mente más que el miedo mismo. Estar ahí quieto, sólo dejando pasar los minutos y las horas, era más de lo que podía soportar. Se sentía capaz de asumir la muerte en las calles. Salir y enfrentarse a aquella pandilla de cretinos descerebrados a los que conocía muy bien, aunque nunca hubiera empleado subalternos en el pasado.

Era consciente de lo que pasaría, por otro lado. Podría con la mayor parte de ellos, pero no sería suficiente. Esos idiotas no tenían tan mala puntería como uno llegaba a esperar. Una bala perdida, un golpe de suerte, un segundo de despiste, bastarían para poner punto y final a la pelea de manera instantánea e irreversible.

Claro que al menos moriría con las botas puestas. Como siempre había pensado que acabaría sus días. Pudriéndose lentamente, devorado por su propia prole, sus armas de guerra dando cuenta de los despojos de su amo.

Se tocó el ojo tuerto con la mano, recordando el momento en que lo perdió, y se preparó para salir. Mejor morir frente al enemigo que de espaldas a él, por indigno que éste sea, aunque sólo vaya a ganar por mera superioridad numérica de sicarios mediocres y de poca monta.

Eso sí, no caería sin más, mostrándose como un pato de feria. Lucharía hasta el último aliento, más por una mera cuestión de orgullo que de ninguna otra cosa.

Curioso. Siempre pensó que acabaría capturado por alguno de los héroes a los que se enfrentaba, o traicionado por algún villano con el que hubiera establecido una frágil y efímera alianza. Nunca pensó que la escala más baja del crimen sería la encargada de quitarle de en medio.

Antes de salir se acercó al centro de la habitación y ordenó a la rata tuerta que se metiera en el bolsillo de su largo abrigo. Puso en marcha el dispositivo de control de los animales, en el otro bolsillo, y cogió un lanzarrayos que reposaba inerte sobre la mesa. Ya no bastarían los trucos de veces anteriores, fingir que estaba muerto a los ojos de sus perseguidores. Nunca usarlo dos veces contra la misma clase de adversario. Y morir en el suelo, de un disparo en la cabeza mientras se hacía el muerto, era más de lo que su tremendo orgullo podía hacerle soportar. Ya consideraba humillantes las ocasiones anteriores en que se había hecho el muerto, de modo que no volvería a pasar por eso. Si le encontraban fiambre sobre el suelo ceniciento, esa vez sería de verdad.

Abrió la puerta, miró al exterior, sopesó sus opciones. Calles estrechas entre paredes de viejas factorías. Existía la posibilidad de pasar desapercibido un tiempo, de aprovechar el entorno a su favor, y también sus habitantes. Pero sabía que una vez dentro del laberinto siempre acababa uno topándose con el callejón en el momento más inoportuno.

En la puerta de salida tenía acceso visual a la misma callejuela que estaba espiando desde la ventana. Antes de bajar ya había tomado las precauciones de sincronizar su descenso con las rondas ocasionales de los matones, para no ser cogido justo por sorpresa. El hecho de que tuvieran imprudentes conversaciones ocasionales jugaba en su favor también. La soledad siempre es la mejor aliada del guerrero traidor y sigiloso.

~---Acabaremos por encontrarle, y si no, quizás el Caído haga por una vez el trabajo en nuestro lugar ~---comentó uno de ellos con aires de saber de lo que estaba hablando.

~---Preferiría que no tuviéramos que verle en acción ~---contestó el otro, con cierto deje ocasional que revelaba un incierto temor.

A Shockman no le gustó eso. Nada. Ni siquiera estaban pensando en él mientras le buscaban y acechaban. Ya no le consideraban una amenaza, sólo un estorbo. Lejos de ser un peligro, había pasado a ser material de tiempos pasados, como tantos otros villanos ya muertos o encerrados bajo una infinidad de llaves.

Pero él seguía ahí. Él era Éxeter. Un villano tan misterioso que apenas ninguno de sus rivales supo nunca en qué consistían exactamente sus poderes.

Se dispuso a dejarse ver cuando, de repente, la rata comenzó a comunicarse con él por medio de su único y peculiar lenguaje. Un sentimiento primitivo, sencillo, pero al mismo tiempo claro y certero.

No estás solo en estas sombras.

Shockman se giró y apuntó el lanzarrayos a la negrura que le rodeaba. Nada. Nada, al menos, a simple vista.

Hasta que escuchó la voz. Esa de la que hablaban los criminales de poca monta en los bares clandestinos de la ciudad, a veces entre susurros, como si temieran invocar su misma presencia.

\emph{Si sales allí fuera acabarás muerto. Y esta vez, de verdad.}

Shockman no pudo evitar una sorpresa al escuchar esas palabras. Lo sabía. Sabía lo que hacía. O al menos, alguna idea tenía al respecto. Como poco tenía contactos y, tal vez, dotes más que sobradas de investigación. Además de ello, la ausencia prolongada de luz que aún se mantenía en la ciudad por las reformas, y que afectaba a muchos distritos con cortes ocasionales y esporádicos, era un indiscutible aliado de una criatura así para de ese modo aparecer donde menos uno la esperara.

O tal vez, simplemente, trabajaba para ellos.

Tal pensamiento hizo que no se lo pensara dos veces a la hora de disparar hacia la oscuridad, el fondo del callejón, lo que alertó a los tipos a los que pretendía tomar por sorpresa. Qué ironía del destino, pensó Shockman. Efectivamente, había encontrado el callejón en el laberinto, pero nada más salir del mismo, y él mismo lo había convertido en una trampa mortal.

No, él no, pensó. Ese ser, esa sombra. Él le había emboscado.

\emph{Deberías confiar en mí} ~---proclamó la sombra de nuevo, como si nada hubiera pasado~---. \emph{Ahora mismo soy tu única oportunidad para que salgas de aquí con vida.}

~---No vendrían hacia aquí si no fuera por ti ~---objetó Shockman.

\emph{¿Quién ha hablado de pelear?} ~---dijo la sombra sin mostrarse aún, aunque podía dejarse entrever su gabardina y parte de su pie deslizándose en el borde de las tinieblas.

Cuando los matones llegaron al callejón se encontraron con que estaba vacío, o al menos, eso era lo que parecía a simple vista. Sin embargo no se convencieron tan fácilmente para salir de allí como si nada hubiera pasado y analizaron la escena con más detenimiento. A su derecha, nada más entrar, había un conglomerado de cubos de basura abollados y rebosantes de fétido contenido. Más adelante notaron que la puerta trasera del almacén estaba entreabierta, como si la acabaran de dejar así por accidente.

~---Debe de haber ido por ahí ~---dijo el timorato, avanzando arma en mano.

Pero su compañero le hizo un gesto para que se detuviera, y señaló a las sombras en lo que ambos buscaban cobertura. Algo parecía moverse allí al fondo, y con lentitud avanzó hasta estar a su misma altura y disparar en cuando tuviera un blanco certero.

Siguió adelantándose, poniéndose en riesgo incluso, aunque siempre atento a toda reacción proveniente del fondo del callejón, cuando notó que sea lo que fuera que estuviera frente a él estaba saliendo de las sombras con pasos muy calmados y silenciosos.

Fue cuando empezó a perfilarse la silueta cuando decidió bajar el arma y le hizo un gesto a su compañero para que se acercara. Sólo era un gato callejero de color ceniciento, el tono que, por motivos lógicos, más abundaba en la ciudad, al mimetizarse a la perfección con el ambiente natural de la misma.

~---Sólo es un gato loco ~---dijo el temeroso disparando en su dirección.

La bala pasó muy cerca del animal, que antes de huir erizó el lomo y bufó en dirección a los sicarios.

~---No dispares, idiota. Guarda tus proyectiles para presas más grandes.

Ambos entraron por la puerta y, durante unos minutos, todo permaneció callado y en silencio en el callejón, hasta que finalmente Shockman salió de las sombras, como si éstas le hubieran tragado y acabaran de vomitarle.

~---¿Cómo has hecho eso, usar las sombras de esa manera?

\emph{Todos tenemos nuestras estrategias, Éxeter. En tu caso, no creo que ese gato saliera justo en el momento adecuado por mera casualidad.}

~---Eso da igual ahora. No tardarán en descubrir mi escondrijo y llamarán a toda la caballería para intentar cercar la zona.

\emph{Una simple emboscada no es suficiente para detenerme} ~---proclamó la sombra con altivez, dejándose ver completamente por vez primera desde que comenzó la conversación.

~---¿Qué harás, teletransportarnos? ¿Hacernos invisibles, tal vez? ~---dijo Shockman con todo de burla.

\emph{Nada de eso. Asómate para asegurarnos de que no regresan aún esos dos.}

Shockman se acercó a la puerta y, sin necesidad de asomar el pescuezo, sólo haciendo caso de la rata, hizo un gesto con la cabeza corroborando que aún estaban solos. Lo bueno de tener animales a las órdenes de uno era que ellos muchas veces funcionaban por instintos muy distintos de los visuales y más útiles en muchas circunstancias.

~---No hay nadie. Ahora quiero ver qué\dots

Lo siguiente que Shockman apreció fue que la sombra le estaba apuntando con un lanzarrayos, o al menos con un arma de características similares y factura, sin duda, única.

~---Sabía que no podía fiarme de ti. Eres como los otros villanos con los que me solía juntar.

La sombra no hizo ningún comentario y se limitó a disparar. Shockman recibió una descarga y cayó al suelo. Acto seguido la sombra cargó con él y se acercó a la pared del extremo opuesto al almacén. Tras unos segundos de precisa y concienzuda exploración, un ladrillo desvencijado se movió más de lo esperable y toda una sección de pared se deslizó con un ruido imperceptible por oídos poco atentos. Después de aquello se limitó a penetrar en el pasadizo penúmbreo justo antes de que la compuerta se cerrara tras de sí, como si nunca hubiera existido en realidad.

\endinput
