Sólo dos días para el final de la partida, y las posiciones avanzaban. Estaban más cerca de conocer a su enemigo, pero al mismo tiempo éste avanzaba siempre un paso por delante de ellos. Era el momento de saltar al vacío, de efectuar complicadas maniobras. Aunque eso implicara adentrarse en el corazón mismo de mayores y más peligrosas intrigas.

\fancyparbreak
La información que Los Caídos habían obtenido de su incursión en la sede de QI podía parecer insuficiente a ojos vista, podía incluso parecer simplemente una confirmación trivial de sospechas más que plausibles. Muchos investigadores no hubieran logrado mucho con tan poco a lo que aferrarse.

Los Caídos no.

Para empezar, estaban empezando a descartar posibilidades. Había habido, en efecto, muertes sospechosas que podían corroborar, sujetos que, por uno u otro motivo, habían sido despedidos de QI, o simpatizaban con gente a la que la que así le había pasado. No lograban ver el porqué de esas muertes, pero algo era algo. Al menos, sabían que habían formado parte importante en la estrategia de su enemigo.

Luego estaba lo del Cancerbero, un artefacto que habían estado diseñando en secreto en QI. Confidencial, sin duda. Por un momento Scream temió que pudiera tener alguna clase de paralelismo con los militares que habían diseñado la servoarmadura que luego adquirió vida propia y se convirtió en el letal y temible Armor. Recordó aquella batalla sobre la superficie lunar, donde por poco se convirtió en combustible orgánico para un vampiro cibernético, y el sudor perló su frente.

Eso no quería decir que considerara la situación actual menos peligrosa, en absoluto. Armor era casi imparable en un combate cuerpo a cuerpo, pero Hades, ese nuevo enemigo, tenía una manera de actuar muy distinta. Tal vez fuera sólo un humano, pero tenía medios, poder y, sobre todo, tenacidad. Era un genuino villano como los de antaño.

Scream se preguntó, por un momento, por qué sólo las cosas malas del pasado eran las que al parecer lograban regresar al presente.

En todo caso, recapituló, además de la certeza de que QI había estado llevando a cabo un proyecto confidencial, tenían una fecha. Y esa era la información crucial, la que podría adelantarles por una vez a los actos de su oponente.

No encontrarían referencia alguna a ese Cancerbero por mucho que investigaran y escarbaran, eso era algo que tenía claro. Pero al menos sí sabían cuándo se suponía que habían estado enfrascados en dicho proyecto, y eso tal vez podía ayudarles a encontrar la verdad oculta que en realidad siempre había estado ante sus ojos.

Fue así que, tras ir a ver a Sky a la comisaría, fingiendo simplemente que se trataba de una visita por motivos de amistad, éste obtuvo en poco tiempo de la base policial de datos una noticia que resultaba ser de gran interés.

~---Más o menos por la fecha que indicas hubo un incendio en una de las subsedes de QI ~---se limitó a comentar, elevando ligeramente la comisura del labio~---. Sólo hubo un herido, una mujer, Tracy Swoop, que sufrió graves quemaduras.

~---¿Hay imágenes? ~---preguntó Scream.

Sky giró la pantalla y Scream pudo ver los restos de lo que parecía ser una factoría más a las afueras de la ciudad. Las llamas habían consumido el edificio con gran precisión.

Demasiada, pensó.

~---Esto huele mal ~---comentó Sky de repente, apoyándose en su respaldo.

~---¿Por qué dices eso?

~---Sólo un herido, para un incendio de tan graves consecuencias.

~---Aquí pone que fue de noche, lo que explicaría la ausencia de gente, pero\dots

~---\dots\ no explica la presencia de ella allí ~---terminó Sky.

~---Al parecer, además, el incendio fue provocado por un empleado que entró en una crisis nerviosa. ¿Y sabes qué es lo mejor?

~---Sorpréndeme.

~---Es uno de los antiguos empleados de QI que murió recientemente ~---acabó Scream saliendo a toda prisa del despacho de Sky.

\parbreak
Nada más llegó a los restos de la factoría, acompañado por el mismo escuadrón que le había acompañado al rascacielos, Scream contempló el solar vacío, desolado y ni siquiera desescombrado, y tuvo la sensación de que allí tal vez encontrarían la respuesta a varias de sus preguntas.

Lo primero que hicieron fue peinar la zona con sensores orgánicos en busca de algún rastro de vida humana, aunque fueran huesos, que esclarecieran los motivos para querer prender fuego a un lugar como aquel. Por supuesto, además, se habían informado sobre la única víctima colateral del incendio. Tracy Swoop, secretaria en la sede central de QI. Vida anodina, ejemplar y no muy reseñable. Tenía contrato, piso, gato y, presumiblemente, proyectos de futuro. Después del incendio se marchó de la ciudad y nadie sabía muy bien dónde podía estar en ese momento. Al menos sabían un lugar en el que no estaba: bajo los escombros de la factoría quemada, pues no detectaron presencia alguna de cadáveres o restos humanos.

Sin embargo, el edificio había estado ocultando otros posibles secretos.

~---Esto parece una trampilla ~---dijo uno de los novatos, quieto frente a un hundimiento de escoria algo más pronunciado que la rasante de alrededor. Apartó la grava triturada a un lado y no tardó en descubrir un tirador.

~---Más pasadizos ~---comentó Scream recordando aquellos que en su momento Armor había fabricado~---. Odio los pasadizos.

~---¿Qué hacemos?

~---Quédate aquí de guardia, con él ~---ordenó Scream señalando al que había hablado y a su compañero de la derecha~---. Nosotros tres bajaremos. Avisadnos de cualquier movimiento sospechoso, nosotros haremos lo propio.

Una vez asignadas las labores Scream bajó el primero y los dos hombres que había designado a tal efecto le siguieron sin perder el ritmo. Todo estaba bastante oscuro, pero gracias a los infrarrojos notaron que estaban en una estancia amplia, tal vez un sótano. Comenzaron a caminar medio a ciegas hasta que encontraron lo que parecía ser un acceso a niveles inferiores de la factoría. Las llamas lo habían devorado todo, aunque aún podían verse los restos de complicados dispositivos cuya función era absolutamente desconocida para los presentes.

~---Sea lo que sea ese Cancerbero parece que era aquí donde lo estaban construyendo ~---dijo Scream en voz alta, examinando los vestigios retorcidos de lo que pareció servir en su momento como un contenedor de piezas.

~---Creo que he encontrado un generador, Capitán.

~---Acciónalo. Y haz el favor de no llamarme Capitán ~---replicó Scream torciendo el gesto.

~---Un débil rescoldo de luz alumbró la ruta de los tres hombres y les permitió darse cuenta de cuán grande debió ser la estancia en la que se encontraban, aunque los muros derrumbados y las montañas de escombros y restos materiales habían dotado al lugar de una cartografía única e irrepetible.

~---Por allí ~---señaló Scream, mirando hacia unas escaleras que llevaban a plantas inferiores~---. Será mejor que nos demos prisa. No creo que esta luz dure demasiado.

Las escaleras estaban consumidas, aunque seguían pudiendo aguantar el peso de una persona, seguramente debido a que sólo habían sido rozadas por las lenguas de fuego. Bajaban en espiral cuadrangular, dejando un ancho hueco que debió pertenecer a alguna clase de robusto montacargas. El camino fue más largo de lo que habían pensado, y no lograron calcular con certeza a cuántas plantas por debajo de la superficie debían estar. En ocasiones localizaban alguna puerta en los descansillos, pero o bien eran incapaces de abrirla por estar atascada, o bien lo conseguían pero sólo para encontrarse con una montaña de roca al otro lado.

Cuando llegaron al final de la escalera se encontraron en la entrada de una cueva que había sido pavimentada pero cuyas paredes se conservaban casi intactas. El suelo tenía las huellas de innumerables rodadas pertenecientes, seguramente, a vehículos de transporte y almacenamiento. Cuando miraron al fondo de la cueva notaron que una fuente de luminosidad nacía desde allí, una luz tenue pero deslumbradora en comparación con aquella que habían activado.

Comenzaron a caminar hacia la luz y apreciaron cómo la decoración del lugar se volvía cada vez más profusa y clásica, con columnas dóricas excavadas en la roca viva de las paredes y mosaicos teselados cada vez más cuidados e impolutos. Al parecer el incendio no había llegado a aquella insólita parte de la factoría.

Cuando llegaron al final del pasillo vieron algo que hizo que se quedaran quietos sin dar un solo paso, por lo desconcertante de hallar algo así en un lugar como ese.

En el suelo yacía una mujer desnuda. Estaba desvanecida y ofrecía la espalda a los recién llegados. Su pelo, largo y lacio, era tan resplandeciente que por un momento Scream llegó a pensar que era la fuente de la luz que les había guiado. Era tal la belleza que irradiaba que resultaba hermosa incluso en su desvalida inconsciencia.

Uno de los dos acompañantes de Scream no pudo mantener la formación por más tiempo y se acercó al lugar donde la mujer se encontraba tumbada. Scream y el otro hombre contuvieron el aliento, esperando una respuesta.

~---Está viva ~---proclamó finalmente, al tiempo que ella misma se despertaba muy lentamente, como si acabara de regresar del mundo de los muertos.

~---Me has salvado, gracias ~---dijo la mujer, incorporándose con extrema timidez pero al mismo tiempo con un aire de sensualidad, protegiendo con un cruce de piernas sus zonas íntimas, pareciendo incluso juguetona al hacerlo. Su cabello liso y sedoso se deslizó por sus senos como el agua cae a lo largo de una brizna de hierba~---. ¿Quién eres?

~---No hay tiempo para eso ~---contestó el aludido, y Scream y su otro compañero contemplaron la escena como si el tiempo se hubiera detenido y fueran incapaces de deshacer el hechizo.

~---¿Quién eres?

~---Me llaman Afrodita, y aunque soy una diosa, para ti seré tu esclava. Acércame tus labios, déjame sentir tu calor, tu aliento de vida.

Scream empezó a ser más que consciente de que algo iba muy mal, pero sentía los músculos igual que si los tuviera agarrotados y fuera una estatua de piedra. No se trataba de que no pudiera moverlos, sino de que algo en su cabeza le impedía hacerlo, como si el mero hecho de intentarlo fuera tan absurdo como lanzarse de cabeza contra una pared para intentar atravesarla.

Lo único que él y aquel que estaba a su lado pudieron hacer es convertirse en testigos mudos de aquella situación que cada vez más parecía adquirir tintes completamente sobrenaturales.

Afrodita besó a aquel que había acudido a auxiliarla, y al tiempo que lo hacía, empezó a besarle en otras partes del cuerpo. Poco a poco, en las mejillas, en el pelo e incluso entre los ojos.

La escena hubiera sido completamente pasional de no ser porque el otro figurante de la misma gritaba de dolor a cada beso que recibía.

Por fin Scream no tardó en comprenderlo del todo, aunque ya estaba más que claro antes del fatal desenlace. Ilusiones. Mentiras de los sentidos. Sus propias neuronas sucumbiendo al engaño, tratando de hacerle razonar, aceptar la evidencia que sólo con la lógica podía enterrar.

Realizando un esfuerzo titánico, sus dedos se acercaron al dispositivo que controlaba los hologramas que ellos mismos proyectaban contra sus enemigos. A cada grito de dolor de su subordinado las yemas se estiraban un poco más, como respondiendo a la llamada de socorro por voluntad propia. Al fin logró acceder a los controles y proyectó una sombra tan grande y aterradora sobre la bellísima mujer que ésta se lanzó hacia atrás como un animal asustado, y al hacerlo se golpeó contra una de las paredes y perdió el conocimiento.

Esa vez, sin embargo, el desvanecimiento sí fue real, y no fingido. Nada más se desmayó las columnas desaparecieron de las paredes y el suelo volvió a adquirir la forma pavimentada que habían visto nada más llegar al final del recorrido espiral. Todo vestigio de una cultura helénica había desaparecido junto con la consciencia de aquella mujer.

Nada más verse libre, el compañero de Scream que había sido presa del mismo engaño corrió a auxiliar a su compañero herido. No pudo por más que enmudecer cuando vio que le habían arrancado trozos del rostro a mordiscos y algunos de ellos estaban aún palpitantes en el suelo.

La visión más horrible, sin embargo, le tocó presenciarla a Scream. Se acercó a la mujer y notó que ya no irradiaba en absoluto la belleza de apenas un instante antes. Seguía desnuda, pero su cuerpo estaba surcado por amorfas cicatrices que lo recorrían como la lava, furiosa, recorre la ladera del volcán del que proviene. Pero sin duda lo peor era su rostro, aún bañado en la sangre que acababa de paladear. Su mera visión le obligó a apartar la mirada y a quitarse la gabardina para cubrirla no por dignidad sino para apartar aquella pesadilla del alcance de su visual.

Comprendió que acababan de encontrar a Tracy Swoop. O al menos, aquello en lo que se había convertido.

\parbreak
Cuando llegaron al Aquerón lo primero que hicieron fue tratar al componente herido del escuadrón, al que tuvieron que dar múltiples puntos de sutura, y que ya por siempre portó aquellas marcas de las que nunca pudo librarse, además de, como penitencia final del destino, soñar a menudo con aquel rostro hermoso que le había surcado el semblante de terribles cicatrices de guerra.

Después de eso encerraron a Afrodita en una de las celdas de máxima seguridad del cuartel y la aislaron completamente, indicando que a nadie se le ocurriera acercarse a ella, fuera lo que fuese lo que escuchara. Y para asegurarse de que se cumplían sus órdenes Scream puso de vigilante a aquel que había sido testigo contra su voluntad del tremendo peligro que aquella mujer suponía, y que logró evitar que varios de los que por allí pasaban cayeran en su trampa por medio de sus palabras expertas y melosas.

La última cosa que quedaba por hacer la hizo Scream en persona en cuanto sacó de la base de datos del cuartel una foto de Tracy Swoop antes del incendio.

No hizo falta confirmación genética alguna para tener la certeza final de que Afrodita y Tracy Swoop eran la misma persona. La amarga ironía provenía del hecho de que en verdad Swoop había sido una mujer muy hermosa, y el aspecto que ofrecía bajo aquellas ilusiones era una versión exagerada de sí misma, sin duda, pero en nada alejada de la realidad bajo mejores circunstancias.

Así pues era capaz de proyectar ilusiones, como ellos con sus dispositivos holográficos. La diferencia era que las de ella eran muchísimo más poderosas, tanto que poco faltó para haberles matado con su uso. Otra diferencia crucial residía en el hecho de que no usaba dispositivo alguno para provocarlas. Parecían ser inherentes a ella misma, y seguramente en ello radicaba el alcance de su poder. Tal vez incluso su cuerpo había desarrollado otros procedimientos instintivos acordes con ellas, del mismo modo que la química corporal es un factor indiscutible a la hora de evaluar la atracción o repulsión entre dos personas de distinto sexo.

De lo que no cabía duda era de que la habían conducido a la locura y las había usado para esconder la realidad de su propia deformidad. Tal vez por culpa de un uso prolongado, un experimento fallido o ambas cosas a la vez. No había manera de saberlo, e interrogarla suponía en ese momento una imposibilidad de base. Si siquiera sabían cómo podrían acercarse a ella sin sufrir su influjo letal, de modo que menos aún enfrentarla en un cara a cara.

Al fin tenían una pieza más del juego pero ni la más remota idea de cómo encajarla en el tablero. ¿Qué era lo que podían hacer, cómo efectuar el siguiente movimiento? La pelota estaba en su tejado, pero no sabían siquiera hacia dónde había que arrojarla.

Fue en ese momento cuando Razorclaw entró corriendo a hablar con él, aún más rápido que cuando fue a avisarle de que Hades estaba emitiendo un comunicado.

~---Tenemos un mensaje ~---dijo al fin, con el aliento entrecortado. Es para ti.

Scream siguió a Razorclaw, que le indicó que se pusiera unos auriculares que estaban sobre uno de los múltiples paneles de mandos del Aquerón.

~---John, aquí Swart. ¿Me oyes?

~---Te escucho, Jim ~---contestó al momento Scream.

~---La tenemos en el punto de mira, pero la niebla nos impide acercarnos más. Está sobre un edificio de los Túneles, y dice que quiere hablar, pero sólo contigo.

~---¿Quién?

~---La mujer que atacó a Sky. La de los objetos invisibles.

~---Mantened la posición, voy para allá. En cuanto llegue, replegaros.

~---Con el debido respeto, tu seguridad\dots

~---No puedo pasar por alto esta ocasión ~---se limitó a contestar Scream, al borde del desvanecimiento por puro cansancio. Segundos después ya estaba enfilando por el laberinto de pasillos que le llevaba hacia los Túneles.

No tuvo que ponerse el traje porque desde que llegó no había tenido ni un solo momento para plantearse siquiera la idea de quitárselo.

\parbreak
Perséfone estaba de pie en el tejado de un edificio de cuatro plantas, de ladrillos grises y paredes melancólicas. Se apoyaba en la pared con indolencia, como si esperara en la cola del cine a que llegara su acompañante. Por un momento podía parecer que era una mujer incluso normal, sencilla. Del montón. Lo que podría haber sido Tracy Swoop de no interponerse en su vida la furia ardiente de la naturaleza.

Colgado de un hombro llevaba el mismo bolso de trucos imposibles que había usado la vez anterior. Era imposible saber con certeza qué podía esconderse entre sus pliegues perversos y asimétricos.

La sombra llegó al lugar donde la estaban esperando, y antes miró hacia abajo. La Nube estaba regresando lentamente a su posición original y los disturbios habían regresado a las calles, aunque muy apaciguados. Irónicamente, dado que tenía que aterrizar a cuatro plantas de altura, iba a notar la niebla tan espesa como lo había sido a ras de suelo en días anteriores.

Se dejó ver vagamente, a unos pocos metros de distancia de donde Perséfone se encontraba, y ésta levantó la vista, como si hubiera quedado a una hora preconcebida y su acompañante llegara con retraso.

\emph{Aquí estoy, mujer. Habla o lo lamentarás.}

~---Tu voz suena sincera, sin duda. Aunque dudo del contenido de tus palabras, del mismo modo que dudo del contenido de las que me han dirigido los otros.

\emph{No hay otros, sólo tú y yo.}

~---No tienes que mantener esa pose de omnipotente conmigo. Sé cuál es tu juego porque aquel a quien sirvo juega un juego similar.

\emph{Hablemos con claridad, entonces. ¿Qué tienes que decirme, o acaso preparas alguna clase de trampa?}

~---Creo que si peleara contra ti la cosa acabaría en empate. No eres como los otros. En ti brilla una decisión que no he visto antes. Vengo a dialogar, sólo eso.

\emph{¿Te manda Hades?}

~---Sí y no. Él me ha dicho que te transmitiera un mensaje, pero lo que quiero contarte es muy distinto.

\emph{Habla por voz de tu amo, entonces, y luego por voz propia} ~---declaró Scream, acercándose hasta estar a poco más de un metro de distancia, sin quitarle el ojo de encima ni un segundo a Perséfone.

~---Quiere que sepas que admira vuestro valor y vuestra dedicación, pero eso no es suficiente. No basta con sembrar el miedo. Hay que sembrar el terror para que surjan los resultados esperados.

\emph{Tú le amas, ¿verdad?} ~---dijo Scream de repente.

~---No es un hecho sorprendente si es que conoces la mitología griega.

\emph{Él no te ama a ti.}

~---¿Por qué dices eso?

\emph{No ama a nadie. Está más allá de todo eso.}

~---Te equivocas. Es perfectamente capaz de amar.

\emph{Pero no a ti.}

Perséfone se echó atrás como si hubiera sido herida en mitad de un combate. Pero era uno que no se desarrollaba con ninguna clase de arma física.

~---Yo lo di todo por él. Todo. Qué sabrás tú de eso. Se nota que siempre has hecho las cosas a tu manera, de acuerdo con tus rígidos e inflexibles preceptos morales. Tanto le quería que le hice algo horrible.

\emph{Tú le volviste invisible, ¿verdad?}

~---Él me lo pidió. Dijo que era necesario para ocultar las cicatrices del pasado, para renacer. Le condené a la invisibilidad eterna, suficiente para volver loco a un hombre, pero no a él.

\emph{¿Fuiste tú quien le hizo también eso a Tracy Swoop?}

~---¿Cómo\dots\ cómo sabes eso?

\emph{Contesta, mujer.}

~---No, no directamente, pero si ayudé en el proceso. Las ilusiones ópticas y la invisibilidad son sólo dos caras de la misma moneda, pero eso apenas nadie lo comprende aún. Él se obsesionó con que lo hiciera, quería ayudarla a toda costa. Lo único que logró fue hundirla del todo en la locura. Pero contesta, ¿cómo sabes eso?

\emph{Ella está bajo nuestra custodia.}

~---¿Habéis encontrado a Afrodita?

\emph{Así es.}

~---¿Dónde estaba?

\emph{No es de tu incumbencia.}

~---Entonces\dots\ por eso es por lo que\dots

\emph{¿Qué estás mascullando?} ~---inquirió Scream, comprendiendo que algo pasaba que se le estaba escapando.

~---Tienes razón. No me ama. Sigue amándola a ella. Siempre la amará a ella.

\emph{¿Por qué concluyes ahora eso?}

~---Porque ha ido a rescatarla.

\emph{¿Qué has dicho?} ~---replicó Scream, pero los movimientos fueron más rápidos que las palabras. Se lanzó furioso hacia Perséfone, pero ésta se echó hacia atrás y logró sacar un objeto invisible de su amplio bolso. Scream logró agarrarlo justo antes de que tuviera tiempo de utilizarlo, y gracias al tacto comprendió que se trataba de una bomba de humo. Trataba de huir. Pero no le resultaría tan fácil como en la ocasión anterior.

Alejó la granada con la mano y con una presa fuerte como una tenaza agarró el bolso y tiró de él con extrema violencia, rompiendo el asa y dispersando su invisible contenido alrededor de ambos. Perséfone cayó al suelo de manera aparatosa, y ya estaba corriendo hacia uno de los objetos cual gacela, deduciendo su posición en base a los vacíos dejados por la grava en el suelo, cuando recibió un puñetazo de Scream que terminó de echarla hacia atrás del todo y por poco la hizo perder el sentido.

Scream se quedó de pie frente a ella, rabioso por lo que acababa de hacer. Estaba a punto de perder el control, de dejar que el instinto animal tomara las riendas de sus actos. Acumulaba tal cansancio que el cuerpo le pedía a gritos que dejase la mente en punto muerto y se limitara a dejar que los gestos hablaran con elocuencia. Pero logró detenerse, pues sabía que estaba siendo presa de una furia que no podía llevarle a ninguna parte recomendable.

Perséfone hizo un gesto con la palma abierta y rogó sin palabras que se detuviera, al tiempo que se tocaba en la cara. Una fea señal que no tardaría en convertirse en moratón la surcaba el lado izquierdo del rostro.

~---Lo siento ~---dijo sin moverse del sitio, bajando las manos poco a poco hasta que estuvieron abiertas y tocando el suelo, en posición vulnerable~---. Él me dijo que te distrajera, que tenía que entrar a por algo importante a vuestro cuartel. Pero no imaginé\dots\ no imaginé que era ella. Me ha ocultado la verdad, y a cambio, te ayudaré. El Cancerbero es un satélite de comunicaciones modificado. Era un proyecto en el que él mismo estaba trabajando. Ahora mismo está en órbita, y es con lo que logró emitir ayer su mensaje.

\emph{¿Cómo logró ponerlo en órbita sin despertar sospechas?}

Perséfone no contestó.

\emph{Entiendo. Lo hiciste pieza a pieza, supongo.}

~---Una por una, pero mi influencia no duraría tanto. Una vez arriba, una tecnología de ilusiones ópticas similar a la que alteró a Afrodita mantuvo el engaño de que era sólo chatarra girando alrededor de la Tierra.

\emph{Eres sólo una marioneta, Perséfone. De fuertes hilos, pero movida por un astuto titiritero.}

~---Él no me haría eso, jamás. Pero no puede tampoco traicionar sus propios sentimientos.

\emph{Eres más ciega de lo que pensaba. No amas a un hombre invisible, sino a uno que nunca existió.}

~---Tienes razón\dots

\emph{Abandónale. Paga por tus delitos, y luego únete a nosotros. Nunca es tarde.}

~---¿Quién es el ciego ahora? ~---dijo Perséfone extendiendo la mano, un poco alejada de la cadera, y agarrando algo que parecía flotar en el aire. Scream tardó un poco en entender qué estaba haciendo, pero al fin lo comprendió.

Estaba introduciendo la mano en un segundo bolso, completamente invisible.

No pudo ver la granada de humo, pero sí sentir sus efectos. Junto con la niebla ambiente bastó para darle a Perséfone la ventaja necesaria para escapar sin dejar rastro. Ella era, sin embargo, el menor de sus problemas. Puso rumbo rápido al Aquerón, pero se permitió diez segundos de respiro en los que miró el cielo.

El final del cuarto día acababa de llegar, y la Nube ya estaba comenzando su ascenso.
