\begin{prev}
    Ellen Gorgon ha caído, y con ella todo su reinado de podredumbre y corrupción. Pero eso no quiere decir en absoluto que los Caídos ya no sean necesarios, ni que la ciudad esté a salvo\dots
\end{prev}

\noindent{}El peligro había pasado, pero sabía que no podrían bajar la guardia. Otros llegarían para ocupar el lugar de los vencidos. Nunca descansaban. Y su misión, por siempre, sería detenerles y proteger la ciudad que era su hogar.\\

\noindent{}Poco a poco la ciudad fue volviéndose un lugar más seguro. Hubo nuevas elecciones, la producción industrial se redujo en pro de la salud de los propios ciudadanos. La Nube, aunque irreversible, disminuyó su masa por centímetro cúbico. Cayó mucha aguaceniza en aquella época, pero todo el mundo sabía que era el presagio de un futuro mejor.

No obstante aún había monstruos en las calles sombrías de Ernépolis~I.

Krexon era uno de ellos. Un cruel alienígena proveniente de un mundo lejano, un fugitivo espacial que había decidido instalarse en la que antaño había sido la capital del crimen, la más corrupta de las Ernépolis. Mucho había hibernado, muchas rutas clandestinas había tenido que atravesar, para descubrir que Ellen Gorgon había caído mientras estaba en el sueño sin sueños. No importaba. Tanto mejor para él. Había mucho donde robar, donde matar, donde amedrentar a aquellos blandos humanos.

Había oído hablar de aquella silueta. Que había surgido de la Nube. Que no podía morir. Que conocía tus mayores temores. Supersticiones de una raza inferior, pensó. Sin embargo, ahora que estaba frente a la realidad del mito, ensombrecido por las mal iluminadas calles de los bajos fondos, reconoció en él algo más que un simple obstáculo en el camino. Había estropeado su rutina de buscar un rehén para tener siempre una opción de emergencia si la policía le encontraba. Cada semana cogía a un transeúnte imprevisto y lo retenía hasta que moría de hambre. Para él resultaba más cómodo así; tardaba menos en conseguir otro que en buscar comida para el mismo. Además, no entendía aquella necesidad que los humanos tenían de alimento orgánico. Con la ceniza y la polución le resultaba más que suficiente.

Y allí estaba aquel famoso ser, interponiéndose entre él y la niña que había seleccionado, siguiéndola hasta que estuviera apartada de las multitudes.

~---Lárgate, hombrecillo ~---dijo a su inmóvil oponente en lo que sus siete ojos violáceos parpadeaban de fuera hacia dentro.

Durante varios segundos no se escuchó nada, y Krexon pensó que tal vez aquella criatura fuera también un alienígena y no entendiera el idioma de los humanos que en su caso tanto le había costado aprender.

\emph{Es mía} ~---se limitó a decir el ser del sombrero y la gabardina. La niña no se movió. Tenía la sensación de que estaba segura junto a aquel misterioso protector.

~---Eso lo veremos ~---una alarma saltó en el cerebro triesférico de Krexon. Se estaba arriesgando mucho por un rehén que podía sustituir por cualquier otro en cualquier otro momento, pero sabía que el orgullo le había dominado. Se concentró y adoptó la forma incorpórea que deseaba.

Avanzó hacia el ser de la gabardina. Éste no se movió, ni siquiera cuando Krexon le atravesó como si no existiera. Nada podía hacer para detenerle. Nada.

Cuando pasó y volvió a su estado tangible, Krexon vio que la niña ya no estaba. No era posible. No se había movido del sitio en ningún momento.

\emph{Has escogido mi ciudad para esconderte. Y eso no me gusta.}

Krexon se giró. Allí estaba aquel fanfarrón, en la penumbra. Corrió hacia él y le golpeó, pero su mano de dos dedos oponibles le atravesó. Podía hacerse intangible también, dedujo.

\emph{Podemos jugar al mismo juego.}

Era de su raza, o al menos originario de su mundo. Era la más inmediata conclusión, la única explicación coherente a que pudiera volverse intangible como él, ya que la evolución había obligado a que en su planeta, debido a la incesante lluvia de meteoritos, todos los seres tuvieran la capacidad de separar sus átomos una distancia imperceptible a simple vista pero suficiente para que nada sólido pudiera tocarles.

Sintió miedo. No podía ser. Él era el único superviviente. Él era único, especial. Krexon, un ser frío y cruel que se vendía al mejor postor.

Cerró sus siete ojos y se concentró en permanecer intangible. Así estuvo mucho tiempo, presa del nerviosismo, hasta que no pudo por más tiempo estar aislado del estado sólido de la materia. Miró a su alrededor, y le vio. No le veía moverse, pero se acercaba cada vez más deprisa. Le estaba rodeando. Como si las distancias no fueran nada para él.

\emph{Estoy aquí} ~---oyó Krexon a su espalda. Se dio la vuelta deprisa y lanzó un golpe, pero sólo acertó al aire. Al momento sufrió un tremendo golpe en su prominente cabeza. Fue lo último que sintió antes de desmayarse sobre el mugroso suelo del mundo humano que tanto detestaba.

La policía llegó poco después, alertada por los padres de la niña. Los testigos decían haber visto a un ser de muchos ojos merodear por los alrededores.

~---Krexon ~---dijo James Sky cuando escuchó a los testigos. Avanzó con sus hombres y lo encontraron atado en un callejón, a dos manzanas de allí. Habría que fabricar una celda especial, pensó Sky, pero no sería problema. Dado que su planeta ya no existía la Tierra se encargaría de que cumpliera su condena.

~---Los testigos hablan de alguien más, Jefe Sky ~---dijo uno de los policías mientras tomaba notas.

~---¿Quién?

~---Adivine ~---comentó el policía con tono de resignación.

~---De acuerdo. Quiero un informe mañana por la mañana.

~---¿No le perseguimos?

Como solía hacer incluso antes de que Brian Wolf dejara el cargo y le nombraran a él Jefe de Policía de Ernépolis~I, James Sky se permitió una furtiva sonrisa.

~---Ya estará muy lejos ~---respondió mientras se metía en el deslizador patrulla.

\begin{next}
    ¡Comienza una nueva saga! Con Ernépolis bajo un clima de guerra espacial, ¡se empezará a tejer una intriga que tendrá a los Caídos en el ojo del huracán!
\end{next}

\endinput
