Nuevas responsabilidades. Nuevos aliados. La naciente posibilidad de hacer las cosas de otra manera. Pero también el nuevo temor a que las esperanzas se desvanezcan sin remedio en el intento.

\fancyparbreak
Alma Espejo no volvió a intentar atrapar a Los Caídos desde aquel encuentro en un tejado solitario que tanto le impresionó. Siempre había pensado que tuvieron algo que ver con el fin de la era de los grandes héroes, con la desaparición repentina y posterior muerte de su padre adoptivo.

Nunca imaginó que estuviera tan cerca de la verdad pero al mismo tiempo de manera tan equivocada.

Eso, por otro lado, no es algo que debiera haberle sorprendido demasiado. Después de que el Detective Starr Miles se hiciera cargo de él, aunque siempre estuvo muy agradecido por todo lo que significó para su educación e integración social, no dudó en marcharse de casa en cuanto tuvo la mayoría de edad. La vida de Starr Miles era compleja y en nada conveniente para tener que cuidar a un crío como él. Además, se metió en bastantes líos durante todos aquellos años, casi tantos como en los que su propio padre adoptivo se había metido a su vez.

Hubo también otros roces, otros problemas de carácter más personal. Enfrentamientos surgidos entre un adolescente de pasado problemático y un curtido investigador con muchas tinieblas interiores que enterrar y ocultar. Muchas cosas dichas que no deberían haberse pronunciado en voz alta jamás.

Pero con todo nunca olvidaría las lecciones que Starr Miles le había dado y lo muchísimo que solía hablarle de los héroes a los que apenas había conocido. Héroes como Reflector, por ejemplo.

A los enemigos de los héroes, sin embargo, sí los conoció bien. Ya no sólo al Jefe Wolf, de cuyas garras le salvó Starr Miles. Ellen Gorgon también estuvo en su camino, hace ya tantos años que apenas podía recordar las circunstancias concretas. El motivo siempre era el mismo: ser un mestizo. Medio humano, medio alien. Aspecto de hombre, mente de hombre, imprevisibles consecuencias genéticas con el paso de los años.

Ponerse a brillar al cumplir los dieciocho, sin ir más lejos.

El motivo más crucial por el que decidió marcharse de la ciudad sin previo aviso. Su padre hubiera tratado de convencerle para que se quedara, le hubiera dicho que estaba afrontando el futuro con cobardía, quizás. Eso ya no importaba. Aquel no era su lugar, y lo sabía bien.

Dio tumbos de colonia en colonia, de mundo en mundo, allá donde pudiera pasar desapercibido más fácilmente. Que nadie se fijara en él, que no le consideran humano. Que pensaran que su brillo permanente era la única manifestación extraña de su poder, que no cayeran en la cuenta de que era un diamante en bruto a la espera de ser explotado. Haciendo de linterna andante en extrañas minas más allá de los Ocho Mundos Coloniales.

Trató de averiguar cuál era la raza a la que pertenecía su padre biológico. Humanoide, sin género de dudas. No especialmente divergente de la humana en términos genéticos, lo que tampoco era acotar mucho teniendo en cuenta que el genoma de una mosca no difiere tanto del humano. Pero su madre, Anne Strud, no había dejado cartas reveladoras ni holos escondidos. Y por ello, posiblemente jamás conocería su origen verdadero.

Hasta que llegó el torneo. Y con él un resquemor interior, la necesidad de retomar el camino donde había sido dejado. Volvió a informarse sobre la ciudad, volvió a tener interés en aquella megalópolis depravada y viciada por la ceniza y la corrupción. Descubrió que su padre adoptivo había muerto. Y escuchó hablar de ese ser, esa sombra deslizante que patrullaba las calles de Ernépolis como un frío y tibio avatar de la propia ciudad.

No tardó en unir hechos. La aparición de esa criatura, la muerte de Starr Miles. Desaparecido desde entonces, pero él lo sabía bien, muerto sin género de dudas. Usurpado. Sustituido. Pensaba que aquella sombra le había matado, había deformado su identidad, borrado todo rastro de su existencia.

Así era, en efecto. No como él había pensado, pero así era.

Scream no tardó en poner al día a Alma Espejo sobre lo sucedido desde entonces, del mismo modo que el propio Alma Espejo le contó todas estas cosas a Scream. A nadie se le escapaba que aquel asunto había pasado a ser personal para el líder de Los Caídos. Álex Miles, hijo adoptivo de Starr Miles. Junto al Juez Supremo Nitram, la única persona que podía hablarle largo y tendido sobre su desaparecido maestro.

Mucho fue lo que escuchó de él, pero no tanto lo que le sacó de dudas sobre su críptica personalidad. Ni siquiera a su hijo le habló demasiado de su pasado, y mucho menos de su pasado como héroe, villano o lo que quiera que hubiera sido. Demasiados hechos sombríos para confesárselos a un adolescente con antecedentes y tendencia a meterse en líos, pensó Scream, a lo que se preguntó si Nitram sabría más al respecto de ello. Pero nuevamente, un callejón sin salida.

Al menos sí que confirmó que Nitram y Miles fueron, si no amigos, al menos sí sujetos que se profesaron respeto mutuo en algún momento. Aparte de ser responsable de agilizar la adopción de Álex Strud compartía con Miles un cierto sentido arcaico de la justicia, de la necesidad de cambiar las cosas de alguna manera importante y contundente. Por desgracia su periodo como embajador le hizo enfocar esa idea en direcciones que Scream ya conocía demasiado bien. Pero Álex nunca olvidó la clase de hombre que fue, por lo que trató de ponerle de su lado aprovechándose del conocimiento privilegiado de su manera de ser, como ya comprobaron Grove y Shockman de primera mano.

Scream le habló de muchos otros enemigos, de tiempos lejanos y no tan lejanos. Le habló de Armor, y le contó los detalles al respecto del primer encuentro que tuvieron con él. Le facilitó también los pormenores acerca del periodo en que los cazarrecompensas irrumpieron en la ciudad.

Hubo otros enemigos de los que prefirió no hablarle, sin embargo. Alma Espejo era joven e impetuoso, y no quería poner frente a sus ojos amenazas frente a las cuales temía aventurar cuál podía ser el resultado. A su debido tiempo, daría detalles más concretos.

Scream era consciente de que ya antes de ser un héroe Álex Miles era un chico en el que muchas personas tenían fijada su atención. No sólo aquellos que pretendían utilizarle, también aquellos que le salvaron. El propio Miles solía darle gran cantidad de consejos y lecciones, piezas de sabiduría que a Scream le recordaron las que él mismo solía decirle en su entrenamiento inicial en Los Caídos, sólo que alteradas de forma y contexto, pero no de fondo. Starr Miles sabía bien que su \emph{hijo} era distinto, y algún día cambiaría de manera imprevisible. ¿Trataba de anticiparse a ese cambio? ¿Fue él quien creó a Alma Espejo, en cierto modo, como había creado a Los Caídos? ¿O la organización nació porque no quiso poner en semejante peligro a su propio hijo?

Había otra posibilidad a tener en cuenta, una que le recordó a Scream lo que Shockman le dijo no hacía mucho tiempo. Tal vez Starr Miles viera en Álex el mismo algo que Shockman mencionó, y por lo que no quiso seguir adelante con él. De algo estaba seguro Scream: su antiguo maestro sabía que Álex iba a dejarle. También sabía cómo encontrarle. Era detective, por favor. Si no le buscó podía deberse a su orgullo, a un sentimiento de no involucrarse en la vida de su hijo si él no lo quería así, o a una compleja e indisoluble mezcla de ambas cosas, como suele pasar con casi todas las emociones humanas.

Lo cierto era que eso ya no tenía marcha atrás. Álex Miles ya era un héroe, y había que seguir hacia delante desde ese punto de partida. Es por ello que Los Caídos en general, y Scream en particular, trataron de convertirse no en aliados de Alma Espejo, sino en mentores, consejeros, guías, apoyo moral y real en el terrible camino que, de manera inconsciente, como ellos cuando decidieron ser defensores de la ciudad, había elegido.

\emph{Nunca te dejes sorprender, nunca te dejes engañar} ~---empezó a decirle Scream en lo que él y el resto de su escuadrón, situado por detrás de ambos, patrullaban la ciudad en una incursión rutinaria. Mientras ellos se movían por las calles, como criaturas hechas con la etérea sustancia de las sombras, Alma Espejo volaba alto, brillante, desde donde ellos, como Satanás, habían sido desterrados para no regresar jamás.

~---Lo capto. Pero es difícil que me sorprendan cuando mi luz no les deja apenas lugar donde esconderse ~---se limitó a objetar con sencillez.

Scream sabía que tenía razón. Y también comprendía que muchos de los consejos que podían darle no eran los que esperaba escuchar. No operaba del mismo modo que ellos. No era una criatura de las tinieblas, nacida para sembrar el terror entre los criminales y la duda entre los honrados ciudadanos. Él\dots

~---Es un héroe como los de antes, más grande quizá ~---proclamó Razorclaw, en asamblea con los demás en el hemiciclo del Aquerón.

~---Puede ser ~---le apoyó Scream~---, pero temo por su seguridad. Cuando le veo sólo vislumbro a la persona que yo era antes de esto, con todas sus virtudes\dots\ pero también con todos sus terribles defectos. Sufrirá decepciones que tal vez le quiebren para el resto de su existencia.

~---Pero no está solo, como lo estuvimos nosotros ~---argumentó Saw, manifestando la opinión general~---. Podemos ayudarle, apoyarle. Ser los cimientos que le permitan construirse a sí mismo como protector de Ernépolis.

~---Y tal vez\dots\ ~---añadió Swind, esperanzado~--- tal vez cuando ese momento llegue, nosotros ya no seamos necesarios.

Grove no podía apenas creer lo que estaba escuchando. Una parte de él sentía absoluto júbilo ante lo que Matthew Swind acababa de sugerir. No más mentiras a Ellie mientras estaba con ella y preguntaba más de lo recomendable, no más excusas furtivas cuando Roy le propusiera un plan de última hora. Pero al mismo tiempo, una incertidumbre nublaba su corazón. Dejar de ser necesarios\dots\ no creía que eso pudiera suceder jamás, y estaba seguro de que Scream opinaba de manera similar, ¿o también, y nunca mejor dicho, se había dejado cegar por la luz de la esperanza?

Todo aquel debate, de repente, se vio interrumpido por una voz al fondo, entre las sombras. Una voz que muchos ignoraban y más aún despreciaban.

~---Patéticos soñadores idealistas ~---se limitó a comentar Shockman dando de comer a su rata.

Nadie dijo nada, aunque muchos estuvieron tentados. Simplemente ignoraron sus palabras. Pero todos no. Cuando acabó la reunión, el líder de Los Caídos le miró con semblante preocupado, del mismo modo que miró también marcharse a Grove, a su manera silenciosa, sin protestar pero también en desacuerdo. Le conocía lo bastante bien como para saberlo.

Por primera vez en mucho tiempo el capitán John Scream, antiguo héroe conocido como Reflector, se planteó si no se estaría haciendo ya viejo para asumir todo aquel ajetreo.

\parbreak
Fueron muchos los criminales a los que Alma Espejo encerró en aquellos días con la ayuda de Los Caídos. El indiscutible poder de uno, unido a la experiencia y contactos de los otros, estrechó el cerco sobre gran cantidad de organizaciones y bandas de todas partes de la ciudad. También cayeron numerosos criminales de mayor categoría, tanto moviéndose por cuenta ajena como contratados por las bandas anteriores. Tal vez no fueron logros tan visualmente impactantes como la derrota de Armor ~---o lo que quiera que fuese exactamente~--- o tan simbólicamente importantes como haber apaciguado al Juez Supremo, pero sin duda contribuyeron de manera notable a restaurar una sensación de seguridad en las calles de Ernépolis, a pesar de que aún no se había restablecido del todo el suministro eléctrico.

Entre Los Caídos hubo propuestas osadas, también. Los hubo que empezaron a pensar que John Scream no tardaría en proclamar oficialmente que dejarían de patrullar callejones para limitarse a servir de apoyo logístico a Alma Espejo, y más aún, hubo quienes pensaron que no sería mala idea dejarse cazar simbólicamente por él, aunque sólo fuera una pantomima, para darle de ese modo el respaldo definitivo que necesitaba.

También había quienes no veían muy claro hacia dónde les estaba llevando aquello, entre los que se encontraban a partes iguales tanto los recién llegados como los novatos que fueron reclutados sin haber sido héroes en el pasado. Y en esos colectivos estaban, obviamente, Shockman y Grove.

Entre el antiguo villano y el voluntarioso joven se había formado cuanto menos una extraña y peculiar relación de mutua confianza, algo que por supuesto Shockman jamás admitiría aunque le fuera la vida en ello. Pero lo cierto era que su desacuerdo común sobre delegar tanta responsabilidad en el nuevo recién llegado, aunque motivada por distintas razones ~---la desconfianza innata en un caso y la sensación de tener aún demasiado que hacer por otro~--- hacía que, al tener algo compartido, algo en lo que estar de acuerdo, eso facilitara multitud de otras conversaciones e intercambio de opiniones. No muy largas, eso sí, pues Shockman no era precisamente hombre de muchas palabras. Pero conversaciones al fin y al cabo.

Hubo un día, sin embargo, que marcó una inflexión en todo aquel asunto. Una prueba de fuego sin precedentes que marcaría un antes y un después en muchos de los miembros de Los Caídos.

Comenzó con una incidencia, cuanto menos, rutinaria y de sencilla indagación. Al parecer circulaba entre los rateros, matones y violadores de poca monta de los Túneles el rumor de que no era conveniente acercarse por un edificio abandonado de los alrededores, pues muchos de los que lo habían hecho no habían sido vueltos a ver de nuevo. Fue el escuadrón de Grove el que dio el aviso sobre ello, y al que le encargaron echar un vistazo para ver qué era lo que pasaba.

\emph{Entendido, cuartel} ~---contestó Grove mientras le indicaba a los suyos que le siguieran hasta las inmediaciones del edificio. Fue entonces cuando recibió una segunda transmisión de alguien que a menudo solía salir también de patrullaje con ellos.

\emph{Voy a haceros de mamaíta, boy scout} ~---contestó Shockman, aún a cierta distancia, pero siguiendo su paso de manera clara y sencilla gracias a que conocía bien las rutas que solían tomar, así como el edificio que mencionaban.

El inmueble en cuestión tenía cuatro plantas y, seguramente, muchas más subterráneas, como sucedía con muchos de los que estaban en los Túneles, construidos por empresarios sin escrúpulos que sólo pensaban en alquilar el mayor número posible de centímetros cuadrados, por decadentes, insalubres y mugrientos que pudieran resultar. La fachada había sufrido mucho por las constantes lluvias de aguaceniza y presentaba un aspecto negro con gran cantidad de ribetes multicolor formados por las capas de pintura deslavadas y mezcladas con la contaminación ambiental proveniente de la Nube.

Sam Grove aterrizó de un salto frente a su misma entrada, saliente hacia el exterior por medio de una antesala cubierta, como si invitara a ser atravesada, si alguien se atrevía a hacerlo. Las ventanas estaban tapiadas o claveteadas, y se fijó en que era una de las pocas casas de los alrededores que no tenía pintadas recientes. En verdad, pensó, parecía una moderna mansión embrujada.

\emph{¿Alguna entrada alternativa, chicos?}

\emph{No, Sam. Ni siquiera el tejado.}

Genial, pensó Grove. No sólo estoy frente a la boca del lobo, de hecho es la única por la que meterse en su interior.

\emph{Iré primero. Seguid el protocolo ensayado para un solo acceso.}

\emph{De acuerdo} ~---contestaron casi al unísono.

El mencionado protocolo consistía en que entrara primero el jefe de escuadrón y, varios minutos después, uno por uno fueran filtrándose los demás miembros, siempre bajo un riguroso camuflaje. El asunto ya no pintaba bien para ellos, pues estaba claro que en un interior estrecho peleaban en desventaja táctica, y un viejo edificio que había sido inicialmente pensado para albergar la mayor cantidad posible de apartamentos, incluso si eso violaba unas cuantas normas no legales pero sí morales sobre viviendas dignas, no era el prototipo de espacio interior diáfano que hubieran elegido para explorar.

Grove no tardó en darse cuenta de que si se dividían recorrerían el edificio con mucha mayor eficacia, pero por otro lado si capturaban a alguno el engaño de la organización se vería seriamente comprometido. Fue por eso por lo que tomó una grave decisión.

\emph{Permaneced en los alrededores. Entraré a explorar y os daré la señal en caso necesario.}

Sus compañeros de escuadrón así hicieron, y Grove siguió adentrándose por el tremendamente oscuro interior del edificio. Aun así sus pasos eran seguidos muy de cerca, concretamente por un aliado temerario que no se veía en la necesidad de obedecer orden alguna de ningún escuadrón.

Grove llegó a las escaleras y, con ello, a la necesidad de tomar una decisión. Subir o bajar. No tardó en concluir que lo malo, lo terrible y lo que uno no espera tener que conocer siempre suele encontrarse bajo tierra. Como ellos mismos, que llenaban de pesadillas las noches de muchos supersticiosos criminales.

Descendió las escaleras lentamente, peldaño a peldaño, y tuvo una mala sensación aparejada con aquel en apariencia tranquilo descenso. Una sensación que no se disipó por mucho que trató de convencerse de que lo más probable era que todo aquello se tratara de una falsa alarma.

Shockman estaba varios metros por detrás de él, siguiendo su rastro con paciencia, aunque una vez dentro del edificio era tan sencillo como seguir las huellas que iba dejando al retirar la espesa capa de ceniza con su avance, huellas que iba respetando una por una para que pareciera que sólo un hombre, o algo que fingía ser tal, había hollado aquel lugar tenebroso y abandonado.

Hasta tres plantas atravesaron de esa manera, y al llegar a la inferior, cuando Shockman todavía estaba a la altura de la mitad, pudo escuchar un murmullo ahogado provenir de la garganta de Grove.

\emph{No puede ser. No puedo creerlo.}

Shockman no comprendió lo que quería decir con eso, pero si él lo había escuchado tan claramente, cualquier otro que estuviera oculto por la zona también lo habría hecho, sin duda. Bajó más deprisa los escalones y todo lo que pudo ver fue que, al otro lado de la puerta del hueco de las escaleras, estaba Grove, apenas entrando en una sala tremendamente grande, limpia y cuidada.

Una sala que tenía una moderna factura de estilo helénico.

Trató de avanzar hacia la posición de Grove, pero de repente una regia compuerta de mármol descendió del techo y se interpuso entre ambos.

~---Tú no tienes permitido el acceso a mi reino, soldado ~---escuchó decir a una voz femenina desde el otro lado.

Grove se quedó solo, allí, en aquella sala de insondable belleza, con pilares dóricos por todas partes por las que mirara, piscinas teseladas con toda clase de nenúfares y otras hermosas flores y la vista lejana de una impresionante cadena de montañas coronadas de blancas y sedosas nubes.

Grove sabía que no era real. Lo sabía tan bien como se comprende cuándo una droga está invadiendo el organismo de una persona. Pero aunque Grove no creyera que fuera real su cerebro sí lo creía, y de ese modo nublaba su mente, le impedía razonar, como en un sueño del que no se puede escapar salvo en el momento de despertarse.

Al fondo de la sala había una mujer. Desnuda, sentada al borde de la piscina. Hermosísima, de piel suave y sedosa. Un objeto de ardiente deseo.

Grove sabía quién era esa mujer. Había leído los informes en el Aquerón. No la había visto en persona, pero ya estaba en la organización cuando todo sucedió. Recordaba su nombre, también.

\emph{Tracy Swoop} ~---dijo con voz débil, como estuviera en el vacío y las palabras no encontraran medio en el que propagarse.

~---¿Quién es esa mujer? ~---preguntó ella, fingiendo mirar a uno y otro lado~---. Aquí sólo estamos tú y yo. Y yo soy Afrodita.

\emph{No, tú\dots\ tú eres\dots\ no\dots\ Ellie\dots} ~---luchó por decir Grove, pero una fascinación incontrolable hacía presa en su interior y comenzó a caminar hacia ella.

Desde el otro lado de la imposible sala, Shockman sólo podía escuchar palabras sueltas de lo que se mencionaba. Aquella maldita compuerta, pensó, estaba en su camino, pero siempre podía mandar a alguien en su lugar. Así dejó que la rata saliera de su bolsillo y buscara por las paredes cochambrosas algún acceso que pudiera utilizar. Constantemente parecía estar despistada y dirigirse hacia la compuerta, pero Shockman la redirigía con una orden y seguía con la labor encomendada.

Mientras tanto Grove ya estaba casi a la altura de Afrodita, incapaz de reaccionar. Sus dedos trataban desesperadamente de acercase al artefacto generador de hologramas, pues por algún motivo sabía que era importante que lo hiciera.

Afrodita se levantó y caminó hacia él a su vez. A un observador externo le hubiera parecido que aquella sombra siniestra y esa mujer de belleza indescriptible pertenecían a universos completamente imposibles de colisionar.

Con mucha lentitud, de manera muy seductora, metió la mano en el bolsillo de la gabardina y cogió el artefacto que tan desesperadamente Grove deseaba utilizar.

~---No busques esto, joven guerrero. Es veneno para tu alma. No niegues el paraíso que en este momento te estoy ofreciendo.

No la mires, trataba de decirse Grove una y otra vez. No mires ese rostro. Ese rostro no es el de verdad. Has visto imágenes del rostro real. Tuviste que apartar la mirada al hacerlo.

Pero era inútil. Sus labios se fueron acercando lentamente a los de Afrodita, hasta que estuvieron a punto de tocarse. Sin embargo, justo en el momento en el que eso iba a suceder, algo insólito ocurrió.

Tracy Swoop, conocida como Afrodita por Los Caídos, notó cómo una rata empezaba a subir por su pierna.

Se apartó asustada, cogida por sorpresa, y se apartó de Grove, aunque éste no se liberó de su maléfico influjo. La rata la mordió, y ella soltó un grito de agonía que alarmó a Grove al instante.

\emph{¡Déjala!} ~---gritó de repente, y trató de apartar a la rata de ella. Pero el animal, intuyendo el peligro inminente, salió huyendo antes de que sufriera daño alguno.

Shockman seguía sin saber qué era lo que estaba pasando al otro lado, aunque tenía claro que Grove luchaba por resistirse a quien estaba tratando de doblegar su voluntad. De repente notó que trataban de contactar con él por el comunicador. Era Scream.

~---¡Shockman! ~---dijo el líder de Los Caídos, claramente preocupado~---. El escuadrón de Grove está entrando para ayudarle. ¿Sabes algo de él?

\emph{Está en apuros, y no creo que puedan hacer nada que no haya tratado de hacer yo} ~---se limitó a comentar Shockman de manera escueta y sintetizada.

~---Él también va hacia allí.

~---¿Él? ~---preguntó Shockman. Pero su respuesta llegó por sí misma cuando notó un enorme estruendo proveniente del otro lado de la compuerta. Grove fue, sin embargo, el que pudo apreciar con todo detalle la espectacular entrada de Alma Espejo, más aún teniendo en cuenta que, bajo su punto de vista alterado, estaba presenciando el descenso de una criatura de luz en un entorno que resultaba ser poco menos que celestial.

Una terrible alarma inconsciente se asentó en la mente aturdida de Grove. Si ella lograba controlarle, dominarle con sus engaños, entonces estaban en gravísimos problemas. Pero para sorpresa suya, Afrodita parecía estar tan sorprendida como él mismo. Se limitó a ver descender a aquella figura de leyenda hasta estar a su altura, y nada más hacerlo trató de acercarse a él, pero no con soltura, sino de manera casi precipitada, impetuosa.

~---Mi señor\dots

Alma Espejo no la dejó ni hablar. Había leído sobre ella, y sabía lo que era capaz de hacer. Por eso se limitó a lanzarla una descarga de luz que, aunque apenas hacía daño, la lanzó contra el suelo con notable violencia. Al mismo tiempo que eso sucedía la ilusión de que estaban en un templo grecorromano se fue desvaneciendo y fluctuando, superpuesta a la realidad de estar en una sucia, mugrienta y tenebrosa planta abandonada de un edificio de apartamentos con gran parte de los tabiques derrumbados o desgastados por el paso de los años y las inclemencias del cruel clima.

La compuerta se volvió etérea ante la mirada de un solo ojo de Shockman, y fue cuando éste comprendió que nunca había estado ahí en realidad, pero su cabeza jamás lo había puesto en duda, ni cuando la rata tuerta trataba de pasar al otro lado cruzando, sencillamente, un umbral que ella no percibía como bloqueado.

Afrodita se levantó con lentitud y su aspecto fluctuó entre la realidad y la ficción, entre la belleza y la más horrenda de las visiones. Fue así como abandonó todo intento de mantener la ilusión sobre el entorno y reservó sus energías en seguir mostrando aquella forma de delicioso ensueño, sólo para tratar de engañar a una mente: la suya propia. Aún desde el suelo, cubriéndose con pudor, habló. Alma Espejo la dejó hacer, sabedor de que ya sólo podía hacerse daño a sí misma.

~---Mi señor, perdóname, por favor. Te haré caso en lo que me digas. Eres un dios, superior a mí en todos los sentidos. Yo sólo sé llevar la falsedad a los hombres. Tú eres, sin embargo, portador de la luz en las más terribles tinieblas.

Shockman se acercó a Afrodita y, sin mediar palabra, la asestó un puñetazo directo en pleno rostro. Alma Espejo comenzó a brillar con plena intensidad y, de una descarga mucho mayor que la que había empleado con ella, lanzó a Shockman contra la pared más cercana. Justo en ese momento, además, llegó el resto de miembros del escuadrón que comandaba Grove, sorprendidos por lo que se estaba desarrollando ante sus atónitas miradas.

~---¿No crees que su mente ya ha sufrido suficiente? ~---dijo agarrando a Afrodita en brazos, inconsciente como se había quedado ante tan directo y nada contenido ataque.

Shockman se levantó, furioso. Por su cabeza pasaron un millar de opciones que en el pasado no hubiera dudado en llevar a cabo pero que en el presente prefirió tratar de olvidar, no por otra cosa más que porque Grove estaba allí presente, aún aturdido pero testigo de todo lo sucedido. Como si la reacción de su compañero le hubiera despertado de su letargo, trató de poner paz entre ellos dos con sus palabras.

~---Escucha ~---dijo quitándose el modulador y dirigiéndose a Alma Espejo~---, mi compañero ha sido brutal con sus acciones, pero ella es uno de los peores enemigos a los que nos hemos enfrentado nunca, y la última vez apenas logramos contenerla.

~---Eso no volverá a pasar ~---proclamó Alma Espejo sin el menor atisbo de duda~---. Tenéis mi palabra.

Después de aquello se limitó a salir volando con ella en brazos y desaparecer por el mismo agujero del techo que había empleado para entrar.

Los compañeros de Grove se acercaron a su director para saber de primera mano qué era exactamente lo que había pasado. Más o menos al mismo tiempo Shockman recibió una comunicación entrante de Scream.

~---¿Está bien Sam? ~---fue lo primero que acertó a preguntar.

\emph{Sí.}

~---¿Qué es lo que ha pasado?

\emph{Mejor que te lo cuente él mismo} ~---dijo justo antes de cortar la transmisión.

~---Espera Shockman, no\dots

No dejó a Scream ni acabar la frase. No sólo porque no tenía ganas de hablar con nadie y porque se sentía herido en su orgullo.

Además de todo eso, prefirió no dar explicaciones porque estaba prácticamente seguro de que no creerían su versión de los hechos por mucho que insistiera en ella.

\endinput
