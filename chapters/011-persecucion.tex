\begin{prev}
    Por fin se conoce quién está tras la servoarmadura: ¡nadie! Armor no es un ser humano, sino producto del virus que infectó los sistemas de esa máquina de precisión letal. Tras usar al Coronel Straxus como huésped para obtener energía, Armor huye no sin anticipar que tiene grandes planes que no tardarán en ponerse en marcha\dots
\end{prev}

\noindent{}El enemigo llevaba ventaja. Su plan, fuera el que fuese, marchaba a la perfección sin salirse en lo más mínimo del esquema que tenía planeado. La derrota había sido cruel. Había jugado con ellos desde el principio.

Pero una guerra no se decide hasta que no tiene lugar la última batalla\dots\\

\noindent{}Felicity Hound estaba nuevamente asomada al balcón de su suite del hotel Andrómeda, mirando al horizonte de velados destellos rojizos, recortados sobre un lienzo celeste de grises enfrentados, como si hubieran sido aplicados con acuarelas a capas desiguales. Aquel cenit era muy distinto del cielo azul del que había sido su hogar tanto tiempo, la colonia de Scorpon. Costaba creer que en un lugar tan ennegrecido como aquel estuvieran más cerca de la libertad que en su mundo natal, poseedor de una atmósfera virgen e incorrupta.

Del mismo modo que costaba creer que la llama de la venganza estuviera empezando a apagarse en su interior.

De repente escuchó un ruido a su espalda, sutil, sencillo, escasamente perceptible. Achacable a una agradable brisa de viento que se hubiera colado por la ventana.

Si es que en Ernépolis hubiera existido tal cosa como una brisa agradable que poder disfrutar asomado en un elevado balcón.

Extendió la mano y dejó que la ceniza se acumulara. Así era como sentía de repente sus ideales. Polvo inerte.

---Él fue quien detuvo a mi madre, ¿sabes? Martin Straxus.

No hubo respuesta, si es que en algún momento alguien estuvo ahí para contestar. Aun así, siguió hablando.

---Al principio todo lo que sentía en mi interior fue odio. Odio puro, irracional. No me importaba el destino de mi mundo, no me da miedo admitirlo. Lo que me enfurecía era que por medio de la fuerza y la amenaza de las armas se podía conseguir cualquier cosa. Que de poco importaba la fuerza de las palabras.

\rquoti{}Luego ese odio fue dosificado. No dejé de pensar en la venganza, pero la racionalicé, la convertí en una excusa, en una causa para mis actos. En mi defensa he de decir que cambié con el tiempo, que realmente me comprometí con mis compatriotas. Si no, nunca hubiera movilizado a tanta gente, jamás me hubieran seguido. El SIL no nació sólo sobre las brasas de un fuego inextinguible.

\rquoti{}Pero sé que en parte me seguían por el odio. No sólo por lo que decía, sino por cómo lo hacía. Más que hablarlo, lo escupía. La diplomacia era un disfraz, hasta para mí misma.

De nuevo silencio. Pero esta vez sí fue interrumpido.

\emph{¿Y ahora qué?}

---¿Qué quieres decir?

\emph{¿Qué ha aprendido de todo esto?}

---Posiblemente nada. Pero eso tampoco importa, puesto que ya no puedo cometer los mismos errores de nuevo. Cada venganza sólo se alcanza una vez en la vida. Pero no has venido aquí a que te hable de mí. ¿Qué es lo que quieres que te cuente?

De nuevo, un silencio. Pasaron varios segundos hasta que el intruso habló.

\emph{El virus.}

---Quieres saber de dónde salió, ¿verdad?

No había necesidad de contestar a esa pregunta.

---Nosotros no lo creamos, si es eso lo que te estás preguntando. Nuestros hombres lograron identificarlo en la red tiempo atrás y consideraron que podía ser un arma poderosa, de una clase muy distinta a la que suelen usar nuestros enemigos.

\emph{Estábais jugando con fuego.}

---Sólo en parte lo sabíamos. No éramos capaces ni siquiera de aislarlo, por eso cuando vimos la ocasión de lanzarlo para defender nuestros intereses no lo dudamos ni un momento. Si no podíamos obtener la armadura para destruirla, entonces decidimos que no sería de nadie.

\emph{¿Acaso no notasteis que era inteligente, que aprendía por sí mismo?}

---¿Conoces el juego de la vida de Conway? Es un claro ejemplo de que la complejidad puede surgir de donde menos se espera. Sobre una cuadrícula blanca coloreamos una serie de cuadros en negro. Sólo hay dos reglas: la primera es que en el siguiente turno coloreamos un cuadro de negro si tiene exactamente tres cuadros vecinos de ese color, incluyendo diagonales. La segunda es que todo cuadro negro que no tenga dos o tres cuadros negros vecinos es borrado automáticamente.

\rquoti{}Si piensas en los cuadros negros como en células y en las dos reglas como la manera que tienen de reproducirse y sobrevivir, descubrirás que probando distintas configuraciones iniciales muy sencillas pueden empezar a generarse comportamientos muy complicados. Pueden organizarse en grupos grandes como si fueran órganos, formar círculos que se agrandan como en una explosión, o incluso desplazarse por el tablero. Este juego es sólo eso, un juego, pero es un ejemplo perfecto de que pueden desarrollarse patrones inesperados a partir de las reglas más sencillas.

\rquoti{}Eso es lo que nos sucedió con el virus. Sólo planteamos su posibilidad de propagación, su capacidad de expandirse en un sistema y destruirlo por dentro, y no tuvimos en cuenta que pudiera estar \emph{evolucionando}, convirtiéndose en otra cosa que éramos incapaces de predecir.

\emph{¿Hay alguna copia?}

---La destruimos en cuanto empezaron los rumores. No estábamos seguros de que hubiera sido eso lo que hubiera pasado, pero no quisimos arriesgarnos. Pudimos intentar estudiarlo para comprender la mente que se había gestado, pero no tuvimos el valor de hacerlo, o quizás fuimos sensatos por primera vez en todo este asunto. Ahora quién sabe lo que querrá esa criatura.

\emph{Aprecio su sinceridad, señorita Hound. Supongo que no esperará más que lo mismo por mi parte, a pesar de lo duras que puedan resultar mis palabras.}

---No querría menos.

\emph{En su esfuerzo por asestar un golpe fatal a sus oponentes, acabó convirtiéndose en lo mismo que ellos. Ambos cometieron errores que poseen ya difícil solución. Pero lo más irónico de todo este asunto es que lograron por fin entenderse, colaborar.}

---¿Colaborar?

\emph{En efecto. En concreto, el ejército terrestre creó el cuerpo del monstruo, y el SIL dio el soplo que le otorgó alma y vida.}\\

\noindent{}---¿Decidme, qué es el miedo para una máquina? ---se planteó Scream en voz alta, en el módulo principal del Aquerón. Había convocado una reunión extraordinaria con todos los directores de escuadrón y los puestos más destacados de la organización.

---Tal vez carezca de miedos ---apuntó Saw, que desde la muerte de Gorgon había pasado a ser el ayudante del nuevo presidente electo Scatter.

---Pero eso es improbable, en realidad ---continuó Scream---. No al menos si está vivo. Porque todo lo que está vivo y tiene la intención de perpetuarse conoce el miedo. El miedo es un mecanismo de defensa más que útil. Los cobardes llegan a viejos. Los valientes mueren jóvenes, a menos que la buena o mala suerte juegue en contra de unos y otros.

---De modo que nuestro enemigo tiene que tener debilidades que ser explotadas ---añadió Razorclaw.

---Sin duda, o de lo contrario no estaría llevando a cabo su plan en el más absoluto secreto. Sabemos que necesita recargarse habitualmente para subsistir. El problema es que ignoramos cómo exactamente funciona este mecanismo de asimilación energética. Debe de poseer sin duda reservas externas, o si no nunca hubiera podido atrapar al Coronel sin albergar un huésped humano. Pero sin duda su maniobra fue muy arriesgada. Puede que nos lleve ventaja, pero Armor es muy consciente de que puede acabársele en cualquier momento.

\rquoti{}Además está aquello que dijo antes de esfumarse, que yo podía aún serle útil. Conocía mi nombre, por lo que había estado investigándome, o tal vez era información que ya poseía en su base de datos.

---¿Qué debemos hacer entonces?

---Esto ---dijo mostrando un enorme mapa de la ciudad, donde el sector oeste estaba ampliado y una zona de tamaño considerable lo rodeada con un círculo--- es un trazado aproximado del radio en el que creemos que puede haber estado operando. Nos hemos basado en la distancia que recorrí cavando bajo tierra, pero por desgracia nuestros datos son imprecisos, pues no llegué a ver el final del túnel. Tampoco sabemos cuánta energía puede llevarse por delante la improvisada prospección que está llevando a cabo. Quiero que todos salvo los escuadrones de emergencia salgáis y patrulléis a lo largo de esta zona, y aviséis de cualquier posible túnel, pasadizo, presencia de agujeros, lo que sea. No entréis en combate directo, si os atrapa eso aumentará su autonomía. Tenemos que intentar aislarle, por difícil que eso suene tal y como lo estoy diciendo. Razorclaw os asignará un cuadrante a cada grupo. Suerte.

Una vez todos se hubieron marchado Scream se quedó a solas con Sky, que no había dicho una sola palabra en toda la reunión.

---Estás preocupado, ¿no es así? Por la ciudad, por tus hombres.

---Así es, John. Tú piensas en los que tienes más cerca, pero yo no puedo hablar con la misma franqueza a los míos. Para no despertar sospechas he tenido que mandar a varios de ellos a ese barrio de mala muerte, sin poder decirles que no sería necesario, que ya habían hecho bastante y otros harían su trabajo en esa ocasión.

---Los militares rastrean cada palmo, pero sus métodos siguen siendo poco ortodoxos. Tus hombres siempre supondrán una ayuda, gracias a que sabes qué clase de consejos deben recibir.

---¿Pero qué son, John? ¿Policías o una célula de baja categoría de Los Caídos?

---Comprendo tus dudas, amigo. Pero piensa que las cosas eran peores cuando la policía estaba en su mayor parte corrompida.

---Eran peores, pero también estaban más claras. En todo caso, John ---dijo Sky moviendo las manos en horizontal, como queriendo zanjar la discusión--- ¿qué crees que está buscando Armor en concreto?

---No lo sé, pero está claro que es un intento a la desesperada, pues no podrá cavar túneles por siempre. Algún día tendrá que salir, y teniendo en cuenta que ha pateado las espinillas de mucha gente y que anda justo de energía no tardarán en localizarle.

Como un presagio a las palabras de Scream, de repente sonaron las alarmas de emergencia. Era el momento de la movilización masiva.

---Parece que le han encontrado ---sentenció Sky.

---Han localizado al enemigo ---dijo Grove, uno de los miembros más novatos, acercándose corriendo hacia ellos.

---¿En qué escuadrón le ha encontrado, Sam? ---preguntó Scream en lo que se ponía en marcha al tiempo que Sky.

---Ninguno, señor.

---¿Cómo le han localizado entonces?

---Porque ha asaltado uno de nuestros cuarteles ---explicó Sam, jadeando debido a la carrera.

---¿Uno de nuestros cuarteles? ---repitió Sky sorprendido---. ¿Qué es lo que estará buscando? ¿Qué crees, John? ¿John?

Pero John estaba callado, como si quisiera desconcertar con el silencio como cuando llevaba el traje de Los Caídos. Armor estaba un paso por delante de ellos, pero al fin comprendía qué era lo que estaba buscando. No podía estar del todo seguro de para qué, pero una terrible sospecha estaba empezando a fraguarse en su interior.

---Debemos irnos cuanto antes ---dijo dejando atrás a Grove, corriendo más deprisa. Incluso Sky tuvo problemas para seguirle.

---¡Pero el chico aún no nos ha dicho dónde debemos ir!

---No hace falta, James. No hace falta. Olvídate de trajes, no tendremos que ir de incógnito, llama al deslizador patrulla. Allá donde vamos, nadie se extrañará de vernos.

---¿A qué se debe eso?

---Se debe a que tú tratas de impedir un robo, y yo me dirijo al lugar donde trabajo.\\

\noindent{}Nada más llegar a las inmediaciones de Gorgon Enterprises y ver el revuelo que se había formado alrededor del edificio Scream comprendió que, como de costumbre, llegaban un paso tarde en el camino trazado por su enemigo. Distintos efectivos policiales y militares acordonaban el lugar, y los corresponsales de prensa se agolpaban para obtener el mejor sitio a la hora de retransmitir la noticia.

---Informe de situación ---preguntó Sky a los suyos, atravesando junto con Scream el cordón policial.

---Las alarmas han saltado desde dentro, y el objetivo está en el hangar. Los militares están a punto de asaltar el recinto. Creemos que\dots

El policía no llegó a terminar la frase. La explosión que reventó la pared este del edificio cogió completamente desprevenido a todo el mundo, incluso al equipo táctico del ejército que se preparaba para entrar. Ya estaban dispuestos a colarse por el agujero cuando Scream trató de detenerles.

---¡No! ---gritó, intentando que le hicieran caso. Pero para ellos no era más que un civil sin experiencia de combate alguna, y su advertencia llegó demasiado tarde.

La nave espacial salió a toda velocidad del agujero, y si bien no llegó a arrollar a ninguno de los soldados, la potencia del despegue les obligó en su mayor parte a lanzarse al suelo, aunque algunos de ellos resultaron derribados por la lluvia de escombros y mobiliario destrozado que la nave había ido arrastrando a su paso.

---El prototipo ---replicó Sky sorprendido---. ¿Cómo lo sabía?

---No lo sabía ---fue la escueta contestación de Scream en lo que entraba corriendo al hangar, atravesando habitaciones seccionadas como si fueran un mendrugo de pan duro---. De hecho, no ha hecho la elección más adecuada a sus necesidades.

---Pero esa nave parece más que adecuada ---argumentó Sky en lo que le seguía a través del caos recién montado---. Dijiste que poseía un emisor de energía al que acoplar armas ondulatorias, y ahora mismo, él es el arma perfecta.

---Sin duda, pero él no busca potencia de fuego.

---¿Qué busca, entonces?

---Velocidad. Y aquí tenemos algo que encaja mucho mejor en esos requisitos ---contestó Scream apretando el botón de su mesa de planos, gracias al que empezó a elevarse la compuerta de un hangar que ambos conocían muy bien.

Debido al boquete de reciente creación, retazos de luz iluminaban la normalmente oscurecida sección donde se albergaba la nave que Scream estaba despertando de su prolongado letargo, y no tardaron en vislumbrarla al completo. Tal vez no sería la más limpia, ni la más elegante, ni tampoco la más estilizada de las naves, pero Scream siempre supo sacar el máximo provecho de hasta el último de los tornillos que la formaban.

La \emph{Trigger}, la nave con la que había surcado los confines del espacio.

---Supongo que llegó el momento que tanto tiempo he temido ---comentó en lo que se ponía el traje espacial a toda velocidad y subía manualmente la carlinga. Nada más sentarse a los mandos un torrente de recuerdos agradables y desagradables, ninguno deseado, le invadió por dentro. El olor a quemado cuando sufrió el accidente en el mundo donde obtuvo los poderes de Reflector. El tacto a nuevo que desprendía cuando la compró con sus ahorros de años de duro trabajo en naves alquiladas.

El aroma del perfume de Aryn.

---Ten cuidado, John. Quién sabe lo que esa cosa puede ser capaz de hacer con una nave espacial entre sus manos.

---Lo tendré, descuida. Ahora, ¡aléjate! ---gritó al tiempo que encendía los motores y éstos estrangulaban todo sonido ambiente. En cuanto Sky estuvo a la distancia adecuada activó las turbinas y se puso en marcha. No había efectuado una puesta a punto, apenas tendría combustible para llegar al otro lado del sistema, pero tendría que apañárselas como pudiera. Si sus sospechas eran fundadas, había mucho en juego en ese momento.

La nave despegó con máxima aceleración y no tardó en atravesar el agujero como si fuera una flecha. Por fortuna no se cruzó con nadie en su camino, o lo hubiera tenido muy difícil para esquivarle.

Segundos le bastaron a Scream para elevar a la \emph{Trigger} hasta la altura de la Nube, donde las explosiones eran más patentes y menos un asunto lejano y olvidado. Se preparó para encontrarse con lo peor mientras atravesaba sus capas de polución opaca y se fiaba exclusivamente de lo que los radares estuvieran dictando.

Cuando atravesó el otro lado, al carecer de puntos de referencia, la sensación de velocidad fue menor, pero si seguía manteniendo los motores estables no debería tardar en volver a tener su objetivo en pantalla. Múltiples escaramuzas se desarrollaban por todas partes, menos de las que parecía desde tierra pues al estar en un entorno tridimensional muchas de las explosiones que se reflejaban en la Nube, como ocurría con las estrellas de las constelaciones, en realidad estaban sucediendo a niveles y distancias muy alejadas entre sí.

Aun así tuvo que maniobrar con extrema precaución para no pasar cerca de ninguno de los focos de batalla, so pena de chocar con alguna nave o ser abatido sin la menor conmiseración por tratarse de un objeto extraño y desconocido. El combustible se agotaba a toda velocidad, y si bien no le inquietaba quedarse varado en el espacio, sabía que si Armor llegaba a su cercano destino habría cosas peores de las que preocuparse.

Aceleró todo lo que pudo, llevando al límite las prestaciones de la nave, hasta que por fin su enemigo volvió a estar en el radar. No tardó, de hecho, en verlo con sus propios ojos.

Con el morro apuntando hacia su nave.

Scream trató de desviarse todo lo que pudo, pero en vez de un misil lo que impactó en su ala derecha fue una potente descarga eléctrica, que sin duda Armor estaba canalizando a través del emisor de energía de la nave. Estaba tocado, pero aun así prosiguió la carrera.

La nave de Armor giró de nuevo y aceleró todo lo que pudo, tomando un tono rojizo, similar al del propio Armor, que indicaba que estaba al límite de su aguante. Scream era consciente de que sus reservas de energía iban muy justas, o de lo contrario hubiera intentado derribarle.

Aquella era la oportunidad que había estado esperando, tal vez la única que se presentaría. Disparó los misiles, y uno de ellos impactó en la parte derecha del juego lateral de turbinas. La nave se desestabilizó y comenzó a volar en trayectoria curvilínea.

Justo a tiempo, pensó mientras la Luna empezaba a aparecer en su campo de visión.

Si bien la nave de Armor iba camino de estrellarse contra la superficie lunar, Scream no podía permitirse el lujo de simplemente mirar. La zona en la que estaba a punto de caer poseía trazas de atmósfera, por lo que podría moverse a mayor velocidad, y el laboratorio de investigación del ejército en la Luna, por desgracia, quedaba cerca del potencial punto de impacto. Fue por eso que no tuvo más elección que lanzarse como un torpedo hacia la superficie lunar y rezar para que lograra aterrizar lo más cerca de su enemigo, y también de la manera menos aparatosa posible. Pero era más fácil decirlo que hacerlo, y como resultado fue incapaz de desplegar el tren de aterrizaje a tiempo, siendo tan exigentes las circunstancias de su vuelo.

Fue así como ambas naves se estrellaron a unos escasos treinta metros de diferencia la una de la otra, provocando un estruendo atronador, como dos poderosos rayos de tormenta que incidieran contra una tierra seca y quebradiza. Scream elevó la carlinga y se puso la mascarilla, pues el aire de la zona era aún demasiado enrarecido para ser del todo respirable.

La otra nave yacía partida en dos, pero a través de la humareda Scream no tardó en vislumbrar aquel visor blanco que empezaba a detestar con todas sus fuerzas.

Armor se incorporó, no sin cierto esfuerzo, y bajó de la cabina de un salto que hundió el suelo varios centímetros. Parecía que algunos sistemas estaban dañados, en especial su brazo derecho, pero sin siquiera prestar la menor atención a la otra nave se giró con la intención de dirigirse al laboratorio.

Se detuvo cuando una ráfaga de ametralladora proveniente de la \emph{Trigger} le disparó por la espalda.

---¡No llegarás, Armor! ---gritó Scream desde la carlinga---. ¡Los dos sabemos que apenas te queda energía!

Como una respuesta siniestra y silenciosa la armadura se quedó quieta, empezó a tambalearse, y se abrió en canal para expulsar un esqueleto, chupado hasta el tuétano, que perteneció en su momento al organismo del Coronel Martin Straxus. Si bien Scream fue testigo de aquel tremendo horror, no pudo disparar al interior expuesto al carecer de ángulo para acertar.

Además, había comprendido por fin cómo podía explotar las debilidades de su enemigo.

Aun sabiendo que era un esfuerzo inútil, ametralló sin piedad a la armadura mientras ésta se plegaba sobre sí misma de nuevo. Una vez compacto de nuevo, Armor se acercó hacia la \emph{Trigger}.

<<Pensé que podría necesitarte para pilotar alguna de las naves que habías diseñado, pero la casualidad ha querido que tenga que darte un uso muy distinto>> sentenció acercándose hacia los restos de la \emph{Trigger}.

Sin moverse del sitio Scream comenzó a apretar botones, preparándose para contener a aquella mole el tiempo que fuera necesario. Al menos, hasta que hubiera calculado bien la jugada. Siguió disparando sin tregua, pero las balas sólo frenaban a Armor, y empezaban a escasear. Por otro lado, la servoarmadura estaba cada vez más magullada y llena de pequeños impactos de metralla.

<<No me detendrás, humano. Una vez llegue al laboratorio, por medio del contacto directo infectaré toda la red interna del ejército y me desharé de este poderoso pero limitado cuerpo robótico. Controlaré todos sus sistemas remotos, y con ellos de mi parte, lentamente, me expandiré por el Cosmos>>.

---No si yo pudo impedirlo, abrelatas ---dijo Scream tratando de enfurecerle, pero era inútil. Armor no respondía a los insultos.

El misil que le impactó de lleno fue más efectivo para lograr esa finalidad.

Si hubiera habido un ser vivo dentro de Armor, tal vez hubiera muerto aplastado al abollarse el torso de la armadura. Aunque ni de eso estaba Scream seguro. Un impacto tan brutal y a tan poca distancia debería haberle reventado por dentro, aunque a buen seguro le habría debilitado.

En todo caso era el único misil que quedaba en la nave, y, una vez pasó de largo las ametralladoras, Armor empezó a escalar por el morro de la nave. Scream comprendió que la suerte estaba echada.

Esperó hasta tenerle casi encima, y pudo ver cómo la armadura empezaba a abrirse de nuevo, lista para fagocitarle.

<<Tu fuerza eléctrica será una gran energía para mí, John Scream>>.

---Tengo una idea mejor ---dijo apretando el botón de la cápsula eyectora de la nave---. Prueba a comerte esta otra.

La cabina salió despedida con Scream dentro, una maniobra pensada en principio para sobrevivir en gravedad cero. En aquel ambiente, sin embargo, y con la carlinga abierta, sabía que el impacto contra el suelo, en el mejor de los casos, iba a doler lo suyo.

Pero desde luego, era mejor que permanecer junto a la nave cuando ésta estaba a punto de autodestruirse.

La explosión, de hecho, le lanzó aún más lejos de lo que la trayectoria parabólica ya estaba haciendo, y reventó la \emph{Trigger} en montones de minúsculos pedazos, lo que siempre había estado buscando para ocultar pruebas en caso necesario.

El módulo aterrizó y la fuerza de la caída le sacó de la cabina, sufriendo en consecuencia un doble impacto contra el suelo. Se levantó dolorido, pero con la mascarilla aún en su sitio y, en términos generales, de una pieza.

Cuando la explosión se aclaró y se acercó al cráter que se había formado, lo primero que vio fue el yelmo de la servoarmadura, inerte y sin brillo, en el borde del mismo. Luego fue cuando vio el resto de los pedazos, espolvoreados como confeti de una fiesta.

Se dejó caer, extenuado, hasta que vio un montón de soldados a lo lejos, provenientes del cercano laboratorio de investigación, que se aproximaron al lugar del doble impacto. Al ver los restos de huesos humanos y los pedazos de armadura que estaban por todos lados, le preguntaron si era el único superviviente.

---Eso espero ---contestó Scream sin dudarlo un solo momento.

\begin{next}
    Las consecuencias del ataque de Armor no se harán esperar, y serán cruciales para el destino de algunos. ¡No te lo pierdas!
\end{next}

\endinput
