El momento de la verdad había llegado. De demostrar de qué pasta estaban hechos, de darlo todo en aquella pelea, tan alejados de su hogar.

Pero sobre todo de demostrar que las cosas podían ser resueltas a su manera, la de los viejos héroes del pasado.

\fancyparbreak
John Scream salió de la nave que les había llevado hasta el espaciopuerto de la colonia Hidra y la abandonó sin la menor intención de mirar atrás. Sabía bien que podía ocurrir que no regresaran en ella porque su enemigo ordenara bloquearla o incluso destruirla, pero lo cierto era que no les importaba en lo más mínimo. De hecho, Scream personalmente opinaba que Hades no haría algo semejante. Su orgullo le impediría comportarse así, además de que siempre podía ser interpretado como un incidente diplomático que le pusiera en entredicho ya teniendo el estatus de nación.

Con respecto a sus tripulantes, eso ya era un asunto distinto. Desde el mismo momento en que Scream pisó el suelo de aquella colonia vestido con el traje de Los Caídos ya no era un ciudadano de Ernépolis, sino un paria sin nombre real ni identidad. Su muerte implicaría nulas acciones diplomáticas y no pocas alegrías por gran parte de sus conciudadanos.

Scream echó un vistazo a su alrededor, al modesto pero aun así majestuoso espaciopuerto, lleno de toda clase de arcos clásicos por los que entraban y salían las naves con gran precisión por parte de los astrocontroladores aéreos. La temperatura ambiente era tibia tirando a fría, pero aun así muy lejana de infiernos congelados como el planeta Sesturm. Gran parte de la esfera terrestre estaba ocupada por el cuerpo conocido como Caronte, mientras que en ese momento Plutón quedaba al otro lado, pero sin duda su inmensidad sería algo digno de espectáculo y que probablemente él mismo podría presenciar, dado que nada más llegar le escoltaron a un sofisticado deslizador tubular que le llevaría a la fortaleza principal de la colonia, antigua sede del gobierno dictatorial y de carácter macroempresarial.

Scream apenas habló en todo el trayecto. Los dos guías que se encargaron de recogerle, por otro lado, tampoco tenían la templanza suficiente para hacerlo. Temían a aquella sombra a la que estaban escoltando como si fuera la mismísima Muerte en persona, a pesar de haber sido aleccionados sobre lo que realmente se escondía tras aquella máscara de tenebrosidad.

El paisaje que recorrieron durante el trayecto fue poco menos que impresionante. Al ser aquella colonia un mundo rocoso sin apenas vida ni interés natural alguno, el esfuerzo por dotarlo de grandes y complejas estructuras arquitectónicas, a muchos niveles y alturas, provocaba una sensación de ramificación completamente inexistente en la ciudad de Ernépolis. Parecía, además, que desde la llegada del nuevo gobernante se había introducido un elemento humanístico y helénico en aquellos diseños, para que contrastaran con la estructura industrial que poseía la mayor parte de los edificios, ya estuvieran diseñados para ser factorías o viviendas, y en muchas ocasiones ambas cosas a la vez. Al mismo tiempo podía verse que aquella ciudad estaba viva, sin duda poblada con gente con anhelos, deseos y ambiciones que poco o nada tenían que ver con las ansias megalómanas de la persona que en ese momento tenía la llave de su destino.

Pero la sorpresa surgió para Scream cuando, habiendo calculado que llevaban recorridos unos cincuenta kilómetros de distancia, alrededor de la mitad de la envergadura de la colonia, seguía sin ver fortaleza alguna en el horizonte. Y fue entonces cuando el recorrido tubular torció hacia abajo y comprendió por qué no lograba vislumbrarla. Principalmente porque la estaba viendo en todo momento, dado que la fortaleza estaba alrededor de todo el satélite, horadándolo incluso de lado a lado, y su gran núcleo estaba en la cara contraria a aquella en la que había aterrizado la nave de Gorgon Enterprises.

Si ver el exterior y su fractal estructura entre moderna y neoclásica era algo poco menos que impresionante, recorrer el interior semirocoso del satélite era una auténtica experiencia sin igual para alguien como Scream, que era más piloto de grandes distancias que de pequeños laberintos. La cantidad de quiebros, giros y curvas extrañas que tomaron era tan numerosa que le hubiera resultado poco menos que imposible comunicar con precisión su posición a sus ocultos compañeros.

Si es que ese hubiera sido el plan, claro.

Cuando el deslizador se detuvo tuvieron ante sí el núcleo central de todo aquel gigantesco complejo inexpugnable, un castillo eléctrico que se elevaba cientos de metros sobre los edificios de su alrededor, rodeado de gran cantidad de arcos voltaicos esporádicos y ennegrecido en muchas de sus incontables paredes metálicas que solían unirse por los lados y con las de diferentes alturas en ángulos más originales e insólitos que los sencillos noventa grados. Scream entró por delante de sus guías y llegado un punto éstos fueron relevados por guardias que vigilaban las puertas. Sobre el conjunto destacaba el planeta Plutón, inmenso y casi místico en un entorno como aquel, a pesar de su grisáceo aspecto, aun con todo lleno de matices. Su presencia, en contra de lo que pudiera parecer, alivió al recién llegado, que se sintió como si una segunda Nube se impusiera sobre su cabeza.

Scream avanzó por una escalinata de docenas de anchos peldaños de desconcertante longitud, y cuando entró en la antesala comprobó que la decoración neoclásica había sido llevada a límites más que insospechados por parte de su nuevo poblador. Poseía amplias ventanas, sin duda para captar la escasísima luz solar que lograba llegar aun a pesar de la débil pero estable atmósfera, aunque en su mayor parte estaba iluminado de manera artificial, dejando muchos huecos en sombra, dotando al lugar de un cierto halo misterioso y fascinante en términos de espacios interiores, poco menos que un museo de luz encerrado en un mundo lejano y desconocido. Una escalera imperial se destacó al final de la antesala, y una vez Scream la recorrió, siempre escoltado, llegó a un balcón superior con una altísima puerta de doble batiente. Una vez la abrieron para él, al otro lado, en un estilo arquitectónico muy similar al anterior pero aún más recargado, encontró a Hades sentado en una especie de trono, tan al fondo de la inmensamente vacía y amplia sala que Scream al principio sólo le distinguió como un punto en mitad del vacío más pomposo y absoluto.

Los guardias se quedaron al pie de la puerta, firmes y sin pestañear. Scream comprobó que iban armados con sendos lanzarrayos, tal vez de alcance mejorado, teniendo en cuenta las dimensiones del lugar. Caminó hasta acercarse a diez metros de su anfitrión, como si alguna clase de ley le impidiera aproximarse un solo palmo más.

~---Bienvenido ~---proclamó Hades, su voz reverberando con un potente eco en la inmensidad de la estancia~---. Supuse que no tardarías en presentarte aquí. ¿Deseas tomar algo, o quizás comer? Ha debido ser un viaje largo.

\emph{No te haces una idea de cuánto, Hades} ~---contestó Scream, solemne, sin dejar de mirar aquellos intensos ojos rojos. La escena en ese momento parecía sacada de la más desbordante mitología.

~---¿Qué ocurrirá ahora, John Scream? Dime, ¿cuál es tu intención al venir hasta aquí a verme en persona? Sé que eres sin duda alguien temperamental, pero hasta tú mismo comprenderás lo arriesgado de tal maniobra.

\emph{No más arriesgado que jugar con el destino de toda una colonia para urdir una trama consistente en hundir económica y moralmente a una ciudad al completo.}

~---Tuve que efectuar un golpe de estado, no lo negaré. Pero los resultados no han podido ser más beneficiosos para este lugar. Ha pasado de la más terrible autocracia a realizar exportaciones a Plutón, Caronte y Nix. El incidente en Ernépolis ha sido ya aclarado en términos políticos y los culpables castigados.

\emph{Castigados} ~---repitió con furia Scream~---. \emph{Castigados a permanecer aquí, libres de las leyes internacionales, supongo.}

~---Veo que seguís con esa estrategia inadecuada de tratar de apuntalar unas ruinas que se están cayendo a trozos ~---prosiguió Hades ignorando el comentario~---. Ernépolis es el paradigma mismo de la miseria moral humana. Ni siquiera puede abordar sus proyectos de futuro sin que entre en juego la corrupción en todos sus sentidos.

\emph{Nosotros ya acabamos con la corrupción de Ellen Gorgon. Sólo hace falta tiempo y paciencia para proseguir con la tarea.}

~---Tiempo es lo que más falta en esta era que nos ha tocado vivir ~---protestó Hades, golpeando el brazo de su trono improvisado, enteramente forjado en metal~---. Es el momento de atacar ahora que se tiene la oportunidad.

\emph{Dime que no eres tan inconsciente como para llevar a este mundo a una guerra. No} ~---se contestó a sí mismo Scream~---, \emph{no todavía, ni en términos legales. Seguirás con subterfugios, tal vez no atentados, tal vez otros planes de aún peor envergadura.}

~---Eso es aún demasiado pronto para revelarlo. Lo que sí es cierto es que vosotros sois un estorbo para mis planes y su pronta consecución. Prueba de ella ha sido el extremo celo con que he tenido que llevar a cabo esta operación. Por eso, intuyo que has venido hasta aquí a ofrecerme una solución rápida. Tus hombres estarán sin duda desperdigados por la colonia, moviéndose a sus anchas. Muy bien. Tus soldados contra los míos, entonces. Yo tengo la ventaja de jugar en casa y tú tienes el sigilo de tu parte. Nos quedaremos aquí, donde nos mantendrán informados a ambos de los respectivos progresos de nuestros bandos.

Scream miró fijamente a Hades. Oficialmente empezaban los juegos, y todo se decidiría movimiento a movimiento. Ya no era una cuestión de grandes estrategias a corto plazo, sino de pequeñas historias que se desarrollarían a lo largo y ancho de la colonia, incluyendo la suya propia.

\parbreak
Al mismo tiempo que John Scream estaba frente al responsable de que Los Caídos hubieran tenido que llegar tan lejos para luchar contra sus peores enemigos, sus mejores hombres avanzaban camuflados a lo largo de las intrincadas y metálicas calles de la colonia, donde el toque de clasicismo no era tan sutil y se habían creado elementos helénicos con materiales modernos, los únicos que se podían emplear a nivel de calle y de manera masiva por todas partes. Si bien al principio se cruzaron con transeúntes y ciudadanos que les obligaban a moverse por caminos y accesos elevados secundarios, no tardaron en notar cómo cada vez las calles estaban cada vez más vacías hasta que se quedaron totalmente despobladas.

Swart iba por delante, y fue el primero en romper el tabú y comentarlo en voz alta.

\emph{Puede que nos hayan encontrado ya} ~---fue todo lo que acertó a decir por medio del comunicador.

\emph{No} ~---contestó Sky, muy serio~---. \emph{La cosa marcha bien. Hades ha dado el toque de queda, sigue sin saber dónde estamos. El plan sigue su trayecto. Hades ha atacado la infraestructura de nuestro mundo. Atacaremos el corazón del suyo, pero a nuestra manera. Sin explosiones, sin muertes. Como sólo nosotros sabemos hacer.}

Ante esa declaración de intenciones los demás se llenaron de coraje y prosiguieron su marcha, silenciosos, cada uno atento a su papel en aquel momento de la batalla. Sabedores de que era mucho lo que se esperaba de cada uno de ellos.

El primer obstáculo llegó cuando empezaron a encontrarse en su camino con los primeros soldados armados. Iban vestidos igual que los primeros sicarios que atacaron Ernépolis~I en la época del descenso de la Nube, lo que trajo no pocos desagradables recuerdos a todos y cada uno de ellos. Durante un tiempo lograron avanzar sin ser vistos ni detectados, pero por un lado no estaban en su terreno natural, y por otro trataban de pasar desapercibidos ante los ojos de una colonia al completo, por lo que no tardaron en encontrarse en una vía muerta, subidos a lo alto de una vieja factoría de tres plantas de altura y paredes retorcidas, oxidadas y enrejadas.

\emph{Llegó el momento de separarnos} ~---dijo Swart deshabilitando el modo de camuflaje.

\emph{Déjame quedarme contigo} ~---insistió Swind~---. \emph{Sigo pensando que tu misión es de las más arriesgadas, si no la que más.}

\emph{No es sólo por la misión, amigo} ~---contestó Swart mirando a las calles metálicas, llenándose de soldados~---. \emph{Esto es personal. Tengo que hacerlo solo, por algo que me ocurrió hace mucho tiempo, en una encrucijada urbana y existencial muy similar.}

Los demás le miraron y no dijeron nada. Cada uno de ellos, de hecho, estaba a punto de enfrentarse a sus propios fantasmas, y sabían bien por lo que estaba pasando, de modo que esperaron a que su compañero se dejara mostrar para conseguirles la distracción que estaban buscando.

Desde el punto de vista de los soldados de Hades la entrada de Swart no pudo ser más espectacular. Cayó desde lo alto como un demonio justiciero, y al aterrizar apenas produjo ruido alguno. Cuando les miró fijamente, sin apartar la vista, sólo pudieron vislumbrar unos ojos blancos, cerrados y temibles que contrastaban con una poderosa silueta negra.

A medida que dio los primeros pasos, los lanzarrayos empezaron a levantarse apuntando en su dirección. Siguió caminando, y los primeros disparos trataron de acertarle, pero todos le atravesaron de lado a lado, como si fuera poco menos que indestructible.

Acto seguido sacó su propia arma, la mostró a sus adversarios y la arrojó al suelo.

\emph{No más disparos por hoy} ~---dijo mostrando los brazos. En uno de ellos llevaba el mismo guantelete que tantos años atrás fue testigo mudo e imperturbable de su caída, brillando bajo aquel cielo sin luz y plagado de estrellas lejanas y misteriosas~---. \emph{Sólo puños} ~---terminó avanzando de un salto hacia sus enemigos.

A tan corta distancia, muchos de los temerosos soldados trataron de apuntar en balde a su enemigo antes de que él mismo les abatiera de un puñetazo, desconcertándoles con aquella mezcla de fuerza bruta y sofisticados trucos de ilusionismo. Cada golpe con aquel guantelete era devastador, y derribaba a varios oponentes de un solo ataque. Aun así Swart no tardó en estar completamente rodeado, hasta tal punto que algunos de los soldados eran incapaces de participar en la contienda porque sus propios compañeros cortaban su línea de visión.

Justo lo que estaba esperando.

Los soldados ya no usaban las armas, pues al ser tantos corrían el riesgo de abatirse unos a otros, por lo que optaron por el cuerpo a cuerpo. Sólo era un hombre, pensaban, y ellos le superaban en una proporción demasiado escalofriante como para poder calcularse con absoluta precisión.

Sin embargo no estaban ante un hombre cualquiera. Estaban ante alguien a quien el destino había otorgado una segunda oportunidad de demostrar su utilidad, alguien que se había entrenado muy duro toda su vida para especializarse en la lucha contra hordas y que de hecho se había sometido a un durísimo perfeccionamiento en días recientes al asalto. Alguien que pensaba que no podría volver a usar ese guantelete nunca jamás, y que aun con todo estaba sufriendo las consecuencias de ello.

~---Hemos logrado mejorarlo, pero será por poco tiempo, y las descargas que emite te producirán un intenso dolor ~---le dijeron en el laboratorio.

~---No será mayor que el que la culpa me ha producido todos estos años ~---agregó Swart sin tener que pensárselo dos veces antes de contestar.

Y allí estaba, repartiendo tortas a diestro y siniestro, tal vez derribando a más gente de la que había abatido jamás mientras fue un héroe, pero aun así a un precio muy alto. Concretamente, no tardó en conocerlo cuando el circuito del guantelete colapsó en su misma mano y le produjo un dolor ya tan insoportable que apenas pudo mantenerse en pie del mismo.

Pero aun así, no gritó. Era una regla sagrada, y no la incumpliría jamás. No gritar, ni mostrarse débil.

Los soldados se apartaron, desconcertados. Aún quedaba una docena de ellos en pie, y por los suelos, magullados, doloridos o inconscientes, había tantos que era imposible contarlos incluso a simple vista. Más que el temor, en aquel momento les invadió una profunda admiración por ese soldado.

Swart se puso en contacto con Scream.

\emph{Te he fallado, John. No he sido capaz de detenerles.}

\emph{Al contrario, Jim} ~---dijo Scream en alto, de modo que Hades también lo pudiera escuchar~---. \emph{No esperaba que acabaras con toda la ciudad, ni mucho menos. Pero tampoco quise ponerte una meta, para que dieras lo máximo posible. Elabora la maniobra de intimidación.}

Hades se quedó intrigado por aquellas palabras, pero concluyó que no tardaría en conocer de mano de sus propios soldados el resultado final de la batalla. Así fue en cuanto un panel surgió del suelo y en él apareció la figura de uno de sus subordinados.

~---¿Sargento? ~---preguntó sin ningún signo de impaciencia o inquietud.

~---Señor\dots\ no hemos podido detenerle. Era como una apisonadora humana, y después de eso él empezó a\dots\ a\dots\ ~---no pudo continuar.

~---Olvídense de ese sujeto. Era sólo una distracción. Otros van con él, avisen a mis dos generales.

~---A la orden, señor.

Hades cortó la comunicación y miró fijamente a Scream.

~---De modo que estáis recurriendo a armas nuevas.

\emph{Según como se mire} ~---dijo Scream con calma.

~---Bien, esperaremos a ver cómo se desarrollan los acontecimientos, entonces.

\emph{Esperaremos} ~---corroboró Scream, quieto y de pie frente a su, por el momento, imperturbable enemigo.

\parbreak
Los deslizadores de la colonia se pusieron en marcha y empezaron a buscar por todas partes de la colonia a los recién llegados a la misma. Eran modelos de mayor tamaño que los que habituales en Ernépolis, y de características militares evidenciadas tanto por su armamento pesado instalado en cabina como su blindaje de gran resistencia y carencia de ventanas. Las mismas que para Matthew Swind suponían en ese momento una ventaja, yendo como iba de polizón en ese mismo momento sobre la carrocería de uno de ellos.

Su misión era simple y al mismo tiempo aterradoramente intrincada: tenía que llegar a un lugar de la colonia, un punto estratégico de la misma, y ponerlo fuera de control. Bien podía ocurrir que la maniobra fuera en balde, pero habían estudiado bien los planos orbitales de los que se disponía y, amén de su conocimiento privilegiado de los recursos y debilidades de su enemigo, habían deducido que allá donde iba encontraría justo aquello que estaba buscando.

No hubo una sola persona en todo el Aquerón que no estuviera de acuerdo en que Swind era la persona indicaba para llevar a cabo tarea tan peligrosa y potencialmente fatal. Sólo él podía tener éxito en semejante hazaña, a tenor de su pasado. Y al propio Swind no le faltaba interés por lanzarse a aquella misión que para otros hubiera sido poco menos que suicida, aun a pesar de no tener la experiencia de su parte en lo relativo a lo que se podía encontrar.

Cuando estuvo bastante cerca de su destino, ya alejándose del corazón de la fortaleza, bajó del deslizador, momento en que sus pilotos notaron la disminución de peso en el mismo. No tuvieron más que virar para localizar a su polizón, pero en cuanto vieron cuál era su destino frenaron en seco y, pálidos, hicieron como si no le hubieran visto, siguiendo con su ruta tal y como estaba establecida.

Swind caminó con calma, sin que nadie apareciera en su camino, hasta estar frente a las puertas de una enorme prisión. Eran tan grandes que parecía que por allí pudieran salir todos los demonios de todos los infiernos de todos los mundos. Pero nada más lejos de la realidad.

De hecho, sólo uno podía escapar de allí en ese momento.

Debido a hechos pasados y no muy alejados en el tiempo, Hades había decidido que era mejor no usar sistemas humanos para vigilar los accesos de la prisión. Los sentidos de sus soldados podían ser engañados para ser empleados en su contra, y por eso la seguridad del complejo había pasado a volverse pasiva, lo que lo volvía muy vulnerable a infiltraciones desde el exterior. Eso, claro, si es que alguien en la colonia estaba lo suficientemente loco como para entrar en aquel lugar, del que circulaban toda clase de chismes y leyendas urbanas, y del que incluso se había decretado como un acto de traición atreverse a circular por los alrededores sin permiso.

Swind se acercó a las puertas y las miró con calma. Otro ser humano hubiera dicho sin reservas que estaban abiertas de par en par. Él no. Las ilusiones eran su terreno, y tenía un sexto sentido para saber olfatearlas. Sólo lamentó que no fuera quien se encontrara con aquel peligro las veces que estuvo suelto por las calles de Ernépolis, pues sin duda lo hubiera podido combatir con mayor facilidad que quienes tuvieron la desgracia de cruzarse en su camino.

Escaló el muro hasta subir a lo algo del edificio y una vez allí buscó un resquicio por el que poder abrir una fisura. No tardó en encontrarlo, y usando el arma reglamentaria del traje abrió un boquete que hizo saltar todas las alarmas. Aquellos dispositivos, sin embargo, no eran rival para él en aquel momento. Si hubiera tenido que pelear la cosa hubiera sido mucho más complicada e inmanejable. Pero si sólo se trataba de emplear las ilusiones en su máxima expresión, entonces aquello marchaba tal como se estaba esperando.

Cuando descendió lo primero que vio a su alrededor fue un increíble templo griego, todo labrado con materiales clásicos y tan auténtico en apariencia como los que en su momento pudieron existir. Era amplio y diáfano, y una brisa fresca soplaba a su espalda, ondeando ligeramente su gabardina mientras caminaba. Un suave olor a amapolas inundaba el lugar. Impresionante, pensó Swind. Aún más perfecto de lo que había imaginado al escuchar los relatos de los testigos.

Al fondo vio un altar que debió ser similar a aquel en el que se erigió una estatua al Dios Zeus en Olimpia, y en él descansaba tumbada de espaldas una hermosa mujer, vestida con una túnica que enseñaba muchas cosas pero tapaba otras aún más apetitosas. Swind había conocido sólo el verdadero aspecto deforme de Afrodita, y no había visto esa fachada que presentaba a la mente de los demás pero, sobre todo, a la suya propia. La tersura de la piel, el movimiento delicado de la túnica, sus cabellos delicados como miel brillante y pura, llenos de energía. Qué gran poder de convicción, pensó Swind, admirado por aquel mundo falso que había sido creado con tanta convicción.

Ojalá él pudiera haber hecho cosas tan increíbles en el pasado, pensó con calma. Pero su estilo de ilusionismo estaba más cercano a la manipulación que a la reconstrucción. Creaba elementos donde no los había o ponía trampas a sus enemigos, como la vista de un pasillo donde sólo había un opaco muro.

Para aquella ocasión, aun así, tenía el truco perfecto debajo de la manga, elaborado tras semanas de arduo trabajo, y sólo para ser usado aquella única vez, en que la sutileza ya no les era ni mucho menos necesaria.

Afrodita siguió de espaldas a Swind, pero comenzó a hablar con ese tono de voz dulce y meloso con el que algunos miembros de Los Caídos solían tener a menudo vívidas pesadillas.

~---¿Has venido a adorarme, guerrero? ¿O estás aquí con la intención de ofrecerte como sacrificio a tu diosa?

~---No, Tracy Swoop ~---dijo con una voz que no era la suya propia ni tampoco la que Los Caídos solían emplear~---, eres tú quien ha sido reclamada para servirme.

Afrodita se giró sobresaltada, como si no fuera ella la creadora del mundo ficticio que les rodeaba. Cuando miró a Swind sólo pudo vislumbrar la silueta brillante de Alma Espejo, que había conocido tiempo atrás y se había visto obligada a tener que abandonar.

~---¿Eres tú, mi señor? ¡Es peligroso que estés aquí!

~---No temas por mí, noble criatura ~---prosiguió hablando Swind, alimentando la fantasía de tan portentosa adversaria~---. Allí fuera me esperan mis enemigos. Son muchos y poderosos, y necesito tus artes para apaciguarlos a todos.

~---No puedo ayudarte, mi señor. Estoy prisionera en esta jaula dorada que han construido para mí.

~---No hay cárcel capaz de retenerme. Sólo espera aquí ~---dijo Swind saliendo por donde había entrado y abriendo desde fuera tras una ardua pelea con múltiples cerraduras electrónicas, aunque bajo el punto de vista de Afrodita resultó ser algo mucho más espectacular.

Salieron de allí, Afrodita por delante, rumbo a la fortaleza, a paso decidido. A medida que se cruzaban con los soldados de Hades Afrodita les hechizaba con sus poderes y les obligaba a seguirles. Cuando había embrujado a suficientes hacía que lucharan entre ellos a mano desnuda y proseguían su camino, provocando más caos alrededor de la ciudad.

Swind se puso en contacto con Scream, dando paso de esa manera a la siguiente fase de aquel ataque frontal.

\parbreak
Nada más Scream terminó de hablar con Swind miró a Hades, que estaba siendo también informado a su vez de una noticia que estaba haciendo cundir el pánico entre sus tropas.

~---Detenedla ~---dijo sin mostrar inflexión alguna en la voz~---. Ya.

Después de eso cortó la transmisión y miró a Scream. En su rostro ya empezaba a asomar el odio en su más pura expresión.

~---Veo que tratáis de atacarme en mis puntos más vulnerables. Bien. Celebro ese espíritu, que es justo el que siempre he pensado que os falta para cumplir con vuestro cometido. Pero esto aún no ha terminado.

\emph{En absoluto} ~---afirmó Scream, sin dejar de encararse en dirección a aquel tirano~---. \emph{Pero tal vez se está desarrollando a más velocidad de la que imaginas.}

~---Hades no dijo nada. Por primera vez en bastante tiempo se estaba empezando a quedar sin contestaciones adecuadas.

\emph{¿Has hablado con tus generales en este intervalo?} ~---continuó Scream, entrecerrando los ojos de manera siniestra, como si fuera él el villano y no sucediera al revés.

\parbreak
Hipnos y Tánatos no tardaron en empezar a moverse por cuenta libre a lo largo de las calles de la ciudad, en lo que sus hombres circulaban de un lado para otro, rastreando a los recién llegados. Algunos de ellos, muertos del miedo, estaban nulificados por completo, y traspasaban ese nerviosismo al resto de sus hermanos de armas. Nada que una mirada de Hipnos, seguida por un toque de Tánatos, no arreglara en un momento, recordándoles a quién debían temer de verdad.

Caminaron y caminaron hasta llegar al borde horizontal del deslizador tubular, donde giraba para atravesar el interior de la colonia. Aquella era la ruta más corta hacia su amo desde allí, y por eso se atrincheraron para defenderla.

Qué poco podían saber ellos, en realidad, que el objetivo de aquellos a quienes buscaban era, ni mucho menos, que derrotarles en su propio terreno por una mera y simple cuestión de revancha.

Antes ellos apareció la oscura silueta del Caído, pero cuando quisieron darse cuenta, y miraron de nuevo, pudieron ver que dos de ellos se presentaron ante sus ojos. Tánatos fue el primero en hablar.

~---Volvemos a vernos ~---dijo observando a la que estaba a la izquierda~---. No hace falta que te vea el rostro para saber que eres tú.

~---En mi caso ~---agregó Hipnos dirigiéndose a la de la derecha~--- me temo que sí tendré que hacerlo para que duermas al fin el sueño eterno.

\emph{Este no es lugar para pelear} ~---dijo Razorclaw mirando a su alrededor, y clavando la vista en una suerte de plaza apartada, con un lago en medio, un par de calles metálicas más al fondo.

Hipnos, a su vez, empezó a caminar en sentido contrario, hasta llegar a una suerte de colina artificial rodeada por un río de aguas depuradas que acababan desembocando en ese mismo lago.

De ese modo dos combates tuvieron lugar al mismo tiempo y en ubicaciones cercanas. La presión añadida ante la posibilidad de que llegaran soldados hizo su parte porque las peleas se desarrollaran a la mayor celeridad posible.

Hipnos llegó a lo algo de la colina y una vez allí se quedó de brazos cruzados.

~---Juguemos al escondite, Ellis Saw, ¿te parece? ~---propuso con insolencia~---. Si llegas donde estoy yo, tú ganas. Si yo te veo antes, tú pierdes\dots\ y Tánatos hace el resto.

\emph{Me parece bien} ~---dijo Saw, y caminó hacia las sombras para fundirse con ellas.

Hipnos comenzó a mirar a su alrededor. Tenía que admitir que esos tipos eran buenos a la hora de esconderse. Muy buenos. No tenía ni la más remota idea de dónde podía estar refugiado.

En fin, tampoco es que le importara, pensó hablando por comunicador en ese momento.

~---Luces a máxima potencia ~---ordenó, y sólo fue cuestión de segundos para que alrededor de la colina los rincones y zonas oscuras se iluminaran como un árbol de Navidad. Todas, salvo una. La única que había sido creada de manera artificial.

Hipnos sacó el lanzarrayos de su bolsillo y disparó a toda velocidad. Supo que había acertado cuando vio la mancha de sangre en el suelo, a sus pies. Después de eso, sólo tuvo que detenerse un momento y su enemigo apareció ante sus ojos, ya sin tratar de esconderse tras manto de tinieblas alguno.

Arrojó el arma y agarró con fuerza el mentón de Saw, tratando de que girara su cabeza para mirarle. Era un forcejeo en el que el ayudante del Presidente tenía todas las de perder, pues cinco segundos de superioridad era lo único que necesitaba su rival para doblegarle. Hipnos se dio cuenta, además, de que Saw estaba herido en el hombro, lo que sin duda le daba aún más ventaja en aquella situación de cuerpo a cuerpo.

~---Buu ~---decía burlonamente, casi sonriendo por su ocurrencia.

Saw dejó una mano libre para volver a agarrar la linterna que le había fallado en las sombras, esperando que la estrategia funcionara. Se lo estaba jugando todo a una carta, pero esperaba no sentirse defraudado por el poder de su antigua arma.

Aunque usarla de esa manera implicaba no volver a usarla nunca jamás.

Levantó como pudo la linterna hacia Hipnos, y éste no pudo evitar hacer un comentario al respecto.

~---¿Vas a envolverme en sombras? ~---dijo con tono chulesco.

\emph{De un modo que ni eres capaz de imaginar} ~---contestó Saw cerrando los ojos, apretando el botón de la linterna y soltando, de una sola vez, la luz que el aparato había acumulado durante todos aquellos años, agotando de ese modo su energía de manera definitiva.

Aquel intenso flash luminoso, tan potente que pudo ser avistado desde algunos emplazamientos del planeta Plutón, detuvo en seco tanto a Razorclaw como a Tánatos, que estaban jugando a su particular juego del ratón y el gato. Aquel destello trajo a la mente de ambos el mismo pensamiento formulado de distinta manera: Ellis ha ganado, Hipnos ha perdido. Lo que ninguno de los dos esperaba es lo que escucharon chillar más allá de su posición, en el margen del río artificial.

~---¡Estoy ciego! ¡Me ha cegado! ~---no hacía más que protestar Hipnos, lloriqueante, justo antes de ser abatido de un sonoro puñetazo.

\emph{El único sitio al que mirará tu hermano a partir de ahora será la tiniebla eterna} ~---proclamó Razorclaw, tratando de minar psicológicamente a su enemigo.

~---Tendrá que aceptarlo, si es que es digno de considerarse un verdadero general ~---fue su única y fría respuesta, tras lo que trató de acercarse de nuevo a Razorclaw, que se movía con gran agilidad, la misma que solía emplear en el pasado, de un lado para otro.

La estrategia de Razorclaw, por otro lado, parecía desarrollarse a la perfección. Sólo necesitaba que su oponente se viera tentado a dar el primer paso. Tentado a actuar. Para ello, trató de empujar a su enemigo a los límites de su paciencia.

Activó el dispositivo y gran cantidad de insectos acudieron a la llamada de Razorclaw, rodeándole así como entorpeciendo a su enemigo. Eran criaturas de lo más extraño y peculiar, importadas a la colonia sin duda de algún viaje espacial lejano en el que no se hizo la cuarentena apropiada. Poseían colores nunca vistos en la Tierra, y cantidad de detalles morfológicos que parecían desafiar la lógica, como demasiadas patas y al mismo tiempo alas dispuestas en posiciones nada similares a las de las moscas o mosquitos. A una nueva orden de Razorclaw se pegaron al cuerpo del general de Hades, pero a los pocos segundos de hacerlo cayeron al suelo como inservibles motas de polvo, igual que si se hubieran visto afectados por una efectiva fumigación instantánea.

~---Se acabó el juego ~---dijo Tánatos sacando el lanzarrayos y disparando a las fuentes de iluminación.

Razorclaw se lanzó hacia él a toda velocidad y trató de derribarle de un puñetazo. Llevaba las manos cubiertas y no había peligro de que le afectara su toque letal, pero sabía que un solo roce en el rostro y volvería a experimentar aquel reino de dolor que le fue mostrado\dots\ y más allá.

Tánatos, aun con todo, logró esquivar los golpes y cumplió con su objetivo: llenar de oscuridad el escenario de la batalla. Sin luces en las que mostrarse, ambos estaban empatados en ese sentido, con la única diferencia de que su ataque sería el último para Razorclaw si lograba efectuarlo primero.

Razorclaw buscó una nueva posición y trató de analizar el sonido a su alrededor. Su adversario era sigiloso hasta lo admirable. Sin duda, pensó, solía emplear esa táctica a menudo, pues tenía mucho que ganar de efectuar un ataque furtivo con éxito. No sabía de dónde podía venir, ni cómo podía atacar.

Ya había ganado, concluyó activando aquel antiguo artefacto de sus tiempos de héroe que a duras penas lograron que funcionara otra vez. Un estridente pitido anunció su posición de manera más clara que si fuera una luciérnaga en la inmensidad de un bosque en plena luna nueva.

El instinto jugó a partir de ese momento a favor de Razorclaw, y le jugó una mala pasada a su letal enemigo. Al escuchar el pitido y percibir así la posición del abogado, Tánatos, que no sabía a qué atenerse en ese momento, se lanzó a dar el primer paso antes de que se le adelantaran. No podía saber que lo mismo que delataba la posición de Razorclaw por medio del oído, delataba la suya propia por medio de mecanismos mucho más sofisticados y precisos que supusieron su inevitable derrota.

Un puñetazo brutal y directo a la mandíbula acabó el trabajo, y Tánatos cayó al suelo como un muñeco de trapo, dando con sus huesos en el suelo frío y metálico.

Al mismo tiempo Razorclaw escuchó a varios soldados aproximarse, alertados tal vez por el grito de Hipnos. Fue hacia la zona de Saw, y le vio subido a lo alto de uno de los edificios colindantes, estrenando el nuevo holograma.

\emph{Será mejor que lo emplees también tú. Además, me muero de ganas por ver cómo se aprecia en plena oscuridad} ~---terminó, y justo cuando Razorclaw comenzó con la pantomima, haciendo que los soldados escaparan despavoridos, Saw se puso en contacto con Scream para decirle que su parte estaba completada.

\parbreak
~---¿Sugieres que mis generales han sido derrotados en tan breve lapso de tiempo? ~---comentó Hades, altivo. Pero Scream no contestó, y Hades se quedó muy serio, demasiado orgulloso como para pedir confirmación de tal noticia pero al mismo tiempo lo bastante listo como para saber que Scream no tenía necesidad de tirarse un farol sobre algo que podía comprobar por sí mismo en un instante.

\emph{Tus carniceros han caído, tu amada está suelta y desbocada por las calles de la ciudad y cunde el caos entre tus soldados} ~---se limitó a resumir Scream, condensando los hechos de manera simplificada y aplastante~---. \emph{Deberías pactar, sería lo más sabio.}

Hades se levantó por primera vez de su asiento y se limitó a reír en voz muy baja.

~---¿Cuál es el plan ahora? ¿Atacarme a mí, en el corazón de mi propio santuario? ¿Qué debería hacer para que todo siguiera su curso? Tal vez ordenar a mis guardias que te atacaran, supongo. Pero estoy ansioso por presenciar por mí mismo un detalle que aún no me ha sido revelado. ¡Guardias!

A una orden de Hades los soldados se acercaron hacia Scream, pero se quedaron clavados en su asiento en cuanto vieron algo que pensaron que era poco menos que imposible.

El cuerpo de Scream comenzó a arder. Una llama perenne, perpetua, que ponía los pelos de punta. Al mismo tiempo, la mirada fría y tenebrosa de Scream no se alteró ni un ápice.

Ningún soldado se atrevió a avanzar. Uno de ellos disparó, y la bala pasó de lado a lado a su objetivo sin hacerle herida visible.

Hades comenzó a aplaudir con lentitud.

~---Bravo. Mis más sinceras felicitaciones por el último truco de feria que habéis preparado. No quisiera estropear la magia del momento, pero será mejor que lo haga. Es sólo un holograma ~---explicó Hades mirando a sus soldados~---. Avisad al resto de las tropas para que se sepa de inmediato.

~---Pero señor, ellos dicen que los objetos que entraban en contacto con ellos ardían de verdad.

En ese momento, fue Hades el que se quedó clavado en su posición, sin saber qué era lo que tenía en mente su enemigo en ese momento. Scream se limitó a acercarse al panel que había surgido junto al trono de Hades y tocarlo con la mano. Las llamas pasaron al instrumento y se encendió como una tea ardiente en la más profunda catacumba.

Los soldados salieron corriendo y dejaron solo a su señor, que empezó a sospechar lo que estaba ocurriendo. El holograma desapareció y comprobó que sólo los brazos de Scream estaban cubiertos en llamas. Un par de aspavientos bien efectuados para apagar el fuego acabaron con el espectáculo.

\emph{Siempre es más real si tiene parte de realidad} ~---explicó.

~---Debo admitir que ha sido muy astuto por vuestra parte. Te has aprovechado de mi conocimiento de que podéis llegar a arder espontáneamente en caso necesario para obligarme a mostrar mi propio desconcierto delante de mis hombres, y así convencerles del todo del engaño.

\emph{Exacto. Esto es entre tú y yo ahora.}

Hades sonrió y sacó su lanzarrayos. Disparó con absoluta precisión, pero no a Scream, sino hacia el techo, e hirió a James Sky en la pierna, que cayó desde lo alto de una de las columnas. Llevaba puesto su antiguo visor.

~---Una vez más voy por delante de vuestros pensamientos ~---explicó Hades, orgulloso~---. Sabía que el Jefe de Policía estaría tan furioso como vosotros, y por eso supuse que se uniría a vuestro burdo y patético plan. Sólo ha sido una cuestión de observación por mi parte, así como analizar tus movimientos involuntarios para acabar sospechando dos o tres ubicaciones donde pudiera estar. Ya imaginaba que te presentarías ante mí solo, sin subterfugios. Tú ya no volverás a ser nunca el héroe que fuiste, tus poderes están para siempre perdidos.

\emph{Impresionante} ~---admitió Scream~---. \emph{Sin duda serías un gran defensor de la ciudad si no hubieras elegido camino tan tortuoso.}

Hades ignoró el comentario como el rey que ignora a un vasallo, dio un par de pasos hacia Sky y recogió el visor del suelo, rompiéndolo con sus propias manos.

~---Este aparato, sin duda, me hubiera puesto en un gran compromiso, al poder localizarme. Lástima que ya no pueda verlo en acción. Pero tengo una duda al respecto. ¿Cómo lograsteis volver a reconstruirlo?

\emph{Te daré una pista} ~---comentó Shockman cayendo de lo alto y golpeando a Hades en pleno rostro~---. \emph{No lo hicimos.}

Hades aguantó el golpe de pie y se quedó un momento calibrando la situación, pensando con calma antes de hablar.

~---Ya veo\dots\ esto no es cosa de ti, sino de él ~---miró a Sky, que se puso en pie, aun herido.

\emph{Te equivocas} ~---explicó Scream~---. \emph{Sabía que Shockman se pondría en acción por su cuenta, y como suponía que tratarías de espiarnos, lo dejé estar. No podías deducir algo así a partir de mis actos y palabras si yo no era realmente consciente de ello.}

\emph{De modo que me usaste} ~---replicó Shockman, enfadado~---. \emph{Bueno, da igual. Al menos ahora podré desquitarme con este payaso} ~---dijo activando su aparato a máxima potencia casi al mismo tiempo que Hades se hacía invisible a los ojos de todos. Sabedor de que trataría de aprovecharse de todos sus puntos débiles, Scream corrió a defender a Sky de un posible ataque, pero Hades le interceptó en el camino con un golpe tan fuerte que se giró en el aire antes de caer al suelo.

Más o menos al mismo tiempo, montones de extraños insectos azules entraron en la habitación. Eran tantos que no se podía apenas ver más allá de unos pocos metros de distancia. Sin embargo, a Shockman le bastaron para distinguir la silueta de Hades.

Se acercó hacia él todo lo rápido que pudo, pues seguía avanzando hacia Sky. Cuando estuvo justo a su altura, sin embargo, recibió un tremendo puñetazo que le derribó también, incapaz de haberlo visto venir.

Hades metió la mano en los bolsillos de la gabardina de Shockman y tras un examen preliminar no tardó en deducir el que era causante de aquel revuelo. Lo apagó y la maraña de insectos azulados buscó huecos abiertos en las paredes por los que escapar, rompiendo varios ventanales en el intento.

~---Vosotros los viejos dinosaurios sois tan previsibles ~---proclamó hablando sin miedo a que descubrieran su posición, debido a la reverberación de la sala~---. ¿Crees que no contaba con imprevistos? Ya suponía que esta bala perdida se presentaría por su cuenta sin hablarlo contigo. He estudiado a todos tus hombres, y conozco sus movimientos.

\rquoti Sabía que el visor era una posible amenaza para mí, e imagino que como no pudisteis reconstruirlo pensaste en un plan B, que involucraba un ataque sorpresa y secreto hasta para ti. Imaginé que Warren Shockman era el hombre que tratarías de manejar para llegar a tal objetivo, por lo que analicé todos sus trucos y trampas con cuidado, así como su impulsivo comportamiento. Pero ya todo se acabó ~---dijo volviendo a hacerse visible de nuevo, y escuchando ruido proveniente del exterior~---. Ahí vienen mis hombres, y no tardarán en cercar este lugar.

Scream levantó la cabeza, tratando de incorporarse. Poco a poco fue recuperando la verticalidad, en un proceso tan lento como tortuoso.

~---Mírate, aún sigues empeñado en pelear. Esto se terminó, John Scream. He logrado ganarte porque he anticipado todos los movimientos de tu organización.

~---Te equivocas ~---protestó Scream, poniéndose en pie con un gran esfuerzo y quitándose el modulador de voz~---. Nosotros hemos ganado. ¿Y sabes por qué?

Hades se quedó callado, intrigado por su enemigo. En ese momento, tenía toda su atención.

Tanta, que no se dio cuenta hasta demasiado tarde de que le habían pegado una granada lapa en plena espalda.

~---Por que no has caído en la cuenta de que los insectos eran azules ~---contestó Scream mientras se tiraba al suelo, al igual que Ellie y los otros, justo antes de que se produjera la explosión. Hades cayó al suelo con la espalda completamente destrozada, emanando humo en una dirección similar a la que se produce cuando se apaga una hoguera. Su lanzarrayos echaba chispas debido a la onda expansiva, igual que su casco y otros aparatos electrónicos que llevaba encima.

~---Tú mismo lo dijiste ~---explicó Scream viendo a Ellie levantarse, algo magullada pero de una pieza~---. Shockman actuó por libre. Por eso tenía que recabar información de la colonia si quería moverse por su cuenta, y no podía pedírsela a nadie de la organización. De modo que le seguí en persona y comprobé que se veía con una chica ajena a nosotros. Luego supe de quién se trataba, lo que por desgracia me llevó a deducir que ella también querría venir hasta aquí, aunque lo acabé corroborando gracias al compartimento de la nave, habilitado con aparatos para más de una persona. Si te hubieras fijado en que los insectos que Shockman usó son propios de este lugar no habrías tardado en deducir que tenía un informante y, por tanto, existía una posible persona ajena a la organización que no entraba en tus cálculos.

Hades se trató de levantar, pero sólo logró quedarse de rodillas y con las manos apoyadas en el frío suelo metálico, al pie de su propio trono vacío.

~---No estabas antes\dots\ los cristales, ¿verdad?

~---Shockman me dijo que sólo entrara si era totalmente necesario ~---explicó Ellie~---. Cuando vi que los insectos rompían la ventana para escapar pensé que le había pasado algo malo. Él me vio ~---señaló a Scream~---, y trató de que ganara tiempo para esto.

~---Veo que aún\dots\ puedes darme alguna sorpresa, John Scream ~---confesó Hades casi como si estuviera bromeando. Un segundo después, Scream sacó su arma y apuntó a la cabeza de Hades.

Sky pudo ver que estaba en modalidad letal.

\emph{John, ¿qué haces?} ~---replicó, preocupado.

~---Hace lo que tiene que hacer ~---contestó Hades por él. En ese instante las puertas se abrieron y entraron docenas de soldados que cubrieron todos los flancos posibles.

~---¡Baje el arma! ~---gritó uno de ellos, con voz trémula.

~---Esto puede acabar ahora, John Scream. Si me disparas, dejaré de ser para siempre una amenaza para tu ciudad. Mis hombres no son rivales para ti, y aunque murieras otros tomarían tu relevo, como ya habrá pasado antes. Si no disparas\dots\ siempre podré acabar volviendo.

Scream miró a los ojos rojos de su invisible enemigo en términos orgánicos. En ellos había grandes dosis de astucia, y también de manipulación. Ganara quien ganara, la victoria era suya, de una u otra manera.

Pero él también podía jugar a conocerle.

~---Bajaré el arma, pero sólo con una condición. Que dejes de una vez por todas Ernépolis en paz. Lárgate con tu fiesta a otra parte.

~---¿Y por qué iba a aceptar tus condiciones?

~---Podría recurrir a tu orgullo. O hacerte dar tu palabra. O también\dots\ recordarte que te hemos ganado sin ninguna clase de objeción ni derramamiento de sangre, y por tanto estamos más cualificados para proteger la ciudad de lo que tú, posiblemente, estarás jamás ~---argumentó con aplastante contundencia.

Hades enmudeció. No solía hacerlo a menudo, pero admitió la lógica rígida de los argumentos del líder de Los Caídos.

~---¡Bajad las armas! ~---ordenó, al tiempo que Scream hacía lo mismo con la propia. Después de eso ayudó a levantarse a Hades, en una escena, cuanto menos, insólita~---. Déjame. No quiero tu ayuda. Largaros tú y los tuyos, y devolved la paz a esta colonia.

Scream enfundó el arma y salió de aquella sala del trono que se había convertido en improvisado terreno de una batalla campal seguido por Sky, Shockman y Ellie. Antes de marcharse se giró y miró de nuevo a su enemigo, tratando de mantener la postura erguida a pesar de estar medio tambaleándose, sin que nadie se atreviera a intentar ofrecerle un apoyo por miedo a las consecuencias.

~---Hubieras sido un gran aliado. Lástima que la vida no siempre sea como uno acaba planeando ~---terminó bajando aquellos largos y extensos escalones inclinados.

\parbreak
Nadie impidió el acceso de Scream y los suyos a la nave en la que habían llegado, por lo que se marcharon de Hidra sin el menor problema, esperando no tener que regresar jamás a aquel lugar. Sky y Saw fueron tratados de sus heridas por un molesto Shockman, que hubiera preferido estar a su aire sin tener que entablar contacto cercano con nadie a su alrededor. Swart pilotaba en calma y Swind le ayudaba con los datos de astronavegación, pero también para otorgarle compañía, entablando alguna que otra conversación ocasional de vez en cuando.

De todos los miembros de la organización sólo Scream permanecía solitario y en silencio, mirando al mal llamado vacío espacial, plagado de estrellas y polvo cósmico. Finalmente Ellie se acercó a hablar con él, una vez se hubo cambiado y puesto de nuevo su propia ropa, escondida en el compartimento de la nave en el que habían hecho el trayecto de ida.

~---Lamento haberte tenido que usar de esa manera ~---comenzó Scream antes de que ella misma pudiera decir nada~---. Suponía que Shockman tenía alguna clase de contacto al margen de nosotros. Lo que no imaginé es que se trataría precisamente de ti.

Ellie permaneció callada por un momento, justo antes de hablar. Quería pensar bien lo que decía para hacerlo de la manera más sencilla posible, sin tener que recurrir a filosofía barata ni improvisados alegatos.

~---Usted le conoció bien, ¿verdad? A Sam, me refiero.

~---Fui quien le recluté en persona. Ahora pienso que nunca debí hacerlo. Él tenía una vida, mucho que perder.

~---Creo que está siendo demasiado duro consigo mismo, y tengo la sensación de que siempre se comporta así. ¿Por qué no madura de una vez?

John Scream se volvió, sorprendido. Nunca antes nadie le había hablado así.

~---No puede cargar con el peso del mundo a sus hombros. Usted no apuntó a Sam con una pistola para que aceptara ni nada parecido. Sé cómo es, y los actos que le motivan por dentro. Aceptó porque quería hacer de la ciudad un lugar seguro para todos. Conociendo lo idealista que es, tal vez pensó en mí al hacerlo.

~---Hablas de Sam en presente, como si aún estuviera vivo.

Ellie bajó el rostro, afligida, pero aun así mantuvo la compostura y siguió hablando.

~---Su muerte no ha sido inútil ni en vano. Usted mismo lo dijo antes. La vida no siempre es como uno la acaba planeando. Ustedes parecían acabados como héroes y resurgieron de sus cenizas, Alma Espejo iba a ser el salvador de la ciudad y acabó convertido en su peor enemigo. Mismamente, yo no estaría aquí y ahora si Sam siguiera vivo.

~---Temo lo que vas a pedirme, y no poseo argumentos para negarme a ello. Debo admitir que eres valiente y decidida, tienes agallas. Además de eso, tus méritos incluyen haber asestado el golpe de la derrota a Hades. En el mundo de los héroes es un excelente currículo para empezar ~---dijo Scream, medio en broma medio en serio.

~---¿Y qué, me admitirá?

~---Hay una cosa que me preocupa, algo que puede suponer un problema. Puede que algún día tengas que enfrentarte al asesino de Sam. Con todo gracias a ti ahora entiendo que no podemos desvincularnos de nuestros propios sentimientos, por muy inquebrantables que queramos parecer, y que es importante que nuestros enemigos no se aprovechen de nuestros puntos débiles, pero más aún levantarnos si alguna vez consiguen hacerlo.

~---¿Estoy dentro, entonces?

~---Bienvenida a Los Caídos ~---contestó Scream, comprendiendo que el futuro tenía aún muchas sorpresas reservadas tanto para la organización como para la ciudad de Ernépolis en su totalidad.

\endinput
