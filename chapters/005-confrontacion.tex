\begin{prev}
    Los Caídos han salido a las calles, y el éxito de la operación ha sido rotundo. Los criminales menores les temen, la policía corrupta no puede detenerles y han logrado hasta que Silenciador fracase en su intento y sea eliminado por Ellen Gorgon en persona. Pero aún queda la pelea más peligrosa de todas\dots{}
\end{prev}

\noindent{}Todo marchaba según lo planeado. Primero cayeron las mafias de los barrios. Después la corrupción policial. Por último los elementos más peligrosos. Entonces ¿por qué tenía la sensación de que lo más difícil estaba por llegar?\\

\noindent{}A pesar de ser un héroe, a pesar de no moverse por venganza, John Scream no pudo evitar alegrarse cuando se enteró del fin de Silenciador. No era su intención más que inutilizarle cuando hizo personalmente el cambio de armas sin que él lo notara, pero siempre estuvo presente la posible represalia de la presidenta de Ernépolis~I. Starr Miles, por el contrario, estaba moderadamente contento. Su porte duro le impedía sentirse satisfecho con los resultados. John Scream y los demás que integraban los Caídos le conocían lo suficiente como para saber que no se sentiría satisfecho hasta haber devuelto la paz a la ciudad. Y ni en ese caso creían seguro que lo estuviera.

Con el paso del tiempo, entre los Caídos fue latente también cierta tensión entre Miles y Scream. El segundo opinaba que debían atacar a Gorgon cuanto antes, mientras que el primero optaba por no precipitarse. De ese modo estuvieron esperando meses hasta que llegó el momento que Starr Miles esperaba, una fiesta celebrada por Gorgon como acto social. El mismo lugar donde todo empezó a terminar para Scream. Tal vez fuera adecuado que el último capítulo se escribiera allí. No obstante, no estaba de acuerdo con la postura estratégica de Miles, y así lo manifestó cuando los Caídos se reunieron en su asamblea semanal, con cientos de personas escuchando un duelo de dos personalidades enfrentadas.

---Creo que debería hacerlo un solo hombre ---dijo Scream en voz alta, para que le oyeran todos los presentes.

---Eso que dices contradice las bases mismas de nuestra idea ---apuntó Miles sin más motivación que exponer su tesis.

---Querrás decir tu idea ---dijo Scream con dureza---. Esto es trabajo de un hombre, y todos lo sabemos.

---No, Scream, no es trabajo de un hombre. Y tú y tu escuadrón lo sabéis bien. Vuestra compenetración es lo que ha hecho que hayáis tenido tantas misiones de tanta importancia.

---Sólo éramos buenos porque el total era la suma de las partes. Pero en este caso, en la boca del lobo, un escuadrón no hace nada.

---Yo creo que Miles tiene razón ---opinó Sky, levantándose de entre los presentes---. Es la ocasión perfecta para la infiltración. Yo estaré invitado, del mismo modo que Razorclaw, por nuestras profesiones de policía y abogado. Y por supuesto Saw estará allí como ayudante personal de Gorgon.

---No guardo buen recuerdo de la última vez que Ellen Gorgon me invitó a una fiesta ---dijo Scream sombrío---. Creo que sospecha algo y se trata de una trampa.

---En todo caso, aun tratándose de una trampa, no podemos mandar a uno solo de nosotros. No pienso arriesgar la vida de nadie de esa manera.

---Pensé que los Caídos era lo más importante ---dijo Scream con ira en su voz, recordando el día que se enfrentaron a Silenciador.

---Si Raid murió no fue más que por tu culpa ---sentenció Miles.

La sala se llenó de murmullos. Todos sabían que Starr Miles tenía razón en cuanto a atacar a Gorgon juntos, pero había sido implacable con John Scream. Le apoyaban, pero en aquel momento lo hicieron por una simple cuestión práctica.

Scream no dijo nada. Sólo se quedó mirando a Miles un buen rato, ante los ojos de todos los presentes, y se marchó.

---¿Dónde vas?

---Voy a hacerlo por libre, Miles. Tranquilo, no estropearé tu operación ---enfatizó el \emph{tu}.

Cuando se marchó, Starr Miles no quiso hacer leña del árbol caído. Sólo se limitó a remarcar una para él importante reflexión.

---Recordad que somos fuertes porque estamos unidos. Scream tiene nobles intenciones, pero actuando solo sus posibilidades de fracasar son mayores. Ahora, todos a vuestros puestos. Los que habéis sido invitados no tardéis en llegar. Recordad que aunque no llevaréis el atuendo podéis ayudar a evacuar al resto de invitados cuando empiece el caos.

La asamblea se disolvió y Starr Miles se dirigió a los calabozos. Siempre, antes de una operación, le relajaba acercarse por allí y ver el fruto de tanto trabajo. Cientos de pasillos con todos aquellos que habían encerrado, que de cara a la sociedad de Ernépolis~I estarían sufriendo innumerables torturas a manos de aquel ser que de las cenizas surgía y carecía de nombre. Nada más lejos de la realidad. Si tenían calabozos era porque hasta que el sistema judicial de la ciudad no dejara de estar podrido deberían ser ellos los propios jueces. Era una tarea complicada tener encerrados a todos aquellos criminales sin que sospecharan todo lo que estaba pasando. Que no vieran a nadie bajo ninguna circunstancia. Una medida extrema, demasiado para el gusto de todos, pero que sabían era inevitable de momento.

Miles paseaba entre las celdas, mirando a través de los cristales que garantizaban que no era a su vez visto por los presos. Al rato Sky se acercó a él.

---Ha sido muy duro, señor. Y lo sabe.

---Sí, James. Lo sé. Pero tú sabes que es necesario. Ahora no te preocupes. Scream volverá, y todo volverá a su cauce. Concéntrate en la operación de hoy.

---Lo haré, señor.\\

\noindent{}A pesar del paso de los años la sala de fiestas seguía gozando del mismo aspecto que el día del último vuelo de Reflector. Sin embargo la Nube, al volverse más oscura, dotaba también al lugar de un halo tenebroso que recordaba que no todo era igual que la última vez. La gente reía, charlaba, bebía; pero había temor en sus corazones. Ya no se podía expresar en voz alta lo que se pensaba. Allá donde se mirara, había guardas gubernamentales por todas partes. James Sky se acercó al Jefe Wolf, ambos de paisano, con cierto aire nervioso.

---Jefe, he oído que él va a aparecer esta noche. ¿Desea que hagamos algo?

Wolf tragó nervioso. Ya que estaba en medio del enfrentamiento, pensó, mejor hacerlo por inacción que por acción. Además, siempre podría convencer a Gorgon de que le tomaron por sorpresa. Claro que, supuso, no podría usar dicha excusa eternamente.

---Sólo son rumores, Sky. Relájese. Todo está en orden.

---Pero señor\dots

---He dicho que se relaje, Sky ---dijo con tono más autoritario, como para recordar quién mandaba.

Sky se marchó y, como era costumbre en él, sonrió cuando nadie le veía justo antes de ponerse en contacto con el Aquerón.

---Todo en orden ---dijo con cuidado de no ser visto---. Wolf piensa que ha tomado la decisión por cuenta propia sin pensar en el empujoncito que acabo de darle.

---De acuerdo.

No dejó de pensar, preocupado, en John Scream y dónde estaría en aquel momento.

Al mismo tiempo, una silueta avanzó por los pasillos cerrados de la sala de fiestas. Aprovechando la penumbra típica de las salas cerradas al público, prosiguió tratando de eliminar los menos obstáculos posibles a lo largo de su camino. Interferir lo mínimo, pensó. Con paciencia. Parecía confirmado que Ellen Gorgon no podría estar entre los invitados hasta última hora, pues tenía asuntos empresariales importantes que tratar. Gorgon. Usaba la sala de fiestas como si fuera suya. Usaba la ciudad entera como si fuera suya. Bien, tal vez todo eso fuera cierto, pensó la sombra afilada. Pero pronto iba a cambiar.\\

\noindent{}Al cabo de un buen rato comenzó la música. El pianista era miembro también de los Caídos, y cuando tocara \emph{Another Day}, una pieza sinfónica de mucho tiempo atrás, sería la señal. Desde su puesto todos esperaban con impaciencia. Algunos entre el público, otros sobre la cúpula, los hologramas preparados, los movimientos perfectamente coordinados, cientos de posibles situaciones improvisadas ensayadas. Un gran truco de magia a escala multitudinaria.

Dos guardias vigilaban, rifles en mano, frente a la puerta de la habitación que Gorgon había convertido en uno de sus múltiples despachos. Oyeron un ruido al fondo y se giraron. Nada. Se dieron la vuelta y una sombra enorme les flanqueó el paso. Antes siquiera de que pudieran disparar dos certeros golpes se hundieron en sus respectivos cuellos y cayeron sin hacer ruido alguno. La sombra desapareció y su propietario se dispuso a entrar en el despacho. Habían estudiado a Gorgon y sabían que estaría sola, hablando a larga distancia con sus mandamases de Gorgon Enterprises. Sin embargo, cuando el hombre de la gabardina entró en la habitación y apuntó a Gorgon con su arma no letal, ésta estaba de espaldas mirando la ventana, las manos cruzadas.

No dijo una palabra. Esperó a que Gorgon se diera la vuelta. Sorprendentemente, ésta no hizo ademán alguno de inmutarse siquiera. Una mujer fría, pensó. Disparó y Gorgon cayó al suelo. No estaba muerta, como quería, pero algo le sorprendió cuando se acercó a ella. Sus dos manos eran humanas. De hecho, esa no era\dots{}

La silueta recibió un tiro por la espalda que la impactó en el costado derecho. Cogido por sorpresa, siguió con la farsa y se volvió aparentando total indiferencia. Sabía que todo dependía de eso.

Ellen Gorgon estaba frente a él, su mano aberrante empuñando el arma con que había sido disparado.

---La diseñaron ex profeso para mí debido a mi\dots{} discapacidad ---dijo Gorgon con indiferencia---. Bueno, supongo que he ganado, ¿no? Ya le puede decir a los suyos que se retiren o le mataré.

La silueta no se inmutó. La reacción que Gorgon esperaba no se produjo. Tiene sangre fría, pensó. Concluyó que hubiera sido un gran aliado.

\emph{No sé de qué está hablando} ---la silueta hizo tremendos esfuerzos para que su voz no sonara entrecortada.

---Oh, vamos, no bromee conmigo. Se acabó. Lo descubrí. Encontré el as debajo de la manga. Usted no es uno. Son muchos. Hacía tiempo que me di cuenta, pero no se lo he dicho a mis hombres. Quería acabar por lo sano.

\emph{Si me mata, volveré.}

---Oh, sí, claro que volverá, siempre vuelve para estropear todos mis negocios, pero no será usted, será otro tipo con otro traje similar. Sin embargo usted es irremplazable. Quién mejor que el líder de su secta, organización o lo que quiera que sea para acabar personalmente conmigo. Sin embargo, ¿qué pasaría si les digo que se rindan a cambio de su vida?

La silueta no contestó. Se limitó a mirar a Gorgon con sus insondables ojos negros.

---Está bien. Se lo diré yo. Sólo son teorías, por supuesto, ni una prueba, pero espero su colaboración para corroborarlas. Si salgo allí, y digo que usted está en mi poder, muchos hombres se ofrecerán para un intercambio. Tal vez invitados incluso. ¿No lo cree?

\emph{Sólo tengo una cosa que añadir} ---dijo la silueta, al tiempo que una gota de sangre caía al suelo.

---Le escucho ---dijo Gorgon impaciente.

\emph{Me subestima.}

Gorgon sonrió.

---Vaya, yo creo que ha sido todo lo contrario. Puede tener un aspecto fantasmal, pero ahora mismo está sangrando, arruinando mi alfombra nueva. Es humano, amigo. Su engaño ha fracasado. Y ahora veré quién es usted.

Gorgon se acercó, arma en mano, a su debilitado enemigo, y cuando se disponía a tirar de su sombrero ocurrió algo que nunca habría imaginado.

La silueta comenzó a arder.

Espontáneamente, como si fuera un acto voluntario. De los pies a la cabeza, sin moverse ni un milímetro, ni un comentario, ni un leve grito de dolor, sólo los ojos oscuros clavando la mirada en ella. Por primera vez en mucho tiempo Ellen Gorgon pensó que se había equivocado. Aquel que tenía frente a ella era más que un hombre. Tal vez estaba hecho de huesos y carne, tal vez sangraba y prendía, pero era más que un hombre.\\

\noindent{}En la sala de fiestas comenzó una tonada que muchos estaban esperando.

\begin{verse}
    \begin{em}
        Live another day,\\
        Climb a little higher\\
        Find another reason to stay\\
        Ashes in your hands\\
        Mercy in your eyes\\
        If you're searching for a silent sky\\
    \end{em}
\end{verse}

%\noindent{}
Las luces se apagaron y acto seguido, como si también estuvieran sincronizados, los invitados comenzaron a correr asustados hacia las salidas. Los guardias se aferraron a sus armas y encendieron las linternas. Muchos de ellos no tuvieron tiempo siquiera de hacerlo. Como un elegante dominó, uno por uno comenzaron a caer de modo que, igual que en un número magistralmente coreografiado, desde fuera daba la sensación de que un solo hombre estaba dando cuenta de todos ellos.\\

\noindent{}En el despacho de Gorgon la silueta cayó como una gran antorcha y prendió la alfombra, por lo que a pesar del apagón había luz en dicha habitación. Pese a estar comenzando un incendio, Gorgon no se movió del sitio. Contempló cómo el cuerpo se consumía, más temprano incluso de lo que había supuesto, y las ropas se oscurecían hasta que sólo un montículo de ceniza quedaba allá donde había estado su mayor problema. Se acercó y esparció la ceniza con el pie, como si pensara que fuera una especie de fénix que va a volver en cualquier momento. Al fin comprendió. Los héroes que había derrotado. Sólo podía ser cosa de ellos.

---Fue una buena idea, amigo mío, pero al final volviste a convertirlo en una pelea cara a cara, héroe contra villano, uno contra uno.

\emph{Siempre fue uno contra uno, Gorgon. Yo contra toda tu ciudad.}

Gorgon se dio la vuelta y disparó. Que la voz coincida puede ser un truco, pensó. Un truco de feria barato. Nada. Sin embargo le parecía que las sombras se movían. Giró en todas direcciones el arma, tratando de no ser cogida por sorpresa.

Fue una acción inútil. Gorgon no era una mujer de acción sino de palabras. De un sutil golpe cayó al suelo.

John Scream se agachó junto a las cenizas de su mentor y se permitió una leve lágrima por un momento. El plan había funcionado. La pelea fingida entre ellos dos por si Gorgon les estaba espiando, un mal menor necesario tanto para asegurar el éxito de la emboscada como las voces de protesta que hubieran bramado en el Aquerón ante la idea de que Miles se iba a sacrificar para atrapar a Gorgon. Se incorporó de nuevo y contactó con Sky, el único que sabía lo que iba a suceder.

\emph{Se acabó, James} ---dijo entristecido---. \emph{Ha empezado un incendio. Gorgon usó a una doble, la dejaré en el pasillo para que os la llevéis de aquí. Evacuad a todo el mundo.}

---¿Y tú qué harás? ---preguntó Sky. Scream se dio cuenta de que se lo preguntaba como si fuera el nuevo dirigente de los Caídos.

\emph{Gorgon está demasiado cerca de la verdad. Tengo que asegurarme de que no la cuente.}

---Ten cuidado, John. Sé lo que hizo, no hagas igual.

\emph{No lo haré, tranquilo. Miles tenía razón. No somos como ellos. Es sólo que esta vez sí que tengo que hacerlo solo.}

---Te veré en el cuartel.

\emph{De acuerdo.}\\

\noindent{}Justo al poco de dejar Scream de hablar con Sky, Gorgon recuperó el sentido. Miró sutilmente a su alrededor y vio la habitación en llamas y el arma cerca de ella. Parecía que podía cogerla sin que su enemigo lo notara. La agarró frenética, con los ojos brillantes, y se levantó, casi perdiendo el equilibrio de la velocidad con que lo hizo. El hombre de la gabardina y el sombrero la estaba mirando. Aquello no la gustó. Le apuntó todo lo deprisa que pudo.

Las llamas crepitaban alrededor, cada vez más cerca de las puertas y ventanas. La tensión resultaba casi palpable.

\emph{No puede matarme, Gorgon. Ya lo ha hecho. Dos veces.}

---No me engañas, prestidigitador. Tendréis una gran voluntad, tendréis poder, pero sois hombres. Hombres. Y ahora lo veremos, cuando tengas que salir de aquí para no morir asfixiado. Un momento ---dijo Gorgon para sí misma, una vez asimiló lo escuchado---. ¿Dos veces?

\emph{Sí, Gorgon. Tal vez ya no lo recuerdes, pero todo empezó aquí también.}

La silueta metió la mano en el bolsillo y sacó una gema que tiró a los pies de Gorgon. Una gema que Ellen Gorgon sí recordaba.

---Tú no eres él. Es otro truco barato. Yo le vi morir. Y no fue ningún engaño. De ser así no veo por qué Reflector habría desaparecido.

\emph{Reflector desapareció. Ahora} ---se quitó el sombrero--- \emph{sólo quedo yo.}

La presidenta de Ernépolis~I no dijo nada, sólo se limitó a mirar aquel rostro de ultratumba con sus ojos negros y muertos. No era un truco, pensó. Era real. Era Scream. El Capitán John Scream.

\emph{Se acabó su reinado. Va convertirse en mi prisionera. No irá a la cárcel, Gorgon. Tampoco la mataré. Tengo otros planes para usted.}

Desde que pertenecía a los Caídos John Scream ya había visto el terror en bastantes ojos, y admitió para sí mismo que su peor enemiga era una mujer que no flaqueaba fácilmente. No al menos de un vistazo superficial.

Gorgon giró la pistola y apuntó a su propia cabeza.

---Ellen Gorgon no será prisionera de nadie.

Acto seguido, sin más preámbulos, disparó.

La estructura de la habitación empezó a venirse abajo. Sin embargo Scream permaneció allí, sólo un minuto más. Miles tenía razón. La venganza no era nada, no daba satisfacciones. La justicia, sin embargo, lo era todo. Miró el cuerpo inerte de Ellen Gorgon, su mano alienígena extendida contra el suelo, y comprendió que era una mujer inteligente. Casi al instante le vino a la memoria la filosófica conversación de dos sabios en la primera colonia espacial. Mirando al planeta Tierra el primero aseveró que la inteligencia sin bondad no servía de nada. El segundo estaba de acuerdo con él, pero le puntualizó. <<La inteligencia sin bondad>>, dijo, <<no sólo no sirve de nada, sino que constituye una amenaza>>.

Abrió la ventana y desapareció de un salto entre las sombras. Acto seguido se acercó a la entrada más próxima al Aquerón.

\begin{next}
    No desconectéis pues esto no ha hecho más que empezar. En el siguiente número, ¡la breve pero intensa aparición de un nuevo enemigo!
\end{next}

\endinput
