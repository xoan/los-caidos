En la hora de asumir las consecuencias de los últimos actos el pasado renacía una vez más, y ellos aprendían que ya no sólo importaba lo que había sucedido mucho, mucho tiempo atrás, antes de ser uno solo a ojos del crimen. También la nueva era traía consigo, no sólo situaciones conflictivas de las que ya no podían escapar, también actos de rebeldía y sentimientos que algunos pensaban que ya nunca volverían a experimentar.

\fancyparbreak
Con la muerte de Alma Espejo como macabro telón de fondo, la fe y confianza de la población se hundió hasta extremos nunca antes sospechados. Su desaparición supuso un trauma social como pocas veces se había visto. Tal vez en tiempos pasados, pero muchos ya no eran capaces de recordarlo.

Hasta en la muerte parecía ser más majestuoso que los héroes de antaño, pensaron algunos en Los Caídos.

Pero pronto ese pensamiento no tardó en desvanecerse. Como una amarga realidad, los que así opinaban a bote pronto recordaron lo que había pasado con aquel aspirante a héroe. Desde un punto de vista objetivo Hades acabó con su vida, fue el enemigo al que jamás logró derrotar. Pero la verdad era mucho más cruda y directa. El peor enemigo de Alma Espejo, al que nunca doblegó, era él mismo y su exceso de confianza. Dijeron de él que era un dios. El problema es que empezó a creer que eso era cierto, y a comportarse acorde con dicha creencia.

Hades también tenía esa opinión de sí mismo. Pero la diferencia con Alma Espejo era que le movían motivaciones muy distintas. A su manera, Hades se veía como un mártir. Estaría dispuesto a darlo todo por defender aquello en lo que creía. Alma Espejo comenzó a pensar sólo en su propio interés, incluso aunque ese interés fuera desinteresado, aunque lo hiciera por proteger a una mujer que, en su opinión, ya había sufrido demasiado.

Pero analizar el comportamiento de aquel ser luminoso cuyo fulgor se había apagado como una vela en una racha de viento era algo más que complicado teniendo en cuenta que, además, ni siquiera los que mejor le conocieron, Los Caídos, tenían todas las piezas del puzle que había creado su personalidad. Su relación con Starr Miles había sido sin duda un factor clave en la misma, y tal vez conocer que ellos también habían sido forjados por su voluntad le empujó a ponerse un poco más allá del límite establecido, a tratar de demostrar que no era sólo un producto de las ideas de un hombre, y menos aún un plan B que había devenido en estrepitoso fracaso.

Quién sabe, pensaba Scream a menudo, que fue de los que mejor conoció a Miles. Tal vez tenía miedo de estar inculcando al chico ideales propios y eso le llevó a dejarle de lado. Tal vez en efecto vio un algo inadecuado en él como para ser la esperanza de la ciudad. Tal vez su juventud, su pasado conflictivo. Ú otra cosa inherente en su personalidad, pues no se es un juguete del pasado en todos y cada uno de los actos que provocamos, existen decisiones y motivaciones que nacen del interior de cada persona.

En todo caso los medios de comunicación, sobre todo El Crepuscular, no tardaron en señalar tanto a Hades como al Caído como los autores materiales de la muerte de Alma Espejo. Elaboraron una teoría según la cual, a pesar de ser enemigos mutuos, se unieron para acabar con el peligro común que su presencia suponía para ambos.

A partir de ahí, más que otra cosa, fue la política la que entró en juego. El Presidente Scatter necesitaba apaciguar los ánimos encendidos de los ciudadanos y ordenó una incursión policial contra el único culpable que aún quedaba en la ciudad y no había huido después de cometer el crimen, como declaró que había hecho Hades y su facción de terroristas. Por fortuna para la organización también había quienes secretamente simpatizaban con ellos, y la policía, a pesar de las órdenes recibidas, hizo la vista gorda en muchos sentidos. Ya no sólo porque James Sky estuviera al mando de la misma, al fin y al cabo no podía tener controlado a todo patrullero en todo momento. Muchos policías, sobre todo los de mente liberal, empezaron a comprender que había gato encerrado detrás de tanta impostura y pretensión de controlar las calles de la ciudad sacando a la escoria de las mismas.

Pero a pesar de ello, la ciudad era un polvorín. Volvían a los viejos tiempos, antes del juicio a Krexon, cuando se convirtieron no en un villano más a los ojos de los ciudadanos, sino el peor, un cáncer viviente que impedía el crecimiento de la ciudad en todo sentido posible.

Pero la situación se pondría más tensa aún. Las horas siguientes serían cruciales para muchos. Amigos se enfrentarían contra amigos.

Y como ocurría en las viejas películas de cine negro, todo empezaría por culpa de una mujer fatal.

\parbreak
Como siempre solía ocurrir en el cuartel, una vez se filtraba un rumor importante y a tener en cuenta John Scream era de los primeros en ser informado, si no el primero, según la situación. El informante fue Jim Swart, que a su vez lo escuchó de uno de los observadores de la familia para la espiaba las calles.

Alguien había regresado a la ciudad. Alguien que no pudo haber escogido peor momento para hacerlo.

~---¿Estás seguro de lo que dices? ~---fue la única pregunta de Scream tras haber escuchado atentamente.

~---Esos tipos suelen estar siempre bien informados. Dicen, además, que entre los maleantes de poca monta la noticia se está extendiendo como la pólvora. Se rumorea que quien tenga el valor de atraparla obtendrá reconocimiento sin igual entre los sindicatos del crimen, además de una considerable suma de dinero que no pregunté, pues se supone que a una criatura que nace de las cenizas no suele interesarle cosa tal como el dinero.

~---Temo saber el motivo por el que ha venido, y creo que no es la prudencia lo que está guiando sus pasos.

~---Creo que la prudencia nunca fue mentor para sus actos, John.

~---No olvides que me salvó la vida una vez y al hacerlo se puso en una situación de grave peligro. Ya pensaré a quién asigno esto, no te preocupes.

~---No lo hago, John ~---fue lo único que Swart añadió tras salir del pequeño despacho. Su mente estaba sembrada de dudas sobre si había hecho lo correcto. Al fin y al cabo, era sólo un rumor lo que él y los suyos habían escuchado. Tal vez estaba llenando la mente de Scream de sospechas innecesarias. Tal vez no.

Pero él conoció a la persona de la que los rumores estaban hablando. Segura de sí misma, lista. Astuta. Pero también le pareció que era alguien tremendamente emocional. Más que por convicciones, actuaba movida por pasiones. Y eso, en el mundo en el que se manejaban, era tan peligroso como una bomba de relojería andante.

Sus pensamientos fueron interrumpidos por alguien a quien era completamente incapaz de tragar desde que se había incorporado al equipo. No sólo por el ineludible estigma que suponía su pasado criminal, también su actitud presente y su aspecto de no importarle nada ni nadie ayudaban para que mejorara su opinión acerca de él.

~---Yo también he escuchado los rumores ~---fue lo único que acertó a comentar así, de repente, para romper el hielo.

~---¿Cómo lo has sabido? ~---dijo Swart, contrariado.

~---Salir solo tiene sus ventajas. No sólo la buscan en los sindicatos del crimen de los bajos fondos. Hades ha puesto precio a su cabeza, y también en los registros de crimen internacional su estatus ha subido de categoría. No en vano trabajaba para el hombre que, junto con nosotros ~---hizo una parada en el tono de voz para subrayar la ironía~--- mató a Alma Espejo.

~---Como si a ti te importara, que no sigues normas y ni siquiera estabas aquí cuando sucedió todo esto.

~---He echado un vistazo a ciertos informes. He leído algo acerca de esa mujer, y he llegado a conclusiones que nunca vendrían en un documento burocrático.

~---¿Entonces por qué no vas a decírselas a John?

~---Ese es el problema. No escuchará.

~---¿Por qué crees eso?

~---No lo creo. Lo sé. Vayamos a comprobarlo, si te parece.

Los dos hombres, que varios años atrás hubieran sido enemigos irreconciliables, regresaron al módulo central y enfilaron en dirección al despacho de Scream. Cuando llegaron al mismo vieron que no había nadie.

~---Estará por algún lado por aquí ~---se limitó a aventurar Swart.

~---Por algún lado, sin duda. Por aquí, seguro que no.

Buscaron a Razorclaw, Saw y Swind, pero los dos primeros estaban en sus respectivos puestos laborales y el tercero estaba con su escuadrón en el exterior. En la sala de entrenamiento de combate encontraron a Grove, practicando con los suyos maniobras de perfeccionamiento de emboscadas sigilosas que no tardaron en ser interrumpidas.

~---Eh, boy scout ~---dijo Shockman con impaciencia~---, deja tus juegos infantiles por un momento. ¿Sabes dónde está el gran jefe?

~---Dijo que saldría un momento a resolver un asunto personal, que nos limitáramos a seguir con el planning establecido.

~---¿El planning establecido? ¿Asuntos personales? ~---repitió Swart sorprendido~---. ¿De qué demonios está hablando? Es incapaz de dar un solo paso sin dejar de pensar en la organización, ni siquiera cuando está con sus diseños de naves espaciales. Muchas veces se los trae al cuartel para seguir con ellos al tiempo que analiza los informes del día y prepara el esquema de acción para el día siguiente.

~---¿Ocurre algo, Sam? ~---preguntó uno de los miembros del escuadrón de Grove que se acercó de repente, inquieto.

~---Todo está bien, señoritas, seguid con vuestra gimnasia rítmica ~---contestó cortante Shockman. Después de eso siguieron hablando.

~---¿Crees que ha ido a advertirla? ~---preguntó Grove, igual de preocupado o más que su compañero de incursiones.

~---Debe de haberlo hecho ~---agregó Swart, sorprendido por el curso de los acontecimientos~---. Pero no pensé que podría hacer algo así. Debemos llamar a los demás y\dots\ ~---Shockman negó con la cabeza.

~---Él quiere solucionarlo a su manera. Sabe que esto es un asunto propio que no debería afectar a la organización. Iré a buscarle. Solo. Necesito algún objeto o prenda reciente suya, y un poco de suerte también.

~---Déjame acompañarte ~---intercedió Grove~---. Tú por un lado y mi escuadrón por otro cubriremos más terreno.

~---No, chico. Este asunto puede ponerse muy turbio, y no estáis preparados para manejarlo con la contundencia adecuada.

~---¿Tú sí, acaso? ~---protestó Swart.

Shockman no dijo nada por un rato. Luego contestó.

~---Al menos yo he conocido las dos caras de la moneda. He tratado con antiguos héroes como John Scream, y también con asesinos desalmados como esa tal Perséfone ~---terminó saliendo de la zona de entrenamiento sin que nadie le siguiera ni detuviera.

\parbreak
La persiguen.

Sabía que no era una buena idea, ni sensata ni recomendable, regresar a Ernépolis~I después de que él hubiera estado por allí. Pero lo que no había podido imaginar era que resultara ser tan nefasta.

Las cosas habían cambiado desde su marcha, como no podía ser de otra manera. No sólo porque las calles de la ciudad parecían más oscuras debido a problemas con la infraestructura, aunque por lo visto esas obras de remodelación ya estaban en su fase final. Se refería a un cambio más sutil, más de los cimientos. Amenazar una ciudad al completo y perpetrar el atentado más bárbaro y espectacular de la historia de la urbe no era algo que fuera a borrarse como las huellas sobre la arena. Se notaba la crispación entre fuerzas del orden y del caos por igual. Imperaba la ley del gatillo fácil en los bajos fondos. Todo el mundo trataba de llamar la atención lo menos posible, de pasar desapercibido y no resultar demasiado destacable.

Bueno, no todo el mundo, pensó. Recientemente había habido alguien que intentó convertirse en el nuevo defensor de aquella ciudad podrida, corrompida y viciada hasta sus mismos cimientos fundacionales. Pero su brillo, por usar un símil preciso y barato, no tardó en apagarse para devolverlo todo de nuevo a la oscuridad.

Y allí estaba ella, Perséfone, que antaño fue la mano derecha de una de las organizaciones más poderosas de todo el sistema, que de hecho seguía siéndolo y, sin duda, creciendo hasta adquirir un estatus que no tardaría en ser comparable al de una pequeña nación. Ella, soldado fiel y eficiente donde los hubiera, desterrada. Relegada, abandonada, desarraigada.

Caída. En el fondo del abismo, sin aliados ni posibilidad de escalar para alcanzar de nuevo la cumbre.

Se suponía que había decidido ir a Ernépolis en cuanto se enteró de que Afrodita había escapado del control del que había sido su antiguo señor. Pensó que si la traía de vuelta lograría ganarse su favor, perdonaría sus errores. Ya no pensaba en obtener su aceptación en términos emocionales, hacía tiempo que había comprendido que Hades ya sólo tenía ojos para Afrodita, Tracy Swoop, aquella a la que llegó a conocer antes de convertirse en lo que se había convertido, un hombre carente de rostro, de nombre y de compasión por los débiles. Aquella a la que tal vez ni siquiera amaba, o lo hacía a su retorcida manera, recordando una y otra vez cómo se había convertido en una figura ilusoria, mentalmente inestable y letal. Por su culpa. No sólo por el incendio, también por dar rienda suelta a sus deseos de venganza, y no comprender que el poder que la regaló se convirtió en la fuente misma de su irreversible perdición.

Pero Perséfone se dio cuenta de que ella también estaba viviendo en un mundo de ilusiones creadas por su propia mente. El favor de su señor. Volver a entrar en su esfera de influencia. Quimeras sin base real alguna. Hades se encargó del asunto personalmente, con resultados fatales para aquel aspirante a héroe que se interpuso en su camino. Para cuando ella llegó ya hacía tiempo que se había marchado y sólo quedaban los rescoldos de su paso.

Pero tuvo la certeza de que daba igual que lo hubiera conseguido, que hubiera doblegado a Afrodita y llevado de vuelta con su señor, éste no la perdonaría jamás. Prueba de ello era que mientras estuvo en la ciudad aprovechó para que los suyos difundieran la noticia que ponía precio a su cabeza. Viva o muerta, eso le daba igual. Primera mala noticia de una larga cola de ellas.

No temía por su vida, desde luego. Perséfone, que había sido la ejecutora de la voluntad de poco menos que un dios, no se dejaba amedrentar por trivialidades como aquella. Aunque eso la obligó a esconderse, claro. A rehuir a los de su condición. Una bala perdida y certera desde un lejano callejón era un plan de asesinato que hasta el mayor de los cobardes podía atreverse a perpetrar.

Se alejó de los barrios grises, si es que alguno no lo era, para mezclarse en distritos menos desafortunados. Su vestimenta elegante pero informal, su amplio bolso, propio de una mujer que tenía dinero para comprar objetos con que llenarlo, consiguieron mimetizarla en el entorno. Habló con otros contactos de mayor amplitud, buscando la mejor y más rápida manera de escapar de allí. Nadie se atrevería a llevar a un muerto viviente, como denominaban en ciertos lugares a aquellos por los que se otorga recompensa, y no podía pagar tanto como para que olvidaran la suma que por ella se requería. Por ese motivo decidió optar por el transporte público y obtuvo un billete para el tren deslizador, la limusina de los desarraigados, la única escapatoria de aquellos que pretenden encontrar una vida mejor en alguna otra polis menos desestructurada.

Aún faltaban horas para que su tren saliera, transcurriendo lentas como vidas enteras. Era lo mejor que había podido encontrar, dada su condición marcada. Además de eso no tardó en averiguar que la policía también se había enterado de que estaba de vuelta, y teniendo en cuenta que su antiguo señor había acabado con la mayor esperanza de justicia que había existido en la ciudad en años, al menos desde el punto de vista del populacho, comprendió que a muchos de ellos no les importaría, llegado el momento, disparar a matar.

Por eso estaba allí, escondida en el tejado de un edificio de viviendas de la zona este de la ciudad, por los alrededores de la estación del tren deslizador. Esperando que la tormenta pasara. Negando una vez más la realidad. A pesar de que sabía que aquel que la había vendido el billete por una gran suma también había delatado dicha venta por una suma aún mayor. A pesar de saber que apestaba a cadáver a cada paso que daba en aquella ya de por sí putrefacta y hedionda ciudad.

A dos edificios de distancia un hombre espiaba atentamente a Perséfone, sentada, mirando a todas partes desconcertada, como un depredador que descubre que acaba de convertirse en inesperada presa. Era un hombre que fingía no serlo, pero no sólo por su gabardina y sombrero oscuros y raídos, sus ojos negros y artificiales, su voz de ultratumba y sus artefactos de prestidigitador. También en su interior, en la intimidad de su conciencia, llevaba tiempo fingiendo que no era un hombre. Que podía no tener vida, ni amigos, ni pareja. Que todo lo que le quedaba podía ser justicia, y que cada acto que elaborara podía estar angustiosamente racionalizado.

Ese hombre, que una vez fue capitán de una nave llamada Trigger y un héroe sin igual llamado Reflector, se había visto recientemente a sí mismo en otros que rodearon su entorno. Alma Espejo fue ciertamente una imagen especular de sí mismo, y vio en él las mismas emociones que le llevaron a la perdición, con mayor suerte, eso sí, que el joven y desdichado Álex Miles. En Warren Shockman encontró la otra cara de la moneda, el después alternativo, lo que podría haber sido sin un mentor que le devolviera, aunque sólo fuera en sentido figurado, a la senda de la luz.

Ese hombre, llamado John Scream, había comprendido que él también tenía emociones que no podía reprimir. Sensaciones contra las que no podía luchar. Errores que estaba dispuesto a cometer.

Y su error personal estaba dos tejados al norte de su posición, donde se dirigía en ese momento.

Era evidente que la mujer notaría su presencia cercana, pero tampoco trató de hacer una entrada espectacular. No hubo teatro, ni sombras estudiadas. Tampoco una frase de presentación, ni insectos huyendo y precediendo su llegada. Sólo se presentó como lo que era, un alma a la deriva perdida en medio de la inmensidad, a la sombra de la eterna Nube, dueña indiscutible de la ciudad en la que se desarrollaban todas las tragedias imaginables.

Scream no esperaba una manifestación de alegría. Tampoco indiferencia. En lugar de eso sólo vislumbró lo que estaba esperando. Una mano que, instintivamente, se introdujo en el bolso de la asesina y se mostró de nuevo aparentemente vacía y desarmada.

Scream no era idiota, y sabía que ella tampoco. Temiendo por su vida, no iba por ahí precisamente con las manos limpias. Estaba seguro de que al darse cuenta de quién era lo que había hecho era cambiar el arsenal por otro de naturaleza no letal, y sabía también que ella jamás reconocería algo así, preferiría morir a tener que admitirlo. Scream, en un arrebato de romanticismo, quiso pensar que lo hizo por algo más que para evitar una potencial condena más severa en caso de ser arrestada.

Perséfone le miró fijamente mientras aterrizaba, desplazando el polvo y la ceniza a su alrededor. Más experimentado, más poderoso, pero también mucho más cansado de aquel juego sin aparente final.

Justo en ese momento empezó a caer ceniza con gran intensidad. Un presagio de terribles acontecimientos venideros, razonaron ambos sabiendo que era también lo que ocupaba el pensamiento del otro.

~---¿Dónde están los demás? ~---preguntó ella con la voz endurecida.

\emph{No hay más} ~---se limitó a decir Scream.

Perséfone relajó la posición. Había pasado demasiados días seguidos sin dormir, y eso acababa pasando factura incluyo a tipos entrenados como ella para resistir tales condiciones.

~---¿Vienes a detenerme? ~---preguntó ella, dejando que la ceniza se filtrara al interior de su amplio bolso de trucos inimaginables.

\emph{¿Qué pasaría si dijera que sí?}

~---No\dots\ no vienes a detenerme. Ojalá vinieras a eso. Vienes a defenderme, ¿verdad? ~---protestó, humillada~---. Qué bajo he caído\dots\ ahora doy lástima a aquellos a los que inspiraba respeto.

\emph{Te matarán. Lo sabes, ¿verdad? Si te entregas la alternativa sería la cárcel\dots\ pero acabarías muerta igualmente. No pudiste elegir peor momento para llegar de visita} ~---añadió Scream tratando de hacer una broma de la que nadie se reiría jamás.

~---Te veo más duro, pero al mismo tiempo más vulnerable. Han pasado muchas cosas, ¿verdad? Nuevos enemigos, tal vez. O esperanzas desvanecidas, por lo que he escuchado. Una vez más tú y los tuyos haciendo las cosas a su manera, sin comprender que no hay leyes ni reglas en este juego.

\emph{No lo vas a conseguir, Perséfone.}

~---¿El qué?

\emph{Enfadarme. Ponerme en tu contra. Sé cómo eres, cómo piensas. No es mucho lo que hemos compartido, pero tú eres más que un soldado, sabes que yo también, y que esa es nuestra mutua perdición. Si fuera como dices no estaría aquí, arriesgándome por tu pellejo. Me limitaría a esperar en mi cuartel a que otros hicieran el trabajo por mí.}

Perséfone se acercó un poco más hacia la posición de Scream.

~---Estás delirando, John Scream. Ves en mí a alguien que nunca seré.

\emph{He visto a otros caer tan hondo como tú o más. Has matado gente, no soy ciego a ello. Pero sé cómo funciona tu retorcida cabeza. Sé que lo veías como misiones que te mandaban ejecutar. Sabes que no vales para asesina profesional. Tal vez ya hayas comprendido que te utilizaron, como a muchos otros antes.}

~---Lárgate, John Scream. Déjame en paz con mis demonios.

Justo cuando se dio la vuelta para alejarse de Scream, éste se acercó a ella y se quedó quieta, a medio girarse, como si aquel hombre pudiera manipular su cuerpo con el poder de la voluntad. Hizo ademán de usar lo que tenía en la mano en ese momento, pero él la agarró de las muñecas, abrió la palma y algo de naturaleza indeterminada cayó al suelo, donde empezó a formarse un vacío redondo sobre el que la ceniza no se depositaba. Scream la besó, y comprobó que sus labios sabían a humo de cigarrillo. Su vida entera era humo desde hacía muchos años, y aquella no iba a ser una excepción.

Perséfone se apartó y Scream aflojó la presa, sin decir ni una palabra. Ella recogió el arma invisible del suelo, la limpió de ceniza con la mano y la guardó en su bolso de nuevo.

~---Una vez me dijiste que no sabía lo que significa que aquel a quien amas muera a consecuencia de tus actos ~---comentó ella con lentitud y se detuvo a tomar aire, como si dudara de lo que estaba a punto de confesar~---. No quiero saberlo ~---aclaró al fin.

Scream no supo interpretar sus palabras. No supo si se refería a que ya no podría sentir afecto por nadie jamás, o que temía por la vida de él si se interponía en aquella vendetta. Tal vez ambas cosas. En todo caso, no contestó. En los juegos de seducción, como en la lucha contra el crimen, todo consiste en saber secuenciar de manera adecuada tanto las hábiles palabras como los elocuentes silencios.

~---¿Cómo supiste dónde estaba? ~---preguntó ella tratando de desviar la conversación.

\emph{Contactos. Gente que sabe mucho en esta ciudad.}

~---La abogada, ¿verdad? ~---dijo ella con plena convicción~---. Algún día se va a enterar de vuestras intrigas, si no las conoce ya. Seguro que tu amigo polizonte ya lo sabe también y tiene sus agentes tras de mí.

\emph{Sin duda. Sin embargo Sky no me parece el problema principal, sino\dots}

No acabó la frase. El disparo vino de un tejado contiguo y les lanzó a ambos al suelo. Era una descarga de lanzarrayos que ennegreció el suelo allá donde alcanzó, generando una hilera de humo que subió lentamente a reunirse con el que siempre estaba por encima de la ciudad. Se levantaron y miraron al frente, donde un nuevo invitado estaba confirmando por sí solo la frase de Scream sin que tuviera que acabarla él mismo. No tenía en mente como escollo principal a Sky ni la policía.

Temía a los cazarrecompensas.

~---Lamento que tengamos que vernos de nuevo en estas circunstancias ~---dijo, al otro lado de la columna de humo, una silueta de seis brazos que Scream conocía muy bien~---. Vengo a atrapar a una fugitiva y traidora.

Scream volvió a recuperar la presencia de ánimo y su silueta creció hasta fundirse con su sombra, como si ninguna dimensión física pudiera poner límites al alcance de su poder. La sombra empezó a deslizarse hasta llegar a los pies de Dobleseis.

~---Eso no va a amedrentarme, ya lo sabes.

\emph{Lo sé, pero tal vez te distraiga.}

~---¿Distraerme?

\emph{De mí, pulpo con patas} ~---dijo Shockman lanzándose sobre el agresor y derribándolo en el suelo~---. \emph{¡Corred!} ~---advirtió.

Scream le dijo a Perséfone que se marcharan.

~---¿Le vas a dejar solo? ~---comentó ella, preguntándose hasta qué punto la amargura había hecho mella en aquel a quien creía conocer bien.

\emph{Dobleseis no le matará. Es un cazarrecompensas, busca dinero legal, no\dots}

~---Dilo ~---inquirió ella~---. Matar a su presa. Como yo.

\emph{No quise decir eso.}

~---Pero es la verdad ~---añadió ella mientras saltaban al edificio más cercano, a menor altura.

Dobleseis trató de hacer una presa a Shockman, pero éste se apartó a tiempo y se interpuso en su camino hacia los fugados, ocupando el lugar de Scream como obstáculo hacia su objetivo.

~---A ti no te conozco, creo ~---dijo sacando armas nuevas, sopesando sus opciones~---. Esa manera agresiva de pelear por la espalda y enzarzarse en un cuerpo a cuerpo a lo loco no es muy propia de los tuyos.

~---Yo tampoco te conocía a ti. Pero últimamente he estado culturizándome. Seis brazos, experto tirador, luchador temible en las distancias cortas\dots\ todo un rival.

~---Gracias, amigo. Lástima que de ti me sepa todos los trucos.

~---En realidad te equivocas. ¿Y sabes otra cosa?

Dobleseis le miró fijamente, pero no podía haber trampa ni cartón en su actitud. Peleaba solo, lo sabía, como Scream hace un momento.

Al menos, se dio cuenta al fin cuando vio la extraña nube de bichos a su espalda, no con el apoyo de otros seres humanos.

~---Si crees que estoy loco, no has visto ni la mitad ~---terminó ordenando a la horda de mosquitos y otros repugnantes insectos que le atacaran sin piedad.

Doblesis se apartó a un lado, completamente impresionado. Se había enfrentado a muchas cosas en su vida, pero una legión de coleópteros furiosos era toda una desagradable novedad. Por fortuna para él su traje y casco le protegían de la mayor parte de las picaduras, pero sabía muy bien por qué su nuevo oponente había jugado esa carta. Cuando empezaron a dejarle en paz no tardó en mirar a su alrededor para darse cuenta de que se había marchado. Sopesó lo sucedido. Encontró al objetivo, y estaba con John Scream, que sorprendentemente, trataba de protegerla a toda costa. Luego fue atacado por uno de sus hombres, que al parecer iba por libre y parece que también andaba tras las huellas de Perséfone, de Scream, o puede que de ambos.

Dobleseis se levantó y, mientras corría y retiraba los insectos que aún tenía pegados al cuerpo y a los cañones de las armas, comprendió que más que una cacería aquello se estaba convirtiendo en una carrera en toda regla.

\parbreak
Scream y Perséfone saltaron de edificio en edificio, sincronizados como si siempre hubieran formado parte de un mismo escuadrón, hasta que decidieron que era el momento de ocultarse a menor altitud. Se detuvieron sobre un bloque mugroso y destartalado en el que apenas había una cuarta parte de las casas habitables. Usaron el acceso del tejado y se escondieron en un apartamento desvencijado y abandonado de cañerías oxidadas y paredes cubiertas con pósteres que evocaban sueños rotos y olvidados.

~---¿Quién demonios era ese de los treinta dedos? ~---preguntó Perséfone, intrigada~---. ¿Es de otra especie o uno de esos mercenarios mutantes de las colonias?

\emph{Eso da igual. El caso es que fuimos enemigos, luego aliados, y por lo que parece ahora somos enemigos otra vez.}

~---Tu problema es que confundes a enemigos contra oponentes. Sólo una cosa os separa en este momento.

\emph{Suficiente para que prefiera no mostrarle mi espalda al descubierto} ~---objetó Scream, sombrío.

~---Esto es un absurdo, lo sabes, ¿verdad, John Scream? Somos de mundos distintos. No puedes pretender ser mi caballero de brillante armadura.

\emph{Entrégate a nosotros. Cumple con tu condena en el Aquerón.}

~---En ese caso podrían pasar tres cosas. O bien escaparía, o bien me dejarías escapar, o bien uno de los tuyos me mataría.

\emph{Nunca ocurriría lo último} ~---matizó Scream, contrariado.

~---Empuja a un hombre al borde de su aguante y será capaz de hacer muchas cosas que nadie podía haber imaginado. Esto sólo puede acabar aquí y ahora. O bien me marcho en ese tren deslizante, o bien me matan en el intento. De cualquiera de las dos maneras, lograré escapar al fin ~---pronunció con voz grave y solemne, agachando la cabeza.

De repente, al levantarla de nuevo, Perséfone se llevó la mano instintivamente al bolso y puso los brazos como barrera contra un potencial intruso que estaba detrás de ambos. Era imposible saber qué clase de artefacto podía estar manejando en ese momento.

Scream no se giró. No le hacía falta. Al fin y al cabo, sabía bien cómo funcionaban sus subordinados.

\emph{Haz caso de lo que dice, Scream. Aunque no lo parezca, busca lo mejor para todos} ~---argumentó Shockman.

\emph{No} ~---replicó Scream, furioso~---. \emph{Lo mejor para ella, no.}

Shockman entró en la habitación y se retiró el sombrero. Perséfone pudo ver su ojo tuerto, así como su semblante endurecido, desprovisto de toda ternura y empatía, o al menos imperceptible desde el exterior.

~---¿Cómo nos has encontrado? ~---preguntó ella, intrigada.

\emph{Yo también tengo mis trucos en el bolsillo} ~---comentó, mientras la rata asomaba la cabeza desde el interior de su gabardina. Scream aprovechó para quitarse el modulador, pues sabía que era extraño para un tercero escuchar a dos personas usar la misma voz.

~---Imagino que te han mandado para detenerme, hacerme entrar en razón. Habrán pensado tal vez que a ti sí te escucharé.

Shockman rió por lo bajo.

\emph{No me tienen en tanta estima como para eso, al menos tus hombres de mayor confianza. Les he convencido de que te den un periodo de tregua.}

~---Eso no cambia lo que acabo de decir. Vienes a detenerme.

\emph{No trataré de ser un obstáculo para ti usando la fuerza, Scream. Tus asuntos son cosa tuya. Pero espero que tengas claro lo que estás haciendo. Que lo hayas pensado bien. Ni siquiera ella} ~---apuntó su único ojo sano en dirección a Perséfone~--- \emph{desea ser ayudada. Esto es una cruzada estúpida por tu parte, tú contra el mundo.}

~---Es curioso que seas tú quien me esté diciendo eso.

\emph{Es curioso que seas tú quien lo esté escuchando} ~---agregó el antiguo villano.

Los dos hombres se miraron sin decir nada. De nuevo, convergiendo en una manera de ver el mundo y divergiendo en otra. El uno, creyendo en la indomable voluntad de redención de aquellos que la buscan con todas sus fuerzas. El otro, severo e inflexible sobre la perversidad inherente de la condición humana.

\emph{Me voy, Scream. Trataré de distraer al seis gatillos para que soluciones esto por tu cuenta. Pero creo que la policía ya está de camino, y un viejo conocido mío va con ellos.}

\emph{No esperaba menos eficiencia por parte de Sky} ~---terminó Scream, volviendo a ajustarse el modulador de voz. De nuevo era una sombra, un ser que fingía no tener nombre ni alma. Pero en realidad, en esos momentos y más que nunca un protector. De alguien que, en realidad, no quería ser protegida. Que tal vez quería pagar por sus fechorías pasadas.

Tal vez.

\parbreak
Volvieron a los tejados, tratando de alejarse de las calles, y comenzaron a correr de edificio en edificio, como fugitivos que estaban estrenando su preciada libertad y huyendo de los barrotes de la indomable prisión que los había sitiado. Faltaba media hora para que el tren llegara a la estación y las primeras sirenas ya acuchillaban el sonido ambiente. Unidas a la eterna oscuridad y la siniestra Nube que flotaba sobre sus cabezas y seguía descargando ceniza de manera implacable, no podían ofrecer un destino más negro para aquellos dos soldados, desarraigados de sus propios bandos.

Al fin llegó la parada final de aquella huída sin sentido. El último edificio al que pudieron llegar estaba ya demasiado alejado de los siguientes como para saltar directamente. Las calles estaban infestadas de deslizadores patrulla, por lo que bajar no era una opción. Al otro lado estaba la vía rápida del tren, un túnel de techo abierto esperando albergar la única escapatoria posible a aquel conflicto, si es que había alguna.

Dar marcha atrás tampoco era una posibilidad, una vez vieron al cazarrecompensas cortándoles el paso en el edificio del que venían.

\emph{¿Ahora colaboras con los policías, Dobleseis?} ~---dijo Scream furioso, más que nada por el mal giro que estaban tomando los acontecimientos.

~---Llegaron a la vez que yo. Si no hubiera sido por tu soldado hubiera llegado antes. Tranquilo, está bien. Magullado, aunque de una pieza. Pero antes de nada, John Scream, porque es ya evidente para mí que tu identidad no es ningún secreto para ella, ¿por qué esta postura? Me sorprende viniendo de alguien como tú.

\emph{¿Alguna vez has dejado de hacer algo y te has arrepentido toda la vida?} ~---dijo Scream con las sirenas de fondo. Dobleseis no contestó~---. \emph{Yo trato de que no me ocurra así.}

~---Eres demasiado visceral ~---proclamó el cazarrecompensas~---. Apuesto que no fue tarea fácil convertirte en lo que eres ahora. Estabas tan seguro de tantas cosas, ¿a que sí? Que siempre ayudarías a la ciudad con tus poderes, que trabajarías de piloto toda la vida, que tu chica siempre estaría a tu lado. Pero dejaste de surcar los cielos, se desvaneció tu posición social, y la mujer a la que amabas fue asesinada delante de tus mismos ojos.

Perséfone se giró de repente en dirección a su espontáneo guardaespaldas, y le miró como si no le hubiera visto nunca antes. Si bien conocía muchos de esos datos, de repente los analizó bajo una nueva óptica que nunca antes había intentado emplear porque no pensó que estuviera a su alcance utilizarla.

Aquello se había acabado. De manera definitiva. Era el momento de plantar cara y que nadie tuviera que jugarse el cuello por ella. Invariablemente, todo acababa allí y en ese momento. Ese cazador de bonificaciones podía tener muchas armas y trucos en la manga.

Pero en ese terreno, ella era la maestra indiscutible.

Hizo un ademán de meter la mano en el bolso, pero Dobleseis la apuntó con sus seis armas, todas ellas distintos y variados modelos de lanzarrayos. Al menos, pensó Scream, no estaba usando la artillería más pesada que tenía. Quién sabe si lo hacía por deferencia hacia él, aunque lo dudaba.

~---Quieta. Deja el bolso en el suelo, y también tu cinturón.

Así hizo Perséfone, sin rechistar.

~---El segundo bolso también. Déjalo caer. Que suene.

~---Veo que me has estudiado ~---comentó haciendo como la mandaban, y dejando su segunda e invisible bolsa de trucos de emergencia yacer junto a la primera, a la vista de todos. La ceniza no tardó en confirmar la presencia allí de algo grande que los ojos no podían discernir.

~---Las manos en alto. Los dos. Bien extendidas.

Así hicieron, y Dobleseis comprobó que uno de los dedos de Perséfone estaba ligeramente arqueado.

~---He dicho bien extendidas ~---replicó.

~---Como prefieras ~---dijo Perséfone haciendo como la mandaban. Por un momento el cazarrecompensas creyó escuchar un ligerísimo clic acompañando ese movimiento, pero fue incapaz de estar seguro desde aquella distancia.

Sólo cuando empezó a escuchar las explosiones a su espalda fue cuando se convenció de que su oído no le había engañado.

Un segundo le bastó para comprender que aquella mujer había ido esparciendo granadas temporizadas tras cada tejado que dejaban atrás, y otro segundo más le bastó para comprobar, gracias al vacío dejado por la ceniza, que tenía una a varios metros de distancia de su posición.

El segundo siguiente tuvo que usarlo para ponerse a cubierto de la inminente explosión.

Gran cantidad de cascotes cayeron a la calle, que estaba en gran medida acordonada por la policía, por lo que no hubo daños personales. Scream trató de aprovechar esa distracción momentánea y se lanzó a por Dobleseis, que tuvo que dejar caer todas sus armas y detuvo los puños de su atacante con las palmas de sus brazos principales. Las otras manos le agarraron de los brazos como si fueran unas tenazas de acero.

~---Admiro tu perseverancia y valor, pero poco puedes hacer contra alguien que a corta distancia posee la fuerza de tres hombres ~---argumentó Dobleseis, imponiéndose poco a poco a Scream, cada vez más inclinado en dirección al suelo. Scream trató de preguntarse dónde estaría Perséfone y si tal vez había aprovechado la situación para largarse, pero comprendió que no había sitio hacia el que escapar.

Y entonces los acontecimientos se precipitaron. A lo lejos escucharon el sonido de la esperanza, el fin de la batalla, los tambores de la rendición. El tren estaba llegando, y por mucho que la policía lo persiguiera y ordenara detenerlo, Scream sabía que Perséfone, si lograba subir al mismo, se las arreglaría para pasar desapercibida y no tardaría en estar fuera de las fronteras de la ciudad, donde la jurisdicción policial no podría alcanzarla.

Pero al mismo tiempo otro sonido, esta vez siniestro y terrible, llegó a los oídos del líder de Los Caídos. Malas noticias que esperaba no tener que escuchar, pero que no tardó en ver con sus propios ojos cuando Dobleseis aflojó la presa y ambos rivales miraron al edificio contiguo, ya olvidada toda pelea, pues el epicentro de la acción estaba en ese momento a un tejado de distancia.

Un aerodeslizador policial estaba flotando sobre las azoteas, empleando un foco deslumbrante y levantando polvo y ceniza con sus rotores en incesante movimiento. Un policía armado con un rifle asomaba de uno de los laterales y apuntaba a Perséfone, quieta, de pie, sin apartar la mirada de él.

Justo en ese momento trataron de comunicarse con Scream, y escuchó una voz que confirmaba sus peores temores.

~---John, soy yo, James. Esto está yendo demasiado lejos. Debido a la explosión he tenido que ordenar al aerodeslizador que suba. Estos hombres están entrenados para disparar a matar si es necesario. Tienes que escucharme y\dots

Scream cortó la comunicación. Sin pensárselo ni un segundo, saltó al edificio de enfrente y llamó la atención del tirador que pasó a apuntarle a él, al verle en posición amenazante.

El tren estaba a punto de llegar. Sólo hacía falta unos segundos más, pensó Scream. Sólo un poco más. Llegado ese punto crearía una distracción, se convertiría en el blanco del disparo y eso bastaría para que ella saltara si hacía falta. Después de eso, minutos era lo que bastaba para que llegara a zona segura, lejos del poder policial, algo más si tuviera que recorrer andando parte de esa distancia. Pero en todo caso, una prórroga a sus planes de abandonar la ciudad.

Perséfone estuvo quieta durante varios segundos. De todos los presentes, todos los actores de aquella terrible función, sólo Dobleseis vio que temblaban las manos de la asesina.

Levantó el brazo y apuntó con un arma invisible hacia el tirador de la policía, que se giró instantáneamente hacia ella.

~---Baje el brazo ~---advirtieron por megafonía desde el vehículo. Perséfone no contestó.

\emph{¡No!} ~---gritó Scream~---. \emph{¡Es mentira! ¡Va desarmada!}

Pero los policías habían sido advertidos sobre la sospechosa, y harían falta más que palabras para convencerles de lo contrario. Hubo una segunda advertencia, a la que ella tampoco reaccionó.

A la tercera advertencia en vano, sobrevino el disparo. Acertó a la asesina en el torso, y se tambaleó hacia atrás hasta perder pie y caer por el margen trasero del edificio. Todo sucedió tan deprisa que los presentes apenas pudieron hacer poco más que observar impotentes, como gárgolas de piedra. El tirador, al ver que Scream corría al borde del edificio, dejó de considerarle una amenaza.

~---¡Sospechoso abatido! ¡Repito, sospechoso abatido! ~---comentaron de nuevo por altavoz. Scream trató de buscar con la mirada a Perséfone, pero no la encontró. Al mismo tiempo el tren ya estaba en la estación, pero al no haber nadie en el andén pasó de largo sin detenerse. ¿Habría llegado a subirse a tiempo? Aun malherida, quizás lo había logrado\dots

Scream comprobó cómo el piloto del aerodeslizador recibía órdenes, presumiblemente para seguir el tren, pero justo en ese momento un certero tiro desestabilizó un rotor y se vio obligado a tomar tierra en un edificio más alejado. Cuando Scream miró hacia el lugar de donde venía el disparo, justo donde había estallado la última granada, sólo pudo encontrar un par de dados de seis caras sumando doce, tirados en el suelo.

El cordón policial comenzó a aflojarse y Scream comprendió que todo había terminado. Perséfone había logrado escapar. Pero el precio a pagar había sido muy alto y a ella podía costarle la vida si la herida de bala no era tratada cuanto antes.

Volvió sobre sus pasos, amargado, solitario, cuando se cruzó con Shockman, de pie en un tejado, herido, tambaleante, pero erguido con orgullo y cierta muestra latente de desprecio hacia el mundo que le rodeaba. Miró a Scream como si no llevara tiempo esperando su regreso, como si aquel fuera un encuentro meramente casual y aleatorio.

Si Warren Shockman hubiera sido Charles Razorclaw o Ellis Saw hubieran dicho algo para llenar el incómodo silencio. Si hubiera sido Sam Grove hubiera tratado de entenderle y decir algo en consecuencia, y si hubiera sido James Sky lo más probable es que lo hubiera logrado. Pero Shockman no era como ellos. En su mundo muchas veces las palabras sobraban. Él no había tenido que entrenar cuándo callar ante los criminales, pues era algo que hacía de manera casi innata. El problema es que muchas veces también lo hacía ante los que estaban de su mismo lado.

Scream agradeció ese silencio, sin embargo. Sólo eso necesitaba en ese momento. No palabras de consuelo, apoyo, ni réplica. Tiempo para reflexionar, para sopesar lo sucedido. Para recordar el pasado y no relegarlo al inconsciente. Para comprender que por mucho que unos hombres traten de aparentar ser desalmados y eficaces hasta el límite de lo racional, las fortalezas y debilidades de las pasiones humanas nunca podrán ser extirpadas por completo de su interior.

Saltó al tejado de enfrente y Shockman le siguió durante un rato en modo camuflaje, sigiloso. Después de varias manzanas de silencioso paseo, decidieron, no obstante y sin palabras, separarse para regresar cada uno por su lado.

\endinput
