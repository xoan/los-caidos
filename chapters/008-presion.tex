\begin{prev}
    Bajo un clima de guerra espacial y tensiones con el ejército, James Sky, tras tener una compleja conversación con la abogada Emma Blades, es testigo de un aparatoso accidente espacial\dots{} que parece tener un misterioso superviviente.
\end{prev}

\noindent{}Ellos estaban próximos a llegar e involucrarse. Interponerse en su tarea, constituir un estorbo para seguir con su esquema de salvaguardia. Y sin embargo, no eran el menor de sus problemas.

Porque algo mucho más tenebroso, más insondable y misterioso, estaba a punto de convertirse en el centro de atención de todo y todos\dots\\

\noindent{}James Sky despertó en una habitación de hospital, uno que, dada su proximidad a la central, había visitado muchas veces antes, no como paciente pero sí para visitar a alguno de sus hombres, herido en acto de servicio. Por todo el cuerpo sentía un dolor lacerante como nunca antes había experimentado, ni siquiera en los duros entrenamientos a los que Starr Miles le había sometido en su momento. Aparte de eso, un desagradable zumbido le impedía pensar con claridad, sin duda un efecto secundario de la explosión.

Con todo, a pesar de las molestias físicas, lo peor fue la visión de aquella cosa.

Aún la recordaba con nitidez, como si la tuviera delante de sus ojos. Si bien era poco lo que había visto, le bastaba para que una luz de alarma se encendiera en su cabeza.

Venía de la nave, sin duda. Y parecía tener cara de pocos amigos. O lo que fuera que tuviera en vez de cara.

Se incorporó con lentitud y se llevó la mano a la frente. Estaba seguro de que el médico le diría que debía descansar, y él mismo se lo decía a sus hombres cuando estaba frente a ellos, al otro lado del espejo. Pero siendo Jefe de Policía y miembro de Los Caídos las cosas se ven de manera muy distinta en términos de bienestar personal.

Ya estaba sentado al borde de la cama cuando vio que alguien entraba en la habitación y encendía la luz. Era Blades.

---Apaga, por favor ---se limitó a mascullar cerrando los ojos y al mismo tiempo apartando el rostro hacia la pared. La penumbra invadió de nuevo la sombría habitación. La ceniza caía por la ventana con parsimonia, lluvia negra para un inquietante porvenir en las calles de Ernépolis.

---¿Qué tal se encuentra? ---preguntó Blades, sentándose en un sofá de invitados que había junto a la cama, en el lado de la ventana.

---Genial. Hacía tiempo que no dormía del tirón ---dijo Sky tratando de levantarse con gran esfuerzo, como si su cuerpo pesara una tonelada.

---Los médicos dicen que ha tenido suerte. Inhaló mucho humo, y el golpe le dejó inconsciente casi al momento. De hecho dicen que su suerte es prodigiosa, poco menos.

---¿Qué quiere decir con eso?

---Quiero decir que cuando le encontraron estaba a salvo en el sótano del local, aún lejos de las llamas. ¿Había bajado a buscar a alguien?

---Así es, en efecto ---contestó Sky, suponiendo cómo había llegado hasta allí---. ¿Qué es lo que ha ocurrido?

---Un accidente con una nave que estaba en órbita. Un transporte militar.

---¿Fue derribada en combate?

---Aún no se conocen más detalles. Creo que el ejército está esperando para hablar con usted en persona.

Blades se quedó callada por un momento. A través del contraluz, Sky pudo apreciar que tenía la boquilla en la mano, sin cigarrillo al final de la misma. Pensó que seguramente el mero hecho de jugar con ella entre los dedos servía para relajarla casi tanto como el propio humo del tabaco.

Se fijó un poco más en ella y vio que tenía trazas de polvo y ceniza por todas partes del cuerpo. La colisión debía haber llenado de escombros las calles colindantes al edificio.

---¿Hay víctimas?

---Los tripulantes de la nave. Todos murieron en el acto.

Puede que no todos, pensó Sky. O eso, o el golpe le jugó a mi cerebro una mala pasada.

---Jefe Sky\dots{} ---continuó Blades.

---Llámeme James.

---Lamento haberle estropeado su día libre.

La declaración fue tan sincera, y a la vez tan peculiar, que Sky no pudo evitar esbozar una de sus sonrisas irónicas, esas que usaba para reírse de una broma que sólo él podía entender. No esa vez, sin embargo.

---En realidad no es que estuviera siendo precisamente una fiesta continua.

---Me gustaría compensarle por lo sucedido. ¿Qué le parece una copa en cuanto salga de aquí?

La primera respuesta que se le pasó a Sky por la cabeza era si también fingiría estar borracha para sacarle información, pero se contuvo. Hacer de tipo duro no era lo suyo, y menos en sus escasos ratos libres.

---Esta vez le dejaré elegir el local a usted, Blades.

---Llámeme Emma.

---Todos en comisaría me llaman Sky o Jefe Sky. Déjeme llamarle Blades.

---¿Por qué motivo?

---Considérelo un intercambio cultural entre polis y abogados ---terminó mirando al horizonte negro y con ribetes rojos que se ofrecía desde la ventana de la habitación.\\

\noindent{}Blades estuvo un tiempo más en la habitación, charlando con Sky de temas intrascendentes. No hubo más menciones a su trabajo como policía ni a su aparente permisividad con el Caído, pero aun así Sky tuvo que ocultar muchos otros detalles de su vida personal y pasada. No en vano una vez fue un héroe con poderes, alguien que luchó sin reservas por la ciudad. Tal vez no tan conocido como otros, pero héroe al fin y al cabo.
Cuando Blades se marchó, dirigió la mirada hacia las sombras picudas del fondo de la habitación.

---Ya se ha ido ---declaró en voz alta.

\emph{Lo sé} ---fue la única respuesta que surgió de las sombras.

---¿Entonces por qué no habías salido aún?

\emph{Tenía la sensación de que querías reflexionar en soledad. ¿Hace cuánto sabías que estaba ahí?}

---En realidad no lo sabía, pero conociéndote intuía que no tardarías en aparecer. Tranquilo. Nuestro equipo de sigilo no está aún comprometido.

\emph{Me alegra escuchar eso} ---dijo la silueta saliendo de las sombras. Si bien ningún subterfugio de ilusionismo adornaba su presencia, sus ropas oscuras y su sombrero de ala ancha, tremendamente estilizado, le otorgaban una presencia más que imponente.

Eso y aquellos siniestros e inhumanos ojos negros.

---¿Quién me ha rescatado? ---preguntó Sky sin andarse con rodeos.

\emph{El escuadrón de Matt. Tenían asignada esa zona. Él mismo entró en persona a rescatarte.}

---¿Ha reportado Matt algo\dots{} anormal? Aparte del hecho de que una nave ha efectuado un picado letal contra un edificio y por poco aterriza en mi cabeza, claro.

\emph{Esperábamos que tú pudieras darnos algo de información de primera mano. ¿Viste algo, tal vez escuchaste las últimas palabras de alguno de los tripulantes?}

---Más que eso. Vi a un superviviente.

\emph{No hay supervivientes.}

---Los hay, John. No sólo sobrevivió a la colisión, parecía estar en bastante buena forma. Debió salir por su propio pie.

\emph{Le diré a Matt que investigue. Tal vez nuestro pasajero misterioso esté desorientado, y ni siquiera sepa dónde está ni qué está haciendo aquí.}

---¿Qué hay de\dots{}?

\emph{Viene alguien} ---dijo Scream de repente, ocultándose de nuevo entre las sombras. Sky se recostó de nuevo sobre la cama.

---Jefe Sky, soy el Coronel Martin Straxus. ¿Está despierto?

---Adelante ---dijo Sky con cierta desgana. No le gustaba demasiado tener que lidiar con los militares, y estaba claro que tarde o temprano aparecerían para ponerle al día de lo que estaba pasando, o al menos de su versión de lo que estaba pasando.

La puerta se abrió y entró un hombre ya entrado en años, pero sin duda perfectamente capaz de dirigir incursiones en terrenos más que desconocidos e inhóspitos para el ser humano. Prueba de ello era que iba armado con su pistola reglamentaria, algo inusual en un alto mando como el suyo. Hizo un gesto a sus dos escoltas para que se quedaran fuera y entró sin más dilación en la habitación.

---Celebro verle consciente.

---¿Qué le trae por aquí?

---Creo que le debemos una explicación, y también necesitamos la colaboración de sus fuerzas del orden. ¿Podemos hablar con franqueza?

---Adelante ---contestó Sky mirando a las sombras---, hable sin reparos.

---Estamos a punto de dar una conferencia de prensa en la que diremos que una de nuestras naves se ha estrellado contra Ernépolis~I como consecuencia de una contienda orbital, sin supervivientes de clase alguna.

---Si eso fuera lo que hubiera pasado no estaría aquí contándomelo en persona. Se habría limitado a mandarme un comunicado por medio de fuentes oficiales ---apuntó Sky sin más.

---Está en lo cierto. En realidad esa nave ni siquiera estaba armada con láseres. Era un módulo que provenía de nuestro laboratorio de investigación en la Luna.

---¿Qué ha ocurrido entonces? ¿Sabotaje?

---Lo ignoramos. Aún es pronto para efectuar un análisis de los restos, que por otro lado no parece que vayan a aportar demasiados datos. La caja negra se ha perdido en órbita, por desgracia.

---¿Y por qué oculta la información a la opinión pública?

---Dos motivos me mueven a ello. El primero es que estamos en guerra, como bien sabe, y no podemos permitirnos aparentar debilidad ante el enemigo. Ocurre que en esa nave viajaban algunos investigadores militares, cuya pérdida es infinitamente mayor que la de meros soldados rasos.

\rquoti{}Existe, además, un superviviente. Una superviviente, para ser más precisos. Y aquí es donde le pedimos ayuda. Queremos encontrarla, pero necesitamos que este asunto se lleve con la mayor discreción.

\rquoti{}Esa mujer, Eileen Drift, era parte de un equipo de investigación que estaba diseñando un prototipo de traje, un modelo único y especial. El traje no ha sido encontrado, por lo que creemos que puede haberlo usado para sobrevivir al siniestro.

---¿Qué es lo que hace ese traje?

---Me temo que eso es información clasificada. Bastará para usted y los suyos con encontrarlo y ponerlo de nuevo bajo custodia de mis tropas. He bajado personalmente de la Luna para esta misión, que es de extrema importancia. Como habrá supuesto, soy más que un militar, de hecho he trabajado personalmente en este proyecto con mis conocimientos de dinámica de altas presiones.

---Todo eso me parece muy bien, pero lo único que me importa es la seguridad de los civiles.

---Una cosa más, Jefe Sky. ¿Qué hay de esos rumores\dots{} ese criminal que parece una sombra y acecha las calles de Ernépolis~I? No creo que sea necesario decirle que mis hombres tienen orden de disparar a matar en cuanto le tengan a tiro.

---No se preocupe por él. Con suerte ---dijo sonriendo y mirando de nuevo hacia las esquinas oscuras de la habitación---, no será tan estúpido como para involucrarse en todo este asunto.\\

\noindent{}Aquella había sido una noche movida para Matthew Swind y su escuadrón. No sólo habían sido testigos, desde los tejados de Ernépolis~I, de un espectacular accidente causado por una nave que había surgido de entre la Nube a tanta velocidad que parecía más un misil que un vehículo tripulado. Además de ello, tuvieron que entrar en el edificio contra el que había colisionado, encontrándose con la desagradable sorpresa de que Sky, uno de los suyos, yacía inconsciente en su interior.

Lograron alejarle por los pelos de ese infierno y llevarle al sótano inferior, una vez cercaron las llamas que estuvieron a punto de atraparles a ellos también. Aquello era una auténtica pesadilla. Nada podía salir vivo de allí.

Nada.

Pero sorprendentemente una comunicación de Scream aseguraba que, de hecho, \emph{alguien} había salido vivo de allí. Que estaban a bordo todos los tripulantes menos uno, y también faltaba una especie de traje especial. No pudo ser más concreto al respecto.

---Al parecer la mujer es científica ---siguió explicando en lo que Matt y su escuadrón le escuchaban desde un tejado---. No sabemos bien lo que está pasando. Puede ser que la propia mujer esté llevando el traje, pero también podría ser que haya sido secuestrada por haber participado en su diseño o, incluso, que nunca haya estado a bordo de la nave. De lo que sí estamos seguros, gracias a Sky, es que el traje sí que está en Ernépolis, por lo que será mejor que lo encontremos antes que el ejército, si es que queremos saber lo que está pasando.

\emph{Recibido, jefe. Ya lo habéis escuchado, chicos. Lo mejor será que nos dispersemos y nos movamos entre las sombras hasta que uno vea algo de interés.}

Se separaron siguiendo una maniobra perfectamente estudiada y no tardaron en cubrir una gran cantidad de superficie y alturas. Swind tenía la sensación de que estaban pisando terreno pantanoso, y no tardó en confirmarse su sospecha en cuanto recibió una transmisión de Rowl, su segundo de a bordo.

\emph{No estamos solos, Matt. Los militares están también por aquí. He pinchado sus comunicaciones.}

\emph{Pásanos su frecuencia} ---ordenó Swind. En un instante todos escucharon la misma corta pero sencilla frase.

---Voy hacia el objetivo, aseguro posición.

\emph{¿Puedes localizar la fuente, Rowl?} ---preguntó Swind al momento.

\emph{Ya lo he hecho. Está a dos manzanas de mi posición.}

\emph{Vamos para allá. Tened mucho cuidado. Estos tipos no son tan supersticiosos como aquellos con los que solemos encontrarnos. Entramos en modo sigiloso. Corto comunicación.}

Nada más llegar al punto de encuentro, una sombría plaza cubierta por complicadas y enrevesadas celosías oxidadas, Swind se colocó en su posición de defensa ---encaramado en alto, abarcando el callejón del fondo, arma a punto--- y analizó el lugar. Su equipo funcionaba por medio de un exhaustivo método de entrenamiento que consistía en conocer todos los recodos y esquinas del lugar por el que merodeaban. Eran el típico escuadrón de patrulleros, y por eso podían permitirse el lujo de resultar mecánicos en sus acciones. Las incursiones a terreno desconocido eran más propias de otros escuadrones más avanzados, así como del propio John Scream.

Desde su posición podía contar hasta cinco soldados armados. Su arma principal era un rifle de asalto para incursiones nocturnas, además de toda clase de armas blancas energéticas. Uno de ellos, además, llevaba una versión modificada de las metralletas que se acoplaban a la parte delantera de las naves de combate.

Estaba claro que esperaban alguna clase de resistencia.

Desde su posición podía vislumbrar la de Rowl y la de Vortex. Slaught quedaba fuera de su alcance, así como Bloff. Por supuesto, no era capaz de ver a ninguno de ellos, ni ellos a él. Pero habían ensayado la coreografía tantas veces que no resultaba necesario hacerlo.

Como director de escuadrón correspondía a Swind efectuar el primer movimiento. Realizó el lento descenso, protegido por toda clase de camuflajes reales y artificiales.

\emph{Estáis en mi territorio} ---fue lo primero que acertaron a escuchar los soldados.

---Lárgate o dispararemos ---contestó el que parecía ser el sargento del grupo.

Swind calló. Era el momento de dejar que la duda les carcomiera por dentro. Que ellos mismos forjaran sus propios miedos.

Y entonces entró en acción el factor inesperado.

Una silueta apareció en medio de la plaza, sin que ninguno de los escuadrones que estaba allí hubiera reparado antes en su presencia. Swind dedujo que parecía poseer un sofisticado camuflaje que superaba con mucho al que ellos mismos habían diseñado tras tanto esfuerzo conjunto.

Se erguía inmensa, unos dos metros y medio de altura, y en términos esenciales era como una armadura medieval modernizada, poseedora de complicadas articulaciones electrónicas allá donde debían juntarse las distintas piezas. Era de color rojizo oscuro, como sangre venal, y parecía ser tremendamente pesada. Poseía gran cantidad de junturas a lo largo del torso, brazos y piernas, dando a entender que seguramente era capaz de adoptar complicadas configuraciones armamentísticas. Sus manos eran enormes, como si llevara guanteletes, y su yelmo no dejaba ver su rostro. En su lugar, tenía una inquietante franja horizontal, blanca y brillante, donde debería estar su mirada.

Los soldados, aun con todo, no se amedrentaron.

---¡Fuego! ---fue la contundente orden del sargento. En cuestión de segundos una andanada de balas sacudió la armadura como cuando la lluvia torrencial impacta contra el suelo, y se vio obligada a apoyarse sobre una rodilla. Una vez se detuvieron y cesó el humo alrededor del recién aparecido comprobaron que seguía en pie, pero parecía inactivo.

La armadura contraatacó, y lo primero que hizo fue disparar una especie de descarga a través de los dedos hacia uno de los soldados. Le dio de lleno, y se quedó en el suelo, convulsionándose. Swind pensó que debía ser una versión a gran escala de un táser. Munición no letal.

No tuvo tiempo de verle hacer nada más. Lo siguiente que los soldados hicieron fue descargar toda la munición de la metralleta contra su objetivo. Aquello empezaba a tomar visos de carnicería.

Junto con la rodilla, la armadura tuvo que hincar la palma de la mano contraria para no perder el equilibrio. El sargento se acercó a ella y la apuntó con su rifle de asalto.

---Avisa al Coronel ---dijo a uno de los suyos---. Tenemos el Proyecto Armor. Y ahora ---acabó acercándose al yelmo--- veamos quién está jugando a las servoarmaduras\dots

Estaba a punto de acercarse a la parte trasera del casco, donde presumiblemente se alojaba el mecanismo que lo abriría, cuando escuchó un grito a su espalda. El sargento se dio la vuelta y se encontró con que dos de sus hombres habían desaparecido sin dejar rastro. Se giró de nuevo y levantó enfurecido el arma.

---¡Sal, cobarde! ¡Pelea como un hombre!

No hubo respuesta. En vez de eso, otro grito resonó en la plaza y otro hombre desapareció.

---¡Alfa 2! ¡Alfa 3! ¡Responded! ---decía inútilmente el sargento, llamando por el comunicador. Ante él sólo estaba el soldado que había sido abatido por la descarga eléctrica, incapaz de reaccionar.

No duró mucho haciéndolo. Una silueta enorme, como una visión fantasmal, apareció frente a él, y vació medio cargador sobre ella. Dejando aflorar un instinto arraigado, giró en todas direcciones y empezó a disparar al frente, espalda y laterales.

El golpe que le dejó inconsciente vino, sin embargo, de arriba, la única dirección a la que casi nadie se molestaba en disparar.

Swind siguió con el teatro y se acercó a la armadura caída, que trataba en vano de levantarse, y cuyo visor había incluso dejado de refulgir.

\emph{¿Quién eres?}

<<Un... recién llegado>> dijo con una voz metálica que sonaba distorsionada.

\emph{Ésta es mi ciudad. Sólo yo aplico la justicia en sus calles, no ellos} ---miró al sargento inconsciente---. \emph{Ahora vendrás conmigo.}

<<... ás>> escuchó de repente Swind, tan bajito que apenas logró entenderlo.

\emph{¿Qué dices?} ---preguntó solemne, acercándose al rostro de la servoarmadura.

<<Divide y vencerás>> repitió con un tono de voz perfectamente audible.

Y entonces el visor volvió a brillar.

Agarró a Swind del cuello y lo alzó en vilo sin ningún esfuerzo. Fue entonces cuando todo el escuadrón comprendió que habían caído en una trampa, y aquel sujeto sólo estaba fingiendo ser menos poderoso de lo que en realidad era.

Una descarga sacudió a Swind por dentro, e inutilizó al instante todos los artefactos que llevaba encima. Empezó a echar espuma por la boca.

<<Ahora observa lo que hago con tu ciudad y los tuyos>> dijo la armadura, con un tono de voz lleno de malevolencia.

Lanzó al suelo a Swind, quien, a pesar de estar próximo a un estado de \emph{shock}, aún pudo ver cómo sus compañeros se lanzaban a por el enemigo, todos a una, conscientes de que no había engaño que mantener, pues posiblemente les había estado monitorizando desde el principio. Pronto pudo descubrir que no sólo era capaz de lanzar descargas eléctricas por contacto cercano y a distancia, sino que sus brazos podían adaptarse y remodelarse para ser sendos escudo y espada de energía, de unas dimensiones mucho mayores que las versiones de acero que solían llevar las armaduras clásicas.

Aparte de eso su tremenda fuerza, unida al terrible hecho, que no tardó en salir a la luz, de que también podía disparar descargas eléctricas letales, no tardaron en confirmar las peores sospechas de Swind.

Aquel enemigo no estaba equilibrando la balanza a su favor. Nunca habían tenido la menor oportunidad de mover siquiera el platillo de su oponente.\\

\noindent{}Cuando Scream llegó con un equipo al lugar de la contienda el espectáculo era poco menos que dantesco. Las paredes de los edificios, en su mayor parte abandonados, parecían un colador. Las armas de los soldados reposaban sobre el suelo, sobre el que había varios montones de ceniza. Scream contó cuatro en total. Cuatro bajas.

Inmóvil en el suelo, despidiendo vapor y oliendo a quemado, pero aún vivo, estaba Swind, en el mismo sitio donde había caído después de padecer aquella brutal electrocución.

---Me dejó vivo para\dots{} para contarlo, John ---decía esforzándose por hablar con claridad, pero emitiendo sólo balbuceos---. Él los mató. Dijo que se llamaba\dots{} Armor.

---Tranquilo Matt, lo entiendo. Tuviste que activar el mecanismo, y reducir su ropa a cenizas.

Matt le miró fijamente. Scream pudo ver en sus ojos el temor. Y entonces habló con una fluidez producto del más sincero pánico.

---No, John. Fue él quien los redujo a cenizas con sus propias manos.

\begin{next}
    ¿Quién es Armor? ¿Qué es lo que desea? Y sobre todo\dots{} ¿cómo puede ser detenido? ¡Nuestros héroes tendrán que jugar a los detectives para averiguar las respuestas a todas estas preguntas!
\end{next}

\endinput
