Hasta ese momento habían conocido múltiples y variados enemigos. Algunos de ellos temibles, perseverantes y ambiciosos. Pero eran conscientes de que no todos los peligros respondían a esos mismos patrones de sed de poder, destrucción o dominación.

Otros se movían por motivos mucho más personales\dots

\fancyparbreak
Bill Deacon era uno de esos sujetos cuya existencia había estado condenada prácticamente desde el primer instante en que comenzó. Abandonado por su padre sólo después de sufrir múltiples abusos por su parte, se fugó de casa a los doce años dejando atrás una madre alcohólica, un hogar mugriento que se caía a pedazos y un casero que les cobraba el alquiler en especies en vez de por medios monetarios. El tirón no tardó en ser el primer método que empleó para procurarse un sustento con el que llegar hasta el final del mes, aparte de establecer los límites de su zona de actuación, donde ningún otro ratero debería inmiscuirse a menos que quisiera recibir una puñalada por la espalda.

La policía le había atrapado muchas veces, por supuesto. Pero sus robos, eufemísticamente catalogados como hurtos, nunca eran tan cuantiosos como para meterle entre rejas, acaso clavarle una multa que jamás se molestaría en pagar.

Alguna vez Deacon había escuchado hablar de ese justiciero que patrullaba por las calles de Ernépolis. En su enajenación mental, creyéndose el amo y señor de sus sucios dominios, pensó que no le amedrentaría sujeto semejante.

Pero mientras corría por esos mismos callejones por los que no toleraba rivales, tropezando en las baldosas desgastadas y cayendo de boca en la ceniza dispersa sólo para levantarse de nuevo, aterrorizado y dolorido, comprendió que la realidad de una situación siempre supera a las expectativas que uno se hace de ella antes de vivirla.

Esa sombra se movía como un auténtico demonio venido del Averno. Parecía no pisar el suelo, y los gatos y otros bichos más desagradables huían despavoridos a su paso. No había piedad en aquellos ojos, ni posibilidad de escapar de esa silueta que parecía rodear y cubrir el callejón con su presencia etérea.

En circunstancias normales, si hubiera tenido un segundo para reflexionar, no hubiera cometido la estupidez de encerrarse a sí mismo en un callejón sin salida. Conocía aquella zona como la palma de su mano, pero con el pánico cristalizando por sus venas, todo esfuerzo por tomar decisiones meticulosas y razonadas con calma se antojaba poco menos que titánico por su parte.

Vio una escalera de emergencia que estaba a varios metros de altura sobre el nivel peatonal. No había manera de agarrarla sólo saltando, pero se dio cuenta de que si se subía en un cubo de basura tal vez eso bastaría para obtener la distancia necesaria.

Era un buen plan. Uno que tal vez hubiera funcionado de no ser por los nervios que agarrotaban todos sus músculos. Aparte de eso, obviando que tenía experiencia más que sobrada a la hora de correr para escapar con el bolso, cartera u otro bien ajeno sustraído, Deacon no era precisamente un atleta consumado, y menos en materia de salto.

Fue por eso que, aunque logró subir a tiempo a su escalón hacia la libertad, no calculó bien y fue a dar con sus huesos de nuevo sobre el frío pavimento.

Ya no había escapatoria. Allí estaba su perseguidor, callado, imperturbable. Por un momento se sintió como si fuera el callejón entero el que se estuviera echando sobre él, cobrando vida y clamando venganza por todos aquellos a los que había robado.

Trató de hablar, dialogar. Humillarse y lloriquear siempre podía ser una estrategia para sobrevivir.

~---Por favor, tío, no me hagas nada, déjame vivir, déjame vivir.

\emph{Demasiado tarde} ~---fue la única respuesta que recibió, en lo que le agarraban del cuello y le levantaban en vilo. Deacon no había sentido en la vida tacto tan gélido como aquel.

~---Robaré para ti, tío, lo que sea, pero déjame ir, por favor.

\emph{No me interesa tu oferta} ~---agregó, comenzando a apretar.

~---Por favor\dots\ por qué yo\dots\ ~---dijo el ladrón, perdiendo el conocimiento.

\emph{Fatalidad. Mala suerte. Pero principalmente\dots}

Los ojos de Deacon empezaron a moverse espasmódicamente, al tiempo que perdía color en el rostro y su lengua se retorcía hacia un lado.

\emph{\dots\ tenía interés en comprobar qué se siente al sembrar el terror en los criminales.}

Lanzó a Deacon al suelo, como un muñeco roto, y se deslizó hacia el exterior del callejón, sin la menor prisa por abandonarlo.

\emph{Y pronto ellos sentirán ese terror} ~---declaró en voz alta, como si declamara~---. \emph{Pronto estarán en el otro lado de la balanza.}

\parbreak
Después de la batalla contra Armor Omega hubo que realizar una dura inversión en la reconstrucción de las calles dañadas de la ciudad. Si bien el ejército ayudó en las labores de retirada de escombros pronto la opinión pública les obligó a marcharse, furiosos por haber sido parte de sus tropas los causantes de aquellos sustanciales destrozos. Los cazarrecompensas también ayudaron en la medida que pudieron, pero pronto empezaron a andar cortos de calderilla y no tardaron en largarse de allí con el sano propósito de llegar a fin de mes como malamente pudieran. Dobleseis y Códec permanecieron algo más de tiempo, pero lejos de tratarse de un asunto solidario, se debió más que otra cosa a que la nave de ambos tardó algo más de tiempo hasta que al menos fue capaz de remontar el vuelo.

El descontento poblaba la ciudad, si es que alguna vez no lo había hecho. Ya no sólo por la destrucción causada por la batalla, el proyecto aeroespacial fue trasladado a Talópolis~IV y, si bien el ciudadano de a pie no tenía el menor interés en conocer los pormenores científicos del mismo, de lo que sí era consciente era del plus económico que hubiera supuesto para las maltrechas arcas del dinero público.

Al menos la construcción de la autopista seguía avanzando, aunque el hecho de que en su tramo subterráneo se escondieran los paramilitares y la urgencia de reparar los daños retrasaron su esquema de desarrollo, y muchos llegaron a poner en entredicho que pudiera alguna vez llegar a ser una realidad.

Pero a pesar de todas las adversidades, y en gran parte debido a la insistencia personal del Presidente Scatter, que veía en ello un excelente acicate electoral, se logró completar un primer tramo de dos kilómetros que circulaba por encima del distrito financiero y unía gran cantidad de sedes oficiales de empresas entre sí. Aquello podía ser una buena noticia de cara a la recuperación de la ciudad, puesto que el precio del terreno en Ernépolis~I empezaba a estar por los suelos, y eso unido a una buena comunicación por tierra y aire podía animar a muchas empresas importantes a trasladar sus oficinas al centro económico de la ciudad.

En cuanto a lo que sucedía en el subsuelo de la ciudad, en el cuartel del Aquerón hacía tiempo que notaban que Scream estaba empezando a mostrar signos visibles de agotamiento. Estaba más que harto de criminales, psicópatas, manipuladores y disidentes que ponían en peligro de manera constante aquella endémica ciudad, y su cansancio era cada vez más difícil de ocultar. Él mismo pensaba a menudo que aquel era sólo un sentimiento pasajero que se le pasaría tarde o temprano. Pero comprobar cómo regresaban aquellos a los que creían haber logrado parar los pies no era lo mejor que podía sucederle para tratar de pensar de manera más optimista en el futuro.

El resto de los miembros de Los Caídos no era ajeno a la aflicción de Scream, ni siquiera los más jóvenes, quienes no cargaban con el peso de haber sido héroes fracasados. Sam Grove no hacía más que mirar a su líder, preocupado, planteándose si el peso de tener que dirigirles no estaría arruinando también su posibilidad de retomar una vida normal algún día. Muchos de los otros miembros de la organización, o no habían perdido del todo el contacto con su pasado, o lo habían logrado retomar. Otros habían decidido que podían hacer más por la ciudad desvinculándose por completo de sus compañeros de armas. Para John Scream nunca había existido tal oportunidad.

Shockman se acercó al chico y le observó con su ojo sano sin hacer ningún comentario. Warren Shockman que observaba a Sam Grove que observaba a John Scream que observaba en lo que se estaba convirtiendo la ciudad. Una cadena de eslabones condenados que podía prolongarse hasta el infinito. Finalmente Grove se dio la vuelta y rompió la transitividad acercándose a su vez al antiguo villano.

~---Nunca le había visto así ~---confesó.

~---Es lo que tienen los héroes, boy scout. Creen que tienen que cargar con el peso del mundo en sus hombros cuando a veces no son capaces ni de cargar con el suyo propio. Me largo. Me asfixio aquí dentro, es todo tan\dots\ rutinario ~---comentó tomando uno de los múltiples y poco iluminados pasillos del Aquerón.

Grove también tenía que salir a patrullar con su escuadrón, pero aún no era hora de que lo hiciera, por lo que él y los suyos se fueron a entrenar como calentamiento previo. A veces se planteaba cómo un sujeto tan independiente y problemático como Shockman podía estar con ellos y otras, sin embargo, lo veía como lo más lógico y normal del mundo, la única manera que tenía de reciclarse si no se contaba la muerte como alternativa, que era lo que esperaba a muchos de su calaña en el mundo en que se movían. Pero en el fondo sabía que Shockman no le temía a la muerte, e incluso secretamente la deseaba. Era un combatiente orgulloso, y no caer en combate sería para él, sin duda, el más duro de los castigos posibles.

Pero aun con todo seguía sorprendiéndose de que su negativa a aceptar las normas no les hubiera metido aún en ningún problema serio, o como poco a ser tenido en cuenta. El peso del pasado es algo muy complicado de soportar, pensó. Sólo esperaba no tener que cargar con una penitencia tan dura como la de aquellos que le rodeaban.

Una vez llegó el momento de que, según el esquema establecido, les tocara salir, Grove y sus compañeros se embutieron en sus trajes y enfilaron hacia los Túneles, tan conflictivos que a veces era posible hasta que dos escuadrones al mismo tiempo operaran en ellos sin que se comprometieran mutuamente. En todo caso, también era uno de los lugares de patrullaje predilecto de todos los miembros, y a menudo el campo de pruebas ideal para nuevas estrategias y modificaciones de dispositivos, la última línea de testeo antes de su implantación al resto de la ciudad.

Tras salir por uno de los múltiples pasadizos, astutamente solitario y apartado, comenzaron a vigilar la zona desde los tejados elevados. Los edificios por allí tampoco eran demasiado altos, con lo que en caso de ser necesitados abajo tardarían poco en descender, además de que en la medida de lo posible dos de los cinco miembros del escuadrón se desplazaban a menor altura, a veces de cornisa en cornisa, siempre adelantados para emplear el máximo sigilo en sus movimientos, ampararse en las sombras, tanto artificiales como naturales, y no romper la ilusión de ser muchos que parecían uno solo.

La noche se presentaba extrañamente tranquila y apacible en un lugar como aquel, que era poco menos que el caldo primitivo de la mayor parte de la podredumbre humana que luego se dedicaba a robar, matar y violar por otros rincones de la ciudad. Apenas un par de disputas a navajazos y un asalto con arma era todo lo que encontraron en su camino. Demasiado tranquilo, de hecho, no hacía más que pensar Grove. Como si los criminales ya tuvieran miedo antes incluso de que salieran a imponérselo.

El primer hecho extraño aconteció cuando vieron a un par de pandilleros correr por las calles a toda velocidad, aterrorizados, casi sin reparar uno en la presencia del otro. En cuanto se acercaron un poco más pudieron observar que eran de hecho de bandas rivales. Debía ser muy chungo aquello de lo que estuvieran escapando para no liarse a tiros allí mismo el uno con el otro.

Grove descendió y el resto de sus compañeros adoptaron la configuración inicial en un caso como aquel, delincuentes a la fuga por calles apartadas y poco transitadas. En cuanto Grove se plantó delante de ellos, aunque no lo sabían, ya estaban virtualmente rodeados por todos los flancos.

La segunda cosa extraña que sucedió es que prácticamente en cuanto le vieron, Grove no tuvo ni tiempo de soltar una frase amedrentadora y ya les tenía tratando de escapar como una exhalación en sentido contrario al de su avance. Como si de repente ya no hubiera nada de lo que estuvieran escapando al otro lado.

Como el protocolo mandaba, Grove desapareció de la vista, pasó a ser un actor secundario y, al otro lado, uno de los suyos ejerció el papel de principal. Uno de los dos tipos se tiró al suelo, histérico, chillando como un cerdo destripado. El otro trató de sacar un arma, pero le temblaban tanto las manos que se le cayó al suelo y prefirió huir a tratar de recogerla. Grove, desconcertado, no trató de detenerles. En ese momento no suponían amenaza alguna para nadie más que para sí mismos, en todo caso.

Se acercó al arma y la cogió para examinarla. El cargador estaba completo. Fuera lo que fuese el motivo de su huída les había cogido desprevenidos.

Para un héroe había dos opciones a partir de ese momento. Podían seguir a los tipos y sonsacarles de qué estaban huyendo, o podían ir a investigar la fuente de semejante comportamiento.

Para Los Caídos, lo bueno era que podían permitirse el lujo de tomar las dos decisiones al mismo tiempo.

Por supuesto, había que tomar las suficientes precauciones para que nadie se diera cuenta que estaban en dos lugares a la vez, y por ello los dos miembros del escuadrón que siguieron a los pandilleros lo hicieron camuflados, sólo esperando poder escuchar aquello que deseaban saber sin necesidad de una intervención directa.

Grove y dos compañeros más, entonces, optaron por ir al origen del problema. Lo cierto era que no estaba tan cerca como hubieran supuesto, si es que seguía esperándoles, lo que o bien quería decir que ya era tarde para localizarle, o bien aquellos dos habían estado corriendo durante bastante más de una o dos manzanas.

Pero no tardaron en localizar, en una calle estrecha y flanqueada por dos edificios muy próximos, los efectos de aquel de quien los pandilleros habían logrado escapar. Logrado, porque allí estaban los demás que no habían tenido tanta suerte en ese sentido.

Cuatro en total, dos de cada bando, lo que sonaba a que los seis estaban allí para comenzar una batalla campal por alguna clase de agravio real o imaginario. Uno de ellos estaba bocabajo entre un montón de basura sin recoger, y no le hizo falta a Grove más que tomarle el pulso para asegurarse de que estaba muerto. Con los demás no fue necesaria tal maniobra. Dos tenían la cabeza brutalmente aplastada contra la pared, habiendo dejado un charco allí donde se había producido el impacto, en la pared de ladrillos de uno de los edificios, llena de toda clase de rebabas y cemento mal secado que contribuyeron a que el impacto fuera infinitamente más doloroso y letal de lo que hubiera resultado sobre una superficie pulida. El tercero había sido, por lo que pudieron ver, asfixiado hasta morir. Sus brazos y piernas estaban retorcidos, y les fue imposible saber si aquello había sido fruto del intento de la víctima por liberarse, viéndose al borde de la muerte, o una macabra acción de su asesino\dots\ ya fuera post o perimortem.

En todo caso, pensó Grove, alguien muy furioso andaba suelto por la ciudad. De repente, uno de los otros dos miembros del escuadrón se comunicó con él a distancia.

\emph{Solicito vía libre para mostrarnos.}

\emph{Sin problemas, estamos solos} ~---fue la rápida respuesta de Grove. Supuso que se encontrarían en alguna situación en la que, para poder obtener alguna clase de información, no les quedaba más remedio que comenzar a actuar. En todo caso, dado el tono sereno con que habían mantenido la veloz conversación, nada parecía andar fuera de lo normal en su lado.

Una vez los cinco se reunieron de nuevo Grove pudo comprobar que el semblante de los dos separados mostraba, cuando menos, sorpresa e incredulidad.

\emph{¿Qué es lo que han dicho?} ~---preguntó.

\emph{Uno de ellos perdió el conocimiento casi al instante de que empezáramos a apretarle las tuercas. El otro no aguantó mucho más en pie, pero logramos sonsacarle que aquello lo había hecho una sola persona, antes siquiera de que se lo preguntáramos.}

\emph{¿Cómo fue que ocurrió eso?} ~---preguntó Grove, intrigado.

~---Más que nada porque estaba plenamente convencido de que había sido una criatura exactamente igual a nosotros ~---explicó con sencillez, quitándose el modulador.

\parbreak
Nada más comunicar Grove las noticias se convocó una reunión de urgencia en el Aquerón. Nunca antes se habían tenido que enfrentar nada similar.

Nunca antes habían tenido que llegar a sospechar que el peligro pudiera estar entre ellos mismos.

~---Se trata sin duda de un imitador ~---fue lo primero que Razorclaw quiso hacer notar~---. Alguien que trata de hacer lo mismo que nosotros, pero a su retorcida manera.

~---No sé qué es lo que me preocupa más, si que sea uno de nosotros o que nosotros hayamos sido el modelo para que un potencial psicópata reparta su propio concepto de justicia por la ciudad ~---dijo Swind tomando la palabra.

Los comentarios se sucedieron a toda prisa, uno detrás de otro, al igual que las preguntas de toda clase a Grove y su escuadrón. Había muchos cabos sueltos, y también muchas posibilidades a considerar.

~---Eran sólo pandilleros ~---explicó el joven director de escuadrón~---. Cualquiera de nosotros habría podido contra ellos.

~---Podría tratarse de un escuadrón al completo ~---agregó Swart.

~---¿Estás insinuando que un director de escuadrón y sus cuatro hombres han cruzado la línea hasta tales extremos de brutalidad delante de nuestras propias narices? ~---fue el airado comentario de Razorclaw. Swart no se amilanó por el ataque.

~---También está la posibilidad de que haya sido alguien de quien podríamos esperar algo así. Alguien que siempre va solo y no se caracteriza por ser precisamente calmado, colaborador ni abierto. Alguien que ahora mismo ni siquiera está aquí.

~---Está fuera, patrullando ~---explicó Grove.

~---Razón de más para sospechar de él.

~---Si le acusas a él tendrías que acusarnos también a nosotros ~---protestó Grove, dando un paso adelante.

~---Escucha, Sam, sé que le has cogido cariño a ese elemento, pero llevo más años que tú en esto y créeme, he perdido la cuenta de cuántas veces he comprobado cómo estos tipos acaban volviendo siempre a sus genuinas raíces.

La discusión empezó a ponerse cada vez más candente, y los murmullos no dejaban de cesar. Al fin, Scream habló.

~---Yo también he salido solo muchas veces, Jim, y he tenido ocasiones sobradas de hacer algo así. También debería ser considerado sospechoso.

~---Tu caso es distinto John, saliste solo por necesidad, para no comprometernos a los demás, y además tú eres\dots

~---¿El qué, Jim? ¿Un héroe? ¿Qué es un héroe, Jim? ¿Qué define la bondad en ellos? Un héroe, igual que un villano, es sólo un ser humano. Sufren, cometen errores y padecen muertes violentas. En eso somos muy parecidos. Yo también deseé matar muchas veces a Ellen Gorgon, ¿sabes? Cuando era un mendigo, cuando lo había perdido todo salvo el odio, fantaseaba a menudo con ello. Pero me sobrepuse a tal deseo. Por eso, si estás aquí hablando conmigo a pesar de que en mi interior se albergaron negros pensamientos, deseo que plantees que Shockman puede ser inocente aun a pesar de su complicado pasado.

Después de eso Scream salió de la habitación sin siquiera dar por concluida la reunión de manera explícita. Grove hubiera deseado decirle algo, pero no supo el qué. Al menos, pensó viéndole salir por la puerta principal del hemiciclo, los valores por los que tanto le admiraba seguían ahí, intocables, intactos.

\parbreak
En la ciudad, mientras tanto, Warren Shockman, conocido en el pasado como Éxeter, caminaba a su aire siendo totalmente ajeno a la polémica que sobre su persona se había levantado. Le comunicaron que había una reunión especial en ese mismo momento, pero como con muchas otras anteriores se limitó a ignorarla y aprovechar para patrullar completamente a su aire, sin nadie más que él en la ciudad en ese momento. Eran los momentos como aquel los que compensaban todas las estupideces que tenía que aguantar en aquella organización, como haber trabajado en los laboratorios al principio o tener que asumir órdenes de mandos superiores, por no hablar ya de castigos. Todo tenía su lado bueno, visto desde ese lado. A cambio de haber compartido sus poderes tenía nuevas habilidades para manejar y toda la impunidad del mundo para usarlas. Ese era, por otro lado, el problema de los héroes. Tenían todos los medios para detener a los criminales y ninguna de las intenciones de emplearlos a fondo. En todo caso, mientras estuviera con ellos aquel juego tendría que ser el suyo también, aunque siempre podía estirar los límites de las normas.

Más abajo de donde se encontraba se encontró con un tipo que parecía estar acosando a una chica ligera de ropa en una avenida solitaria. No era la clase de trabajo para lucirse, pero Shockman detestaba especialmente a los tipos que abusaban de su posición de poder, y aquel era sin duda un buen ejemplo de ello.

Por eso se preparó para descender cuando de repente, sin previo aviso, se sintió como si el mundo entero se hubiera oscurecido a su alrededor. Perdió la noción del equilibrio y empezó a notarse muy mareado, y justo antes de perder el conocimiento notó como si un frío gélido le recorriera todo el cuerpo.

Lo último que percibió justo antes de desmayarse fue cómo su rata salía del bolsillo a toda velocidad, como si acabara de ver al mismísimo Diablo.

\endinput
