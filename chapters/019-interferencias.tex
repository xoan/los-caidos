Tras la crisis, la reflexión. La ciudad planteándose a sí misma qué hacer a partir de ese momento. El tiempo de tomar decisiones importantes había llegado para muchas personas, y también el momento de disfrutar de lo que se tiene, de reclamar a la vida lo que uno desea de ella.

Sobre todo porque nunca se sabe cuándo podía comenzar una nueva espiral de locura\dots

\fancyparbreak
No resultó ningún misterio que en Ernépolis~I empezara a crearse un sentimiento de animadversión social hacia el Caído, aquella sombra que hasta hacía no mucho había empezado a ser considerada como protectora de la urbe. Voces de protesta se alzaron desde muchos sectores, y dado que sus quejas no podían hacer mella aparente en un ser que parecía estar más allá de rencillas y asuntos humanos, fueron dirigidas hacia los poderes establecidos que con su permisividad parecían estar siguiendo el juego a aquel renegado de la justicia.

Voces muy radicales, en algunos casos, que reclamaban dimisiones, uso prolongado de la fuerza e implantación de toques de queda. Voces que habían permanecido extrañamente calladas cuando aquel ser penúmbreo desmanteló pieza por pieza el imperio de Ellen Gorgon, y que sin embargo se alzaban como un coro ante la amenaza que parecía cernirse sobre los pilares del mundo en el que creían sin reservas, al margen de que estuviera comandado por héroes o villanos.

Uno de los cargos con el que fueron especialmente virulentas fue con el del Jefe de Policía. James Sky había tenido que hacer frente en el pasado a acusaciones que le tildaban de advenedizo traidor hacia el equipo del anterior Jefe, Brian Wolf, así como de permitir que los rescoldos de la Guerra de Las Ocho Colonias, en la forma de un letal experimento militar, se pasearan por la ciudad. Lo último en echársele en cara fue su fracaso en impedir ni más ni menos que la destrucción de un rascacielos por un peligroso fundamentalista del que poco o nada pudieron averiguar para detenerle. Como si aquellas cosas pasaran por vez primera en la historia de la humanidad, y fueran inéditas para las mentes inocentes de sus contemporáneos.

Pero sí eran inéditas, en cierto modo. Lo eran porque habían pasado en otro lugar, en otro tiempo, en otros mundos. El horror nunca llegaba a casa. La guerra se desarrollaba en selvas plagadas de monstruos alienígenas inimaginables, las explosiones siempre suceden demasiado lejos para poder escucharse con nitidez, distanciadas lo suficiente como para que se llegue a dudar de su existencia.

Pero no había duda alguna en lo que había pasado. Ernépolis había sido atacada con ferocidad por todos los frentes posibles. En términos del crimen organizado, en términos de corrupción, en términos de escaramuzas bélicas, en términos de terrorismo a escala global. Hasta la misma Nube había descendido temporalmente para tomar posesión de ella.

Pero aun así, Ernépolis no se rendía. Ernépolis quería vivir.

Quítale todo lo que tiene a un ser humano y éste desarrollará un instinto y una fuerza más allá de toda elucubración previa. Ésta era una premisa que Los Caídos conocían muy bien, y no hacía más que evidenciar que debían seguir cumpliendo con su misión. Porque a pesar de las penurias, de los problemas, la ciudad seguía adelante. Y no lo hacía porque unos cuantos líderes de palabras huecas la estuvieran levantando con el esfuerzo de sus medidas y promesas, sino porque era la propia gente, los ladrillos que la mantenían en pie, los que lo estaban haciendo a título individual.

Es por eso que a nadie le extrañó que el ocio invadiera el mercado de Ernépolis. La gente quería disfrutar. La gente quería pasárselo bien. Ya no sólo imperaban las salas clandestinas de juego, ni las grandes fiestas donde el alcohol corría entre manos adineradas como la sangre en una cruenta batalla. Ernépolis era un terreno por explorar, donde ofrecer múltiples ofertas de ocio. Siempre había sido fundamentalmente exportadora. Era el momento de importar. No piezas, ni productos de primera necesidad, sino aquello que más demandaba y solicitaba: una oportunidad para olvidar.

Sam Grove era uno de los que deseaba aferrarse a esa oportunidad. A pesar de ser miembro de Los Caídos su reclutamiento fue posterior a la era de Gorgon, y de hecho aún seguía siendo relativamente novato. Los más veteranos le decían que disfrutara de esa inocencia mientras pudiera, mientras aún no hubiera experimentado demasiado.

Demasiado, pensó con ironía. Bien cierto era que no había vivido las épocas en las que se añadía a las aleaciones de metal fundido al que hablaba más de la cuenta, y la desaparición de los grandes héroes del pasado fue algo que sólo vio a través de periódicos y en escuetas noticias. Pero ya había presenciado bastante como para tener una idea clara de a qué se estaban refiriendo. En su currículo figuraba un ser artificial capaz de exterminar batallones enteros él solito, además de un tirano egocéntrico eternamente invisible, una depredadora deforme con la habilidad de parecer la criatura más hermosa existente y una asesina de manos desnudas pero que siempre tenía el truco perfecto oculto a la vista. Era bastante como para pasar por fin el estatus de novato, sobre todo porque a algunos de esos seres los había conocido de primera mano. Evidentemente nunca podría equipararse con la experiencia del Capitán Scream, que había peleado en persona con aquellos demonios con aspecto humano o casi humano, pero al menos sí era digno de ser mencionado.

Pero no. Grove seguía siendo el novato, el que tenía aún mucho que aprender, el que no había conocido a Starr Miles. A pesar de que había ascendido a director de escuadrón, de todo lo que había demostrado conocer.

Supuso que el respeto en Los Caídos sólo se ganaba con sufrimiento personal. Pero de eso, para su fortuna, aún no sabía lo suficiente.

Grove era joven, también. Y como tal era uno más de los ciudadanos de Ernépolis que tenía el derecho y las ganas de disfrutar esa efimera juventud, antes de que otro Armor ú otro Hades vinieran a estropeársela. A veces, de hecho, se preguntaba si no era una maldición pertenecer a Los Caídos. Si no hubiera sido mejor que no le hubieran reclutado.

\emph{Tengo interés en ti.}

Esas fueron las palabras que la sombra le dirigió aquel día, en aquel callejón. Palabras que pudo haber ignorado ú olvidado. Palabras que, luego lo supo, no se le decían a cualquiera. Y menos de boca del propio John Scream en persona.

Pero eso ya era el pasado. Lo hecho, hecho estaba. Eligió una rama del árbol de su destino, y las otras se podaron al instante. Por delante sólo tenía las ramificaciones de aquella en la que estaba posado.

Con todo lo malo, Grove era un chico jovial. Un tipo con ganas de exprimir la vida. Y por ese motivo fue el primero en alegrarse cuando se enteró de que The Jammers iban a irrumpir en la ciudad.

Si le hubieran preguntado a John Scream o a James Sky por The Jammers hubiese sido bastante poco lo que podrían haber contestado. En el caso de Grove, la información era bastante más completa. No es que se tratara del grupo más famoso del sistema en ese momento, pero había alcanzado cierta notoriedad en el mundillo underground de la música electrónica.

Lo primero interesante de The Jammers era que nadie había visto a The Jammers. Nadie al menos en círculos más allá de lo oficial. Decían que ellos mismos se habían conocido a distancia al principio y poco a poco fueron construyendo su música por medio de la acumulación, primero componiendo unos, luego añadiendo otros, enviando, recibiendo, mezclando pistas, remezclando, discutiendo, tanteando, en un proceso, cuanto menos, largo y costoso pero también curioso pues de ese modo la música era muy distinta a la alternativa estándar de discutir los arreglos en persona.

Ni siquiera sus nombres eran asunto conocido. Usaban seudónimos que reflejaban bastante bien el estilo que les definía. A la guitarra eléctrica, ese misterioso componente de una banda que a veces centraba las miradas más que el cantante, estaba el ecléctico y minimalista Overdrive. El bajo, siempre injustamente olvidado pero esencial en una formación, el técnicamente perfecto Delay. De los teclados y fraseos ocasionales se encargaba Echo, con su estilo denso y cargado de acordes profundos y góticos.

Fase estaba al mando de la batería, realizando curiosos e interesantes duelos rítmicos con Echo en muchas ocasiones. Y finalmente, a las voces, el más misterioso de todos los componentes, el grave y tremendista Distorsión, con su voz fluctuante que hubiera hecho las delicias de David Gahan y su arcaico grupo conocido como Depeche Mode.

El estilo de The Jammers era cuanto menos peculiar y muy apreciado por muchos colectivos de fans distintos gracias a que entre ellos mismos profesaban influencias muy dispares, que traían consigo desde Lacrimosa a Heaven Denied, pasando por Warreh Spawn, Disaster Area y Recoil. Su música era sencilla, clara y de mensaje directo: electricidad sumada a emociones magnificadas. Una combinación que había traído consigo múltiples hits en el pasado lejano y no tan lejano, y legiones incondicionales de seguidores que sentían que esa clase de sonido reflejaba la parte más oscura y siniestra que albergaba su propio interior.

Grove comprendía bien ese sonido y sus amigos también. Aunque, pensó, por motivos muy distintos.

El concierto que The Jammers iba a ofrecer en Ernépolis~I era el primero que darían oficialmente en toda su trayectoria, lo cual hizo que hubiera no poca expectación. Iba a ser poco menos que el acontecimiento de moda de la temporada entre los fanáticos del electrorock, aquello que uno no podía perderse si quería contar algo interesante a sus hijos o sacar un tema de conversación para ligarse a la chica de turno.

En este último aspecto Grove ya tenía los deberes hechos y fue al concierto con Ellie, la que había sido colega del grupo de toda la vida y no tardó en pasar de novia ocasional a más o menos permanente. Al menos estaban bastante más unidos de lo que cabía esperar por aquel entonces de una relación a temprana edad.

~---¿Crees que tocarán Black Lips? ~---preguntaba Ellie, indudablemente entusiasmada por el inminente acontecimiento que estaban a punto de presenciar. The Jammers habían escogido uno de los estadios más grandes de la ciudad para darse a conocer, y como teloneros estaban tocando los Bebop Cowboys, otra de las formaciones de moda del momento.

~---Eso espero ~---añadió Grove contemplando a Ellie, y lo guapa que estaba con aquella falda y medias rotas que tan bien le quedaban~---. Aunque espero que hagan la versión del tema Mysterons que en su momento les dio a conocer.

~---¿La de Portishead? Estás hecho un anticuado ~---bromeó ella cogiéndole de las manos y besándole. Los conciertos creaban cierto magnetismo animal en los presentes, y resultaban ser un deshinibidor de emociones mucho más potente y menos arriesgado que cualquiera de las drogas legales y no legales que circulaban por la ciudad con tanta fluidez como el viento de madrugada entre calles solitarias.

Los teloneros no tardaron en terminar de explotar sus minutos de gloria a la sombra de los verdaderos héroes de la noche, y tras unos minutos de tensa espera, los primeros acordes de Black Lips empezaron a sonar. El público estaba completamente entregado, y la pareja se dirigió de manera instintiva hacia el escenario. Por algún lado estaban los amigos de Grove disfrutando también del concierto, pero habían quedado en verse después. Aquel era un momento para disfrutar junto a Ellie, olvidarse de patrullas a altas horas de la mañana en calles tenebrosas y cubiertas por un manto de ceniza.

Los primeros versos fraseados de la canción no se hicieron esperar. Pronto los asistentes se dieron cuenta de que por detrás de la densa niebla del escenario los componentes del grupo ya estaban en sus puestos de combate, dispuestos a hacer vibrar los amplificadores al límite de su potencia. Y para empezar, habían elegido un tema que era puro ritmo y entrega en cada una de sus estrofas, corto pero intenso en su concepción.

\begin{verse}
    \begin{em}
        Beneath every church of dust\\
        Beneath every soundless wind\\
        Beneath every path to void\\
        I abandoned the corruption for your holy sight
    \end{em}
\end{verse}

Mucha gente, después de asistir al concierto, fue preguntada sobre qué les había parecido ese tema en directo. Gran parte de ellos fueron incapaces de contestar, pues al mismo tiempo que el tema era interpretado, pudieron ver por primera vez a sus creadores en carne y hueso, y eso obnubiló por completo su atención.

Ya muchos lo sospechaban, pero Overdrive era un alienígena. Llevaba una guitarra de dos mástiles, las mismas que Jimmy Page popularizó, pero con la salvedad de que su brazo izquierdo acababa en dos manos y podía extraer de ella toda una armonía de sonidos nunca antes imaginada que explicaba muchas cosas de su estilo.

Delay era humano, un tipo de aspecto curioso, con el detalle de que llevaba unas gafas de piloto espacial colgando del cuello, y llevaba también mitones como los de los combatientes en la Guerra de las Ocho Colonias. Echo, como todos ya sabían gracias a sus fraseos, era una chica. Su detalle más característico era una gorra de visera que tapaba su pelo largo y castaño, y en la que ponía Balamb Garden, otro de sus grupos favoritos. De Fase lo más destacable era el tatuaje que atravesaba todo su brazo derecho, difícil de leer dado que no paraba de moverlo arriba y abajo, baqueta en mano, pero que luego en fotos pudo verse que ponía \textsc{Ídolo Binario}.

Distorsión, sin embargo, era el que acaparaba los flashes de todas las cámaras. Llevaba una cazadora y su rostro estaba cubierto por un holograma que imitaba la nieve de una antigua televisión estropeada, casero y sencillo de elaborar pero más que efectista sobre un escenario. Ya era bien conocida su obsesión por la imagen y el anonimato, pero no se pensó que pudiera llevar esa idea hasta el extremo de llegar a ocultar su propia cara ante miles de personas, y que no se supiera con claridad si él era humano o alien.

El concierto no fue especialmente largo debido a que el repertorio de The Jammers constaba de un único disco, aunque ofrecieron varias versiones, entre ellas la que Grove esperaba escuchar, y una canción inédita llamada Reset. Tuvieron, además, el detalle de invitar de nuevo al escenario a los Bebop Cowboys para tocar juntos una última canción sobre el escenario.

Cuando el concierto terminó Grove esperó un rato hasta que se hubo despejado la zona y después salieron con calma del lugar. Tras reunirse él y Ellie con el resto del grupo, fueron a tomar una copa al primer local no demasiado masificado que encontraron y allí estuvieron charlando acera de la actuación y el misterio que rodeaba a aquel peculiar conjunto musical.

~---He escuchado decir que se tapan el rostro porque tienen cuentas pendientes con la justicia ~---comentó Roy, uno de los amigos más cercanos de Grove.

~---Entonces han venido al sitio adecuado. Aquí todo el mundo tiene algún asunto pendiente, de una u otra manera ~---agregó con ironía Ellie tras paladear su bebida.

Grove estaba algo abstraído, debido principalmente a que en breve tendría que excusarse diciendo que debía marcharse ya, pues llegaba el momento de sacar a los chicos a patrullar. De hecho él mismo había solicitado aquella zona dado que sabía que estaría por allí, al igual que los suyos, obligados también a dejar a medias una noche de ocio y disfrute en la siempre convulsa ciudad de Ernépolis~I.

Su mirada se posó en el televisor del local, completamente lleno de nieve. En Ernépolis aún era común que existieran muchos aparatos de señal analógica, debido a la cantidad de problemas en la red de comunicaciones que provocaba la Nube.

~---Parece que alguien está homenajeando al cantante de The Jammers ~---bromeó Roy, mirando hacia donde Grove llevaba tiempo haciéndolo.

Como siempre sucede cuando uno se lo está pasando bien el tiempo se le pasó volando a Grove y no tardó en llegar la hora de ahuecar el ala. No tuvo que elaborar demasiado su excusa, dado que a menudo tenía que ausentarse antes de lo debido, y de ese modo pronto estuvo listo para lanzarse de lleno a la acción.

Se acercó a una de las múltiples entradas a los pasillos del Aquerón y, tras emplearla de ese modo que sólo Los Caídos conocían, estuvo resguardado y a salvo de miradas inquisitivas. Avanzó por las ramificaciones del pasillo hasta que llegó a su destino, uno de los subcuarteles que en su momento perteneció a otro de tantos héroes desaparecidos, en los que había equipo suficiente para que un escuadrón se lanzara a la acción. Como era menester en un director llegó el primero, aunque el resto del equipo no se hizo esperar. Cuando los demás llegaron no se dijeron demasiadas palabras. Todos, en su fuero interno, estaban disgustados por tener que dejar atrás las vidas que tanto esfuerzo les estaba costando disfrutar. Otra actitud indisciplinada de los novatos que algunos veteranos desdeñaban en ocasiones. Aun así, antes de salir al exterior hubo algunos momentos dedicados a la charla informal sobre el concierto mientras se ponían la gabardina, el sombrero, preparaban los anuladores de fotones y probaban los moduladores de voz. Como si aquello fuera, realmente, un trabajo más.

Después de eso, una vez salieron al exterior, ya eran otra cosa. Eran un solo hombre, una sombra. Un ser que se deslizaba, oscuro y tremendo, por los dobleces de la realidad.

Grove no tardó en avistar problemas. Atraco a mano armada. Sujeto con arma, chica amenazada. Aquella escoria era como la ceniza del pavimento. Nunca desaparecía, por mucha que quitaras.

~---Preparados para descender ~---dijo Grove.

~---\dots\ rados ~---le contestaron.

~---No te capto.

~---Preparados ~---le dijeron de nuevo.

Grove descendió y se puso frente al ladrón. La parte divertida de ser director de escuadrón, al menos, era que se podían ejercitar las dotes teatrales en el desempeño de la labor.

El plan era sencillo. Elaborarían una clásica maniobra de frase cortada. Empezar en un lado, acabar en otro. Desconcertante y efectivo con la baja ralea.

\emph{Si no te alejas de mis calles\dots} ~---empezó Grove. Justo después, gracias a un holo distorsionador, desapareció del punto de vista de su compañero.

Su segundo se dejó ver al fondo del callejón, siendo la ilusión general parecida a como si se hubiera transportado. Sin embargo, no habló. No dijo una sola palabra, y la frase comenzada por Grove quedó a medias en el callejón.

~---¿Qué pasa, tío? ¿Se te ha comido la lengua el gato? ~---fue la respuesta del matón, envalentonado. Sacó el arma, y disparó.

En otras circunstancias, nada habría pasado en ese momento. Todo estaba medido al milímetro, y esa clase de actitudes siempre eran habituales entre los que se enfrentaban. El problema de base fue que falló la coordinación, y eso desestabilizó al escuadrón por completo.

En concreto, la bala dio en el hombro del segundo al mando de Grove, y por un momento trastabilló, sin llegar a caer al suelo. Al menos, algo no falló en la planificación, y era lo que cada hombre había aprendido individualmente. Pues no gritó, ni se tapó la herida con la mano. Se limitó a seguir avanzando, en silencio, hacia su atacante.

La imagen, vista desde fuera, resultaba perturbadora, pero la tensión era palpable entre los miembros del escuadrón. Porque si aquello fallaba, si al sujeto de gatillo feliz le daba por disparar de nuevo, no habría segunda oportunidad de fingir nada.

\emph{Cubridle con un holograma. Sustituidle. ¿Me oís?}

Pero los demás no le escuchaban. Las comunicaciones no funcionaban y nadie se atrevía a hacer nada pues, si dos miembros del escuadrón tenían la desgracia de coincidir al actuar, se desvelaría el pastel al completo.

La sombra se acercó hacia el sujeto, muy lentamente, como si no tuviera prisa. El ladrón levantó, tembloroso, el arma. La chica estaba tirada en el suelo, casi más temerosa de su salvador que de su atacante.

~---Te he disparado, tío. Te he disparado ~---fue todo lo que dijo el criminal, negando con la cabeza.

La sombra le cogió del cuello y el arma cayó al suelo lleno de ceniza. Aunque no se pudo escuchar entonces, cuatro voces suspiraron al unísono.

\emph{Pronto desearás no haberlo hecho} ~---fue su única declamación antes de noquearle, cubrirle con la gabardina y desaparecer como un escapista de su orquestada prisión.

Grove trató de ponerse en contacto con la policía para llevarles hacia el sujeto y así no llenar el Aquerón de criminales de poca monta, pero fue imposible hablar con ellos y tuvieron que llevárselo por los tenues pasillos.

La herida de su compañero era bastante fea. Perdía sangre por montones. Un escuadrón tendría que ir luego a limpiar las huellas del encuentro, ya que las criaturas inmortales no deben sangrar.

Cuando el escuadrón llegó al Aquerón lo primero que Grove hizo fue llevar corriendo a su soldado a que le trataran de inmediato. Nada más llegar examinaron la herida y le pusieron a Grove la mano en el hombro.

~---Lo has hecho bien ~---le explicaron~---. Un poco más y hubiera entrado en estado crítico.

Pero Grove no sentía que lo hubiera hecho bien. Había cometido un error. Un error terrible de coordinación que le hubiera podido costar todo a Los Caídos. Pidió al resto de su escuadrón que le dejara solo y se sentó a reflexionar sentado en una esquina del hemiciclo, en el módulo central.

No tardó en acercársele una figura más que familiar.

~---He fallado, señor. Le he decepcionado.

~---No, Sam. No es así, y aunque así hubiera sido, no tendrías nada de lo que avergonzarte.

~---¿Por qué me miente, señor? He cometido un error imperdonable, no revisar las comunicaciones.

~---Sam\dots, tu escuadrón ha tenido la desgracia de ser el primero en padecer algo que no podíamos predecir.

~---¿Qué quiere decir, señor?

~---Ahora mismo lo sabrás ~---dijo con calma, tomando asiento a su lado.
